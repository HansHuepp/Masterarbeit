% !TeX spellcheck = en-US
% !TeX encoding = utf8
% !TeX program = pdflatex
% !BIB program = biber
% -*- coding:utf-8 mod:LaTeX -*-


% vv  scroll down to line 200 for content  vv


\let\ifdeutsch\iffalse
\let\ifenglisch\iftrue
% EN: This file is loaded before the \documentclass command in the main document

% EN: The following package allows \\ at the title page
%     For more information see https://github.com/latextemplates/scientific-thesis-cover/issues/4
\RequirePackage{kvoptions-patch}

\ifenglisch
  \PassOptionsToClass{numbers=noenddot}{scrbook}
\else
  %()Aus scrguide.pdf - der Dokumentation von KOMA-Script)
  %Nach DUDEN steht in Gliederungen, in denen ausschließlich arabische Ziffern für die Nummerierung
  %verwendet werden, am Ende der Gliederungsnummern kein abschließender Punkt
  %(siehe [DUD96, R3]). Wird hingegen innerhalb der Gliederung auch mit römischen Zahlen
  %oder Groß- oder Kleinbuchstaben gearbeitet, so steht am Ende aller Gliederungsnummern ein
  %abschließender Punkt (siehe [DUD96, R4])
  \PassOptionsToClass{numbers=autoendperiod}{scrbook}
\fi

% Warns about outdated packages and missing caption declarations
% See https://www.ctan.org/pkg/nag
\RequirePackage[l2tabu, orthodox]{nag}

%DE: Neue deutsche Trennmuster
%    Siehe http://www.ctan.org/pkg/dehyph-exptl und http://projekte.dante.de/Trennmuster/WebHome
%    Nur für pdflatex, nicht für lualatex
\RequirePackage{ifluatex}
\ifluatex
  % do not load anything
\else
  \ifdeutsch
    \RequirePackage[ngerman=ngerman-x-latest]{hyphsubst}
  \fi
\fi

\documentclass[
  % fontsize=11pt is the standard
  a4paper,  % Standard format - only KOMAScript uses paper=a4 - https://tex.stackexchange.com/a/61044/9075
  twoside,  % we are optimizing for both screen and two-side printing. So the page numbers will jump, but the content is configured to stay in the middle (by using the geometry package)
  bibliography=totoc,
  %               idxtotoc,   %Index ins Inhaltsverzeichnis
  %               liststotoc, %List of X ins Inhaltsverzeichnis, mit liststotocnumbered werden die Abbildungsverzeichnisse nummeriert
  headsepline,
  cleardoublepage=empty,
  parskip=half,
  %               draft    % um zu sehen, wo noch nachgebessert werden muss - wichtig, da Bindungskorrektur mit drin
  draft=false
]{scrbook}
% !TeX encoding = utf8
% -*- coding:utf-8 mod:LaTeX -*-

% EN: This file includes basic packages and sets options. The order of package
%     loading is important

% DE: In dieser Datei werden zuerst die benoetigten Pakete eingebunden und
%     danach diverse Optionen gesetzt. Achtung Reihenfolge ist entscheidend!


% EN: Styleguide:
% - English comments are prefixed with "EN", German comments are prefixed with "DE"
% - Prefixed headings define the language for the subsequent paragraphs
% - It is tried to organize packages in blocks. Bocks are separated by two empty lines.

% DE: Styleguide:
%
% Ein sehr kleiner Styleguide. Packages werden in Blöcken organisiert.
% Zwischen zwei Blöcken sind 2 Leerzeilen!


% EN: Enable copy and paste of text from the PDF
%     Only required for pdflatex. It "just works" in the case of lualatex.
%     mmap enables mathematical symbols, but does not work with the newtx font set
%     See: https://tex.stackexchange.com/a/64457/9075
%     Other solutions outlined at http://goemonx.blogspot.de/2012/01/pdflatex-ligaturen-und-copynpaste.html and http://tex.stackexchange.com/questions/4397/make-ligatures-in-linux-libertine-copyable-and-searchable
%     Trouble shooting outlined at https://tex.stackexchange.com/a/100618/9075

\ifluatex
\else
  \usepackage{cmap}
\fi


% EN: File encoding
% DE: Codierung
%     Wir sind im 21 Jahrhundert, utf-8 löst so viele Probleme.
%
% Mit UTF-8 funktionieren folgende Pakete nicht mehr. Bitte beachten!
%   * fancyvrb mit §
%   * easylist -> http://www.ctan.org/tex-archive/macros/latex/contrib/easylist/
\ifluatex
  % EN: See https://tex.stackexchange.com/a/158517/9075
  %     Not required, because of usage of fontspec package
  %\usepackage[utf8]{luainputenc}
\else
  \usepackage[utf8]{inputenc}
\fi


% DE: Parallelbetrieb tex4ht und pdflatex

\makeatletter
\@ifpackageloaded{tex4ht}{
  \def\iftex4ht{\iftrue}
}{
  \def\iftex4ht{\iffalse}
}
\makeatother


% EN: Mathematics
% DE: Mathematik
%
% DE: Viele Mathematik-Sachen. Siehe https://texdoc.net/pkg/amsmath
%
% EN: Options must be passed this way, otherwise it does not work with glossaries
% DE: fleqn (=Gleichungen linksbündig platzieren) funktioniert nicht direkt. Es muss noch ein Patch gemacht werden:
\PassOptionsToPackage{fleqn,leqno}{amsmath}
%
% DE: amsmath Muss nicht mehr geladen werden, da es von newtxmath automatisch geladen wird
% \usepackage{amsmath}


%% EN: Fonts
%% DE: Schriften
%%
%% !!! If you change the font, be sure that words such as "workflow" can
%% !!! still be copied from the PDF. If this is not the case, you have
%% !!! to use glyphtounicode. See comment at cmap package


% EN: Times Roman for all text
\ifluatex
  \RequirePackage{amsmath}
  \RequirePackage{unicode-math}
  \setmainfont{TeX Gyre Termes}
  \setmathfont{texgyretermes-math.otf}
  \setsansfont[Scale=.9]{TeX Gyre Heros}
  \setmonofont[StylisticSet={1,3},Scale=.9]{inconsolata}
\else
  \RequirePackage{newtxtext}
  \RequirePackage{newtxmath}
  % EN: looks good with times, but no equivalent for lualatex found,
  %     therefore replaced with inconsolata
  %\RequirePackage[zerostyle=b,scaled=.9]{newtxtt}
  \RequirePackage[varl,scaled=.9]{inconsolata}

  % DE: Symbole
  % unicode-math scheint für die meisten schon etwas anzubieten
  %
  %\usepackage[geometry]{ifsym} % \BigSquare

  % EN: The euro sign
  % DE: Das Euro Zeichen
  %     Fuer Palatino (mathpazo.sty): richtiges Euro-Zeichen
  %     Alternative: \usepackage{eurosym}
  \newcommand{\EUR}{\ppleuro}
\fi


% DE: Noch mehr Symbole
%\usepackage{stmaryrd} %fuer \ovee, \owedge, \otimes
%\usepackage{marvosym} %fuer \Writinghand %patched to not redefine \Rightarrow
%\usepackage{mathrsfs} %mittels \mathscr{} schoenen geschwungenen Buchstaben erzeugen
%\usepackage{calrsfs} %\mathcal{} ein bisserl dickeren buchstaben erzeugen - sieht net so gut aus.

% EN: Fallback font - if the subsequent font packages do not define a font (e.g., monospaced)
%     This is the modern package for "Computer Modern".
%     In case this gets activated, one has to switch from cmap package to glyphtounicode (in the case of pdflatex)
% DE: Fallback-Schriftart
%\usepackage[%
%    rm={oldstyle=false,proportional=true},%
%    sf={oldstyle=false,proportional=true},%
%    tt={oldstyle=false,proportional=true,variable=true},%
%    qt=false%
%]{cfr-lm}

% EN: Headings are typset in Helvetica (which is similar to Arial)
% DE: Schriftart fuer die Ueberschriften - ueberschreibt lmodern
%\usepackage[scaled=.95]{helvet}

% DE: Für Schreibschrift würde tun, muss aber nicht
%\usepackage{mathrsfs} %  \mathscr{ABC}

% EN: Font for the main text
% DE: Schriftart fuer den Fliesstext - ueberschreibt lmodern
%     Linux Libertine, siehe http://www.linuxlibertine.org/
%     Packageparamter [osf] = Minuskel-Ziffern
%     rm = libertine im Brottext, Linux Biolinum NICHT als serifenlose Schrift, sondern helvet (von oben) beibehalten
%\usepackage[rm]{libertine}

% EN: Alternative Font: Palantino. It is recommeded by Prof. Ludewig for German texts
% DE: Alternative Schriftart: Palantino, Packageparamter [osf] = Minuskel-Ziffern
%     Bitte nur in deutschen Texten
%\usepackage{mathpazo} %ftp://ftp.dante.de/tex-archive/fonts/mathpazo/ - Tipp aus DE-TEX-FAQ 8.2.1

% DE: Schriftart fuer Programmcode - ueberschreibt lmodern
%     Falls auskommentiert, wird die Standardschriftart lmodern genommen
%     Fuer schreibmaschinenartige Schluesselwoerter in den Listings - geht bei alten Installationen nicht, da einige Fontshapes (<>=) fehlen
%\usepackage[scaled=.92]{luximono}
%\usepackage{courier}
% DE: BeraMono als Typewriter-Schrift, Tipp von http://tex.stackexchange.com/a/71346/9075
%\usepackage[scaled=0.83]{beramono}

% EN: backticks (`) are rendered as such in verbatim environments.
%     See following links for details:
%     - https://tex.stackexchange.com/a/341057/9075
%     - https://tex.stackexchange.com/a/47451/9075
%     - https://tex.stackexchange.com/a/166791/9075
\usepackage{upquote}

% EN: For \texttrademark{}
\usepackage{textcomp}

% EN: name-clashes von marvosym und mathabx vermeiden:
\def\delsym#1{%
  %  \expandafter\let\expandafter\origsym\expandafter=\csname#1\endcsname
  %  \expandafter\let\csname orig#1\endcsname=\origsym
  \expandafter\let\csname#1\endcsname=\relax
}

%\usepackage{pifont}
%\usepackage{bbding}
%\delsym{Asterisk}
%\delsym{Sun}\delsym{Mercury}\delsym{Venus}\delsym{Earth}\delsym{Mars}
%\delsym{Jupiter}\delsym{Saturn}\delsym{Uranus}\delsym{Neptune}
%\delsym{Pluto}\delsym{Aries}\delsym{Taurus}\delsym{Gemini}
%\delsym{Rightarrow}
%\usepackage{mathabx} - Ueberschreibt leider zu viel - und die \le-Zeichen usw. sehen nicht gut aus!


% EN: Modern font encoding
%     Has to be loaded AFTER any font packages. See https://tex.stackexchange.com/a/2869/9075.
\ifluatex
\else
  \usepackage[T1]{fontenc}
\fi
%


% EN: Character protrusion and font expansion. See http://www.ctan.org/tex-archive/macros/latex/contrib/microtype/
% DE: Optischer Randausgleich und Grauwertkorrektur

\usepackage[
  babel=true, % EN: Enable language-specific kerning. Take language-settings from the languge of the current document (see Section 6 of microtype.pdf)
  expansion=alltext,
  protrusion=alltext-nott, % EN: Ensure that at listings, there is no change at the margin of the listing
  final % EN: Always enable microtype, even if in draft mode. This helps finding bad boxes quickly.
        %     In the standard configuration, this template is always in the final mode, so this option only makes a difference if "pros" use the draft mode
]{microtype}


% EN: \texttt{test -- test} keeps the "--" as "--" (and does not convert it to an en dash)
\DisableLigatures{encoding = T1, family = tt* }

% DE: fuer microtype
% DE: tracking=true muss als Parameter des microtype-packages mitgegeben werden
% DE: Deaktiviert, da dies bei Algorithmen seltsam aussieht

%\DeclareMicrotypeSet*[tracking]{my}{ font = */*/*/sc/* }%
%\SetTracking{ encoding = *, shape = sc }{ 45 }
% DE: Hier wird festgelegt,
%     dass alle Passagen in Kapitälchen automatisch leicht
%     gesperrt werden.
%     Quelle: http://homepage.ruhr-uni-bochum.de/Georg.Verweyen/pakete.html
%    Deaktiviert, da sonst "BPEL", "BPMN" usw. wirklich komisch aussehen.
%     Macht wohl nur bei geisteswissenschaftlichen Arbeiten Sinn.


% EN: amsmath teaks


% EN: Fixes bugs in AMS math
%     Corrently conflicts with unicode-math
% \usepackage{mathtools}

%\numberwithin{equation}{section}
%\renewcommand{\theequation}{\thesection.\Roman{equation}}

% EN: work-around ams-math problem with align and 9 -> 10. Does not work with glossaries, No visual changes.
%\addtolength\mathindent{1em}


% EN: For theorems, replacement for amsthm
\usepackage[amsmath,hyperref]{ntheorem}
\theorempreskipamount 2ex plus1ex minus0.5ex
\theorempostskipamount 2ex plus1ex minus0.5ex
\theoremstyle{break}
\newtheorem{definition}{Definition}[section]


% CTAN: https://ctan.org/pkg/lccaps
% Doc: http://texdoc.net/pkg/lccaps
%
% Required for DE/EN \initialism
\usepackage{lccaps}


% EN: Defintion of colors. Argument "hyperref" is not used as we do not want to change border colors of links: Links are not colored anymore.
% DE: Farbdefinitionen
\usepackage[dvipsnames]{xcolor}


% EN: Required for custom acronyms/glossaries style.
%     Left aligned Columns in tables with fixed width.
%     See http://tex.stackexchange.com/questions/91566/syntax-similar-to-centering-for-right-and-left
\usepackage{ragged2e}


% DE: Wichtig, ansonsten erscheint "No room for a new \write"
\usepackage{scrwfile}


% EN: Support for language-specific hyphenation
% DE: Neue deutsche Rechtschreibung und Literatur statt "Literature"
%     Die folgende Einstellung ist der Nachfolger von ngerman.sty
\ifdeutsch
  % DE: letzte Sprache ist default, Einbindung von "american" ermöglicht \begin{otherlanguage}{amercian}...\end{otherlanguage} oder \foreignlanguage{american}{Text in American}
  %     Siehe auch http://tex.stackexchange.com/a/50638/9075
  \usepackage[american,main=ngerman]{babel}
  % Ein "abstract" ist eine "Kurzfassung", keine "Zusammenfassung"
  \addto\captionsngerman{%
    \renewcommand\abstractname{Kurzfassung}%
  }
  \ifluatex
    % EN: conditionally disable ligatures. See https://github.com/latextemplates/scientific-thesis-template/issues/54
    %     for a discussion
    \usepackage[ngerman]{selnolig}
  \fi
\else
  % EN: Set English as language and allow to write hyphenated"=words
  %     `american`, `english` and `USenglish` are synonyms for babel package (according to https://tex.stackexchange.com/questions/12775/babel-english-american-usenglish).
  %      "english" has to go last to set it as default language
  \usepackage[ngerman,main=english]{babel}
  % EN: Hint by http://tex.stackexchange.com/a/321066/9075 -> enable "= as dashes
  \addto\extrasenglish{\languageshorthands{ngerman}\useshorthands{"}}
  \ifluatex
    % EN: conditionally disable ligatures. See https://github.com/latextemplates/scientific-thesis-template/issues/54
    %     for a discussion
    \usepackage[english]{selnolig}
  \fi
\fi
%


% EN: For easy quotations: \enquote{text}
%     This package is very smart when nesting is applied, otherwise textcmds (see below) provides a shorter command
%     Note that this package results in a warning when it is loaded before minted (actually fvextra).
% DE: Anführungszeichen
%     Zitate in \enquote{...} setzen, dann werden automatisch die richtigen Anführungszeichen verwendet.
%     Dieses package erzeugt eine Warnung, wenn es vor minted (genauer fvextra) geladen wird.
\usepackage{csquotes}


% EN: For even easier quotations: \qq{text}.
%     Is not smart in the case of nesting, but good enough for the most cases
\usepackage{textcmds}
\ifdeutsch
  % EN: German quotes are different. So do not use the English quotes, but the ones provided by the csquotes package.
  \renewcommand{\qq}[1]{\enquote{#1}}
\fi


% EN: extended enumarations
% DE: erweitertes Enumerate
\usepackage{paralist}


% DE: Gestaltung der Kopf- und Fußteilen

\usepackage[automark]{scrlayer-scrpage}

\automark[section]{chapter}
\setkomafont{pageheadfoot}{\normalfont\sffamily}
\setkomafont{pagenumber}{\normalfont\sffamily}

% DE: funktioniert nicht: Alle Linien sind hier weg
%\setheadsepline[.4pt]{.4pt}


% DE: Intelligentes Leerzeichen um hinter Abkürzungen die richtigen Abstände zu erhalten, auch leere.
%     Siehe commands.tex \gq{}
\usepackage{xspace}
% DE: Macht \xspace und \enquote kompatibel
\makeatletter
\xspaceaddexceptions{\grqq \grq \csq@qclose@i \} }
\makeatother


\newcommand{\eg}{e.\,g.,\ }
\newcommand{\ie}{i.\,e.,\ }


% EN: introduce \powerset - hint by http://matheplanet.com/matheplanet/nuke/html/viewtopic.php?topic=136492&post_id=997377
\DeclareFontFamily{U}{MnSymbolC}{}
\DeclareSymbolFont{MnSyC}{U}{MnSymbolC}{m}{n}
\DeclareFontShape{U}{MnSymbolC}{m}{n}{
  <-6>    MnSymbolC5
  <6-7>   MnSymbolC6
  <7-8>   MnSymbolC7
  <8-9>   MnSymbolC8
  <9-10>  MnSymbolC9
  <10-12> MnSymbolC10
  <12->   MnSymbolC12%
}{}
\DeclareMathSymbol{\powerset}{\mathord}{MnSyC}{180}


% EN: Package for the appendix
% DE: Anhang
\usepackage{appendix}
%[toc,page,title,header]
%


% EN: Graphics
% DE: Grafikeinbindungen
%
% EN: The parameter "pdftex" is not required
\usepackage{graphicx}
\graphicspath{{\getgraphicspath}}
\newcommand{\getgraphicspath}{graphics/}


% EN: Enables inclusion of SVG graphics - 1:1 approach
%    This is NOT the approach of https://ctan.org/pkg/svg-inkscape,
%     which allows text in SVG to be typeset using LaTeX
%     We just include the SVG as is.
\usepackage{epstopdf}
\epstopdfDeclareGraphicsRule{.svg}{pdf}{.pdf}{%
  inkscape -z -D --file=#1 --export-pdf=\OutputFile
}


% EN: Enables inclusion of SVG graphics - text-rendered-with-LaTeX-approach
%     This is the approach of https://ctan.org/pkg/svg-inkscape,
\newcommand{\executeiffilenewer}[3]{%
  \IfFileExists{#2}
  {
    %\message{file #2 exists}
    \ifnum\pdfstrcmp{\pdffilemoddate{#1}}%
      {\pdffilemoddate{#2}}>0%
      {\immediate\write18{#3}}
    \else
      {%\message{file up to date #2}
      }
    \fi%
  }{
    %\message{file #2 doesn't exist}
    %\message{argument: #3}
    %\immediate\write18{echo "test" > xoutput.txt}
    \immediate\write18{#3}
  }
}
\newcommand{\includesvg}[1]{%
  \executeiffilenewer{#1.svg}{#1.pdf}%
  {
    inkscape -z -D --file=\getgraphicspath#1.svg %
    --export-pdf=\getgraphicspath#1.pdf --export-latex}%
  \input{\getgraphicspath#1.pdf_tex}%
}


% EN: Enable typesetting values with SI units.
\ifdeutsch
  \usepackage[mode=text,group-minimum-digits=4]{siunitx}
  \sisetup{locale=DE}
\else
  \usepackage[mode=text,group-minimum-digits=4,group-separator={,}]{siunitx}
  \sisetup{locale=US}
\fi


% EN: Extensions for tables
% DE: Tabellenerweiterungen
\usepackage{array} %increases tex's buffer size and enables ``>'' in tablespecs
\usepackage{longtable}
\usepackage{dcolumn} %Aligning numbers by decimal points in table columns
\ifdeutsch
  \newcolumntype{d}[1]{D{.}{,}{#1}}
\else
  \newcolumntype{d}[1]{D{.}{.}{#1}}
\fi
\setlength{\extrarowheight}{1pt}


% DE: Eine Zelle, die sich über mehrere Zeilen erstreckt.
%     Siehe Beispieltabelle in Kapitel 2
\usepackage{multirow}


% DE: Fuer Tabellen mit Variablen Spaltenbreiten
%\usepackage{tabularx}
%\usepackage{tabulary}


% EN: Links behave as they should. Enables "\url{...}" for URL typesettings.
%     Allow URL breaks also at a hyphen, even though it might be confusing: Is the "-" part of the address or just a hyphen?
%     See https://tex.stackexchange.com/a/3034/9075.
% DE: Links verhalten sich so, wie sie sollen
%     Zeilenumbrüche bei URLs auch bei Bindestrichen erlauben, auch wenn es verwirrend sein könnte: Gehört der Bindestrich zur URL oder ist es ein Trennstrich?
%     Siehe https://tex.stackexchange.com/a/3034/9075.
\usepackage[hyphens]{url}
%
%  EN: When activated, use text font as url font, not the monospaced one.
%      For all options see https://tex.stackexchange.com/a/261435/9075.
% \urlstyle{same}
%
% EN: Hint by http://tex.stackexchange.com/a/10419/9075.
\makeatletter
\g@addto@macro{\UrlBreaks}{\UrlOrds}
\makeatother


% DE: Index über Begriffe, Abkürzungen
%\usepackage{makeidx} makeidx ist out -> http://xindy.sf.net verwenden


% DE: lustiger Hack fuer das Abkuerzungsverzeichnis
%     nach latex durchlauf folgendes ausfuehren
%     makeindex ausarbeitung.nlo -s nomencl.ist -o ausarbeitung.nls
%     danach nochmal latex
%\usepackage{nomencl}
%    \let\abk\nomenclature %Deutsche Ueberschrift setzen
%          \renewcommand{\nomname}{List of Abbreviations}
%        %Punkte zw. Abkuerzung und Erklaerung
%          \setlength{\nomlabelwidth}{.2\hsize}
%          \renewcommand{\nomlabel}[1]{#1 \dotfill}
%        %Zeilenabstaende verkleinern
%          \setlength{\nomitemsep}{-\parsep}
%    \makenomenclature


% EN: Logic for TeX - enables if-then-else in commands
% DE: Logik für TeX
%     FÜr if-then-else @ commands.tex
\usepackage{ifthen}


% EN: Code Listings
% DE: Listings
\usepackage{listings}
\lstset{language=XML,
  showstringspaces=false,
  extendedchars=true,
  basicstyle=\footnotesize\ttfamily,
  commentstyle=\slshape,
  % DE: Original: \rmfamily, damit werden die Strings im Quellcode hervorgehoben. Zusaetzlich evtl.: \scshape oder \rmfamily durch \ttfamily ersetzen. Dann sieht's aus, wie bei fancyvrb
  stringstyle=\ttfamily,
  breaklines=true,
  breakatwhitespace=true,
  % EN: alternative: fixed
  columns=flexible,
  numbers=left,
  numberstyle=\tiny,
  basewidth=.5em,
  xleftmargin=.5cm,
  % aboveskip=0mm, %DE: deaktivieren, falls man lstlistings direkt als floating object benutzt (\begin{lstlisting}[float,...])
  % belowskip=0mm, %DE: deaktivieren, falls man lstlistings direkt als floating object benutzt (\begin{lstlisting}[float,...])
  captionpos=b
}

\ifluatex
\else
  % EN: Enable UTF-8 support - see https://tex.stackexchange.com/q/419327/9075
  \usepackage{listingsutf8}
  \lstset{inputencoding=utf8/latin1}
\fi

\ifdeutsch
  \renewcommand{\lstlistlistingname}{Verzeichnis der Listings}
\fi


% EN: Alternative to listings could be fancyvrb. Can be used together.
% DE: Alternative zu Listings ist fancyvrb. Kann auch beides gleichzeitig benutzt werden.
\usepackage{fancyvrb}
%
% EN: Font size for the normal text
% DE: Groesse fuer den Fliesstext. Falls deaktiviert: \normalsize
%\fvset{fontsize=\small}
%
% DE: Somit kann im Text ganz einfach §verbatim§ text gesetzt werden.
%     Disabled, because UTF-8 does not work any more and lualatex causes issues
%\DefineShortVerb{\§}
%
% EN: Shrink font size of listings
\RecustomVerbatimEnvironment{Verbatim}{Verbatim}{fontsize=\footnotesize}
\RecustomVerbatimCommand{\VerbatimInput}{VerbatimInput}{fontsize=\footnotesize}
%
% EN: Hack for fancyvrb based on http://newsgroups.derkeiler.com/Archive/Comp/comp.text.tex/2008-12/msg00075.html
%     Change of the solution: \Vref somehow collidated with cleveref/varioref as the output of \Vref{} was "Abschnitt 4.3 auf Seite 85"; therefore changed to \myVref -- so completely removed
%     See https://tex.stackexchange.com/q/132420/9075 for more information.
\newcommand{\Vlabel}[1]{\label[line]{#1}\hypertarget{#1}{}}
\newcommand{\lref}[1]{\hyperlink{#1}{\FancyVerbLineautorefname~\ref*{#1}}}


% EN: Tunings of captions for floats, listings, ...
% DE: Bildunterschriften bei floats genauso formatieren wie bei Listings
%     Anpassung wird unten bei den newfloat-Deklarationen vorgenommen
%     https://www.ctan.org/pkg/caption2 is superseeded by this package.
\usepackage{caption}


% EN: Provides rotating figures, where the PDF page is also turned
% DE: Ermoeglicht es, Abbildungen um 90 Grad zu drehen
%     Alternatives Paket: rotating Allerdings wird hier nur das Bild gedreht, während bei lscape auch die PDF-Seite gedreht wird.
%     Das Paket lscape dreht die Seite auch nicht
\usepackage{pdflscape}


% EN: Required for proper environments of fancyvrb and lstlistings
%    There is also the newfloat pacakge (recommended by minted), but we currently have no expericene with that
% DE: Wird für fancyvrb und für lstlistings verwendet
\usepackage{float}
%
% EN: Alternative to float package
%\usepackage{floatrow}
% DE: zustäzlich für den Paramter [H] = Floats WIRKLICH da wo sie deklariert wurden paltzieren - ganz ohne Kompromisse
%     floatrow ist der Nachfolger von float
%     Allerdings macht floatrow in manchen Konstellationen Probleme. Deshalb ist das Paket deaktiviert.
%
% EN: See http://www.tex.ac.uk/cgi-bin/texfaq2html?label=floats
% DE: floats IMMER nach einer Referenzierung platzieren
%\usepackage{flafter}


% EN: Put footnotes below floats
%     Source: https://tex.stackexchange.com/a/32993/9075
\usepackage{stfloats}
\fnbelowfloat


% EN: For nested figures
% DE: Fuer Abbildungen innerhalb von Abbildungen
%     Ersetzt die Pakete subfigure und subfig - siehe https://tex.stackexchange.com/a/13778/9075
\usepackage[hypcap=true]{subcaption}


% EN: Extended support for footnotes
% DE: Fußnoten
%
%\usepackage{dblfnote}  %Zweispaltige Fußnoten
%
% Keine hochgestellten Ziffern in der Fußnote (KOMA-Script-spezifisch):
%\deffootnote[1.5em]{0pt}{1em}{\makebox[1.5em][l]{\bfseries\thefootnotemark}}
%
% Abstand zwischen Fußnoten vergrößern:
%\setlength{\footnotesep}{.85\baselineskip}
%
% EN: Following command disables the separting line of the footnote
% DE: Folgendes Kommando deaktiviert die Trennlinie zur Fußnote
%\renewcommand{\footnoterule}{}
%
\addtolength{\skip\footins}{\baselineskip} % Abstand Text <-> Fußnote
%
% Fußnoten immer ganz unten auf einer \raggedbottom-Seite
% fnpos kommt aus dem yafoot package
\usepackage{fnpos}
\makeFNbelow
\makeFNbottom


% EN: Variable page heights
% DE: Variable Seitenhöhen zulassen
\raggedbottom


% DE: Falls die Seitenzahl bei einer Referenz auf eine Abbildung nur dann angegeben werden soll,
%     falls sich die Abbildung nicht auf der selben Seite befindet...
\iftex4ht
  %tex4ht does not work well with vref, therefore we emulate vref behavior
  \newcommand{\vref}[1]{\ref{#1}}
\else
  \ifdeutsch
    \usepackage[ngerman]{varioref}
  \else
    \usepackage{varioref}
  \fi
\fi


% EN: More beautiful tables if one uses \toprule, \midrule, \bottomrule
% DE: Noch schoenere Tabellen als mit booktabs mit http://www.zvisionwelt.de/downloads.html
\usepackage{booktabs}
%
%\usepackage[section]{placeins}


% EN: Graphs and Automata
%
% TODO: Since version 3.0 (2013-10-01), it supports pdflatex via the auto-pst-pdf package
%       Requires -shell-escape
%\usepackage{gastex}


%\usepackage{multicol}

% DE: kollidiert mit diplomarbeit.sty
%\usepackage{setspace}


% DE: biblatex statt bibtex
\usepackage[
  backend       = biber, %biber does not work with 64x versions alternative: bibtex8
  %minalphanames only works with biber backend
  sortcites     = true,
  bibstyle      = alphabetic,
  citestyle     = alphabetic,
  giveninits    = true,
  useprefix     = false, %"von, van, etc." will be printed, too. See below.
  minnames      = 1,
  minalphanames = 3,
  maxalphanames = 4,
  maxbibnames   = 99,
  maxcitenames  = 2,
  natbib        = true,
  eprint        = true,
  url           = true,
  doi           = true,
  isbn          = true,
  backref       = true]{biblatex}

% enable more breaks at URLs. See https://tex.stackexchange.com/a/134281.
\setcounter{biburllcpenalty}{7000}
\setcounter{biburlucpenalty}{8000}

\bibliography{bibliography}
%\addbibresource[datatype=bibtex]{bibliography.bib}

%Do not put "vd" in the label, but put it at "\citeauthor"
%Source: http://tex.stackexchange.com/a/30277/9075
\makeatletter
\AtBeginDocument{\toggletrue{blx@useprefix}}
\AtBeginBibliography{\togglefalse{blx@useprefix}}
\makeatother

%Thin spaces between initials
%http://tex.stackexchange.com/a/11083/9075
\renewrobustcmd*{\bibinitdelim}{\,}

%Keep first and last name together in the bibliography
%http://tex.stackexchange.com/a/196192/9075
\renewcommand*\bibnamedelimc{\addnbspace}
\renewcommand*\bibnamedelimd{\addnbspace}

%Replace last "and" by comma in bibliography
%See http://tex.stackexchange.com/a/41532/9075
\AtBeginBibliography{%
  \renewcommand*{\finalnamedelim}{\addcomma\space}%
}

\DefineBibliographyStrings{ngerman}{
  backrefpage  = {zitiert auf S\adddot},
  backrefpages = {zitiert auf S\adddot},
  andothers    = {et\ \addabbrvspace al\adddot},
  %Tipp von http://www.mrunix.de/forums/showthread.php?64665-biblatex-Kann-%DCberschrift-vom-Inhaltsverzeichnis-nicht-%E4ndern&p=293656&viewfull=1#post293656
  bibliography = {Literaturverzeichnis}
}

% EN: enable hyperlinked author names when using \citeauthor
%     source: http://tex.stackexchange.com/a/75916/9075
\DeclareCiteCommand{\citeauthor}
{\boolfalse{citetracker}%
  \boolfalse{pagetracker}%
  \usebibmacro{prenote}}
{\ifciteindex
  {\indexnames{labelname}}
  {}%
  \printtext[bibhyperref]{\printnames{labelname}}}
{\multicitedelim}
{\usebibmacro{postnote}}

% EN: natbib compatibility
%\newcommand{\citep}[1]{\cite{#1}}
%\newcommand{\citet}[1]{\citeauthor{#1} \cite{#1}}
% EN: Beginning of sentence - analogous to cleveref - important for names such as "zur Muehlen"
%\newcommand{\Citep}[1]{\cite{#1}}
%\newcommand{\Citet}[1]{\Citeauthor{#1} \cite{#1}}

% DE: Blindtext. Paket "blindtext" ist fortgeschritterner als "lipsum" und kann auch Mathematik im Text (http://texblog.org/2011/02/26/generating-dummy-textblindtext-with-latex-for-testing/)
%     kantlipsum (https://www.ctan.org/tex-archive/macros/latex/contrib/kantlipsum) ist auch ganz nett, aber eben auch keine Mathematik
%     Wird verwendet, um etwas Text zu erzeugen, um eine volle Seite wegen Layout zu sehen.
\usepackage[math]{blindtext}


% EN: Make LaTeX logos available by commands. E.g., \lualatex
%     Disabled, because currently causes \not= already defined
%\usepackage{dtk-logos}

% quick replacement:
\newcommand{\LuaLaTeX}{Lua\LaTeX\xspace}
\newcommand{\lualatex}{\LuaLaTeX}

% DE: Neue Pakete bitte VOR hyperref einbinden. Insbesondere bei Verwendung des
%     Pakets "index" wichtig, da sonst die Referenzierung nicht funktioniert.
%     Für die Indizierung selbst ist unter http://xindy.sourceforge.net
%     ein gutes Tool zu erhalten.
%     Hier also neue packages einbinden.
% EN: Add new packages at this place.


% EN: Provides hyperlinks
%     Option "unicode" fixes umlauts in the PDF bookmarks - see https://tex.stackexchange.com/a/338770/9075
%
% DE: Erlaubt Hyperlinks im Dokument.
%     Alle Optionen nach \hypersetup verschoben, sonst crash
%     Siehe auch: "Praktisches LaTeX" - www.itp.uni-hannover.de/~kreutzm
\usepackage[unicode]{hyperref}


% EN: Define colors
% DE: Da es mit KOMA 3 und xcolor zu Problemen mit den global Options kommt MÜSSEN die Optionen so gesetzt werden.
%     Eigene Farbdefinitionen ohne die Namen des xcolor packages
\definecolor{darkblue}{rgb}{0,0,.5}
\definecolor{black}{rgb}{0,0,0}


% EN: Define color of links and more
\hypersetup{
  % have both title and number hyperlinking to content
  linktoc=all,
  bookmarksnumbered=true,
  bookmarksopen=true,
  bookmarksopenlevel=1,
  breaklinks=true,
  colorlinks=true,
  pdfstartview=Fit,
  pdfpagelayout=SinglePage, % DE: Alterntaive: TwoPageRight -- zweiseitige Darstellung: ungerade Seiten rechts im PDF-Viewer - siehe auch http://tex.stackexchange.com/a/21109/9075
  %pdfencoding=utf8, % EN: This is probably the same as passing the option "unicode" at \usepackage{hyperref}
  filecolor=darkblue,
  urlcolor=darkblue,
  linkcolor=black,
  citecolor=black
}


% EN: Abbreviations - has to be loaded after hyperref
% DE: Abkürzungsverzeichnis - muss nach hyperref geladen werden
%
% DE: siehe http://www.dickimaw-books.com/cgi-bin/faq.cgi?action=view&categorylabel=glossaries#glsnewwriteexceeded
\usepackage[acronym,indexonlyfirst,nomain]{glossaries}
\ifdeutsch
  \addto\captionsngerman % DE: siehe https://tex.stackexchange.com/a/154566
  {%
    \renewcommand*{\acronymname}{Abkürzungsverzeichnis}
  }
\else
  \renewcommand*{\acronymname}{List of Abbreviations}
\fi
\renewcommand*{\glsgroupskip}{}
%
% EN: Removed Glossarie as a table as a quick fix to get the template working again
%     See http://tex.stackexchange.com/questions/145579/how-to-print-acronyms-of-glossaries-into-a-table
%
\makenoidxglossaries


% EN: Extensions for references inside the document (\cref{fig:sample}, ...)
% DE: cleveref für cref statt autoref, da cleveref auch bei Definitionen funktioniert
\usepackage[capitalise,nameinlink,noabbrev]{cleveref}
\ifdeutsch
  \crefname{table}{Tabelle}{Tabellen}
  \Crefname{table}{Tabelle}{Tabellen}
  \crefname{figure}{\figurename}{\figurename}
  \Crefname{figure}{Abbildung}{Abbildungen}
  \crefname{equation}{Gleichung}{Gleichungen}
  \Crefname{equation}{Gleichung}{Gleichungen}
  \crefname{theorem}{Theorem}{Theoreme}
  \Crefname{theorem}{Theorem}{Theoreme}
  \crefname{listing}{\lstlistingname}{\lstlistingname}
  \Crefname{listing}{Listing}{Listings}
  \crefname{section}{Abschnitt}{Abschnitte}
  \Crefname{section}{Abschnitt}{Abschnitte}
  \crefname{paragraph}{Abschnitt}{Abschnitte}
  \Crefname{paragraph}{Abschnitt}{Abschnitte}
  \crefname{subparagraph}{Abschnitt}{Abschnitte}
  \Crefname{subparagraph}{Abschnitt}{Abschnitte}
\else
  \crefname{listing}{\lstlistingname}{\lstlistingname}
  \Crefname{listing}{Listing}{Listings}
\fi


% DE: Zur Darstellung von Algorithmen
%     Algorithm muss nach hyperref geladen werden
\usepackage[chapter]{algorithm}
\usepackage[]{algpseudocode}


% DE: Links auf Gleitumgebungen springen nicht zur Beschriftung,
%     Doc: http://mirror.ctan.org/tex-archive/macros/latex/contrib/oberdiek/hypcap.pdf
%     sondern zum Anfang der Gleitumgebung
\usepackage[all]{hypcap}


% DE: Deckblattstyle
%
\ifdeutsch
  \PassOptionsToPackage{language=german}{scientific-thesis-cover}
\else
  \PassOptionsToPackage{language=english}{scientific-thesis-cover}
\fi


% EN: Bugfixes packages
%\usepackage{fixltx2e} %Fuer neueste LaTeX-Installationen nicht mehr benoetigt - bereinigte einige Ungereimtheiten, die auf Grund von Rueckwaertskompatibilitaet beibahlten wurden.
%\usepackage{mparhack} %Fixt die Position von marginpars (die in DAs selten bis gar nicht gebraucht werden}
%\usepackage{ellipsis} %Fixt die Abstaende vor \ldots. Wird wohl auch nicht benoetigt.


% EN: Settings for captions of floats
% DE: Formatierung der Beschriftungen
%
\captionsetup{
  format=hang,
  labelfont=bf,
  justification=justified,
  %single line captions should be centered, multiline captions justified
  singlelinecheck=true
}


% EN: New float environments for listings and algorithms
%
% \floatstyle{ruled} % TODO: enabled or disabled causes no change - listings and algorithms are always ruled
%
\newfloat{Listing}{tbp}{code}[chapter]
\crefname{Listing}{Listing}{Listings}

\newfloat{Algorithmus}{tbp}{alg}[chapter]
\ifdeutsch
  \crefname{Algorithmus}{Algorithmus}{Algorithmus}
\else
  \crefname{Algorithmus}{Algorithm}{Algorithms}
  \floatname{Algorithmus}{Algorithm}
\fi



% EN: Various chapter styles
% DE: unterschiedliche Chapter-Styles
%     u.a. Paket fncychap

% Andere Kapitelueberschriften
% falls einem der Standard von KOMA nicht gefaellt...
% Falls man zurück zu KOMA moechte, dann muss jede der vier folgenden Moeglichkeiten deaktiviert sein.

%\usepackage[Sonny]{fncychap}

%\usepackage[Bjarne]{fncychap}

%\usepackage[Lenny]{fncychap}

%DE: Zur Aktivierung eines der folgenden Möglichkeiten ein Paar von "\iffalse" und "\fi" auskommentieren

\iffalse
  \usepackage[Bjarne]{fncychap}
  \ChNameVar{\Large\sf} \ChNumVar{\Huge} \ChTitleVar{\Large\sf}
  \ChRuleWidth{0.5pt} \ChNameUpperCase
\fi

\iffalse
  \usepackage[Rejne]{fncychap}
  \ChNameVar{\centering\Huge\rm\bfseries}
  \ChNumVar{\Huge}
  \ChTitleVar{\centering\Huge\rm}
  \ChNameUpperCase
  \ChTitleUpperCase
  \ChRuleWidth{1pt}
\fi

\iffalse
  \usepackage{fncychap}
  \ChNameUpperCase
  \ChTitleUpperCase
  \ChNameVar{\raggedright\normalsize} %\rm
  \ChNumVar{\bfseries\Large}
  \ChTitleVar{\raggedright\Huge}
  \ChRuleWidth{1pt}
\fi

\iffalse
  \usepackage[Bjornstrup]{fncychap}
  \ChNumVar{\fontsize{76}{80}\selectfont\sffamily\bfseries}
  \ChTitleVar{\raggedright\Large\sffamily\bfseries}
\fi

% EN: Complete different chapter style - self made

% Innen drin kann man dann noch zwischen
%   * serifenloser Schriftart (eingestellt)
%   * serifenhafter Schriftart (wenn kein zusaetzliches Kommando aktiviert ist) und
%   * Kapitälchen wählen
\iffalse
  \makeatletter
  %\def\thickhrulefill{\leavevmode \leaders \hrule height 1ex \hfill \kern \z@}

  %Fuer Kapitel mit Kapitelnummer
  \def\@makechapterhead#1{%
    \vspace*{10\p@}%
    {\parindent \z@ \raggedright \reset@font
      %Default-Schrift: Serifenhaft (gut fuer englische Dokumente)
      %A) Fuer serifenlose Schrift:
      \fontfamily{phv}\selectfont
      %B) Fuer Kapitaelchen:
      %\fontseries{m}\fontshape{sc}\selectfont
      %C) Fuer ganz "normale" Schrift:
      %\normalfont
      %
      \Large \@chapapp{} \thechapter
      \par\nobreak\vspace*{10\p@}%
      \interlinepenalty\@M
      {\Huge\bfseries\baselineskip3ex
        %Fuer Kapitaelchen folgende Zeile aktivieren:
        %\fontseries{m}\fontshape{sc}\selectfont
        #1\par\nobreak}
      \vspace*{10\p@}%
      \makebox[\textwidth]{\hrulefill}%    \hrulefill alone does not work
      \par\nobreak
      \vskip 40\p@
    }}

  %Fuer Kapitel ohne Kapitelnummer (z.B. Inhaltsverzeichnis)
  \def\@makeschapterhead#1{%
    \vspace*{10\p@}%
    {\parindent \z@ \raggedright \reset@font
      \normalfont \vphantom{\@chapapp{} \thechapter}
      \par\nobreak\vspace*{10\p@}%
      \interlinepenalty\@M
      {\Huge \bfseries %
        %Default-Schrift: Serifenhaft (gut fuer englische Dokumente)
        %A) Fuer serifenlose Schrift folgende Zeile aktivieren:
        \fontfamily{phv}\selectfont
        %B) Fuer Kapitaelchen folgende Zeile aktivieren:
        %\fontseries{m}\fontshape{sc}\selectfont
        #1\par\nobreak}
      \vspace*{10\p@}%
      \makebox[\textwidth]{\hrulefill}%    \hrulefill does not work
      \par\nobreak
      \vskip 40\p@
    }}
  %
  \makeatother
\fi


% DE: Minitoc-Einstellungen
%\dominitoc
%\renewcommand{\mtctitle}{Inhaltsverzeichnis dieses Kapitels}


% EN: Nicer paragraph line placement:
%     - Disable single lines at the start of a paragraph (Schusterjungen)
%     - Disable single lines at the end of a paragraph (Hurenkinder)
%     Normally, this is clubpenalty and widowpenalty, but using a package, it feels more non-hacky
\usepackage[all,defaultlines=3]{nowidow}
%
\displaywidowpenalty = 10000


% EN: Try to get rid of "overfull hbox" things and let text flow batter
%     See also
%       - http://groups.google.de/group/de.comp.text.tex/browse_thread/thread/f97da71d90442816/f5da290593fd647e?lnk=st&q=tolerance+emergencystretch&rnum=5&hl=de#f5da290593fd647e
%       - http://www.tex.ac.uk/cgi-bin/texfaq2html?label=overfull
\tolerance=2000
%
% EN: This could be increased to 20pt
\setlength{\emergencystretch}{3pt}
%
% EN: Suppress hbox warnings if less than 1pt
\setlength{\hfuzz}{1pt}


% EN: Fix names for algorithms in German
% DE: fuer algorithm.sty: - falls Deutsch und nicht Englisch.
\ifdeutsch
  \floatname{algorithm}{Algorithmus}
  \renewcommand{\listalgorithmname}{Verzeichnis der Algorithmen}
\fi




% Float-placements - http://dcwww.camd.dtu.dk/~schiotz/comp/LatexTips/LatexTips.html#figplacement
% and http://people.cs.uu.nl/piet/floats/node1.html
\renewcommand{\topfraction}{0.85}
\renewcommand{\bottomfraction}{0.95}
\renewcommand{\textfraction}{0.1}
\renewcommand{\floatpagefraction}{0.75}
%\setcounter{totalnumber}{5}

% EN: ensure that floats covering a whole page are placed at the top of the page
%    see http://tex.stackexchange.com/a/28565/9075
\makeatletter
\setlength{\@fptop}{0pt}
\setlength{\@fpbot}{0pt plus 1fil}
\makeatother



% DE: Bei Gleichungen nur dann die Nummer zeigen, wenn die Gleichung auch referenziert wird
%     Funktioniert mit MiKTeX Stand 2012-01-13 nicht. Deshalb ist dieser Schalter deaktiviert.
%
%\mathtoolsset{showonlyrefs}


% EN: Margins
% DE: Ränder
%     Viele Moeglichkeiten, die Raender im Dokument einzustellen.
%
%     Satzspiegel neu berechnen. Dokumentation dazu ist in "scrguide.pdf" von KOMA-Skript zu finden
%     Optionen werden bei \documentclass[] in ausarbeitung.tex mitgegeben.
% \typearea[current]{current} %neu berechnen, da neue Schrift eingebunden

%\usepackage{a4}
%\usepackage{a4wide}
%\areaset{170mm}{277mm} %a4:29,7hochx21mbreit

%Wer die Masse direkt eingeben moechte:
%Bei diesem Beispiel wird die Regel nicht beachtet, dass der innere Rand halb so gross wie der aussere Rand und der obere Rand halb so gross wie der untere Rand sein sollte
%\usepackage[inner=2.5cm, outer=2.5cm, includefoot, top=3cm, bottom=1.5cm]{geometry}

% EN: Package geometry to enlarge on page
%
%     Normally, geometry should not be used as the typearea package calculates the margins perfectly for printing
%     However, we want better screen-readable documents where the content does not "jump"
%     Thus, we fix the margins left and right to the same value
%
%     Source: http://www.howtotex.com/tips-tricks/change-margins-of-a-single-page/
%
\usepackage[
  left=3cm,right=3cm,top=2.5cm,bottom=2.5cm,
  headsep=18pt,
  footskip=30pt,
  includehead,
  includefoot
]{geometry}


% EN: Provides todo notes
% DE: schoene TODOs
\ifdeutsch
  \usepackage[colorinlistoftodos,ngerman]{todonotes}
\else
  \usepackage[colorinlistoftodos]{todonotes}
\fi
\setlength{\marginparwidth}{2,5cm}

\let\xtodo\todo
\renewcommand{\todo}[1]{\xtodo[inline,color=black!5]{#1}}
\newcommand{\utodo}[1]{\xtodo[inline,color=green!5]{#1}}
\newcommand{\itodo}[1]{\xtodo[inline]{#1}}


% EN: Enable footnotes in tables.
%     This package superseeds the 1997 package "footnote"
\usepackage{footnotehyper}
% TODO: The footnotehyper author recommends to enclose the respective area with \begin{savenotes} ... \end{savenotes}
\makesavenoteenv{tabular}
\makesavenoteenv{table}
% Reuse of footnotes, see http://tex.stackexchange.com/questions/10102/multiple-references-to-the-same-footnote-with-hyperref-support-is-there-a-bett
\crefformat{footnote}{#2\footnotemark[#1]#3}


% EN: pgfplots (optional if the ppackage is installed)
%     PGFPlots draws high-qual­ity func­tion plots in nor­mal or log­a­rith­mic scal­ing
\IfFileExists{pgfplots.sty}{
  \usepackage{pgfplots}
  % EN: highest version supported by overleaf as of 2018-03-16
  \pgfplotsset{compat=1.14}
}{}


% EN: pgfplotstable (optional if the ppackage is installed)
%     PGFPlots generates tables from csv files
\IfFileExists{pgfplotstable.sty}{
  \usepackage{pgfplotstable}
}{}


% EN: Package for creating graphics programmatically
\usepackage{tikz}


% EN: Package for creating uml diagramms
\usepackage{tikz-uml}


% EN: Forest: apgf/TikZ-based package for drawing linguistic trees - https://ctan.org/pkg/forest
\usepackage{forest}


% EN: Enable PlantUML listings in the environment "plantuml"
\IfFileExists{plantuml.sty}{
  \usepackage[output=latex]{plantuml}
}{}


% EN: Layout: bottoms of pages not aligned to each other
% DE: Der untere Rand darf "flattern"
\raggedbottom


% DE: Wie tief wird das Inhaltsverzeichnis aufgeschlüsselt
% 0 --\chapter
% 1 --\section % fuer kuerzeres Inhaltsverzeichnis verwenden - oder minitoc benutzen
% 2 --\subsection
% 3 --\subsubsection
% 4 --\paragraph
\setcounter{tocdepth}{1}


% EN: Fixes wrong spacing in the TOC.
%     Source: https://tex.stackexchange.com/a/33842/9075 -> comment by esdd
\RedeclareSectionCommand[tocnumwidth=2.8em]{section}


% DE: Angaben in die PDF-Infos uebernehmen
\makeatletter
\hypersetup{
  pdftitle={}, %Titel der Arbeit
  pdfauthor={}, %Author
  pdfkeywords={}, % CR-Klassifikation und ggf. weitere Stichworte
  pdfsubject={}
}
\makeatother


% EN: Higher compression of the output PDF
\pdfcompresslevel=9


% EN: Required for recent version of komascript, as some packges are not that compatible with KOMAScript as they should be
%     Has to be loaded at the *very* end, so we use "\AtEndPreamble" by etoolsbox
\usepackage{etoolbox}
\AtEndPreamble{\usepackage{scrhack}}


% EN: Provide tables over multiple pages
\usepackage{longtable}


% EN: Show LaTeX commands and their results in the document
%     Enables the command \PrintDemo
% See https://github.com/latextemplates/scientific-thesis-template/issues/82 for further discussion
\usepackage{latexdemo}


% DE: Fuer deutsche Texte: Weniger Silbentrennung, mehr Abstand zwischen den Woertern
\ifdeutsch
  \setlength{\emergencystretch}{3em} % Silbentrennung reduzieren durch mehr frei Raum zwischen den Worten
\fi



\usepackage[
  title={Getacar: A Privacy-Preserving Platform for Ride-Pooling},
  author={Hans Hüppelshäuser},
  type=master,
  institute=iaas, % or other institute names - or just a plain string using {Demo\\Demo...}
  course={M.Sc. Wirtschaftsinformatik},
  examiner={Prof.\ Dr.\ Marco Aiello},
  supervisor={Robin Pesl \ M.Sc.},
  startdate={March 27, 2023},
  enddate={Septeber 27, 2023}
]{scientific-thesis-cover}

% Hier stehen alle Abkürzungen
\newacronym{pow}{PoW}{Proof of Work}
\newacronym{pos}{PoS}{Proof of Stake}
\newacronym{dpos}{DPoS}{Delegated Proof of Stake}
\newacronym{sumo}{SUMO}{Simulation of Urban MObility}
\newacronym{dapp}{DApp}{Decentralised Applications}
\newacronym{ipfs}{IPFS}{Interplanetary File System}
\newacronym{rsus}{RSUs}{Road Side Units}
\newacronym{osi}{OSI}{Open Source Initiative}


\makeindex

\begin{document}

%tex4ht-Konvertierung verschönern
\iftex4ht
  % tell tex4ht to create picures also for formulas starting with '$'
  % WARNING: a tex4ht run now takes forever!
  \Configure{$}{\PicMath}{\EndPicMath}{}
  %$ % <- syntax highlighting fix for emacs
  \Css{body {text-align:justify;}}

  %conversion of .pdf to .png
  \Configure{graphics*}
  {pdf}
  {\Needs{"convert \csname Gin@base\endcsname.pdf
      \csname Gin@base\endcsname.png"}%
    \Picture[pict]{\csname Gin@base\endcsname.png}%
  }
\fi

%\VerbatimFootnotes %verbatim text in Fußnoten erlauben. Geht normalerweise nicht.

% DE: wird fuer Tabellen benötigt (z.B. >{centering\RBS}p{2.5cm} erzeugt einen zentrierten 2,5cm breiten Absatz in einer Tabelle
\newcommand{\RBS}{\let\\=\tabularnewline}

% EN: To avoid issues with Springer's \mathplus
%     See also http://tex.stackexchange.com/q/212644/9075
\providecommand\mathplus{+}

% DE: typoraphisch richtige Abkürzungen
\newcommand{\zB}{z.\,B.\xspace}
\newcommand{\bzw}{bzw.\xspace}
\newcommand{\usw}{usw.\xspace}
\renewcommand{\dh}{d.\,h.\xspace}

% EN: from hmks makros.tex - \indexify
\newcommand{\toindex}[1]{\index{#1}#1}

% DE: Tipp aus "The Comprehensive LaTeX Symbol List"
\newcommand{\dotcup}{\ensuremath{\,\mathaccent\cdot\cup\,}}

% DE: Anstatt $|x|$ $\abs{x}$ verwenden.
%     Die Betragsstriche skalieren automatisch, falls "x" etwas größer sein sollte...
\newcommand{\abs}[1]{\left\lvert#1\right\rvert}

% DE: für Zitate
\newcommand{\citeS}[2]{\cite[S.~#1]{#2}}
\newcommand{\citeSf}[2]{\cite[S.~#1\,f.]{#2}}
\newcommand{\citeSff}[2]{\cite[S.~#1\,ff.]{#2}}
\newcommand{\vgl}{vgl.\ }
\newcommand{\Vgl}{Vgl.\ }

% EN: For the algorithmic package
\newcommand{\commentchar}{\ensuremath{/\mkern-4mu/}}
\algrenewcommand{\algorithmiccomment}[1]{\hfill $\commentchar$ #1}

% DE: Seitengrößen - Gegen Schusterjungen und Hurenkinder...
\newcommand{\largepage}{\enlargethispage{\baselineskip}}
\newcommand{\shortpage}{\enlargethispage{-\baselineskip}}

\newcommand{\initialism}[1]{%
  \ifdeutsch%
    \textsc{#1}\xspace%
  \else%
    \textlcc{#1}\xspace%
  \fi%
}
\newcommand{\OMG}{\initialism{OMG}}
\newcommand{\BPEL}{\initialism{BPEL}}
\newcommand{\BPMN}{\initialism{BPMN}}
\newcommand{\UML}{\initialism{UML}}

\pagenumbering{arabic}
\Titelblatt

%Eigener Seitenstil fuer die Kurzfassung und das Inhaltsverzeichnis
\deftriplepagestyle{preamble}{}{}{}{}{}{\pagemark}
%Doku zu deftriplepagestyle: scrguide.pdf
\pagestyle{preamble}
\renewcommand*{\chapterpagestyle}{preamble}



%Kurzfassung / abstract
%auch im Stil vom Inhaltsverzeichnis
\ifdeutsch
  \section*{Kurzfassung}
\else
  \section*{Abstract}
\fi

<Short summary of the thesis>

\cleardoublepage


% BEGIN: Verzeichnisse

\iftex4ht
\else
  \microtypesetup{protrusion=false}
\fi

%%%
% Literaturverzeichnis ins TOC mit aufnehmen, aber nur wenn nichts anderes mehr hilft!
% \addcontentsline{toc}{chapter}{Literaturverzeichnis}
%
% oder zB
%\addcontentsline{toc}{section}{Abkürzungsverzeichnis}
%
%%%

%Produce table of contents
%
%In case you have trouble with headings reaching into the page numbers, enable the following three lines.
%Hint by http://golatex.de/inhaltsverzeichnis-schreibt-ueber-rand-t3106.html
%
%\makeatletter
%\renewcommand{\@pnumwidth}{2em}
%\makeatother
%
\tableofcontents

% Bei einem ungünstigen Seitenumbruch im Inhaltsverzeichnis, kann dieser mit
% \addtocontents{toc}{\protect\newpage}
% an der passenden Stelle im Fließtext erzwungen werden.

\listoffigures
\listoftables

%Wird nur bei Verwendung von der lstlisting-Umgebung mit dem "caption"-Parameter benoetigt
%\lstlistoflistings
%ansonsten:
\ifdeutsch
  \listof{Listing}{Verzeichnis der Listings}
\else
  \listof{Listing}{List of Listings}
\fi

%mittels \newfloat wurde die Algorithmus-Gleitumgebung definiert.
%Mit folgendem Befehl werden alle floats dieses Typs ausgegeben
\ifdeutsch
  \listof{Algorithmus}{Verzeichnis der Algorithmen}
\else
  \listof{Algorithmus}{List of Algorithms}
\fi
%\listofalgorithms %Ist nur für Algorithmen, die mittels \begin{algorithm} umschlossen werden, nötig

% Abkürzungsverzeichnis
\printnoidxglossaries

\iftex4ht
\else
  %Optischen Randausgleich und Grauwertkorrektur wieder aktivieren
  \microtypesetup{protrusion=true}
\fi

% END: Verzeichnisse


% Headline and footline
\renewcommand*{\chapterpagestyle}{scrplain}
\pagestyle{scrheadings}
\pagestyle{scrheadings}
\ihead[]{}
\chead[]{}
\ohead[]{\headmark}
\cfoot[]{}
\ofoot[\usekomafont{pagenumber}\thepage]{\usekomafont{pagenumber}\thepage}
\ifoot[]{}


%% vv  scroll down for content  vv %%































%%%%%%%%%%%%%%%%%%%%%%%%%%%%%%%%%%%%%%%%%%%%%%%%%%%%%%%%%%%%%%%%%%%%%%%%%%%%%%
%
% Main content starts here
%
%%%%%%%%%%%%%%%%%%%%%%%%%%%%%%%%%%%%%%%%%%%%%%%%%%%%%%%%%%%%%%%%%%%%%%%%%%%%%%


\chapter{Introduction}
\section{Problem Statement}
With the prospect of autonomous driving becoming a reality in the foreseeable future, a surge in traffic is anticipated due to the increased accessibility and convenience offered by this technology.

In recent years, developments surrounding transportation have changed dramatically with the surge in technologies such as autonomous driving. While experts and industry stakeholders advertise \gls{av} for their ability to introduce efficiency, convenience, and potential safety to our roadways, there lies an inherent problem that may impair current traffic conditions.

The rapid integration and adoption of autonomous driving, as projected, would inevitably lead to a massive upswing in vehicular traffic. This increase stems from the upsides that make an \gls{av} appealing, their ease and convenience. With the absence of the need to drive manually, more individuals might be inclined to choose a personal \gls{av} over other modes of transport, creating stagnation and traffic jams. Therefor, while the technological transition promises a number of advantages, without proper intervention, it may also have negative effects on  travel, above all in the form of  increased road traffic.

Ride-pooling, or shared mobility, emerges as a potential solution to this problem. The principle of ride-pooling revolves around utilising a single vehicle to transport multiple passengers headed in similar directions, ensuring optimal usage of a car's seating capacity, reducing the number of individual trips, and consequentially, decreasing overall traffic. However, in practice ride-pooling faces a number of challenges.

A fundamental challenge lies in the potential monopolization of the ride-pooling market. <Text on my why monopolisation in the industry happens>. Such centralization not only puts a vast amount of data and power into the hands of one or few entities but also makes it harder for new market entrants to compete. New competitors and smaller businesses find themselves blocked out, leading to diminished innovation, potential price inflation, and reduced consumer choice.

Furthermore, centralized platforms, by their nature, tend to consolidate vast amounts of personal and transactional data. This leads to concerns regarding user privacy. If a user’s ride details, routes, payment information, and behavior on the platform are stored without adequate privacy measures, it exposes them to potential risks.

Given this backdrop, the concept of a decentralized, trust-based platform for ride-pooling becomes not just preferable but essential. The \gls{dlt}, commonly known as blockchain, has demonstrated its potential in recent years as a tool for creating decentralized systems. By design, \gls{dlt} offers transparency, immutability, and decentralization – attributes that can address many of the concerns posed by centralized systems. On the other hand, factors such as the  transparency of data running through a distributed ledger poses challenges in regards to the privacy of personal data and the anonymity of user activity on the platform.<Need to connect the end of this paragraph better to the beginning of the next one>

Therefore, this study seeks to conceptualize, design, and evaluate a platform for ride-pooling based on such distributed technology. By doing so, it endeavors to craft a system where users and providers can interact seamlessly, ensuring the privacy of transactions and enabling an environment where neither party has direct knowledge of the other.

In summary, the ubiquity of autonomous driving presents challenges that threaten to reverse its benefits. The situation demands innovative solutions, one of which is ride-pooling. However, the successful execution of ride-pooling mandates a departure from traditional centralized platforms to decentralized ones that preserve privacy and ensure trust. This research aims to contribute to the creation of such platforms.





\section{Objectives}
The landscape of transportation is on the cusp of a transformative leap. As elucidated in the problem statement, while autonomous driving is set to redefine the way we perceive mobility, it simultaneously brings forth a myriad of challenges. Chief among them is the anticipated surge in traffic and the privacy and monopolistic concerns surrounding the conventional ride-pooling solutions. As researchers and stakeholders invested in the future of transportation, it becomes our prerogative to address these challenges head-on. This section delineates the primary objectives we seek to accomplish in this study, ensuring that the benefits of autonomous mobility are maximized while minimizing potential drawbacks.

\subsection{Prototypical Realization of the Decentralized Platform}

The heart of our research lies in the creation of a decentralized platform for ride-pooling. Unlike traditional systems where power and control are concentrated, decentralized platforms spread power across nodes, ensuring equitable control and reducing the risk of any single entity's dominance.

The objective is twofold:

Conceptualization: Before diving into implementation, we will lay down a theoretical framework for the platform. This involves establishing parameters for participation, mapping out user journeys, and ensuring the system's robustness against potential threats.

Prototyping: Armed with a comprehensive theoretical design, we will proceed to construct a prototype of the platform. By building a tangible system, we can simulate real-world scenarios, understand unforeseen challenges, and refine the design in response to these challenges.

\subsection{Design of an Interaction Protocol between the Platform, Users, and Providers}

Every platform, at its core, is an ecosystem of interactions. Within the context of our decentralized ride-pooling platform, these interactions encompass:

Users seeking rides
Providers offering ride services
Transactions facilitating the above exchanges
A streamlined, secure, and efficient protocol for these interactions is paramount for the success of the platform. The objective here is to develop a protocol that:

Ensures Seamless Integration: The protocol should allow new users and providers to easily join the platform and existing ones to leave without causing disruptions.

Facilitates Trustworthy Transactions: Every transaction must ensure that the provider is fairly compensated and the user receives the promised ride.

Preserves Privacy: Given the emphasis on privacy, the protocol should guarantee that the interactions between users and providers remain untraceable by external entities.

\subsection{Design of a Trust Mechanism for Users and Providers}

Trust is the linchpin of any successful platform. In decentralized systems, where there isn't a central authority to arbitrate disputes or verify participants, the importance of trust is magnified. Our objective is to instate mechanisms that:

Verify Participants: A mechanism to authenticate and verify new entrants to the platform to prevent malicious actors.

Enable Feedback: Allow users and providers to rate and review each other, ensuring a self-regulating community.

Incentivize Good Behavior: Introduce rewards or recognitions for participants who consistently adhere to platform guidelines and receive positive feedback.

Penalize Misconduct: Conversely, there should be deterrents and penalties for those engaging in fraudulent or malicious activities.

\subsection{Evaluation of User and Provider Anonymity in Transactions}

An integral part of our platform's promise is the assurance that users and providers can transact without directly knowing each other's identity. This objective involves:

Anonymity Verification: Rigorous testing to ensure that the implemented measures successfully obfuscate the identities of participants during transactions.

Privacy Audits: Periodic checks to ascertain that no external or internal entities can trace or link transactions back to individual participants.

\subsection{Proposal of Solutions for Physical Issues, such as Vehicular Damage}

While much of our research leans heavily on the digital side of the platform, we cannot ignore the real-world, tangible issues. A prominent concern in ride-pooling, especially in an autonomous setting, is the potential damage to vehicles. Thus, we aim to:

Develop Damage-Reporting Mechanisms: Allow users to report any damages or issues they encounter during their ride.

Ensure Accountability: While preserving user privacy, create a system that holds individuals accountable for any damages they cause.

Integrate Insurance or Damage Control Measures: Explore partnerships with insurance providers or other mechanisms to safeguard against substantial damages.

In conclusion, the objectives outlined here form the backbone of this research. They not only signify our commitment to advancing the realm of transportation but also underscore our dedication to ensuring that advancements cater to societal well-being, both in terms of mitigating traffic congestion and safeguarding user privacy and rights. Through these objectives, we hope to sculpt a future where autonomous driving and ride-pooling coalesce harmoniously, amplifying the strengths of each other while concurrently addressing their individual challenges.
\section{Methodology}
It is important to utilise a structured approach when designing the decentralised ride-pooling platform. The following methodology provides a step-by-step process where each step builds upon the previous one, ensuring the platform viability, resulting in the design and prototypical implementation of a decentralised, privacy-preserving ride-pooling platform that showcases the current state of technology and scientific research in that field.


\begin{enumerate}

    \item \textbf{Literature Research about Current Solutions}: 
    First, it is important to understand the current state of scientific research. Therefore, this paper will conduct an in-depth analysis of the current state of scientific literature, reviewing academic papers, industry reports, and white papers about ride-pooling, decentralised systems, and related technologies. The output of this step is a comprehensive overview of what has been achieved in the area of decentralised ride-pooling so far.

    \item \textbf{Identification of Shortcomings}: 
    Building upon the previous stage, this research will work out potential shortcomings of the current research landscape, point them out and propose solutions to balance out these shortcomings. This allows for the decentralised ride-pooling platform built through this research to not only be a gathering of existing research findings but also to provide added value to the research landscape.

    \item \textbf{Proposal of a Solution Design}:
    The next step is to create a comprehensive design for the platform based on the findings of the research analysis. This phase includes designing the architecture of the decentralised platform, defining interaction flows and outlining trust and privacy mechanisms. 
    

    \item \textbf{Prototypical Implementation of the Solution Design}: 
    To prove the viability of the design, it is necessary to build a prototype that can showcase that the core functions and components work as intended. Therefore, this step includes the programming of smart contracts, user interfaces and other components and enabling them to communicate with each other. The finished prototype allows for real-world testing and iterative refinement.

    \item \textbf{Evaluation Whether the Previously Set Requirements are met}: 
    Based on the results from the working prototype, it is then important to analyse if all research objectives set for the decentralised ride-pooling platform are met. Based on this evaluation, it is possible to determine if the research is a success and can provide a significant contribution to research.

    \item \textbf{Identification of Limitations and Proposal of Possible Improvements}:
    The evaluation is also meant to bring up shortcomings of the research and aspects of the platform that require further investigation. These shortcomings and possible improvements are clearly pointed out so that future researchers can take the findings of this work and use it as the base for their research.

\end{enumerate}

In summary, this iterative methodology ensures a scientific step-by-step approach to developing the  privacy-preserving ride-pooling platform. The research approach, therefore, helps meet all research objectives set for this work.

\chapter{Background Information}
\section{Autonomous Driving and Ride-Pooling}
The advent of autonomous driving, once the stuff of science fiction, is now on the cusp of revolutionizing the transportation sector. Coupled with the growing significance of ride-pooling, this technological development promises a transformative impact on urban mobility, energy consumption, and even the very layout of our cities. This section delves into the intricacies of both concepts, exploring their origins, developments, and the potential synergy they hold for the future of transportation.

Autonomous Driving:
Autonomous, or self-driving vehicles, utilize a blend of hardware and software to navigate and control the car without human intervention. Classified into levels 0 to 5, with 5 being fully autonomous, these vehicles rely on intricate systems of sensors, cameras, lidars, and radars. They constantly gather data about their environment, which is then processed by advanced algorithms to make driving decisions.

Historically, the concept of a car driving itself can be traced back to as early as the 1920s, but tangible progress started in the latter half of the 20th century. Projects like the EUREKA Prometheus Project in the 1980s and the DARPA Grand Challenge in the early 2000s played pivotal roles in advancing autonomous technology. Today, major tech firms and automobile companies are in a race, not just to refine this technology, but also to ascertain the ethical, legal, and infrastructural changes required for mass adoption.

The potential benefits of autonomous driving are vast:

Safety: Human error, responsible for a majority of road accidents, could be drastically reduced.
Efficiency: Optimal driving by autonomous cars might reduce traffic congestion and lead to more streamlined traffic flows.
Accessibility: Those unable to drive due to age, disability, or other factors can enjoy independent mobility.
Economic Impact: A reduction in accidents implies decreased costs in healthcare and vehicle repairs.
However, challenges persist. Technical hurdles, legislative barriers, ethical dilemmas (like decision-making in unavoidable accidents), and public skepticism need addressing for a broader acceptance.

Ride-Pooling:
Ride-pooling, distinct from ride-sharing, involves multiple riders sharing a single vehicle trip, where each passenger's destination is likely different, but their routes are similar. It's an evolution of the traditional carpooling concept, made more efficient and scalable by modern technology.

Platforms like UberPool and Lyft Line have popularized ride-pooling in urban environments. The appeal of such services lies in their promise of reduced commuting costs for passengers, decreased traffic congestion, lower carbon emissions due to fewer cars on the road, and the potential for a significant reduction in the need for parking spaces in urban areas.

However, ride-pooling isn't without its set of challenges. Efficient route optimization to ensure minimal detours, balancing demand and supply, and ensuring passenger safety are areas that companies constantly grapple with.

The Synergy of Autonomous Driving and Ride-Pooling:
When these two paradigms converge, we witness a compelling vision for the future of urban mobility.

Efficiency and Cost: With autonomous vehicles at the helm, ride-pooling can achieve unparalleled efficiency. Vehicles can be operational 24/7, reducing the per-trip cost and, by extension, the fare for passengers. The absence of a driver also implies more space for passengers or cargo.

Environmental Impact: Electric autonomous vehicles, combined with ride-pooling, can substantially reduce carbon emissions. Fewer cars would be on the road, and those in operation would be used more efficiently and likely be eco-friendlier.

Urban Planning: The convergence could reshape cities. With fewer vehicles on the road and less need for parking, vast tracts of land could be repurposed for green spaces, recreation, or housing.

Accessibility and Inclusivity: An autonomous ride-pooling system ensures mobility for all, regardless of age, disability, or socioeconomic status, potentially democratizing transportation.

However, the integration is not without potential pitfalls. Job losses, especially for drivers in the ride-sharing industry, the challenge of retrofitting infrastructure to accommodate autonomous vehicles, and the need to build robust, hack-proof systems are among the concerns that need to be addressed.

In conclusion, both autonomous driving and ride-pooling represent not just technological advancements but shifts in our societal approach to transportation. Their convergence holds the promise of a more efficient, eco-friendly, and inclusive transportation landscape. However, the journey there requires careful navigation, balancing the immense potential benefits with the inherent challenges. The next chapters will delve deeper into the technological backbone that can make this vision a reality.
\section{Blockchain Technology and Smart Contracts}
Blockchain technology is often considered as a revolutionary breakthrough, its applications span from the realm of finance to the intricate world of supply chain management, where it has been leveraged to address challenges like the lack of unified green standards and effective information disclosure mechanisms~\cite{Zhou.2023}. At its essence, a blockchain operates as a distributed ledger or database, meticulously recording transactions across a vast network of computers. This decentralised approach ensures that data modifications necessitate the unanimous approval of all system participants~\cite{Zhang.2022}. Such a structure is pivotal in guaranteeing that once information is incorporated, it solidifies as an immutable component of the database, thereby offering unparalleled levels of privacy, security, and data integrity~\cite{Pingale.2021}. The decentralised nature of blockchain technology eliminates the need for intermediaries, such as banks or servers, which are typically required in centralised systems for transactions between users~\cite{Pingale.2021}. This transformative technology, with its cryptographic consensus algorithms, stands as a testament to the evolution of secure and transparent data management in the digital age~\cite{Pingale.2021}.

\subsection{Introduction to Blockchain}
The beginning of blockchain technology can be traced back to the creation of Bitcoin in 2008~\cite{Nakamoto.2009}. Satoshi Nakamoto, the pseudonymous individual or group behind Bitcoin, introduced the concept as a solution to the double-spending problem in digital currencies~\cite{Nakamoto.2009}. Before Bitcoin, digital currencies struggled with the challenge of ensuring that a digital token was not spent more than once~\cite{Chiu.2022}. Nakamoto's solution was to create a decentralised system where transactions are verified by network participants through a consensus mechanism, eliminating the need for a central authority~\cite{Chiu.2022}. This was a groundbreaking innovation, as it allowed for the creation of decentralised monetary systems, free from governmental or institutional control.

One of the outstanding features of blockchain technology is its decentralisation~\cite{Gencer.2018}. Unlike traditional databases, such as an SQL database operated by a central entity, blockchains operate on a peer-to-peer network~\cite{Gencer.2018}. Every participant (or node) has access to the entire database and the complete history of all transactions. This means that no single participant has control over the data, and all participants collectively maintain the integrity of the data.

Immutability is another critical feature of blockchains. Once a transaction is recorded on the blockchain, it becomes extremely difficult to alter~\cite{Pilkington.}. This is because each block contains a cryptographic hash of the previous block, creating a chain of blocks~\cite{Pilkington.}. To change a single block, one would need to alter all subsequent blocks, which is computationally impractical, especially in large networks~\cite{ContedeLeon.2017}.

Transparency is inherently built into the system due to its open-source nature~\cite{Banupriya.2021}. Every transaction on the blockchain is visible to anyone who chooses to view it, ensuring full transparency in the network~\cite{Banupriya.2021}. However, personal information about the users conducting the transactions remains private as each user is commonly represented through some form of Public Key~\cite{Wei.2022}. This ensures a balance between transparency and privacy~\cite{Wei.2022}.

Security is a necessity in blockchain systems~\cite{Nguyen.2019}. Transactions must be approved through consensus mechanisms, like proof-of-work or proof-of-stake, before they are added to the blockchain~\cite{Nguyen.2019}. Additionally, the cryptographic nature of the technology ensures that data is secure and tamper-proof~\cite{Akbar.2021}.

While the foundational principles of blockchains remain consistent, there are different types tailored to specific needs~\cite{Ghosh.2021}. Public blockchains, like Bitcoin and Ethereum, are open to anyone and are simply secured by their cryptographic algorithms~\cite{Ghosh.2021}. In contrast, private blockchains, like the Hyperledger Blockchain Projects, can be restricted to a specific group of participants, often used by businesses for internal processes~\cite{Lu.2023}. Private Blockchains can also be used as Consortium blockchains, or federated blockchains, operated under the leadership of a group rather than a single entity or the publics~\cite{Lu.2023}. They provide a balance between the openness of public blockchains and the restrictions of private ones~\cite{Lu.2023}.

\subsection{How Blockchain Works}
Diving deeper into the mechanics of blockchain technology shows the interplay of cryptographic principles, network theory, and consensus algorithms~\cite{Xiong.2022}. At the core of this technology are blocks, which are essentially records of transactions. Each block typically contains a timestamp, a reference to the previous block (known as the parent block), and a list of transactions. These transactions are represented as cryptographic hashes, which are fixed-size strings of characters generated from input data of any size. The advantage of these hashes is that even a small change in the input data results in a completely different hash, ensuring the integrity of transaction records, that can be traced back to the very first block, known as the genesis block~\cite{Xiong.2022}.

Central to the operation of a blockchain is the concept of consensus mechanisms~\cite{Tahir.2022}. These are protocols that ensure all participants in the network agree on the validity of transactions. The most well-known consensus mechanism is Proof of Work (PoW). In PoW, participants, often referred to as "miners", solve complex mathematical problems to validate transactions and create new blocks~\cite{Kairaldeen.2021}. This process requires significant computational power and energy. An alternative mechanism, Proof of Stake (PoS), determines the creator of a new block based on their stake or ownership of the cryptocurrency. It's seen as a more energy-efficient alternative to PoW. Delegated Proof of Stake (DPoS) further refines this by allowing coin holders to vote for a few trusted nodes to validate transactions, streamlining the process and reducing the energy footprint~\cite{KUCHKOVSKY.2021}.

The blockchain network is maintained by nodes, which are computers participating in the network~\cite{Xiong.2022}. In general, there are two primary types of nodes: full nodes and light nodes~\cite{Mitra.2021}. Full nodes store the entire blockchain and validate all transactions and blocks. They serve as the backbone of the network, ensuring data integrity and consistency. Light nodes, on the other hand, store only a subset of the blockchain and rely on full nodes for transaction validation and other heavy operations~\cite{Mitra.2021}. Their primary role is to facilitate faster and more efficient interactions with the blockchain.

Transactions are used to interact with the blockchain. When a user initiates a transaction, it is broadcast to the network and placed in a pool of unconfirmed transactions. Nodes pick up these transactions, validate them against the blockchain's history to ensure legitimacy, and then place them in a block. Once the block is full, it's presented to the network for verification through the consensus mechanism. Upon successful verification, the block is added to the chain, and the transaction becomes a permanent part of the blockchain's history~\cite{Xiong.2022}.

\subsection{Smart Contracts}
As one part of blockchain technology, smart contracts have emerged as an advanced tool, extending the use-cases of blockchains beyond the record keeping of transactions~\cite{UchaniGutierrez.2023}. A smart contract is a self-executing contract where the terms of agreement or conditions are represented through written lines of code~\cite{Zhou.2022}. They are protocols that verify and enforce credible transactions without the need for third parties~\cite{Zhou.2022}. At their core, they are digital contracts that automatically execute actions when predefined conditions are met.

The concept of smart contracts is not entirely new, but its practical application gained momentum with the advancements of blockchain technology~\cite{Pierro.}. While Bitcoin introduced a rudimentary scripting language that allowed for some conditions to be set for transactions, it was Ethereum, launched in 2015, that demonstrated the potential of smart contracts~\cite{Pierro.}. Ethereum's platform is designed specifically to create and execute smart contracts, providing a more flexible scripting language and a platform for decentralised applications (DApps)~\cite{Pierro.}. Since Ethereum's creation, a number of other blockchains have integrated smart contract capabilities, offering unique features and optimisations.

Once deployed, smart contracts operate without human intervention, ensuring that actions are carried out correctly when conditions are met~\cite{UchaniGutierrez.2023}. This allows for interactions among parties that are not required to trust each other. Since the contract is on a blockchain, all parties can verify the contract's code and monitor its execution~\cite{UchaniGutierrez.2023}. The decentralised nature of blockchains also ensures that smart contracts are secure from tampering, providing an added layer of security~\cite{Zhou.2022}.

Understanding the life cycle of a smart contract provides insights into its operational particularities~\cite{Pierro.}. The journey begins with its creation, where the contract's terms are defined and encoded. Once the code is written and tested, it is deployed onto the blockchain, showing up as an immutable part of the ledger~\cite{Pierro.}. After the deployment, the contract is now active and can start receiving and processing information. Execution occurs when the conditions specified in the contract are met, triggering the actions encoded in the contract~\cite{Pierro.}. While many smart contracts are designed to run without a predefined end, there are scenarios where they might have a termination condition, ending the contract's active state on the blockchain.

Even though smart contracts have many advantages, they also come with their own set of limitations and challenges~\cite{.2019}. One notable challenge in the Ethereum network is the concept of gas fees. Every operation, from contract deployment to execution, requires computational resources. Users pay for these resources using so-called gas and with increased network activities, these fees can rise. Scalability remains a concern as well. As more complex smart contracts and DApps are developed, there is a growing demand for blockchains to process more transactions per second without compromising on security or decentralisation~\cite{.2019}. Lastly, smart contracts are only as good as the code they're written in. Coding errors or oversights can lead to vulnerabilities, potentially allowing malicious actors to exploit the contract~\cite{Zhou.2022}.

In conclusion, smart contracts represent a significant leap in how agreements and transactions can be managed on a decentralised network~\cite{.2019}. While they offer numerous advantages, it is essential to navigate their challenges effectively to utilize their full potential~\cite{.2019}. As the blockchain ecosystem evolves, it is expected that smart contracts will become even more integrated into various sectors, reshaping traditional processes and systems~\cite{.2019}.

\subsection{Blockchain Security}
Blockchain's decentralised nature, which is often used for its resilience and transparency, also presents unique security challenges~\cite{Leng.2022}. One of the most discussed vulnerabilities is the 51 Percent attack. In such an attack, if a single entity gains control of more than half of the network's mining power, they can potentially double-spend coins and halt or reverse transactions. This undermines the trust and integrity of the blockchain~\cite{Singh.2021}. Similarly, Sybil attacks occur when a single adversary controls multiple nodes, aiming to flood the network with false transactions or undermine mechanisms that rely on redundancy and trust.

Smart contracts have their own set of security concerns~\cite{Alkhalifah.2021}. Reentrancy attacks are a prime example, where an attacker drains funds from a contract by repeatedly calling its functions before the initial function call is completed~\cite{Alkhalifah.2021}. Issues like overflow and underflow, where variable values exceed their set limits, can also be exploited, leading to unintended consequences in contract execution~\cite{Guo.2022}. These vulnerabilities underscore the importance of rigorous code audits and testing before deploying smart contracts on a live network.

Privacy, a cornerstone of blockchain's appeal, is nuanced~\cite{Kus.2022}. Most blockchains offer pseudonymity, where transactions are linked to a cryptographic address rather than personal identities. However, with sophisticated analysis, patterns can emerge, potentially de-anonymizing users~\cite{Kus.2022}. True anonymity, where transaction details and participants are obscured, remains a challenge. Innovations like zero-knowledge proofs, which allow one party to prove to another that a statement is true without revealing any specific information, are promising solutions~\cite{Deng.2021}. Additionally, private blockchains, restricted to specific participants, can offer enhanced privacy but at the cost of decentralisation~\cite{Deng.2021}.

In essence, while blockchain offers robust security mechanisms at the core of its design, it is not impenetrable. As the technology matures, addressing these vulnerabilities will be an important part for ensuring its general adoption and trustworthiness~\cite{Singh.2021}.

\section{Decentralized Platforms and Data Privacy}
Decentralised platforms, at their core, are systems where components, like resources, or operations, are not controlled or managed by a single, central entity. Instead, they are distributed across multiple nodes, with each having equal authority and autonomy. This is in direct contrast with traditional centralised systems, where a single entity or a group of entities holds all the power and control. In the following chapter we will introduce the core concepts behind decentralised platforms and showcase how data privacy can be handled in a system without a central authority ~\cite{Tverdokhlib.2022}.


\subsection{Introduction to Decentralised Platforms}

One of the primary characteristics of decentralised platforms is that it does not have a central point of control. This means that no single entity has the authority to make decisions on their own or changes without consensus from the majority of the network's participants~\cite{SEFRAOUI.2022}. This leads to enhanced security, as the absence of a single point of failure makes the system more resilient to attacks~\cite{Maffiola.2022}. Additionally, decentralised platforms often employ cryptographic techniques to ensure data integrity, privacy, and authentication. This ensures that transactions and interactions on the platform are secure, verifiable, and immutable~\cite{SEFRAOUI.2022}.

Comparing decentralised platforms with centralised systems reveals strong differences in their operational philosophies. Centralised systems, such as traditional databases or web servers, are controlled by a single entity. This central authority has the power to set rules, make changes and grant or deny access~\cite{Maffiola.2022}. While this centralisation can lead to efficiencies in terms of decision-making and simpler system architectures, it also presents vulnerabilities. A single point of failure in a centralised system can lead to the entire system collapse. Moreover, centralisation often results in data silos, where a single entity has control over large amounts of data ~\cite{Maffiola.2022}.

On the other hand, decentralised platforms operate on the principles of democracy and transparency~\cite{SEFRAOUI.2022}. Decisions are made based on consensus algorithms, ensuring that no single participant can dominate or manipulate the system. This democratisation of control can result in trust among users, as the platform operations are transparent ~\cite{Hasan.2022}. Data in decentralised systems is typically stored across multiple nodes, ensuring redundancy and resilience. Even if one or more nodes fail, the system can continue to operate seamlessly ~\cite{Hasan.2022}.

There are several potential benefits for decentralised platforms~\cite{Hasan.2022}. Firstly, they offer enhanced security and resilience due to their distributed nature. The risk of system-wide failures or attacks is significantly reduced~\cite{Maffiola.2022}. Secondly, they promote transparency and trust among users, as decisions are made collectively and openly~\cite{Hasan.2022}. Additionally, decentralised platforms lead to innovations in peer-to-peer transactions, like smart contracts, and decentralised applications, that allow for new business models.

However decentralised platforms come with a set of challenges~\cite{Hasan.2022}. The lack of a central authority can lead to slower decision-making, as achieving consensus can be time-consuming. Additionally, the technology enabling decentralised platforms, such as blockchain, is still maturing, leading to scalability and performance issues~\cite{Hasan.2022}. Interoperability between different decentralised platforms is also a challenge.

\subsection{Data Privacy: Definition and Importance}
Data privacy, at its core, refers to the right of individuals to control or influence what information about them is collected and how it is used. It centres around the rules put in place to protect personal information and ensure that individuals remain in control of it~\cite{Covert.2020}. This concept is crucial for several reasons.

Data privacy is directly linked to personal autonomy and dignity. Personal data can reveal intimate details about an individual's life, preferences, and habits. Ensuring that such information is not misused or mishandled is vital~\cite{Covert.2020}. Without robust data privacy measures, individual rights can be violated, leading to a loss of trust in digital systems and platforms.

Furthermore, in the context of businesses and services, data privacy is important for maintaining consumer trust. Companies that fail to protect user data or misuse it can face significant reputational damage, legal consequences, and financial losses~\cite{Li.2019}. In sectors like decentralised ride-sharing, where users share location data, payment details, and personal preferences, ensuring data privacy can be the difference between a successful platform and one that users abandon due to trust issues~\cite{Li.2019}.

While data privacy is a critical concept, it is essential to differentiate it from related terms like data security and data protection, as they are often used interchangeably but have distinct meanings.
Data security refers to the protective measures and technologies used to secure data from unauthorised access. It focuses on defending data from malicious threats, like hackers, malware, or other cyber-attacks. For instance, using encryption to secure data to prevent unauthorised access is an example of data security practices.

On the other hand, data protection is a broader concept that includes both data privacy and data security. It refers to the policies, procedures, and legal measures designed to ensure that data is collected, stored, and used in a way that respects individual rights and complies with relevant laws and regulations~\cite{Covert.2020}. 

In conclusion, data privacy is the right of individuals to control their personal information and its usage~\cite{Covert.2020}. As we continue to integrate digital platforms into our everyday life, understanding and prioritising data privacy will become even more important.

\subsection{Mechanisms of Data Privacy in Decentralised Blockchain Systems}
Decentralised blockchain systems are a revolutionary technology, offering data transparency, immutability and security. However, the nature of public blockchains, which are open and transparent, creates significant privacy challenges. Every transaction and its associated data are visible to anyone who accesses the blockchain, leading to potential privacy breaches and exposure of sensitive information.

Encryption plays an important role in addressing these challenges. At its core, encryption involves converting data into a code to prevent unauthorised access. In the context of blockchains, wallets consisting of a public key (often linked to an address on the blockchain) and private key are widely used. A public key, visible to everyone, is used to encrypt data, while a private key, known only to the owner, is used to decrypt it. This ensures that only the intended recipient can access the information. Furthermore, end-to-end encryption ensures that data remains encrypted during its entire journey from the sender to the recipient, preventing potential eavesdroppers from accessing the information during transmission.

One approach to establishing  an encrypted connection over a public network is the Diffie-Hellman Key Exchange. The Diffie-Hellman Key Exchange, introduced by Whitfield Diffie and Martin Hellman in 1976, is a cryptographic protocol that allows two parties to independently generate a shared secret key over an insecure communication channel. The protocol is based on the mathematical properties of modular arithmetic and discrete logarithm problems. Specifically, given a prime number \( p \) and a base \( g \) (where \( g \) is a primitive root modulo \( p \)), each party selects a private key and computes a public key. The public keys are then exchanged, and each party uses the other's public key along with their own private key to compute the shared secret. The security of the protocol relies on the difficulty of the discrete logarithm problem: while it is computationally easy to generate the public key from the private key, the reverse operation is considered infeasible with current technology when large prime numbers are used.

Another advanced cryptographic technique employed in blockchains is zero-knowledge proofs (ZKPs). ZKPs allow one party to prove to another that a statement is true without revealing any specific information about the statement itself. For instance, in a transaction, a user can prove they have sufficient funds without revealing the exact amount. This ensures transaction validity while preserving user privacy.

Homomorphic encryption offers another layer of privacy. It allows computations to be performed on encrypted data without first decrypting it. The result, when decrypted, remains accurate. This means that blockchain systems can process transactions and maintain data integrity without exposing the actual data, a boon for privacy-centric applications.

While public blockchains offer transparency, they create challenges for the development of applications, especially those requiring higher levels of privacy. Private and consortium blockchains emerge as alternatives in such scenarios. Private blockchains restrict participation to selected entities, while consortium blockchains involve multiple organisations governing the network. Both these types limit data visibility to only authorised participants, enhancing data privacy.

Off-chain storage is another solution to the privacy challenges. Instead of storing all data on the blockchain, only essential information is kept on-chain, while the rest is stored off-chain in secure databases. This reduces the amount of data exposed on the public ledger, ensuring privacy.

Lastly, layer 2 solutions, built on top of the primary blockchain, offer scalability and privacy improvements. By processing transactions off the main chain and only writing the final state on-chain, these solutions can ensure faster transactions and enhanced privacy. In conclusion, while decentralised blockchain systems present certain inherent privacy challenges, a combination of cryptographic techniques and architectural solutions can effectively address these concerns.


\section{Current Solutions and Shortcomings}
Due to the diverse literature regarding decentralised ride-pooling platforms, the proven approach of a systematic literature search according to Vom Brocke was chosen~\cite{vomBrocke.2009}. In this way, quality criteria such as traceability and reproducibility can be ensured through a clearly defined processes. Two common, cross-publisher research databases and one common publisher database were used for the literature search.
The selection of several cross-publisher research databases is intended to ensure that the search provides a representative overview of existing research on decentralised ride-pooling platforms. The selection of the database of a publisher with a focus on information technology is intended to show how the research topic is treated in the literature from a primarily information technology perspective. The cross-publisher research databases used are Scopus and Ebscohost. The publisher database is IEEE Xplore.
The goal is to obtain research literature as a search result that deals with the development of decentralised ride-pooling platforms. To obtain results covering mainly decentralised platforms the search phrase ``decentralised'' was used. The following three synonyms were used to obtain search results that deal with the topic of ride-pooling: ``ride-sharing'', ``ride-pooling'' and ``ride-hailing''. Initial tests have shown that results with this search phrase return suitable research papers without noticeable gaps in regards to the topics covering decentralised ride-pooling.

The complete search phrase looks as followed:
\begin{verbatim}
("decentralised" AND ("ride-pooling" OR "ride-sharing" OR "ride-hailing"))
\end{verbatim}

For Scopus, Epscohost (all selectable databases included) and IEEE Xplore, the search phrase was applied to the title, abstract and keywords of the publications. Initial tests have shown that restricting the search to title, abstract and keywords is the best compromise between the quantity and quality of the search results. Only literature that was published in the after 2014 (2015 – 2023) was considered for the literature search. This is to ensure that the specialist literature found is of current relevance without overly restricting the scope of the existing research literature. Likewise, after the initial compilation of the search results, all duplicates were removed. In this way it is avoided that publications are counted twice because they are listed in several literature databases.

\subsection{Selection of the findings}
The literature search was carried out between the 28. July and the 18. of August 2023, resulting in 86 hits. A criteria-based selection was made beyond the search phrase and the time limit for the publication of the specialist literature. The exclusion criteria used in the criteria-based selection are no publications in languages other than English, no panels and comments, and no literature dealing with decentralised platforms or ride-pooling. In addition, publications that are not freely available or accessible via a license from the University of Stuttgart had to be excluded. The inclusion criteria used are Only publications in English, only publications from 2015 onwards, and only papers discussing the technical development of decentralised ride-pooling platforms~\cite{Bandara.2015}.
Following Bandara, a first check of the actual relevance of the hits for answering the research question was carried out by screening the title, keywords and abstract. A full-text analysis was then carried out on the literature that was still considered relevant after the initial screening. Applying the inclusion and exclusion criteria in the initial screening and the subsequent full-text analysis, 10 relevant publications were identified from the 86 search hits for answering the research question. Additionally, two more relevant papers could be identified by following citations from the relevant literature.  Table \ref{tab:litSearchResults} shows how the relevant research literature is distributed across the research databases. 


\begin{table}[h]
\centering
\caption{Results of the Literature Search}
\label{tab:litSearchResults}
\begin{tabular}{|l|c|c|c|}
\hline
Scientific Database & Search Results & Excluded Literature & Included Literature \\ \hline
Scopus & 54 & 49 & 5 \\ \hline
Epscohost & 2 & 2 & 0 \\ \hline
IEEE & 30 & 25 & 5 \\ \hline
Citation search &  &  & 2 \\ \hline \hline
Total & 86 & 73 & 12 \\ \hline
\end{tabular}
\end{table}

The analysis of the publications shows that many different approaches are discussed in the scientific literature on how decentralised ride-pooling platforms can be built. For the results of the literature analysis to be evaluated and interpreted, the results must first be structured. For this purpose, a concept matrix approach, according to Webster and Watson is pursued~\cite{Webster.2002}. Based on the concept matrix approach, the specialist literature identified as relevant is assigned to eight topics relevant for the creation of a decentralised ride-pooling platform. These eight topics are derived from a general analysis of the topics covered by the scientific literature combined with topics relevant to fulfilling the research objectives:

\begin{itemize}
    \item \textbf{Blockchain Utilisation}: Blockchain is the underlying technology used for the creation of the decentralised ride-pooling platform. The literature needs to show in detail how blockchain technology is utilised by smart contracts and cryptocurrencies to build a ride-pooling platform.
    
    \item \textbf{Payments and Service Fees}: The decentralised ride-pooling platform must manage ride payments and general service fees. Therefore it is important for the literature to show how these financial transactions can be implemented and how to ensure that ride providers are compensated fairly for their services inside the decentralised ride-pooling platform.
    
    \item \textbf{Privacy and Anonymity}: Using blockchain technology demands a robust architecture that ensures privacy and anonymity for all users inside the platform. The scientific literature must showcase how users can engage with the platform and other users without revealing their identity directly or implicitly by sharing too much personal data with the platform over a longer time period.
    
    \item \textbf{Security and Resilience}: For a decentralised platform to gain widespread adoption, it must guarantee the safety and security of all parties. While the blockchain itself already provides many security features by design, it is important for the literature to show how the off-chain components are hardened and how to prevent the off-chain components from providing false information to the on-chain components.
    
    \item \textbf{Trust Mechanisms}: As decentralised platforms can not rely on a central trusted authority, robust trust mechanisms become essential. The research papers must explain how community trust mechanisms can be successfully implemented into a decentralised platform.
    
    \item \textbf{Off-Chain Edge Cases}: It is impossible to handle every edge case through the decentralised platform. As there is no central authority, it is important to provide alternative solutions to solve these problems without contradicting the decentralised nature of the platform. The research needs to recognise the existence of these edge cases and has to provide solutions to handle them.
    
    \item \textbf{Customer and Ride Provider Interaction Flow}: The customer and ride provider interaction flow stands in the centre of the decentralised ride pooling platform. The literature needs to provide insides into how this flow should look to utilise the advantages of blockchain technology.
    
    \item \textbf{Prototypical Realisation}: Before building a market-ready version, the decentralised ride-pooling platform should be built as a prototype that showcases the most important aspects of the platform and proves its feasibility. Therefore it is important for the literature to include a prototypical realisation of the platform that provides important insides that are not which cannot be derived from the architecture alone.
\end{itemize}


As a result, the concept matrix shows the frequency with which the concepts dealt with in the specialist literature are distributed over the nine topics of decentralised ride-pooling. The assignment of the concepts on the x-axis to authors of the relevant specialist literature on the y-axis can be seen in Table \ref{tab:litSearchResultsMatrix}. If a research paper covers a topic in detail, it is marked with $\checkmark$ $\checkmark$. if a research paper covers some aspects of a topic, it is marked with a $\checkmark$ . If a paper does not cover a topic at all or in a way that does not align with the objectives of this research, it is marked with a $\times$. 


\begin{longtable}{p{5cm}l|l|l|l|l|l|l|l|l}
\caption{Results of the Literature Search} \\
\label{tab:litSearchResultsMatrix}
Research Paper / Topic & 
&
\rotatebox{90}{Blockchain Utilisation} & 
\rotatebox{90}{Customer and Ride Provider Interaction Flow} & 
\rotatebox{90}{Payments and Service Fees} & 
\rotatebox{90}{Privacy and Anonymity} & 
\rotatebox{90}{Security and Resilience} & 
\rotatebox{90}{Trust Mechanisms} & 
\rotatebox{90}{Off-Chain Edge Cases} & 
\rotatebox{90}{Prototypical Realisation} \\ 
\hline
\endfirsthead

\multicolumn{9}{c}%
{{\bfseries \tablename\ \thetable{} -- continued from previous page}} \\
\hline
Research Paper / Topic & 
&
\rotatebox{90}{Blockchain Utilisation} & 
\rotatebox{90}{Customer and Ride Provider Interaction Flow} & 
\rotatebox{90}{Payments and Service Fees} & 
\rotatebox{90}{Privacy and Anonymity} & 
\rotatebox{90}{Security and Resilience} & 
\rotatebox{90}{Trust Mechanisms} & 
\rotatebox{90}{Off-Chain Edge Cases} & 
\rotatebox{90}{Prototypical Realisation} \\ 
\hline
\endhead

\hline \multicolumn{9}{r}{{Continued on next page}} \\
\endfoot

\hline
\endlastfoot


B-Ride: Ride Sharing With Privacy-Preservation, Trust and Fair Payment Atop Public Blockchain & ~\cite{Baza.2021} & $\checkmark$ $\checkmark$ & $\checkmark$ $\checkmark$ & $\checkmark$  & $\times$ & $\times$ & $\checkmark$ & $\times$ & $\checkmark$  \\
\hline
Application of Blockchain Technology to Smart City Service: A Case of Ridesharing & ~\cite{Chang.} & $\checkmark$ & $\checkmark$ & $\times$ & $\checkmark$  & $\checkmark$ $\checkmark$ & $\times$ & $\times$ & $\times$ \\
\hline
Ride-Hailing for Autonomous Vehicles: Hyperledger Fabric-Based Secure and Decentralize Blockchain Platform & ~\cite{Shivers.} & $\checkmark$ & $\checkmark$ & $\times$ & $\checkmark$ & $\times$ & $\times$ & $\times$ & $\checkmark$  \\
\hline
RiderS: Towards a Privacy-Aware Decentralized Self-Driving Ride-Sharing Ecosystem & ~\cite{Bathen.} & $\checkmark$ & $\checkmark$ & $\checkmark$ & $\checkmark$$\checkmark$ & $\checkmark$ & $\times$ & $\times$ & $\checkmark$ \\
\hline
A Decentralized Ride-Hailing Mode Based on Blockchain and Attribute Encryption & ~\cite{Zhang.} & $\checkmark$& $\checkmark$ & $\checkmark$ $\checkmark$ & $\checkmark$ & $\times$ & $\times$ & $\times$ & $\checkmark$  \\
\hline
Enhancing Blockchain-based Ride-Sharing Services using IPFS & ~\cite{Mahmoud.2022} & $\checkmark$$\checkmark$ & $\checkmark$$\checkmark$ & $\checkmark$ & $\checkmark$ & $\checkmark$ & $\times$ & $\times$ & $\checkmark$  \\
\hline
BlockWheels - A Peer to Peer Ridesharing Network & ~\cite{Joseph.} & $\checkmark$ & $\checkmark$$\checkmark$ & $\checkmark$ & $\checkmark$ & $\times$  & $\times$  & $\times$ & $\times$  \\
\hline
A Light Blockchain-Powered Privacy-Preserving Organization Scheme for Ride Sharing Services & ~\cite{Baza.52520205282020} & $\checkmark$ & $\checkmark$ $\checkmark$ & $\checkmark$ & $\checkmark$ & $\times$ & $\times$ & $\times$ & $\times$ \\
\hline
\pagebreak
BlockV: A Blockchain Enabled Peer-Peer Ride Sharing Service & ~\cite{Pal.} & $\checkmark$ & $\checkmark$ & $\checkmark$ & $\times$  & $\checkmark$ & $\checkmark$  &  $\checkmark$ &  $\checkmark$ \\
\hline
Blockchain-Based Ride-Sharing System with Accurate Matching and Privacy-Preservation & ~\cite{Badr.} &  $\checkmark$ & $\checkmark$ & $\times$ & $\checkmark$  & $\checkmark$ $\checkmark$ & $\times$ & $\times$ & $\times$ \\
\hline
Towards Blockchain-based Ride-sharing Systems & ~\cite{Vazquez.} & $\checkmark$ $\checkmark$ & $\checkmark$ & $\checkmark$ & $\checkmark$ & $\checkmark$ & $\times$ & $\times$ & $\checkmark$  \\
\hline
Co-utile P2P ridesharing via decentralization and reputation management & ~\cite{Sanchez.2016} &  $\checkmark$ &  $\checkmark$ &  $\checkmark$ &  $\checkmark$  $\checkmark$ &  $\checkmark$ &  $\checkmark$ & $\times$ & $\times$   \\
\hline
\end{longtable}

\subsection{Scientific Literature findings}
The concept matrix \ref{tab:litSearchResultsMatrix} shows that the literature review did not identify a single paper that provides detailed coverage of all topics and would thereby allow us to answer all research objectives. The matrix also shows that while many of the papers discuss multiple topics, they often remain on a conceptual level without the goal of developing a feature-complete platform. It is still very important to take a detailed look at the identified literature to discuss their approaches on developing a decentralised ride-pooling platform. In the following we will take a look at the outstanding features that are proposed in each paper and evaluate how they can support the creation of our feature complete ride polling service.


``B-Ride: Ride Sharing With Privacy-Preservation, Trust and Fair Payment Atop Public
introduces'' B-Ride, a decentralised ride-sharing service built on public Blockchain~\cite{Baza.2021}. B-Ride ensures ride data privacy for both drivers and riders. To counter malicious users exploiting blockchain's anonymity, the system introduces a time-locked deposit protocol using smart contracts and zero-knowledge set membership proof. This ensures trust and commitment from all participants. A unique "pay-as-you-drive" methodology is proposed for fair payment, where drivers are compensated based on the distance covered. This 
system has many advantages. It ensures that the ride provider gets paid for the driven distance, and the customer does not have to deposit more money than necessary at once. The problem with this approach is, that it requires so called Location Prover. These hardware devices ensure that the car provides honest location information about its position. While this technology is superior to systems that do not relay on Location Prover, a global network of Location Provers is currently not feasible. Therefore our platform will utilise an upfront deposit of the expected ride cost by the user that can be claimed by the ride provider after completing the ride.
Additionally, B-Ride features a decentralised reputation management mechanism, rating drivers on past behaviour, incentivizing them to maintain good conduct. The system was successfully implemented and tested on the Ethereum blockchain, highlighting its real-world applicability. While a rating system is needed to ensure trust on the platform, B-Rides implementation also relies on Location Provers. Therefore we will look at other research papers and their approaches in regards to rating mechanisms.

The authors, Shuchih Ernest Chang and Chi-Yin Chang, highlight in their research paper ``Application of Blockchain Technology to Smart City Service: A Case of Ridesharing'' ~\cite{Chang.} the challenges faced by traditional ridesharing platforms. To address the challenges of traditional ridesharing platforms, the SmaRi system leverages blockchain technology and smart contracts. This approach not only ensures secure and automated transactions but also promotes decentralised decision-making. The research emphasises the potential of blockchain in reshaping ridesharing services. A notable design decision by the authors is to use an off-chain authentication service called social networking service. This service allows users to utilise social media accounts to share rides with friends and to authenticate against the platform. While this concept is not covered in depth it provides insides into the many advantages of an off-chain authentication service.

The paper ``Ride-Hailing for Autonomous Vehicles: Hyperledger Fabric-Based Secure and Decentralize Blockchain Platform'' addresses the problems of centralised ride-sharing platforms. The authors propose a decentralised approach using blockchain technology, allowing individual AV owners to contribute their vehicles to a community-driven fleet when not in use.~\cite{Shivers.} The chosen blockchain platform for this endeavour is Hyperledger Fabric. The paper is notable for utilising a private blockchain to tackle the problems in regards to anonymity and privacy, which are inherent downsides of public blockchains. The decision between a public and a private blockchain is one of the core architectural decisions for our own ride-sharing platform.  After taking the arguments by ~\cite{Shivers.} as well as other research papers into consideration, we decided to go forward with a public blockchain for our platform. With privacy being a focus of our ride pooling platform, there should be no possibility to trace individual user activity by monitoring the chain activities, even if it is public. Therefore we prioritise the increased decentralisation of public chains. Using a public chain allows us to utilise generic public nodes to handle smart contracts. Thereby we do not need to build a private network of independent node providers to build a private blockchain. Other research papers also prove the feasibility of decentralised ride pooling platforms on public chains ~\cite{Mahmoud.2022} ~\cite{Joseph.} ~\cite{Baza.52520205282020}

Another research paper introduces ``RiderS, a groundbreaking decentralized self-driving ride-sharing ecosystem'' ~\cite{Bathen.}.Central to this is the emphasis on user privacy, achieved through a privacy-first biometric technology. Instead of traditional passwords, users become their own unique identifier, ensuring genuine system interactions. To fortify this ecosystem, blockchain technology is employed, offering benefits like decentralisation and auditability. Each participant, whether a rider or an autonomous vehicle (AV), accesses the system via a ''Wallet''. This software client manages credentials, facilitates transactions, and serves as the primary gateway into the blockchain. Monetary exchanges within this ecosystem utilise a stable coin named ''Mobi'', anchored to various cryptocurrencies and fiat currencies. This system is very usefull and should be adapted by our platform. By introducing a Crypto Exchnage to the platform we allow the users to pay with a verity of different currencies including fiat currencies while still utilising the advantages of crypto currencies   in our platform. A standout feature is the emphasis on privacy. Users can generate single-use addresses, ensuring anonymity for each ride. This also should be adapted by our platform. Even though the wallet owner is anonymous on the chain, it prevents wallet tracking over a long period of time, which could lead the exposure of the wallet holder.

The research paper ``A Decentralized Ride-Hailing Mode Based on Blockchain and Attribute Encryption'' presents a novel ride-hailing approach using blockchain and attribute encryption.~\cite{Zhang.} 
The system includes a decentralised Blockchain-Based Ride-Hailing Mode: This mode has roles such as the Passenger, who generates encrypted ride details; the Driver, who decrypts and decides on ride acceptance; the Location Prover (LP), verifying the driver's location; and the Authentication Center, distributing keys and authenticating identities. Thereby the paper introduces a number of concepts that help us create our privacy preserving ride pooling platform. First of all the concept of creating a shared secret between customer and ride provider should be used  to share sensitive information on chain, like exact coordinates. With the  Authentication Center the paper also introduces an off chain authentication service, which further promotes the concept of an off chain authority that can verify  wallets to handle on chain interactions with the ride pooling platform.

The authors of ``Enhancing Blockchain-based Ride-Sharing Services using IPFS''  propose a decentralised ride-sharing system to address challenges in centralised services, such as security concerns and single points of failure.~\cite{Mahmoud.2022}  The solution integrates blockchain with the Interplanetary File System (IPFS). Instead of storing all ride-sharing data on the blockchain, the system moves this data to IPFS and only retains a compact hash on the blockchain. This approach reduces data storage on the blockchain, leading to faster processing and lower costs. The system uses smart contracts on the Ethereum platform for management, and experimental results highlight its scalability and efficiency. This concept should be utilised if the prototype implementation or future iterations of the platforms should struggle with managing the amounts of data necessary to hanlde rides, resulting in high gas prices or slow blockchain performance.


The paper ``BlockWheels - A Peer to Peer Ridesharing Network a ridesharing'' system built on the Ethereum blockchain. ~\cite{Joseph.} introduces a sophisticated ride-matching system, utilising geolocation tools to pair riders with nearby drivers. While out platform is utilising an auction based approach to match customers with ride providers, this paper showcases the advantages of an off chain matching approach to handle the complex matching with an on chain ride handelig that tracks the actual ride.

The paper ``A Light Blockchain-Powered Privacy-Preserving Organization Scheme for Ride Sharing Services'' introduce a decentralised system using public blockchain, eliminating the central third-party vulnerabilities.~\cite{Baza.52520205282020} This system ensures location and time privacy by employing spatial and temporal cloaking techniques, allowing riders and drivers to share generalised locations and time intervals instead of exact details. This approach should also be utilised with our platform. With a location matching based on approximated data we can ensure that the customer only needs to share their exact location with the ride provider that will fulfill the ride request.
BlockWheels participants also uses changing pseudonyms for each trip, ensuring untraceability. With BlockWheels also promoting this concept it shows that this approach to ensuring untraceability is a best practice in regards to on chain user activities. The entire scheme has been practically implemented and tested on the Ethereum platform, showcasing its feasibility and effectiveness in real-world scenarios.

The authors ~\cite{Pal.}  introduces a decentralized ride-sharing solution using blockchain. BlockV ensures fairness in ride-sharing in two main ways:
Payment Fairness: It allows any network peer to compute the ride cost based on path details.
Ride Fairness: In case of disputes, the system collaborates with Road Side Units (RSUs) to determine and penalize any malicious activity by drivers or riders.
The BlockV system involves four key participants: the DRIVER, RIDER, BlockV, and RSUs. The process starts with riders selecting a route and fare from a decentralised database. Once chosen, they confirm the ride and lock in the fare. At the ride's end, riders can either complete the ride, releasing funds, or raise complaints if unsatisfied. The system then verifies complaints using RSUs and takes appropriate action.
With the RSUs BlockV provides a solution to the problem on how to manage edge-cases like customer complaints. While this concept relies on the existence of RSUs and mainly focuses on the handling of false routes taken by the ride provider it showcases the importance of robust edge case handling.

The paper ``Blockchain-Based Ride-Sharing System with Accurate Matching and Privacy-Preservation'' proposes a method of dividing the ride-sharing coverage area into small cells using overlapping grids ~\cite{Badr.}. This ensures that customers and ride providers are matched with location accuracy, as they report their locations by cell numbers. When their exact locations coincide within a common cell across any grid, a match is made. While this approach does not utilises the planned auction system proposed by our platform it promotes a grid based approach that can help to ensure that potential matching services can be bound to specific areas. By assinging matching services to single tiles in a grid we can assure that each customer can find their local matching service and no matching service can collect data for areas that are too large.


The paper ``Towards Blockchain-based Ridesharing Systems'' addresses privacy concerns, by also utilising spatial cloaking and an off chain matching service.~\cite{Vazquez.} When a passenger requests a ride, an off-blockchain algorithm matches them with suitable drivers based on this cloaked data. To foster a sense of trust, both parties, the ride provider and the customer, post a deposit fee through a smart contract. This deposit acts as a commitment, and if either party defaults, the other is automatically compensated. This flow very much aligns with our vision of interaction flow of our decentralised ride pooling platform. The main advantage of this approach is that it allows for more complex matching algorithms without dramatically increasing gas fees, while still utilising the advantages of blockchain by tracking the actual ride and related payments on chain. 

The research paper ``Co-utile P2P ridesharing via decentralization and reputation management'' focuses on preserving user privacy~\cite{Sanchez.2016}. In practice, this means that only when a driver's and passenger's trips align will they be privy to each other's identity, desired trip details, and reputation. This selective disclosure ensures that personal data remains confidential. This also aligns with the research objectives of our decentralised ride pooling platform and needs to be considered in the final design.
Addressing privacy alone isn't enough; trust is equally important. The authors tackle this by weaving in a decentralised reputation management mechanism. Post a shared ride, both drivers and passengers have the liberty to rate each other. This allows peers to gauge the aggregated reputation of others, based on historical ratings, in a manner that's both transparent and trustworthy. This is a common best practice even with centralised ride sharing platforms. For our decentralised platform the rating should also be managed on chain, as it profits from the tamper proof nature of blockchain.

\subsection{Conclusion}
The detailed literature review show that there are many different approaches on how a decentralised ride pooling platform should be designed, with different authors focusing on different aspects of the platform. While there are many common best practices in regards to safety and user privacy there is also no uniform approach to designing the different components of the platform. While some papers suggest to handle all interactions with the platform on chain others suggest taking some elements off chain to allow for more complex flows. Therefore we can not rely on simply combining the platforms from the research papers into a single, feature complete platform.

Therefore, to create a feature complete ride pooling platform, it will be necessary to make design decisions that will contradict the suggested approaches of some papers to embrace design decisions made by other papers. These decisions will be made based on our research objectives, which state that the maximisation of privacy, security and transparency is the underlying goal of our platform.




\chapter{Proposed Solution}
\section{Conceptual Design of the Decentralized Platform}
\section{Interaction Protocol between Platform, Users, and Providers}
\section{Trust Mechanism Design}
\section{Privacy Measures}

\chapter{Implementation of the Decentralized Platform}
\section{Smart Contracts}
\section{User Frontend}
\section{Matching Service}
\section{Virtual Car}

\chapter{Evaluation}
\section{Validation against Requirements}
\section{Privacy Considerations}
\section{Testing and Results}

\chapter{Conclusion \& Outlook}
\section{Identified Limitations}
\section{Outlook}

\printbibliography

All links were last followed on March 17, 2018.

\appendix


\pagestyle{empty}
\renewcommand*{\chapterpagestyle}{empty}
\Versicherung
\end{document}
