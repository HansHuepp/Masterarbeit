% !TeX spellcheck = en-US
% !TeX encoding = utf8
% !TeX program = pdflatex
% !BIB program = biber
% -*- coding:utf-8 mod:LaTeX -*-


% vv  scroll down to line 200 for content  vv


\let\ifdeutsch\iffalse
\let\ifenglisch\iftrue
\input{pre-documentclass}
\documentclass[
  % fontsize=11pt is the standard
  a4paper,  % Standard format - only KOMAScript uses paper=a4 - https://tex.stackexchange.com/a/61044/9075
  twoside,  % we are optimizing for both screen and two-side printing. So the page numbers will jump, but the content is configured to stay in the middle (by using the geometry package)
  bibliography=totoc,
  %               idxtotoc,   %Index ins Inhaltsverzeichnis
  %               liststotoc, %List of X ins Inhaltsverzeichnis, mit liststotocnumbered werden die Abbildungsverzeichnisse nummeriert
  headsepline,
  cleardoublepage=empty,
  parskip=half,
  %               draft    % um zu sehen, wo noch nachgebessert werden muss - wichtig, da Bindungskorrektur mit drin
  draft=false
]{scrbook}
\input{config}


\usepackage[
  title={GETACAR: A Privacy-Preserving Platform for Ride-Pooling},
  author={Hans Hüppelshäuser},
  type=master,
  institute=iaas, % or other institute names - or just a plain string using {Demo\\Demo...}
  course={M.Sc. Wirtschaftsinformatik},
  examiner={Prof.\ Dr.\ Marco Aiello},
  supervisor={Robin Pesl, \ M.Sc.},
  startdate={March 27, 2023},
  enddate={Septeber 27, 2023}
]{scientific-thesis-cover}

% Hier stehen alle Abkürzungen
\newacronym{pow}{PoW}{Proof of Work}
\newacronym{pos}{PoS}{Proof of Stake}
\newacronym{dpos}{DPoS}{Delegated Proof of Stake}
\newacronym{sumo}{SUMO}{Simulation of Urban MObility}
\newacronym{dapp}{DApp}{Decentralised Applications}
\newacronym{ipfs}{IPFS}{Interplanetary File System}
\newacronym{rsus}{RSUs}{Road Side Units}
\newacronym{osi}{OSI}{Open Source Initiative}


\makeindex

\begin{document}

%tex4ht-Konvertierung verschönern
\iftex4ht
  % tell tex4ht to create picures also for formulas starting with '$'
  % WARNING: a tex4ht run now takes forever!
  \Configure{$}{\PicMath}{\EndPicMath}{}
  %$ % <- syntax highlighting fix for emacs
  \Css{body {text-align:justify;}}

  %conversion of .pdf to .png
  \Configure{graphics*}
  {pdf}
  {\Needs{"convert \csname Gin@base\endcsname.pdf
      \csname Gin@base\endcsname.png"}%
    \Picture[pict]{\csname Gin@base\endcsname.png}%
  }
\fi

%\VerbatimFootnotes %verbatim text in Fußnoten erlauben. Geht normalerweise nicht.

\input{commands}
\pagenumbering{arabic}
\Titelblatt

%Eigener Seitenstil fuer die Kurzfassung und das Inhaltsverzeichnis
\deftriplepagestyle{preamble}{}{}{}{}{}{\pagemark}
%Doku zu deftriplepagestyle: scrguide.pdf
\pagestyle{preamble}
\renewcommand*{\chapterpagestyle}{preamble}



%Kurzfassung / abstract
%auch im Stil vom Inhaltsverzeichnis

  \section*{Abstract}
The widespread adoption of autonomous vehicles is expected to lead to an overall increase in traffic. Ride-pooling can counter this downside of an otherwise promising technology, but the majority of current ride-pooling platforms utilise centralised designs that allow companies to collect vast amounts of user data. To solve this problem, we propose the decentralised ride-pooling platform GETACAR that focuses on privacy-preservation. GETACAR utilises blockchain technology to allow for the transparent and immutable tracking of ride processes without exposing personal information to other participants or the platform itself. To realise the platform, we develop its design, define its interactions and create a prototypical implementation. GETACAR is comprised of several components, including a customer and ride provider frontend allowing humans and autonomous vehicles to interact with GETACAR. We introduce an off-chain matching service to find the best possible match between customers and ride providers via a Vickrey auction.  GETACAR also connects with crypto exchanges that allow the platform to use cryptocurrencies for internal transactions while users can still handle payments via fiat currencies.
An authentication service verifies all parties wishing to participate on the platform, ensuring accountability and impeding the use of multiple identities.  To ensure safety across the platforms, a robust rating system is in place that allows all parties to rate each other. In addition, a number of privacy mechanisms are in place to minimise the exposure of personal information, including location cloaking, pseudonyms, and frequently changing wallets. A prototype validates the GETACAR platform design by showcasing the platform's key features, including a fully realised user frontend, the matching service, and smart contracts running on the Ethereum blockchain. All these components are connected and working together, allowing for a customer to request a ride with multiple ride providers bidding on the ride. The implemented matching service determines the winner, and smart contracts manage the overall ride, including the rating process. Both the design of the platform and the prototype showcase the potential of blockchain technology to create next-generation ride-pooling platforms that ensure transparency while preserving privacy.

\cleardoublepage

  \section*{Kurzfassung}
Es ist zu erwarten, dass die Verbreitung autonomer Fahrzeuge zu einem allgemeinen Anstieg des Verkehrsaufkommens führen wird. Ride-Pooling kann diesem Nachteil einer ansonsten vielversprechenden Technologie entgegenwirken, aber aktuelle Ride-Pooling-Plattformen nutzen zentralisierte Designs, die es Unternehmen ermöglichen, große Mengen an Benutzerdaten zu sammeln. Um dieses Problem zu lösen, setzt die dezentrale Ride-Pooling-Plattform GETACAR auf den Schutz der Privatsphäre. GETACAR nutzt die Blockchain-Technologie, um eine transparente und nicht manipulierbare Verfolgung von Fahrten zu ermöglichen. Die Plattform besteht aus einem Kunden- und Fahranbieter-Frontend, das es Menschen und autonomen Fahrzeugen ermöglicht, mit GETACAR zu interagieren. Es wird ein Off-Chain Matching-Service eingeführt, um über eine Vickrey Auktion die Kunden bestmöglich mit passenden Fahranbietern zusammenzubringen. Die Plattform ist mit einem Authentifizierungsdienst ausgestattet, der Pseudonyme generiert und Krypto-Wallets verifizieren kann, um sicherzustellen, dass Benutzeridentitäten nicht auf der Plattform preisgegeben werden. GETACAR ist mit Krypto-Börsen verbunden, die es der Plattform ermöglichen, Kryptowährungen für interne Transaktionen zu verwenden, während Benutzer weiterhin Zahlungen über Fiat-Währungen abwickeln können. Um die Sicherheit auf allen Plattformen zu gewährleisten, gibt es ein robustes Bewertungssystem, das es allen Parteien ermöglicht, sich gegenseitig zu bewerten. Ein Prototyp validiert das GETACAR-Plattformdesign, indem er die wichtigsten Funktionen der Plattform präsentiert, darunter ein vollständig realisiertes Benutzer-Frontend, den Matching-Service und Smart Contracts, die auf der Ethereum Blockchain laufen. Alle diese Komponenten sind miteinander verbunden und arbeiten miteinander, sodass ein Kunde eine Fahrt bei mehreren Fahranbietern anfordern kann und diese auf die Fahrt bieten können. Der implementierte Matching-Service ermittelt den Gewinner und Smart Contracts verwalten die gesamte Fahrt, einschließlich des Bewertungsprozesses. Sowohl das Design der Plattform als auch der Prototyp verdeutlichen das Potenzial der Blockchain-Technologie zur Schaffung von Ride-Pooling-Plattformen der nächsten Generation, die Transparenz gewährleisten und gleichzeitig die Privatsphäre wahren.


\cleardoublepage


% BEGIN: Verzeichnisse

\iftex4ht
\else
  \microtypesetup{protrusion=false}
\fi

%%%
% Literaturverzeichnis ins TOC mit aufnehmen, aber nur wenn nichts anderes mehr hilft!
% \addcontentsline{toc}{chapter}{Literaturverzeichnis}
%
% oder zB
%\addcontentsline{toc}{section}{Abkürzungsverzeichnis}
%
%%%

%Produce table of contents
%
%In case you have trouble with headings reaching into the page numbers, enable the following three lines.
%Hint by http://golatex.de/inhaltsverzeichnis-schreibt-ueber-rand-t3106.html
%
%\makeatletter
%\renewcommand{\@pnumwidth}{2em}
%\makeatother
%
\tableofcontents

% Bei einem ungünstigen Seitenumbruch im Inhaltsverzeichnis, kann dieser mit
% \addtocontents{toc}{\protect\newpage}
% an der passenden Stelle im Fließtext erzwungen werden.

\listoffigures
\listoftables

%Wird nur bei Verwendung von der lstlisting-Umgebung mit dem "caption"-Parameter benoetigt
%\lstlistoflistings
%ansonsten:
\ifdeutsch
  \listof{Listing}{Verzeichnis der Listings}
\else
  \listof{Listing}{List of Listings}
\fi

%mittels \newfloat wurde die Algorithmus-Gleitumgebung definiert.
%Mit folgendem Befehl werden alle floats dieses Typs ausgegeben

%\listofalgorithms %Ist nur für Algorithmen, die mittels \begin{algorithm} umschlossen werden, nötig

% Abkürzungsverzeichnis
\printnoidxglossaries

\iftex4ht
\else
  %Optischen Randausgleich und Grauwertkorrektur wieder aktivieren
  \microtypesetup{protrusion=true}
\fi

% END: Verzeichnisse


% Headline and footline
\renewcommand*{\chapterpagestyle}{scrplain}
\pagestyle{scrheadings}
\pagestyle{scrheadings}
\ihead[]{}
\chead[]{}
\ohead[]{\headmark}
\cfoot[]{}
\ofoot[\usekomafont{pagenumber}\thepage]{\usekomafont{pagenumber}\thepage}
\ifoot[]{}


%% vv  scroll down for content  vv %%































%%%%%%%%%%%%%%%%%%%%%%%%%%%%%%%%%%%%%%%%%%%%%%%%%%%%%%%%%%%%%%%%%%%%%%%%%%%%%%
%
% Main content starts here
%
%%%%%%%%%%%%%%%%%%%%%%%%%%%%%%%%%%%%%%%%%%%%%%%%%%%%%%%%%%%%%%%%%%%%%%%%%%%%%%


\chapter{Introduction}

With autonomous vehicles becoming a reality in the near future, an increase in overall traffic on the roads can be expected. The mass utilisation of ride-pooling platforms can counter this development, but the current ride-pooling platforms are built centralised and are lacklustre in regards to data privacy. Therefore, this research aims to create a decentralised, privacy-preserving ride-pooling platform that provides a feasible alternative to the current platform landscape.
\section{Problem Statement}
With the prospect of autonomous driving becoming a reality in the foreseeable future, a surge in traffic is anticipated due to the increased accessibility and convenience offered by this technology.

In recent years, developments surrounding transportation have changed dramatically with the surge in technologies such as autonomous driving. While experts and industry stakeholders advertise \gls{av} for their ability to introduce efficiency, convenience, and potential safety to our roadways, there lies an inherent problem that may impair current traffic conditions.

The rapid integration and adoption of autonomous driving, as projected, would inevitably lead to a massive upswing in vehicular traffic. This increase stems from the upsides that make an \gls{av} appealing, their ease and convenience. With the absence of the need to drive manually, more individuals might be inclined to choose a personal \gls{av} over other modes of transport, creating stagnation and traffic jams. Therefor, while the technological transition promises a number of advantages, without proper intervention, it may also have negative effects on  travel, above all in the form of  increased road traffic.

Ride-pooling, or shared mobility, emerges as a potential solution to this problem. The principle of ride-pooling revolves around utilising a single vehicle to transport multiple passengers headed in similar directions, ensuring optimal usage of a car's seating capacity, reducing the number of individual trips, and consequentially, decreasing overall traffic. However, in practice ride-pooling faces a number of challenges.

A fundamental challenge lies in the potential monopolization of the ride-pooling market. <Text on my why monopolisation in the industry happens>. Such centralization not only puts a vast amount of data and power into the hands of one or few entities but also makes it harder for new market entrants to compete. New competitors and smaller businesses find themselves blocked out, leading to diminished innovation, potential price inflation, and reduced consumer choice.

Furthermore, centralized platforms, by their nature, tend to consolidate vast amounts of personal and transactional data. This leads to concerns regarding user privacy. If a user’s ride details, routes, payment information, and behavior on the platform are stored without adequate privacy measures, it exposes them to potential risks.

Given this backdrop, the concept of a decentralized, trust-based platform for ride-pooling becomes not just preferable but essential. The \gls{dlt}, commonly known as blockchain, has demonstrated its potential in recent years as a tool for creating decentralized systems. By design, \gls{dlt} offers transparency, immutability, and decentralization – attributes that can address many of the concerns posed by centralized systems. On the other hand, factors such as the  transparency of data running through a distributed ledger poses challenges in regards to the privacy of personal data and the anonymity of user activity on the platform.<Need to connect the end of this paragraph better to the beginning of the next one>

Therefore, this study seeks to conceptualize, design, and evaluate a platform for ride-pooling based on such distributed technology. By doing so, it endeavors to craft a system where users and providers can interact seamlessly, ensuring the privacy of transactions and enabling an environment where neither party has direct knowledge of the other.

In summary, the ubiquity of autonomous driving presents challenges that threaten to reverse its benefits. The situation demands innovative solutions, one of which is ride-pooling. However, the successful execution of ride-pooling mandates a departure from traditional centralized platforms to decentralized ones that preserve privacy and ensure trust. This research aims to contribute to the creation of such platforms.





\section{Objectives}\label{sec:objectives}
The landscape of transportation is transforming rapidly. As stated in the problem statement, while autonomous driving is set to redefine the way we perceive mobility, it simultaneously brings forth a number of challenges. Chief among them is the anticipated surge in traffic and privacy and centralisation concerns surrounding the conventional ride-pooling solutions. As we are coming closer to this future of autonomous transportation, it becomes necessary to address these challenges head-on. This section outlines the primary objectives I seek to accomplish in this study, ensuring that the benefits of autonomous mobility are maximized while minimizing potential drawbacks.

\subsection{Prototypical Realization of the Decentralized Platform}

The main goal of this research lies in the creation of a decentralized platform for ride-pooling. Unlike traditional systems where power and control are concentrated, decentralized platforms spread power across nodes, ensuring distributed control and reduced risk of  single entity dominance.

The objective is twofold:

Conceptualization: Before diving into implementation, I will lay down a theoretical framework for the platform. This involves establishing parameters for participation, mapping out user flows, and ensuring the system's robustness against potential threats.

Prototyping: Based on the theoretical design, I will proceed to construct a prototype of the platform. By building the system, I can simulate real-world scenarios, understand unforeseen challenges, and refine the design in response to these challenges.

\subsection{Design of an Interaction Protocol between the Platform, Users, and Providers}

The platform, at its core, is an ecosystem of interaction flows. These interactions include:

\begin{itemize}
    \item Users seeking rides
    \item Providers offering ride services
    \item Transactions facilitating the above exchanges
    \item <TODO: Complete List>
\end{itemize}

A streamlined, secure, and efficient protocol for these interactions is necessary for the success of the platform. The objective here is to develop a protocol that:

Ensures Seamless Ride Booking: The protocol should allow an easy to use ride booking flow for users as well as for ride providers which can compete with existing, centralised platforms.

Facilitates Trustworthy Transactions: Every transaction must ensure that the provider is fairly compensated and the user receives the promised ride.

Preserves Privacy: Given the emphasis on privacy, the protocol should ensure that the interactions between users and providers remain untraceable by external entities.

<TODO: Complete List>

\subsection{Design of a Trust Mechanism for Users and Providers}

Trust is a necessity for any successful platform. In decentralized systems, where there is not a central authority to manage disputes or verify participants, the importance of trust is magnified. The objective is to instate mechanisms that:

Verify Participants: A mechanism to authenticate and verify new entrants to the platform to prevent malicious actors.<Might need to rephrase this because i am not doing the Auth Service>

Incentivize Good Behavior: Introduce incentives for platform participants who consistently adhere to platform guidelines and receive positive ratings.

Penalize Misconduct: Conversely, there should be negative effects for those engaging in fraudulent or malicious activities.

<TODO: Complete List>

\subsection{Evaluation of User and Provider Anonymity in Transactions}

An integral part of the platform's advantage is the assurance that users and providers can transact without directly knowing each other's identity. This objective involves:

Anonymity Verification: Showing how the platform hides the identities of participants during transactions.

Data Collection Prevention: Insuring that no party is able to collect critical amounts of data.

<TODO: Complete List>

\subsection{Proposal of Solutions for Physical Issues and Edge Cases}

While much of the research focuses on the digital side of the platform, the handling of unpredictable issues that accrue in the real-world can not be ignored. Prominent concerns in ride-pooling, especially with autonomous vehicles, are the potential damage to vehicles by passengers or inappropriate actions by individuals towards other passengers. Therefor, the aim is to:

Ensure Accountability: While preserving user privacy, create a system that holds individuals accountable for any damages or misconduct.

Develop Reporting Mechanisms: Allow users and ride providers to report any damages or misconduct they encounter during their ride.

<TODO: Complete List>

 Through these objectives, the goal is to construct a platform that shows that the concept of a decentralised, privacy protecting ride-pooling platform is feasible.
\section{Methodology}
It is important to utilise a structured approach when designing the decentralised ride-pooling platform. The following methodology provides a step-by-step process where each step builds upon the previous one, ensuring the platform viability, resulting in the design and prototypical implementation of a decentralised, privacy-preserving ride-pooling platform that showcases the current state of technology and scientific research in that field.


\begin{enumerate}

    \item \textbf{Literature Research about Current Solutions}: 
    First, it is important to understand the current state of scientific research. Therefore this paper will conduct an in-depth analysis of the current state of scientific literature, reviewing academic papers, industry reports, and white papers about ride-pooling, decentralised systems, and related technologies. The output of this step is a comprehensive overview of what has been achieved in the area of decentralised ride-pooling so far.

    \item \textbf{Identification of Shortcomings}: 
    Building upon the previous stage, this research will work out potential shortcomings of the current research landscape, point them out and propose solutions to balance out these shortcomings. This allows for the decentralised ride-pooling platform built through this research to not only be a gathering of existing research findings but also to provide added value to the research landscape.

    \item \textbf{Proposal of a Solution Design}:
    The next step is to create a comprehensive design for the platform based on the findings of the research analysis. This phase includes designing the architecture of the decentralised platform, defining interaction flows and outlining trust and privacy mechanisms. 
    

    \item \textbf{Prototypical Implementation of the Solution Design}: 
    To prove the viability of the design, it is necessary to build a prototype that can showcase that the core functions and components work as intended. Therefore this step includes the programming of smart contracts, user interfaces and other components and enabling them to communicate with each other. The finished prototype allows for real-world testing and iterative refinement.

    \item \textbf{Evaluation Whether the Previously Set Requirements are met}: 
    Based on the results from the working prototype, it is then important to analyse if all research objectives set for the decentralised ride-pooling platform are met. Based on this evaluation, it is possible to determine if the research is a success and can provide a significant contribution to research.

    \item \textbf{Identification of Limitations and Proposal of Possible Improvements}:
    The evaluation is also meant to bring up shortcomings of the research and aspects of the platform that require further investigation. These shortcomings and possible improvements are clearly pointed out so that future researchers can take the findings of this work and use it as the base for their research.

\end{enumerate}

In summary, this iterative methodology ensures a scientific step-by-step approach to developing the  privacy-preserving ride-pooling platform and meeting all research objectives set for this work.

\chapter{Background Information}
The following chapter elaborates on relevant concepts and technologies that are necessary to understand for this research and will present an overview over the current state of academic literature in the field of decentralised ride sharing platforms. 
\section{Autonomous Driving and Ride-Pooling}
 The advantages of autonomous driving, coupled with the growing significance of ride-pooling, promise having a transformative impact on urban mobility~\cite{Stamadianos.2023}. This section dives into the intricacies of both concepts, exploring their origins, developments, and the potential synergy they hold for the future of transportation.

\subsection{Autonomous Driving}
Autonomous, or self-driving vehicles, utilize a blend of hardware and software to navigate and control the car without human intervention~\cite{Stamadianos.2023}. Classified into levels 0 to 5, with 5 being fully autonomous, these vehicles rely on intricate systems of sensors, cameras, lidars, and radars. They constantly gather data about their environment, which is then processed by advanced algorithms to make driving decisions~\cite{Hacohen.2022}.
Historically, the concept of a car driving itself can be traced back to as early as the 1920s, but tangible progress started in the latter half of the 20th century. Projects like the EUREKA Prometheus Project in the 1980s and the DARPA Grand Challenge in the early 2000s played important roles in development of autonomous technologies~\cite{Hacohen.2022}. Today, major tech and automobile companies are competing to build fully autonomous vehicles for mass adoption~\cite{Stamadianos.2023}.
The potential benefits of autonomous driving are vast:
\begin{itemize}
    \item Safety: Human error, responsible for a majority of road accidents, could be drastically reduced~\cite{Hacohen.2022}.
    \item Efficiency: Optimal driving by autonomous cars might reduce traffic congestion and lead to more streamlined traffic flows~\cite{Stamadianos.2023}.
    \item Accessibility: Those unable to drive due to age, disability, or other factors can enjoy independent mobility~\cite{Hacohen.2022}.
    \item Economic Impact: A reduction in accidents implies decreased costs in healthcare and vehicle repairs~\cite{Stamadianos.2023}.
\end{itemize}

However, challenges still exist. Technical complications, legal barriers, ethical questions (like decision-making in unavoidable accidents), and public skepticism need to be addressed for a broader acceptance~\cite{Hacohen.2022}.

\subsection{Ride-Pooling}
Ride-pooling, distinct from ride-sharing, involves multiple riders sharing a single vehicle on a trip, where each passenger's destination is likely different, but their routes are similar~\cite{Perivier.}. Platforms like UberPool and Lyft Line have popularized ride-pooling in urban environments. The appeal of such services lies in their promise of reduced cost of travel for passengers, decreased overall traffic, lower carbon emissions, and the potential reduction of occupied parking spaces~\cite{Shaheen.}.
However, ride-pooling is not without its challenges. Efficient route optimization to ensure minimal detours, balancing demand and supply, and ensuring passenger safety are areas that ride-pooling providers constantly struggle with~\cite{Perivier.}.

The Synergy of Autonomous Driving and Ride-Pooling offer a promising vision of future of urban mobility~\cite{Stamadianos.2023}.
When considering efficiency and cost, the concept of autonomous ride-pooling offers great advantages. The possibility of vehicles being available round the clock presents an unprecedented level of efficiency. This constant operation results in a decline in the per-trip cost, which is directly beneficial for passengers in the form of lower prices. This efficiency is further increased by the absence of a driver, which decreases operating costs and also provides added space in the vehicle for additional passengers or cargo, increasing the utility of the vehicles~\cite{Hacohen.2022,Stamadianos.2023}.
In terms of environmental impact, the combination of electric and autonomous vehicles, when integrated with ride-pooling services, allow  for trips with reduced ecological impact. Fewer cars on the roads result in less pollution\cite{Hacohen.2022,Stamadianos.2023}. The concept of autonomous ride sharing extends into urban planning as well.  Cities are currently designed with a focus on  moving vehicles and parking. This could undergo a significant transformation. With the reduced need for roads and parking spaces, there is potential for vast areas to be repurposed for green spaces, recreational areas, or new housing units~\cite{Stamadianos.2023}.
Lastly, at the core of these advancements lies an increase in accessibility and inclusivity.  Autonomous ride-pooling systems lower barriers created by by age, disability, or socioeconomic status.~\cite{Hacohen.2022}.
However, the integration is not without potential downsides. Job losses, especially for drivers in the ride-sharing industry, the challenge of adjusting infrastructure to accommodate autonomous vehicles, and the need to build robust and save systems, are concerns that need to be addressed~\cite{Hacohen.2022}.

In conclusion, autonomous ride-pooling platforms represent not just technological advancements but can change our approach towards transportation~\cite{Shaheen.}, promising a more efficient, eco-friendly, and inclusive transportation landscape~\cite{Stamadianos.2023}. However, the development of such platforms brings up the challenge of balancing the immense potential benefits with the inherent challenges. 
\section{Blockchain Technology and Smart Contracts}

The blockchain concept represents a technological breakthrough. The decentralised, immutable ledger  at the core of the technology allows blockchains to be utilised in a number of industries, including finance and supply chain management, where the integrity and immutability of information play an important role ~\cite{Zhou.2023}. The decentralised consensus approach of blockchain ensures that data modifications are only possible through the unanimous approval of all participating systems. This ensures that information that is written onto the ledger becomes immutable. To provide all the important information on blockchain for this research, we provide an overview of the current state of blockchain, explain the technical concepts behind blockchain in more detail, discuss the applications of smart contracts and take a closer look at the security of blockchain~\cite{Tran.2022b}.

\subsection{Introduction to Blockchain}
Blockchain technology as we know it today originated with the introduction of Bitcoin in 2008 ~\cite{Nakamoto.2009}. 
Satoshi Nakamoto, the pseudonymous individual or group behind Bitcoin, introduced the concept as a solution to the double-spending problem in digital currencies~\cite{Nakamoto.2009}. Digital currencies before Bitcoin faced the problem that it was very difficult to ensure that no token could be spent more than once. The solution to this problem presented by Nakamoto is a decentralised ledger, where every transaction gets verified by a network of nodes through a consensus mechanism ~\cite{Tran.2022b}. This technological breakthrough was groundbreaking because it allowed the creation of decentralised currencies that are not controlled by a single entity like a government.


One of the outstanding features of blockchain technology is its decentralisation~\cite{Gencer.2018}. Unlike traditional databases, such as an SQL database operated by a central entity, blockchains operate on a peer-to-peer network~\cite{Gencer.2018}. Every participant (or node) has access to the entire database and the complete history of all transactions. This means that no single participant has control over the data, and all participants collectively maintain the integrity of the data.

Immutability is another critical feature of blockchains. Once a transaction is recorded on the blockchain, it becomes extremely difficult to alter~\cite{Pilkington.}. This is because each block contains a cryptographic hash of the previous block, creating a chain of blocks~\cite{Pilkington.}. To change a single block, one would need to alter all subsequent blocks, which is computationally impractical, especially in large networks~\cite{ContedeLeon.2017}.

Transparency is inherently built into the system due to its open-source nature. Every transaction on the blockchain is visible to anyone who chooses to view it, ensuring full transparency in the network~\cite{Banupriya.2021}. However, personal information about the users conducting the transactions remains private as each user is commonly represented through some form of Public Key~\cite{Wei.2022}. This ensures a balance between transparency and privacy.

While the foundational principles of blockchains remain consistent, there are different types tailored to specific needs~\cite{Ghosh.2021}. Public blockchains, like Bitcoin and Ethereum, are open to anyone and are simply secured by their cryptographic algorithms~\cite{Ghosh.2021}. In contrast, private blockchains, like the Hyperledger Blockchain Projects, can be restricted to a specific group of participants, often used by businesses for internal processes. Private Blockchains can also be used as consortium blockchains or federated blockchains, operated under the leadership of a group ~\cite{Lu.2023}. They provide a balance between the openness of public blockchains and the restrictions of private ones.

\subsection{How Blockchain Works}
Diving deeper into the mechanics of blockchain technology shows the interplay of cryptographic principles, network theory, and consensus algorithms~\cite{Xiong.2022}. At the core of this technology are blocks, which are essentially records of transactions. Each block typically contains a timestamp, a reference to the previous block (known as the parent block), and a list of transactions. These transactions are represented as cryptographic hashes, which are fixed-size strings of characters generated from input data of any size. The advantage of these hashes is that even a small change in the input data results in a completely different hash, ensuring the integrity of transaction records that can be traced back to the very first block, known as the genesis block~\cite{Xiong.2022}.

Central to the operation of a blockchain is the concept of consensus mechanisms~\cite{Tahir.2022}. These are protocols that ensure all participants in the network agree on the validity of transactions. The most well-known consensus mechanism is \gls{pow}. In \gls{pow}, participants, often referred to as "miners", solve complex mathematical problems to validate transactions and create new blocks. This process requires significant computational power and energy. An alternative mechanism, \gls{pos}, determines the creator of a new block based on their stake or ownership of the cryptocurrency. It is seen as a more energy-efficient alternative to \gls{pow}. \gls{dpos} further refines this by allowing coin holders to vote for a few trusted nodes to validate transactions, streamlining the process and reducing the energy footprint~\cite{KUCHKOVSKY.2021}.

The blockchain network is maintained by nodes, which are computers participating in the network~\cite{Xiong.2022}. In general, there are two primary types of nodes: full nodes and light nodes~\cite{Mitra.2021}. Full nodes store the entire blockchain and validate all transactions and blocks. They serve as the network's backbone, ensuring data integrity and consistency. Light nodes, on the other hand, store only a subset of the blockchain and rely on full nodes for transaction validation and other heavy operations~\cite{Mitra.2021}. Their primary role is facilitating faster and more efficient interactions with the blockchain.

Transactions allow users to interact with the blockchain. Once a transaction is initiated, it is broadcast across the blockchain network and placed in a pool of unconfirmed transactions. Worker nodes then take these transactions from the pool and validate them against the ledger history to ensure that they are valid. If a transaction is determined as valid, it is placed in a block, together with other valid transactions. Once the block is full, it is shared with the network for verification through the consensus mechanism. After this, the block is added to the chain, and the transaction becomes a permanent part of the ledger history~\cite{Xiong.2022}.

\subsection{Smart Contracts}
As one part of blockchain technology, smart contracts have emerged as an advanced tool, extending the use-cases of blockchains beyond the record keeping of transactions~\cite{UchaniGutierrez.2023}. A smart contract is a self-executing contract where the terms of agreement or conditions are represented through written lines of code~\cite{Zhou.2022}. They are protocols that verify and enforce credible transactions without the need for third parties~\cite{Zhou.2022}. At their core, they are digital contracts that automatically execute actions when predefined conditions are met.

The concept of smart contracts is not new, but its practical application gained popularity with the advancements of blockchain technology~\cite{Pierro.}. Ethereum, launched in 2015, demonstrated the potential of smart contracts~\cite{Pierro.}. Ethereum's platform is designed specifically to create and execute smart contracts, providing a more flexible scripting language and a platform for creating a \gls{dapp}~\cite{Pierro.}. Since Ethereum's creation, a number of other blockchains have integrated smart contract capabilities, offering unique features and optimisations.

Once deployed, smart contracts operate without human intervention, ensuring that transactions are carried out correctly when conditions are met~\cite{UchaniGutierrez.2023}. This allows for interactions among parties that are not required to trust each other. Since the contract is on a blockchain, all parties can verify the contract's code and monitor its execution~\cite{UchaniGutierrez.2023}. The decentralised nature of blockchains also ensures that smart contracts are secure from tampering, providing an added layer of security~\cite{Zhou.2022}.

Understanding the life cycle of a smart contract provides insights into its operational model. The journey begins with its creation, where the contract's terms are defined and encoded. Once the code is written and tested, it is deployed onto the blockchain, appearing as an immutable part of the ledger~\cite{Tran.2022b}. After the deployment, the contract is now active and can start receiving and processing information. Execution occurs when the conditions specified in the contract are met, triggering the actions encoded in the contract~\cite{Pierro.}. While many smart contracts are designed to run without a predefined end, there are scenarios where they might have a termination condition, ending the contract's active state on the blockchain ~\cite{Tran.2022b}.

Even though smart contracts have many advantages, they also come with their own set of limitations and challenges ~\cite{.2019}. One notable challenge in the Ethereum network is the concept of Gas fees. Every operation, from contract deployment to execution, requires computational resources. Users pay for these resources using so-called Gas, and with increased network activities, these fees can rise. Scalability remains a concern as well. As more complex smart contracts and DApps are developed, there is a growing demand for blockchains to process more transactions per second without compromising on security or decentralisation ~\cite{Tran.2022b}. Lastly, smart contracts are only as good as the code they're written in. Coding errors or oversights can lead to vulnerabilities, potentially allowing malicious actors to exploit the contract~\cite{Zhou.2022}.

In conclusion, smart contracts represent a significant leap in how agreements and transactions can be managed on a decentralised network~\cite{.2019}. While they offer many advantages, it is necessary to consider their challenges to utilise their full potential~\cite{.2019}.

\subsection{Blockchain Security}
Blockchain's decentralised nature, which is often used for its resilience and transparency, also presents unique security challenges. One of the most discussed vulnerabilities is the 51 Percent attack. In such an attack, if a single entity gains control of more than half of the network's nodes, it can potentially double-spend coins and stop or reverse transactions. This undermines the trust and integrity of the blockchain. Similarly, Sybil attacks occur when a single party controls multiple nodes, aiming to flood the network with false transactions or undermine mechanisms that rely on redundancy and trust ~\cite{Singh.2021}.

Smart contracts have their own set of security concerns~\cite{Alkhalifah.2021}. Reentrancy attacks are a prime example, where an attacker drains funds from a contract by repeatedly calling its functions before the initial function call is completed~\cite{Alkhalifah.2021}. Issues like overflow and underflow, where variable values exceed their set limits, can also be exploited, leading to unintended consequences in contract execution~\cite{Guo.2022}. These vulnerabilities underscore the importance of rigorous code audits and testing before deploying smart contracts on a live network.

Most blockchains offer pseudonymity, where transactions are linked to a cryptographic address rather than personal identities. However, through analysis, patterns can emerge, potentially de-anonymizing users~\cite{Kus.2022}.

In essence, while blockchain offers robust security mechanisms at the core of its design, it is not impenetrable. As the technology matures, addressing these vulnerabilities will be an important part of ensuring its general adoption and trustworthiness.

\section{Decentralised Platforms and Data Privacy}
In the digital age, the concept of decentralisation has emerged as a transformation paradigm, reshaping the way we understand and interact with systems and networks~\cite{Tverdokhlib.2022}. Decentralised platforms, at their core, are systems where components, be it data, resources, or operations, are not controlled or managed by a single, central entity. Instead, they are distributed across multiple nodes or participants, each having equal authority and autonomy. This contrasts sharply with traditional centralised systems, where a single entity or a group of entities holds the power and control. In the following chapter we will introduce the core concepts behind decentralised platforms and showcase how data privacy can be handled in a system without a central authority.


\subsection{Introduction to Decentralised Platforms}

One of the primary characteristics of decentralised platforms is the absence of a central point of control. This means that no single entity has the authority to make unilateral decisions or changes without consensus from the majority of the network's participants~\cite{SEFRAOUI.2022}. This leads to enhanced security, as the absence of a single point of failure makes the system more resilient to attacks~\cite{Maffiola.2022}. Additionally, decentralised platforms often employ cryptographic techniques to ensure data integrity, privacy, and authentication. This ensures that transactions and interactions on the platform are secure, verifiable, and tamper-proof~\cite{SEFRAOUI.2022}.

Comparing decentralised platforms with centralised systems reveals strong differences in their operational philosophies and outcomes~\cite{Maffiola.2022}. Centralised systems, such as traditional databases or web servers, are controlled by a single entity. This central authority has the power to dictate rules, make changes, grant or deny access, and is also the primary custodian of the data~\cite{Maffiola.2022}. While this centralisation can lead to efficiencies in terms of decision-making and streamlined operations, it also presents vulnerabilities. A single point of failure in a centralised system can lead to the entire system's collapse~\cite{Maffiola.2022}. Moreover, centralisation often results in data silos, where information is trapped within one part of the system, inaccessible to others.

On the other hand, decentralised platforms operate on the principles of democracy and transparency~\cite{SEFRAOUI.2022}. Decisions are made based on consensus algorithms, ensuring that no single participant can dominate or manipulate the system. This democratisation of control can lead to enhanced trust among users, as they are assured that the platform operates in a fair and transparent manner~\cite{Hasan.2022}. Moreover, data in decentralised systems is typically stored across multiple nodes, ensuring redundancy and resilience~\cite{Hasan.2022}. Even if one or more nodes fail, the system can continue to operate seamlessly.

The potential benefits of decentralised platforms are manifold~\cite{Hasan.2022}. Firstly, they offer enhanced security and resilience due to their distributed nature. The risk of system-wide failures or attacks is significantly reduced~\cite{Maffiola.2022}. Secondly, they promote transparency and trust among users, as decisions are made collectively and openly~\cite{Hasan.2022}. This can lead to increased user adoption and loyalty. Furthermore, decentralised platforms can lead to innovations in peer-to-peer transactions, smart contracts, and decentralised applications, opening up new avenues for business and social interactions.

However, like any transformative technology, decentralised platforms come with their set of challenges~\cite{Hasan.2022}. The lack of a central authority can sometimes lead to slower decision-making, as achieving consensus can be time-consuming. Additionally, the technology underpinning decentralised platforms, such as blockchain, is still maturing, leading to scalability and performance issues~\cite{Hasan.2022}. Interoperability between different decentralised platforms is also a concern, as it can hinder seamless integration and communication.

In conclusion, decentralised platforms represent a paradigm shift in the way we design and interact with digital systems~\cite{Tverdokhlib.2022}. While they offer numerous advantages in terms of security, transparency, and innovation, they also present challenges that need to be addressed~\cite{Hasan.2022}. As the technology matures and evolves, it will be fascinating to witness the transformative impact of decentralised platforms on various sectors of the economy and society at large.


\subsection{Data Privacy: Definition and Importance}
In the digital age, where vast amounts of data are generated and exchanged every second, understanding the nuances of data privacy becomes paramount. Data privacy, at its core, refers to the right of individuals to control or influence what information about them is collected and how it is used. It encompasses the practices, safeguards, and binding rules put in place to protect personal information and ensure that individuals remain in control of it~\cite{Covert.2020}. This concept is crucial for several reasons.

Firstly, data privacy is intrinsically linked to personal autonomy and dignity. In a world where personal data can reveal intimate details about an individual's life, preferences, and habits, ensuring that such information is not misused or mishandled is vital~\cite{Covert.2020}. Without robust data privacy measures, individuals may feel violated, leading to a loss of trust in digital systems and platforms.

Secondly, data privacy is essential for safeguarding fundamental human rights, such as the right to freedom of expression and the right to seek and receive information. When individuals are unsure about the privacy of their data, they may hesitate to express their views or access information freely, fearing surveillance or repercussions.

Furthermore, in the context of businesses and services, data privacy is crucial for maintaining consumer trust. Companies that fail to protect user data or misuse it can face significant reputational damage, legal consequences, and financial losses~\cite{Li.2019}. In sectors like decentralised ride-sharing, where users share location data, payment details, and personal preferences, ensuring data privacy can be the difference between a successful platform and one that users abandon due to trust issues~\cite{Li.2019}.

While data privacy is a critical concept, it is essential to distinguish it from related terms like data security and data protection, as they are often used interchangeably but have distinct meanings.

Data security refers to the protective measures and technologies used to safeguard data from unauthorised access, breaches, or theft. It focuses on defending data from malicious threats, be it hackers, malware, or other cyber-attacks. For instance, using encryption to secure data or firewalls to prevent unauthorised access are examples of data security practices.

On the other hand, data protection is a broader concept that encompasses both data privacy and data security. It refers to the policies, procedures, and legal measures designed to ensure that data is collected, stored, and used in a way that respects individual rights and complies with relevant laws and regulations~\cite{Covert.2020}. Data protection laws, such as the General Data Protection Regulation (GDPR) in the European Union, set out the principles and guidelines that organisations must follow when handling personal data.

In conclusion, data privacy is the right of individuals to control their personal information and its usage~\cite{Covert.2020}. Its importance cannot be overstated, given its implications for personal autonomy, human rights, and business trust~\cite{Li.2019}. While closely related, it differs from data security, which focuses on defending data from threats, and data protection, which encompasses the overarching policies and laws governing data handling~\cite{Covert.2020}. As we continue to integrate digital platforms into every facet of our lives, understanding and prioritising data privacy will remain of paramount importance.

\subsection{Mechanisms of Data Privacy in decentralised Blockchain Systems}
Decentralised blockchain systems have emerged as a revolutionary technology, offering the promise of transparency, immutability, and disintermediation. However, the very nature of public blockchains, which are open and transparent, poses significant privacy challenges. Every transaction and its associated data are visible to anyone who accesses the blockchain, leading to potential privacy breaches and exposure of sensitive information.

Encryption plays a pivotal role in addressing these challenges. At its core, encryption involves converting data into a code to prevent unauthorised access. In the context of blockchains, public key and private key encryption is widely used. A public key, visible to everyone, is used to encrypt data, while a private key, known only to the owner, is used to decrypt it. This ensures that only the intended recipient can access the information. Furthermore, end-to-end encryption ensures that data remains encrypted during its entire journey from the sender to the recipient, preventing potential eavesdroppers from accessing the information during transmission.

Another advanced cryptographic technique employed in blockchains is zero-knowledge proofs (ZKPs). ZKPs allow one party to prove to another that a statement is true without revealing any specific information about the statement itself. For instance, in a transaction, a user can prove they have sufficient funds without revealing the exact amount. This ensures transaction validity while preserving user privacy.

Homomorphic encryption offers another layer of privacy. It allows computations to be performed on encrypted data without first decrypting it. The result, when decrypted, remains accurate. This means that blockchain systems can process transactions and maintain data integrity without exposing the actual data, a boon for privacy-centric applications.

Secure multi-party computation (SMPC) is a cryptographic protocol that allows multiple parties to jointly compute a function over their inputs while keeping those inputs private. In the context of blockchains, SMPC can enable decentralised applications to function without revealing user data to other participants, ensuring data privacy and security.

While public blockchains offer unparalleled transparency, they might not be suitable for all applications, especially those requiring higher levels of privacy. Private and consortium blockchains emerge as alternatives in such scenarios. Private blockchains restrict participation to selected entities, while consortium blockchains involve multiple organisations governing the network. Both these types limit data visibility to only authorised participants, enhancing data privacy.

Another approach to establish  an encrypted connection over a public network is the Diffie-Hellman Key Exchange. The Diffie-Hellman Key Exchange, introduced by Whitfield Diffie and Martin Hellman in 1976, is a cryptographic protocol that allows two parties to independently generate a shared secret key over an insecure communication channel. The protocol is based on the mathematical properties of modular arithmetic and discrete logarithm problems. Specifically, given a prime number \( p \) and a base \( g \) (where \( g \) is a primitive root modulo \( p \)), each party selects a private key and computes a public key. The public keys are then exchanged, and each party uses the other's public key along with their own private key to compute the shared secret. The security of the protocol relies on the difficulty of the discrete logarithm problem: while it is computationally easy to generate the public key from the private key, the reverse operation is considered infeasible with current technology when large prime numbers are used.

Off-chain storage is another solution to the privacy challenges. Instead of storing all data on the blockchain, only essential information is kept on-chain, while the rest is stored off-chain in secure databases. This reduces the amount of data exposed on the public ledger, ensuring privacy.

Lastly, layer 2 solutions, built on top of the primary blockchain, offer scalability and privacy improvements. By processing transactions off the main chain and only settling the final state on-chain, these solutions can ensure faster transactions and enhanced privacy.

In conclusion, while decentralised blockchain systems present certain inherent privacy challenges, a combination of cryptographic techniques and architectural solutions can effectively address these concerns, paving the way for a more secure and private decentralised future.

\subsection{Regulatory and Legal Challenges in Decentralised Blockchain Systems}
Decentralised blockchain systems, while revolutionary in their potential to disrupt traditional centralized models, face a myriad of regulatory and legal challenges. At the forefront of these challenges is the General Data Protection Regulation (GDPR), a regulation that has reshaped the landscape of data protection in the European Union and has significant implications for decentralised systems.

The GDPR emphasises the rights of individuals over their personal data, including the right to access, rectify, and erase their data. In a centralised system, complying with these rights is straightforward, as there's a single entity controlling the data. However, in decentralised blockchain systems, data is distributed across a network of nodes, making modifications or deletions challenging~\cite{Schwerin.2018}. Once data is added to the blockchain, it becomes immutable, meaning it cannot be altered or deleted. This immutability clashes with the GDPR's "right to be forgotten," where individuals can request their data to be erased~\cite{BelenSaglam.}. Thus, blockchain developers and operators must tread carefully, ensuring that personal data is either kept off-chain or encrypted in a way that it can be rendered inaccessible if required.

Beyond the GDPR, other regional data protection regulations further complicate the landscape. For instance, California's Consumer Privacy Act (CCPA) and Brazil's General Data Protection Law (LGPD) have their own sets of requirements, some of which may conflict with the decentralised nature of blockchains. These regulations, while primarily designed to protect consumers, can pose challenges for decentralised systems that operate across multiple jurisdictions. Ensuring compliance with a patchwork of regional regulations requires a deep understanding of each jurisdiction's nuances and a flexible system architecture that can adapt to diverse requirements.

Legal challenges in decentralised blockchain systems are not limited to data protection. The very nature of decentralisation means there's often no single entity to hold accountable, making it difficult to enforce regulations or resolve disputes. For instance, in a decentralised ride-sharing platform, if a dispute arises between a driver and a passenger, the absence of a central authority like traditional ride-sharing companies complicates resolution. Moreover, the pseudonymous nature of blockchain transactions can pose challenges in identifying parties in legal proceedings.

However, the decentralised community is not without potential solutions. One approach is the use of ''smart contracts'' – self-executing contracts with the terms directly written into code. These can automate and streamline compliance processes, ensuring that transactions adhere to regional regulations. Additionally, off-chain storage solutions can be employed to store sensitive personal data, linking it to the blockchain through secure cryptographic hashes. This ensures data can be modified or deleted as required, without compromising the immutability of the blockchain~\cite{Schwerin.2018}.

In conclusion, while decentralised blockchain systems offer transformative potential, they must navigate a complex web of regulatory and legal challenges. By understanding these challenges and innovating solutions that respect both the spirit of decentralisation and the need for regulatory compliance, blockchain systems can pave the way for a more decentralised and equitable digital future.

\section{Current Solutions and Shortcomings}
Due to the diverse literature regarding decentralised ride-pooling platforms, the proven approach of a systematic literature search according to Vom Brocke was chosen~\cite{vomBrocke.2009}. In this way, quality criteria such as traceability and reproducibility can be ensured through a clearly defined processes. Two common, cross-publisher research databases and one common publisher database were used for the literature search.
The selection of several cross-publisher research databases is intended to ensure that the search provides a representative overview of existing research on decentralised ride-pooling platforms,. The selection of the database of a publisher with a focus on information technology is intended to show how the research topic is treated in the literature from a primarily information technology perspective. The cross-publisher research databases used are Scopus and Ebscohost. The publisher database is IEEE Xplore.
The goal is to obtain research literature as a search result that deals with the development of decentralised ride-pooling platforms. To obtain results covering mainly decentralised platforms the search phrase ``decentralised'' was used. The following three synonyms were used to obtain search results that deal with the topic of ride-pooling: ``ride-sharing'', ``ride-pooling'' and ``ride-hailing''. Initial tests have shown that results with this search phrase return suitable research papers without noticeable gaps in regards to the topics covering decentralised ride-pooling.

The complete search phrase looks as followed:
\begin{verbatim}
("decentralised" AND ("ride-pooling" OR "ride-sharing" OR "ride-hailing"))
\end{verbatim}

For Scopus, Epscohost (all selectable databases included) and IEEE Xplore, the search phrase was applied to the title, abstract and keywords of the publications. Initial tests have shown that restricting the search to title, abstract and keywords is the best compromise between the quantity and quality of the search results. Only literature that was published in the after 2014 (2015 – 2023) was considered for the literature search. This is to ensure that the specialist literature found is of current relevance without overly restricting the scope of the existing research literature. Likewise, after the initial compilation of the search results, all duplicates were removed. In this way it is avoided that publications are counted twice because they are listed in several literature databases.

\subsection{Selection of the findings}
The literature search was carried out in July 2023, resulting in 86 hits. A criteria-based selection was made beyond the search phrase and the time limit for the publication of the specialist literature. The exclusion criteria used in the criteria-based selection are no publications in languages other than English, no panels and comments, and no literature dealing with decentralised platforms or ride-pooling. In addition, publications that are not freely available or accessible via a license from the University of Stuttgart had to be excluded. The inclusion criteria used are Only publications in English, only publications from 2015 onwards, and only papers discussing the technical development of decentralised ride-pooling platforms~\cite{Bandara.2015}.
Following Bandara, a first check of the actual relevance of the hits for answering the research question was carried out by screening the title, keywords and abstract. A full-text analysis was then carried out on the literature that was still considered relevant after the initial screening. Applying the inclusion and exclusion criteria in the initial screening and the subsequent full-text analysis, 10 relevant publications were identified from the 86 search hits for answering the research question. Additionally, two more relevant papers could be identified by following citations from the relevant literature.  Table \ref{tab:litSearchResults} shows how the relevant research literature is distributed across the research databases. 


\begin{table}[h]
\centering
\caption{Results of the Literature Search}
\label{tab:litSearchResults}
\begin{tabular}{|l|c|c|c|}
\hline
Scientific Database & Search Results & Excluded Literature & Included Literature \\ \hline
Scopus & 54 & 49 & 5 \\ \hline
Epscohost & 2 & 2 & 0 \\ \hline
IEEE & 30 & 25 & 5 \\ \hline
Citation search &  &  & 2 \\ \hline \hline
Total & 86 & 73 & 12 \\ \hline
\end{tabular}
\end{table}

The analysis of the publications shows that many different approaches are discussed in the scientific literature on how decentralised ride-pooling platforms can be built. For the results of the literature analysis to be evaluated and interpreted, the results must first be structured. For this purpose, a concept matrix approach, according to Webster and Watson is pursued~\cite{Webster.2002}. Based on the concept matrix approach, the specialist literature identified as relevant is assigned to eight topics relevant for the creation of a decentralised ride-pooling platform. These eight topics are derived from a general analysis of the topics covered by the scientific literature combined with topics relevant to fulfilling the research objectives:

\begin{itemize}
    \item \textbf{Blockchain Utilisation}: Blockchain is the underlying technology used for the creation of the decentralised ride-pooling platform. The literature needs to show in detail how blockchain technology is utilised by smart contracts and cryptocurrencies to build a ride-pooling platform.
    
    \item \textbf{Payments and Service Fees}: The decentralised ride-pooling platform must manage ride payments and general service fees. Therefore it is important for the literature to show how these financial transactions can be implemented and how to ensure that ride providers are compensated fairly for their services inside the decentralised ride-pooling platform.
    
    \item \textbf{Privacy and Anonymity}: Using blockchain technology demands a robust architecture that ensures privacy and anonymity for all users inside the platform. The scientific literature must showcase how users can engage with the platform and other users without revealing their identity directly or implicitly by sharing too much personal data with the platform over a longer time period.
    
    \item \textbf{Security and Resilience}: For a decentralised platform to gain widespread adoption, it must guarantee the safety and security of all parties. While the blockchain itself already provides many security features by design, it is important for the literature to show how the off-chain components are hardened and how to prevent the off-chain components from providing false information to the on-chain components.
    
    \item \textbf{Trust Mechanisms}: As decentralised platforms can not rely on a central trusted authority, robust trust mechanisms become essential. The research papers must explain how community trust mechanisms can be successfully implemented into a decentralised platform.
    
    \item \textbf{Off-Chain Edge Cases}: It is impossible to handle every edge case through the decentralised platform. As there is no central authority, it is important to provide alternative solutions to solve these problems without contradicting the decentralised nature of the platform. The research needs to recognise the existence of these edge cases and has to provide solutions to handle them.
    
    \item \textbf{Customer and Ride Provider Interaction Flow}: The customer and ride provider interaction flow stands in the centre of the decentralised ride pooling platform. The literature needs to provide insides into how this flow should look to utilise the advantages of blockchain technology.
    
    \item \textbf{Prototypical Realization}: Before building a market-ready version, the decentralised ride-pooling platform should be built as a prototype that showcases the most important aspects of the platform and proves its feasibility. Therefore it is important for the literature to include a prototypical realisation of the platform that provides important insides that are not which cannot be derived from the architecture alone.
\end{itemize}


As a result, the concept matrix shows the frequency with which the concepts dealt with in the specialist literature are distributed over the nine topics of decentralised ride-pooling. The assignment of the concepts on the x-axis to authors of the relevant specialist literature on the y-axis can be seen in Table \ref{tab:litSearchResultsMatrix}. If a research paper covers a topic in detail, it is marked with $\checkmark$ $\checkmark$. if a research paper covers some aspects of a topic, it is marked with a $\checkmark$ . If a paper does not cover a topic at all or in a way that does not align with the objectives of this research, it is marked with a $\times$. 


\begin{longtable}{p{5cm}l|l|l|l|l|l|l|l|l}
\caption{Results of the Literature Search} \\
\label{tab:litSearchResultsMatrix}
Research Paper / Topic & 
&
\rotatebox{90}{Blockchain Utilisation} & 
\rotatebox{90}{Customer and Ride Provider Interaction Flow} & 
\rotatebox{90}{Payments and Service Fees} & 
\rotatebox{90}{Privacy and Anonymity} & 
\rotatebox{90}{Security and Resilience} & 
\rotatebox{90}{Trust Mechanisms} & 
\rotatebox{90}{Off-Chain Edge Cases} & 
\rotatebox{90}{Prototypical Realization} \\ 
\hline
\endfirsthead

\multicolumn{9}{c}%
{{\bfseries \tablename\ \thetable{} -- continued from previous page}} \\
\hline
Research Paper / Topic & 
&
\rotatebox{90}{Blockchain Utilisation} & 
\rotatebox{90}{Customer and Ride Provider Interaction Flow} & 
\rotatebox{90}{Payments and Service Fees} & 
\rotatebox{90}{Privacy and Anonymity} & 
\rotatebox{90}{Security and Resilience} & 
\rotatebox{90}{Trust Mechanisms} & 
\rotatebox{90}{Off-Chain Edge Cases} & 
\rotatebox{90}{Prototypical Realization} \\ 
\hline
\endhead

\hline \multicolumn{9}{r}{{Continued on next page}} \\
\endfoot

\hline
\endlastfoot


B-Ride: Ride Sharing With Privacy-Preservation, Trust and Fair Payment Atop Public Blockchain & ~\cite{Baza.2021} & $\checkmark$ $\checkmark$ & $\checkmark$ $\checkmark$ & $\checkmark$  & $\times$ & $\times$ & $\checkmark$ & $\times$ & $\checkmark$  \\
\hline
Application of Blockchain Technology to Smart City Service: A Case of Ridesharing & ~\cite{Chang.} & $\checkmark$ & $\checkmark$ & $\times$ & $\checkmark$  & $\checkmark$ $\checkmark$ & $\times$ & $\times$ & $\times$ \\
\hline
Ride-Hailing for Autonomous Vehicles: Hyperledger Fabric-Based Secure and Decentralize Blockchain Platform & ~\cite{Shivers.} & $\checkmark$ & $\checkmark$ & $\times$ & $\checkmark$ & $\times$ & $\times$ & $\times$ & $\checkmark$  \\
\hline
RiderS: Towards a Privacy-Aware Decentralized Self-Driving Ride-Sharing Ecosystem & ~\cite{Bathen.} & $\checkmark$ & $\checkmark$ & $\checkmark$ & $\checkmark$$\checkmark$ & $\checkmark$ & $\times$ & $\times$ & $\checkmark$ \\
\hline
A Decentralized Ride-Hailing Mode Based on Blockchain and Attribute Encryption & ~\cite{Zhang.} & $\checkmark$& $\checkmark$ & $\checkmark$ $\checkmark$ & $\checkmark$ & $\times$ & $\times$ & $\times$ & $\checkmark$  \\
\hline
Enhancing Blockchain-based Ride-Sharing Services using IPFS & ~\cite{Mahmoud.2022} & $\checkmark$$\checkmark$ & $\checkmark$$\checkmark$ & $\checkmark$ & $\checkmark$ & $\checkmark$ & $\times$ & $\times$ & $\checkmark$  \\
\hline
BlockWheels - A Peer to Peer Ridesharing Network & ~\cite{Joseph.} & $\checkmark$ & $\checkmark$$\checkmark$ & $\checkmark$ & $\checkmark$ & $\times$  & $\times$  & $\times$ & $\times$  \\
\hline
A Light Blockchain-Powered Privacy-Preserving Organization Scheme for Ride Sharing Services & ~\cite{Baza.52520205282020} & $\checkmark$ & $\checkmark$ $\checkmark$ & $\checkmark$ & $\checkmark$ & $\times$ & $\times$ & $\times$ & $\times$ \\
\hline
\pagebreak
BlockV: A Blockchain Enabled Peer-Peer Ride Sharing Service & ~\cite{Pal.} & $\checkmark$ & $\checkmark$ & $\checkmark$ & $\times$  & $\checkmark$ & $\checkmark$  &  $\checkmark$ &  $\checkmark$ \\
\hline
Blockchain-Based Ride-Sharing System with Accurate Matching and Privacy-Preservation & ~\cite{Badr.} &  $\checkmark$ & $\checkmark$ & $\times$ & $\checkmark$  & $\checkmark$ $\checkmark$ & $\times$ & $\times$ & $\times$ \\
\hline
Towards Blockchain-based Ride-sharing Systems & ~\cite{Vazquez.} & $\checkmark$ $\checkmark$ & $\checkmark$ & $\checkmark$ & $\checkmark$ & $\checkmark$ & $\times$ & $\times$ & $\checkmark$  \\
\hline
Co-utile P2P ridesharing via decentralization and reputation management & ~\cite{Sanchez.2016} &  $\checkmark$ &  $\checkmark$ &  $\checkmark$ &  $\checkmark$  $\checkmark$ &  $\checkmark$ &  $\checkmark$ & $\times$ & $\times$   \\
\hline
\end{longtable}

\subsection{Scientific Literature findings}
The concept matrix \ref{tab:litSearchResultsMatrix} shows that the literature review did not identify a single paper that provides detailed coverage of all topics and would thereby allow us to answer all research objectives. The matrix also shows that while many of the papers discuss multiple topics, they often remain on a conceptual level without the goal of developing a feature-complete platform. It is still very important to take a detailed look at the identified literature to discuss their approaches on developing a decentralised ride-pooling platform. In the following we will take a look at the outstanding features that are proposed in each paper and evaluate how they can support the creation of our feature complete ride polling service.


``B-Ride: Ride Sharing With Privacy-Preservation, Trust and Fair Payment Atop Public
introduces'' B-Ride, a decentralized ride-sharing service built on public Blockchain~\cite{Baza.2021}. B-Ride ensures ride data privacy for both drivers and riders. To counter malicious users exploiting blockchain's anonymity, the system introduces a time-locked deposit protocol using smart contracts and zero-knowledge set membership proof. This ensures trust and commitment from all participants. A unique "pay-as-you-drive" methodology is proposed for fair payment, where drivers are compensated based on the distance covered. This 
system has many advantages. It ensures that the ride provider gets paid for the driven distance, and the customer does not have to deposit more money than necessary at once. The problem with this approach is, that it requires so called Location Prover. These hardware devices ensure that the car provides honest location information about its position. While this technology is superior to systems that do not relay on Location Prover, a global network of Location Provers is currently not feasible. Therefore our platform will utilise an upfront deposit of the expected ride cost by the user that can be claimed by the ride provider after completing the ride.
Additionally, B-Ride features a decentralized reputation management mechanism, rating drivers on past behavior, incentivizing them to maintain good conduct. The system was successfully implemented and tested on the Ethereum blockchain, highlighting its real-world applicability. While a rating system is needed to ensure trust on the platform, B-Rides implementation also relies on Location Provers. Therefore we will look at other research papers and their approaches in regards to rating mechanisms.

The authors, Shuchih Ernest Chang and Chi-Yin Chang, highlight in their research paper ``Application of Blockchain Technology to Smart City Service: A Case of Ridesharing'' ~\cite{Chang.} the challenges faced by traditional ridesharing platforms. To address the challenges of traditional ridesharing platforms, the SmaRi system leverages blockchain technology and smart contracts. This approach not only ensures secure and automated transactions but also promotes decentralized decision-making. The research emphasizes the potential of blockchain in reshaping ridesharing services. A notable design decision by the authors is to use an off-chain authentication service called social networking service. This service allows users to utilise social media accounts to share rides with friends and to authenticate against the platform. While this concept is not covered in depth it provides insides into the many advantages of an off-chain authentication service.

The paper ``Ride-Hailing for Autonomous Vehicles: Hyperledger Fabric-Based Secure and Decentralize Blockchain Platform'' addresses the problems of centralised ride-sharing platforms. The authors propose a decentralized approach using blockchain technology, allowing individual AV owners to contribute their vehicles to a community-driven fleet when not in use.~\cite{Shivers.} The chosen blockchain platform for this endeavour is Hyperledger Fabric. The paper is notable for utilizing a private blockchain to tackle the problems in regards to anonymity and privacy, which are inherent downsides of public blockchains. The decision between a public and a private blockchain is one of the core architectural decisions for our own ride-sharing platform.  After taking the arguments by ~\cite{Shivers.} as well as other research papers into consideration, we decided to go forward with a public blockchain for our platform. With privacy being a focus of our ride pooling platform, there should be no possibility to trace individual user activity by monitoring the chain activities, even if it is public. Therefore we prioritise the increased decentralisation of public chains. Using a public chain allows us to utilize generic public nodes to handle smart contracts. Thereby we do not need to build a private network of independent node providers to build a private blockchain. Other research papers also prove the feasibility of decentralised ride pooling platforms on public chains ~\cite{Mahmoud.2022} ~\cite{Joseph.} ~\cite{Baza.52520205282020}

Another research paper introduces ``RiderS, a groundbreaking decentralized self-driving ride-sharing ecosystem'' ~\cite{Bathen.}.Central to this is the emphasis on user privacy, achieved through a privacy-first biometric technology. Instead of traditional passwords, users become their own unique identifier, ensuring genuine system interactions. To fortify this ecosystem, blockchain technology is employed, offering benefits like decentralization and auditability. Each participant, whether a rider or an autonomous vehicle (AV), accesses the system via a "Wallet." This software client manages credentials, facilitates transactions, and serves as the primary gateway into the blockchain. Monetary exchanges within this ecosystem utilize a stable coin named "Mobi," anchored to various cryptocurrencies and fiat currencies. This system is very usefull and should be adapted by our platform. By introducing a Crypto Exchnage to the platform we allow the users to pay with a verity of different currencies including fiat currencies while still utilising the advantages of crypto currencies   in our platform. A standout feature is the emphasis on privacy. Users can generate single-use addresses, ensuring anonymity for each ride. This also should be adapted by our platform. Even though the wallet owner is anonymous on the chain, it prevents wallet tracking over a long period of time, which could lead the exposure of the wallet holder.

The research paper ``A Decentralized Ride-Hailing Mode Based on Blockchain and Attribute Encryption'' presents a novel ride-hailing approach using blockchain and attribute encryption.~\cite{Zhang.} 
The system includes a decentralized Blockchain-Based Ride-Hailing Mode: This mode has roles such as the Passenger, who generates encrypted ride details; the Driver, who decrypts and decides on ride acceptance; the Location Prover (LP), verifying the driver's location; and the Authentication Center, distributing keys and authenticating identities. Thereby the paper introduces a number of concepts that help us create our privacy preserving ride pooling platform. First of all the concept of creating a shared secret between customer and ride provider should be used  to share sensitive information on chain, like exact coordinates. With the  Authentication Center the paper also introduces an off chain authentication service, which further promotes the concept of an off chain authority that can verify  wallets to handle on chain nteractions with the ride pooling platform.

The authors of ``Enhancing Blockchain-based Ride-Sharing Services using IPFS''  propose a decentralized ride-sharing system to address challenges in centralized services, such as security concerns and single points of failure.~\cite{Mahmoud.2022}  The solution integrates blockchain with the Interplanetary File System (IPFS). Instead of storing all ride-sharing data on the blockchain, the system moves this data to IPFS and only retains a compact hash on the blockchain. This approach reduces data storage on the blockchain, leading to faster processing and lower costs. The system uses smart contracts on the Ethereum platform for management, and experimental results highlight its scalability and efficiency. This concept should be utalised if the prototype implementation or future iterations of the platforms should struggle with managing the amounts of data necessary to hanlde rides, resulting in high gas prices or slow blockchain performance.


The paper ``BlockWheels - A Peer to Peer Ridesharing Network a ridesharing'' system built on the Ethereum blockchain. ~\cite{Joseph.} introduces a sophisticated ride-matching system, utilizing geolocation tools to pair riders with nearby drivers. While out platform is utilising an auction based approach to match customers with ride providers, this paper showcases the advantages of an off chain matching approach to handle the complex matching with an on chain ride handelig that tracks the actual ride.

The paper ``A Light Blockchain-Powered Privacy-Preserving Organization Scheme for Ride Sharing Services'' introduce a decentralized system using public blockchain, eliminating the central third-party vulnerabilities.~\cite{Baza.52520205282020} This system ensures location and time privacy by employing spatial and temporal cloaking techniques, allowing riders and drivers to share generalized locations and time intervals instead of exact details. This approach should also be utilised with our platform. With a location matching based on approximated data we can ensure that the customer only needs to share their exact location with the ride provider that will fulfill the ride request.
BlockWheels participants also uses changing pseudonyms for each trip, ensuring untraceability. With BlockWheels also promoting this concept it shows that this approach to ensuring untraceability is a best practice in regards to on chain user activities. The entire scheme has been practically implemented and tested on the Ethereum platform, showcasing its feasibility and effectiveness in real-world scenarios.

The authors ~\cite{Pal.}  introduces a decentralized ride-sharing solution using blockchain. BlockV ensures fairness in ride-sharing in two main ways:
Payment Fairness: It allows any network peer to compute the ride cost based on path details.
Ride Fairness: In case of disputes, the system collaborates with Road Side Units (RSUs) to determine and penalize any malicious activity by drivers or riders.
The BlockV system involves four key participants: the DRIVER, RIDER, BlockV, and RSUs. The process starts with riders selecting a route and fare from a decentralized database. Once chosen, they confirm the ride and lock in the fare. At the ride's end, riders can either complete the ride, releasing funds, or raise complaints if unsatisfied. The system then verifies complaints using RSUs and takes appropriate action.
With the RSUs BlockV provides a solution to the problem on how to manage edge-cases like customer complaints. While this concept relies on the existence of RSUs and mainly focuses on the handling of false routes taken by the ride provider it showcases the importance of robust edge case handling.

The paper ``Blockchain-Based Ride-Sharing System with Accurate Matching and Privacy-Preservation'' proposes a method of dividing the ride-sharing coverage area into small cells using overlapping grids ~\cite{Badr.}. This ensures that customers and ride providers are matched with location accuracy, as they report their locations by cell numbers. When their exact locations coincide within a common cell across any grid, a match is made. While this approach does not utilises the planned auction system proposed by our platform it promotes a grid based approach that can help to ensure that potential matching services can be bound to specific areas. By assinging matching services to single tiles in a grid we can assure that each customer can find their local matching service and no matching service can collect data for areas that are too large.


The paper ``Towards Blockchain-based Ridesharing Systems'' addresses privacy concerns, by also utilising spatial cloaking and an off chain mathcing service.~\cite{Vazquez.} When a passenger requests a ride, an off-blockchain algorithm matches them with suitable drivers based on this cloaked data. To foster a sense of trust, both parties, the ride provide and the customer, post a deposit fee through a smart contract. This deposit acts as a commitment, and if either party defaults, the other is automatically compensated. This flow very much aligns with our vision of interaction flow of our decentralised ride pooling platform. The main advantage of this approach is that it allows for more complex matching algorithms without dramatically increasing gas fees, while still utilising the advantages of blockchain by tracking the actual ride and related payments on chain. 

The research paper ``Co-utile P2P ridesharing via decentralization and reputation management'' focusses on preserving user privacy~\cite{Sanchez.2016}. In practice, this means that only when a driver's and passenger's trips align will they be privy to each other's identity, desired trip details, and reputation. This selective disclosure ensures that personal data remains confidential. This also aligns with the research objectives of our decentralised ride pooling platform and needs to be considered in the final design.
Addressing privacy alone isn't enough; trust is equally important. The authors tackle this by weaving in a decentralized reputation management mechanism. Post a shared ride, both drivers and passengers have the liberty to rate each other. This allows peers to gauge the aggregated reputation of others, based on historical ratings, in a manner that's both transparent and trustworthy. This is a common best practice even with centralised ride sharing platforms. For our decentralised platform the rating should also be managed on chain, as it profits from the tamper proof nature of blockchain.

\subsection{Conclusion}
The detailed literature review show that there are many different approaches on how a decentralised ride pooling platform should be designed, with different authors focusing on different aspects of the platform. While there are many common best practices in regards to safety and user privacy there is also no uniform approach to designing the different components of the platform. While some papers suggest to handle all interactions with the platform on chain others suggest taking some elements off chain to allow for more complex flows. Therefore we can not rely on simply combining the platforms from the research papers into a single, feature complete platform.

Therefore, to create a feature complete ride pooling platform, it will be necessary to make design decisions that will contradict the suggested approaches of some papers to embrace design decisions made by other papers. These decisions will be made based on our research objectives, which state that the maximisation of privacy, security and transparency is the underlying goal of our platform.




\chapter{Proposed Solution}\label{chap:ProposedSolution}
Having delved into the details of the current research landscape surrounding decentralised ride-sharing platforms, we can now construct our own platform based on the findings of the scientific literature. The goal is not to replicate existing platforms but to set a benchmark that encapsulates best practices from various research papers while also proposing improvements to current methodologies.

We want to look at the platform holistically and incorporate all the attributes in our design that are necessary to operate the platform, including Blockchain utilisation, transparency, user and provider interaction protocols, payments and service fees, privacy and anonymity, security and resilience, trust mechanisms, off-chain edge cases, and a prototypical realisation.

Therefore, in this chapter, we provide an overview of the conceptual design of the decentralised platform, discuss the inner workings of each component that is part of the platform in detail and showcase the data privacy and rust mechanisms of the platform.
Through these explanations, we aim to provide a comprehensive blueprint for a decentralised ride-sharing platform that meets and exceeds the expectations set by the current academic and industry standards.

For convenience, the proposed platform will, from now on, be called GETACAR. We chose the name for its ability to describe the core offering of the platform, to get-a-car ride, while also being short, recognisable and easy to remember.

(The name also sounds phonetically similar to the title of the 1997 science fiction movie Gattaca, starring Ethan Hawke, Uma Thurman, and Jude Law, which the author of this paper immensely enjoyed.)
\section{Conceptual Design of the Decentralised Platform}\label{sec:DesignOfThePlatform}
\subsection{User Experience}
In the realm of ride-pooling platforms, the user experience is paramount. From an end-user perspective, the conceptual design of the GETACAR platform is intentionally straightforward. Drawing inspiration from the research on decentralized ride-pooling platforms, GETACAR aligns its user flow with established centralized solutions such as Uber Pool and Lyft.
Therefore the flow should look like this: Open the App, Login/Signup, Set Pickup Location, Set Destination, Choose Ride Requirements, Request Ride, Wait for Driver Match, Confirm Booking, Track the Driver, Ride, Payment, Rating \& Feedback

The rationale behind this design choice is evident. Platforms like Uber and Lyft have already invested significant resources into refining and optimizing the user flow. Over the years, they have garnered invaluable insights and established best practices that have proven effective in ensuring a seamless user experience. It would be counterproductive to reinvent the wheel when such tried-and-true models exist. Instead, by basing the user flow on these best practices, GETACAR aims to provide an experience that is not only familiar to users but also efficient and intuitive. 
One of the primary objectives of GETACAR is to offer an experience that rivals, if not surpasses, the current centralized solutions. By emulating the user flow of these established platforms, GETACAR aspires to incentivize users to transition from centralized platforms to its decentralized counterpart. The promise is twofold: a user experience that mirrors what users are already accustomed to, coupled with the added benefits of enhanced privacy, transparency, and more competitive pricing. The latter is achieved by offering reduced platform fees, which in turn can translate to better prices for the end-users. 
Similarly, the platform's design also caters to the needs of ride providers. The process of offering rides and interacting with the platform is designed to be as straightforward and user-friendly as possible. Yet, this simplicity does not come at the expense of privacy and transparency. Both are upheld as core tenets, ensuring that all parties—be it users or providers—can operate with confidence and trust. This emphasis on ease of use for providers is crucial. Therefore a simplified provider flow would look like this: Wait for fitting Ride Requests, Bid on Ride Requests, Accept Ride Request, Navigate to Pickup Location, Confirm Passenger Pickup, Start the Ride, Navigate to Destination, End the Ride and Confirm Payment, Rate the Passenger, Wait for Next Ride or Go Offline

To compete with existing centralized platforms, which boast vast fleets of ride providers, GETACAR needs to ensure a substantial influx of ride providers. By offering a platform that is both simple to use and transparent, GETACAR aims to attract and retain a large pool of providers, making it a formidable competitor in the ride-pooling landscape.<Aspects of self driving cars>

\subsection{Component Overview}
- Now that the goal for the GETACAR UX is clear we can focus on how our requirements towards privacy, transparency and price advantage and how these can  be achieved without deteriorating the UX.

-To achieve this a number of components need to interact with each other.
- These are the components:

1. User Frontend: Provides Users the ability to make ride requests, rate ride-providers and passengers and adjust general settings like the preferred min. rating of ride providers
2. Ride Provider Frontend: Provides Ride Providers with the ability to search for / bid on open ride request, 
3. Authentication Service: The Authentication Service manages the user accounts and ratings and is the only component that can match user pseudonyms with the real account owner 
4. Matching Service: The Matching Service has the job to ensure that each user gets matched with its optimal ride provider to a fair price. 
5. Ride Contract Service
6. Crypto Exchnage 


\section{Privacy Measures and Trust Mechanism Design}\label{sec:PrivacyAndTrustMechanism}
In the journey of conceptualizing the GETACAR platform, paramount importance has been accorded to privacy measures, ensuring the protection of user data at every juncture. The following chapter will focus on how data is orchestrated and disseminated across the platform's multifaceted components.


\subsection{Customer Privacy}

This table delineates a compilation of both conventional and GETACAR-specific data points typically exchanged in ride-sharing platforms.

To facilitate clarity and ease of interpretation, each data point, in the context of each service, is represented using specific symbols:

\begin{itemize}
    \item \textbf{$\checkmark$}: Symbolizes that the service possesses access to the respective information.
    \item \textbf{$\bullet$}: Denotes the necessity for the information to be shared amongst parties.
    \item \textbf{$\times$}: Indicates the service's lack of access to the particular information.
    \item \textbf{($\times$)}: Indicates that the service can access parts of the information.
\end{itemize}

The ensuing table provides a detailed breakdown of these data points in relation to the various services within the platform.

The GETACAR platform has been meticulously designed with a primary focus on user privacy. Through its architecture, the platform ensures that user data is managed with the utmost discretion, granting access only where necessary across its specific components. The table provided offers a detailed overview of this data sharing mechanism.

When it comes to basic personal details such as the full name, gender, and date of birth, these are exclusively accessible by the customer themselves and the authentication service. This design choice ensures that these personal identifiers remain shielded and aren't exposed to other facets of the platform. Similarly, contact details like the email address, phone number, and home address are safeguarded, with access limited to the customer and the authentication service. This setup ensures that personal contact details remain undisclosed to ride providers or other services.

In the realm of payment information, details like credit/debit card numbers and bank account specifics are primarily shared between the customer and the crypto exchange. This arrangement facilitates smooth transactions while upholding the confidentiality of the user's financial data. Moreover, the payment history, which is vital for transparency in transactions, is accessible to the customer and, to some extent, by the crypto exchange.

Ride details form a crucial part of the platform's operations. Information such as pickup and drop-off locations, the date and time of rides, and specific ride preferences are shared between the customer and the ride provider. This sharing ensures that the ride provider is equipped with all necessary details to fulfill the ride request, all the while maintaining the user's privacy. Additionally, location data, including real-time location during a ride, frequent visitation points, and the route taken during the ride, are shared between the customer and the ride provider. This is pivotal for ensuring safety and efficiency during the journey.

Device-related information, which encompasses details like the device type, operating system, app version, and unique device identifiers, remains confidential, with access restricted solely to the customer. This ensures that device-specific data remains shielded. Communication, especially in-app messages exchanged between the driver and the rider, is kept private, confined to just the two parties, ensuring that all communication remains confidential.

Behavioral data, which includes insights into app usage patterns, click patterns within the app, and features that are frequently accessed, is kept private, with access limited exclusively to the customer. This ensures that user behavior within the app remains undisclosed. Safety and security data, particularly records of any incidents or disputes during rides, are shared between the customer and the ride provider, ensuring a prompt and effective resolution to any issues that may arise.

Ratings form an integral part of the platform's feedback mechanism. Ratings that customers provide about drivers are accessible to both the customer and the authentication service. On the flip side, ratings that drivers give about the user are accessible to the ride provider and the authentication service. This dual-access system ensures transparency in the rating process while upholding user privacy.

Lastly, user preferences and settings, which include language preferences and notification settings, are kept confidential, with access limited solely to the customer.

In essence, the table and its accompanying explanation underscore the GETACAR platform's unwavering commitment to safeguarding user data. By judiciously limiting access and ensuring that only necessary data is shared, the platform promises a secure and privacy-centric ride-sharing experience.


\begin{table}[h]
\centering
\small % Reduce font size
\begin{tabular}{|l|c|c|c|c|c|c|}
\hline
\textbf{Data} & \rotatebox{90}{\textbf{Customer}} & \rotatebox{90}{\textbf{Ride Provider}} & \rotatebox{90}{\textbf{Matching Service}} & \rotatebox{90}{\textbf{Crypto Exchange}} & \rotatebox{90}{\textbf{Authentication Service}} & \rotatebox{90}{\textbf{Publicly Available}} \\
\hline
\multicolumn{1}{|l|}{\textbf{Basic Personal Details:}} & \multicolumn{1}{c}{} & \multicolumn{1}{c}{} & \multicolumn{1}{c}{} & \multicolumn{1}{c}{} & \multicolumn{1}{c}{} & \multicolumn{1}{c|}{} \\
\hline
Full name & $\checkmark$$\bullet$ &  $\times$ & $\times$ & $\times$ & $\checkmark$$\bullet$ & $\times$ \\
\hline
Gender & $\checkmark$$\bullet$ &  $\times$ & $\times$ & $\times$ & $\checkmark$$\bullet$ & $\times$ \\
\hline
Date of birth & $\checkmark$$\bullet$ &  $\times$ & $\times$ & $\times$ & $\checkmark$$\bullet$ & $\times$ \\
\hline
\multicolumn{1}{|l|}{\textbf{Contact Information:}} & \multicolumn{1}{c}{} & \multicolumn{1}{c}{} & \multicolumn{1}{c}{} & \multicolumn{1}{c}{} & \multicolumn{1}{c}{} & \multicolumn{1}{c|}{} \\
\hline
Email address & $\checkmark$$\bullet$ &  $\times$ & $\times$ & $\times$ & $\checkmark$$\bullet$ & $\times$ \\
\hline
Phone number & $\checkmark$$\bullet$ &  $\times$ & $\times$ & $\times$ & $\checkmark$$\bullet$ & $\times$ \\
\hline
Home address & $\checkmark$$\bullet$ &  $\times$ & $\times$ & $\times$ & $\checkmark$$\bullet$ & $\times$ \\
\hline
\multicolumn{1}{|l|}{\textbf{Payment Information:}} & \multicolumn{1}{c}{} & \multicolumn{1}{c}{} & \multicolumn{1}{c}{} & \multicolumn{1}{c}{} & \multicolumn{1}{c}{} & \multicolumn{1}{c|}{} \\
\hline
Credit/debit card details & $\checkmark$$\bullet$ & $\times$ & $\times$ & $\checkmark$$\bullet$ & $\times$ & $\times$ \\
\hline
Bank account details & $\checkmark$$\bullet$ & $\times$ & $\times$ & $\checkmark$$\bullet$ & $\times$ & $\times$  \\
\hline
Payment history & $\checkmark$$\bullet$ & $\times$ & $\times$ & ($\times$) &$\checkmark$ & $\times$ \\
\hline
Billing address & $\checkmark$$\bullet$ & $\times$ & $\times$ & $\checkmark$$\bullet$ & $\times$ & $\times$ \\
\hline
\multicolumn{1}{|l|}{\textbf{Ride Details:}} & \multicolumn{1}{c}{} & \multicolumn{1}{c}{} & \multicolumn{1}{c}{} & \multicolumn{1}{c}{} & \multicolumn{1}{c}{} & \multicolumn{1}{c|}{} \\
\hline
Pickup and drop-off locations & $\checkmark$$\bullet$ & $\checkmark$$\bullet$ &  ($\times$) & $\times$ & $\times$ & $\times$ \\
\hline
Date and time of rides & $\checkmark$$\bullet$ & $\checkmark$$\bullet$ & $\checkmark$ & $\times$ & $\checkmark$ & $\times$ \\
\hline
Ride preferences & $\checkmark$$\bullet$ & $\checkmark$$\bullet$ & $\checkmark$ & $\times$ & $\times$ & $\times$ \\
\hline
Ride history & $\checkmark$$\bullet$ & $\times$ & $\times$ & $\times$ & $\times$ & $\times$ \\
\hline
\multicolumn{1}{|l|}{\textbf{Location Data: }} & \multicolumn{1}{c}{} & \multicolumn{1}{c}{} & \multicolumn{1}{c}{} & \multicolumn{1}{c}{} & \multicolumn{1}{c}{} & \multicolumn{1}{c|}{} \\
\hline
Real-time location during a ride & $\checkmark$$\bullet$ & $\checkmark$$\bullet$ & $\times$ & $\times$ & $\times$ & $\times$ \\
\hline
Frequent locations &$\checkmark$$\bullet$ & $\times$ & $\times$ & $\times$ & $\times$ & $\times$ \\
\hline
Route taken during the ride & $\checkmark$$\bullet$ & $\checkmark$$\bullet$ & $\times$ & $\times$ & $\times$ & $\times$ \\
\hline
\multicolumn{1}{|l|}{\textbf{Device Information: }} & \multicolumn{1}{c}{} & \multicolumn{1}{c}{} & \multicolumn{1}{c}{} & \multicolumn{1}{c}{} & \multicolumn{1}{c}{} & \multicolumn{1}{c|}{} \\
\hline
Device type & $\checkmark$$\bullet$ & $\times$ & $\times$ & $\times$ & $\times$ & $\times$ \\
\hline
Operating system & $\checkmark$$\bullet$ & $\times$ & $\times$ & $\times$ & $\times$ & $\times$ \\
\hline
App version & $\checkmark$$\bullet$ & $\times$ & $\times$ & $\times$ & $\times$ & $\times$ \\
\hline
Device identifiers & $\checkmark$$\bullet$ & $\times$ & $\times$ & $\times$ & $\times$ & $\times$ \\
\hline
\multicolumn{1}{|l|}{\textbf{Communication Data: }} & \multicolumn{1}{c}{} & \multicolumn{1}{c}{} & \multicolumn{1}{c}{} & \multicolumn{1}{c}{} & \multicolumn{1}{c}{} & \multicolumn{1}{c|}{} \\
\hline
In-app messages between driver and rider & $\checkmark$$\bullet$ & $\checkmark$$\bullet$ & $\times$ & $\times$ & $\times$ & $\times$ \\
\hline
\multicolumn{1}{|l|}{\textbf{Behavioral Data: }} & \multicolumn{1}{c}{} & \multicolumn{1}{c}{} & \multicolumn{1}{c}{} & \multicolumn{1}{c}{} & \multicolumn{1}{c}{} & \multicolumn{1}{c|}{} \\
\hline
App usage patterns & $\checkmark$$\bullet$ & $\times$ & $\times$ & $\times$ & $\times$ & $\times$ \\
\hline
Click patterns within the app & $\checkmark$$\bullet$ & $\times$ & $\times$ & $\times$ & $\times$ & $\times$ \\
\hline
Features frequently used & $\checkmark$$\bullet$ & $\times$ & $\times$ & $\times$ & $\times$ & $\times$ \\
\hline
\multicolumn{1}{|l|}{\textbf{Safety and Security Data: }} & \multicolumn{1}{c}{} & \multicolumn{1}{c}{} & \multicolumn{1}{c}{} & \multicolumn{1}{c}{} & \multicolumn{1}{c}{} & \multicolumn{1}{c|}{} \\
\hline
Records of any incidents or disputes during rides & $\checkmark$$\bullet$ & $\checkmark$$\bullet$ & $\times$ & $\times$ & $\times$ & $\times$ \\
\hline
\multicolumn{1}{|l|}{\textbf{Ratings: }} & \multicolumn{1}{c}{} & \multicolumn{1}{c}{} & \multicolumn{1}{c}{} & \multicolumn{1}{c}{} & \multicolumn{1}{c}{} & \multicolumn{1}{c|}{} \\
\hline
Ratings provided about drivers & $\checkmark$$\bullet$ & $\times$ & $\times$ & $\times$ & $\checkmark$ & $\times$ \\
\hline
Ratings received from drivers about the user & $\times$ & $\checkmark$$\bullet$ & $\times$ & $\times$ & $\checkmark$ & $\times$ \\
\hline
\multicolumn{1}{|l|}{\textbf{Preferences and Settings: }} & \multicolumn{1}{c}{} & \multicolumn{1}{c}{} & \multicolumn{1}{c}{} & \multicolumn{1}{c}{} & \multicolumn{1}{c}{} & \multicolumn{1}{c|}{} \\
\hline
Language preference & $\checkmark$$\bullet$ & $\times$ & $\times$ & $\times$ & $\times$ & $\times$ \\
\hline
Notification settings & $\checkmark$$\bullet$ & $\times$ & $\times$ & $\times$ & $\times$ & $\times$ \\
\hline
\end{tabular}
\caption{Customer Data Privacy Matrix}
\label{tab:your_label_here}
\end{table}






\begin{table}
\centering
\small % Reduce font size
\begin{tabular}{|l|c|c|c|c|c|c|}
\hline
\textbf{Data} & \rotatebox{90}{\textbf{Ride Provider}} & \rotatebox{90}{\textbf{Customer}} & \rotatebox{90}{\textbf{Matching Service}} & \rotatebox{90}{\textbf{Crypto Exchange}} & \rotatebox{90}{\textbf{Authentication Service}} & \rotatebox{90}{\textbf{Publicly Available}} \\
\hline
\multicolumn{1}{|l|}{\textbf{Basic Personal Details:}} & \multicolumn{1}{c}{} & \multicolumn{1}{c}{} & \multicolumn{1}{c}{} & \multicolumn{1}{c}{} & \multicolumn{1}{c}{} & \multicolumn{1}{c|}{} \\
\hline
Full name & $\checkmark$$\bullet$ & $\checkmark$$\bullet$ & $\times$ & $\times$ & $\checkmark$$\bullet$ & $\times$ \\
\hline
Gender & $\checkmark$ & $\checkmark$ & $\times$ & $\times$ & $\times$ & $\times$ \\
\hline
Date of birth & $\checkmark$ & $\checkmark$ & $\times$ & $\times$ & $\times$ & $\times$ \\
\hline
\multicolumn{1}{|l|}{\textbf{Contact Information:}} & \multicolumn{1}{c}{} & \multicolumn{1}{c}{} & \multicolumn{1}{c}{} & \multicolumn{1}{c}{} & \multicolumn{1}{c}{} & \multicolumn{1}{c|}{} \\
\hline
Email address & $\checkmark$$\bullet$ & $\checkmark$$\bullet$ & $\checkmark$$\bullet$ & $\times$ & $\checkmark$$\bullet$ & $\times$ \\
\hline
Phone number & $\checkmark$$\bullet$ & $\checkmark$$\bullet$ & $\times$ & $\times$ & $\checkmark$$\bullet$ & $\times$ \\
\hline
Home address & $\checkmark$ & $\checkmark$ & $\times$ & $\times$ & $\times$ & $\times$ \\
\hline
\multicolumn{1}{|l|}{\textbf{Payment Information:}} & \multicolumn{1}{c}{} & \multicolumn{1}{c}{} & \multicolumn{1}{c}{} & \multicolumn{1}{c}{} & \multicolumn{1}{c}{} & \multicolumn{1}{c|}{} \\
\hline
Credit/debit card details & $\times$ & $\checkmark$$\bullet$ & $\times$ & $\checkmark$$\bullet$ & $\times$ & $\times$ \\
\hline
Bank account details & $\times$ & $\checkmark$$\bullet$ & $\times$ & $\checkmark$$\bullet$ & $\times$ & $\times$  \\
\hline
Payment history & $\times$ & $\checkmark$ & $\times$ & $\times$ & $\times$ & $\times$ \\
\hline
Billing address & $\times$ & $\checkmark$ & $\times$ & $\times$ & $\times$ & $\times$ \\
\hline
\multicolumn{1}{|l|}{\textbf{Ride Details:}} & \multicolumn{1}{c}{} & \multicolumn{1}{c}{} & \multicolumn{1}{c}{} & \multicolumn{1}{c}{} & \multicolumn{1}{c}{} & \multicolumn{1}{c|}{} \\
\hline
Pickup and drop-off locations & $\checkmark$$\bullet$ & $\checkmark$$\bullet$ & $\checkmark$$\bullet$ & $\times$ & $\times$ & $\times$ \\
\hline
Date and time of rides & $\checkmark$$\bullet$ & $\checkmark$$\bullet$ & $\checkmark$$\bullet$ & $\times$ & $\times$ & $\times$ \\
\hline
Ride preferences & $\checkmark$ & $\checkmark$ & $\times$ & $\times$ & $\times$ & $\times$ \\
\hline
Ride history & $\checkmark$$\bullet$ & $\checkmark$$\bullet$ & $\times$ & $\times$ & $\times$ & $\times$ \\
\hline
Driver ratings and feedback & $\checkmark$$\bullet$ & $\checkmark$$\bullet$ & $\times$ & $\times$ & $\times$ & $\checkmark$$\bullet$ \\
\hline
\multicolumn{1}{|l|}{\textbf{Location Data: }} & \multicolumn{1}{c}{} & \multicolumn{1}{c}{} & \multicolumn{1}{c}{} & \multicolumn{1}{c}{} & \multicolumn{1}{c}{} & \multicolumn{1}{c|}{} \\
\hline
Real-time location during a ride & $\checkmark$$\bullet$ & $\checkmark$$\bullet$ & $\times$ & $\times$ & $\times$ & $\times$ \\
\hline
Frequent locations & $\checkmark$ & $\checkmark$ & $\times$ & $\times$ & $\times$ & $\times$ \\
\hline
Route taken during the ride & $\checkmark$ & $\checkmark$ & $\times$ & $\times$ & $\times$ & $\times$ \\
\hline
\multicolumn{1}{|l|}{\textbf{Device Information: }} & \multicolumn{1}{c}{} & \multicolumn{1}{c}{} & \multicolumn{1}{c}{} & \multicolumn{1}{c}{} & \multicolumn{1}{c}{} & \multicolumn{1}{c|}{} \\
\hline
Device type & $\checkmark$ & $\checkmark$ & $\times$ & $\times$ & $\times$ & $\times$ \\
\hline
Operating system & $\checkmark$ & $\checkmark$ & $\times$ & $\times$ & $\times$ & $\times$ \\
\hline
IP address & $\checkmark$ & $\checkmark$ & $\times$ & $\times$ & $\times$ & $\times$ \\
\hline
App version & $\checkmark$ & $\checkmark$ & $\times$ & $\times$ & $\times$ & $\times$ \\
\hline
Device identifiers & $\checkmark$ & $\checkmark$ & $\times$ & $\times$ & $\times$ & $\times$ \\
\hline
\multicolumn{1}{|l|}{\textbf{Communication Data: }} & \multicolumn{1}{c}{} & \multicolumn{1}{c}{} & \multicolumn{1}{c}{} & \multicolumn{1}{c}{} & \multicolumn{1}{c}{} & \multicolumn{1}{c|}{} \\
\hline
In-app messages between driver and rider & $\checkmark$$\bullet$ & $\checkmark$$\bullet$ & $\times$ & $\times$ & $\times$ & $\times$ \\
\hline
\multicolumn{1}{|l|}{\textbf{Behavioral Data: }} & \multicolumn{1}{c}{} & \multicolumn{1}{c}{} & \multicolumn{1}{c}{} & \multicolumn{1}{c}{} & \multicolumn{1}{c}{} & \multicolumn{1}{c|}{} \\
\hline
App usage patterns & $\checkmark$ & $\checkmark$ & $\times$ & $\times$ & $\times$ & $\times$ \\
\hline
Click patterns within the app & $\checkmark$ & $\checkmark$ & $\times$ & $\times$ & $\times$ & $\times$ \\
\hline
Features frequently used & $\checkmark$ & $\checkmark$ & $\times$ & $\times$ & $\times$ & $\times$ \\
\hline
\multicolumn{1}{|l|}{\textbf{Safety and Security Data: }} & \multicolumn{1}{c}{} & \multicolumn{1}{c}{} & \multicolumn{1}{c}{} & \multicolumn{1}{c}{} & \multicolumn{1}{c}{} & \multicolumn{1}{c|}{} \\
\hline
Records of any incidents or disputes during rides & $\checkmark$$\bullet$ & $\checkmark$$\bullet$ & $\times$ & $\times$ & $\times$ & $\times$ \\
\hline
\multicolumn{1}{|l|}{\textbf{Ratings: }} & \multicolumn{1}{c}{} & \multicolumn{1}{c}{} & \multicolumn{1}{c}{} & \multicolumn{1}{c}{} & \multicolumn{1}{c}{} & \multicolumn{1}{c|}{} \\
\hline
Ratings provided about drivers & $\checkmark$$\bullet$ & $\checkmark$$\bullet$ & $\times$ & $\times$ & $\times$ & $\checkmark$$\bullet$ \\
\hline
Ratings received from drivers about the user & $\checkmark$$\bullet$ & $\checkmark$$\bullet$ & $\times$ & $\times$ & $\times$ & $\checkmark$$\bullet$ \\
\hline
\multicolumn{1}{|l|}{\textbf{Preferences and Settings: }} & \multicolumn{1}{c}{} & \multicolumn{1}{c}{} & \multicolumn{1}{c}{} & \multicolumn{1}{c}{} & \multicolumn{1}{c}{} & \multicolumn{1}{c|}{} \\
\hline
Language preference & $\checkmark$ & $\checkmark$ & $\times$ & $\times$ & $\times$ & $\times$ \\
\hline
Notification settings & $\checkmark$ & $\checkmark$ & $\times$ & $\times$ & $\times$ & $\times$ \\
\hline
\end{tabular}
\caption{Data access from the perspective of the Ride Provider}
\label{tab:ride_provider_perspective}
\end{table}


\chapter{Implementation of the Decentralised Platform}\label{chap:PrototypeImplementation}
To validate the design of the GETACAR Platform, it is important to showcase that an actual implementation of the services is feasible. Therefore, in the following chapter, we showcase how each component is built. The goal of this implementation is not to create a ready-to-release platform to prove that the core features and functions of the platform work as previously described. Because of that, the aim is to build this prototype with as many standardised and commonly utilised technologies as possible. While some specialist or upcoming technologies like newer cutting-edge blockchain protocols might allow to tackle shortcomings of the platform or allow for additional features, this is not the focus of this prototype. The goal is to present a prototype that can be used as a blueprint to build and release a marked-ready decentralised ride-pooling platform. Therefore, it should be easy to adapt the design of the GETACAR platform with a variety of different underlying tech stacks.

The prototype we describe on the following pages does cover the smart contracts that enable the ride flow, the matching service and both the frontend for the customer and the interface for the ride provider. All components interact with each other and provide a complete ride experience. Not part of the prototype is the authentication service. As described in the introduction of this paper, the realisation of a decentralised authentication service is not part of the scope of this research. Additionally, the prototype does not fully implement the Crypto Exchange component, as it would require corporations with multiple companies to provide crypto exchange services. Therefore this prototype utilises a crypto wallet with an integrated crypto exchange to allow for the manual simulation of the buying and selling crypto currency on the GETACAR platform.

\section{Smart Contracts}\label{sec:SmartContracts}
Smart contracts make up the backbone of the GETACAR ride pooling platform. They allow for the secure and transparent ride ordering flow that is the standout feature of the platform. Therefore we will start with the construction of these smart contracts for the prototype. 

Before writing the contracts it is necessary do decide on a programming language. This decision is crucial because it will also effect the compatibility of the smart contracts with the available blockchain platforms. 

Looking at the smart contract programming languages used by the research papers and the adaption of smart contract programming languages by blockchains the decision is clear: Solitidy is a broadly adapted smart contract programming language that is not only utilised by the Ethereum blockchain, but also other popular blockchains like the Binance Smart Chain, Polkadot, Tron and Avalanche. 

Solidity is a high-level programming language originally tailored for Ethereum blockchain's smart contracts. Influenced by JavaScript, Python, and C++, its syntax allows developers to craft self-executing contracts where terms are coded directly. These contracts are compiled to bytecode for the Ethereum Virtual Machine (EVM). Given blockchain's immutable nature and financial implications, Solidity emphasizes security and exception handling. 

Selecting this smart contract programming language for the development of the GETACAR prototype ensures easy adaption by future developers. Additionally the high adaption rate of Solidity ensures a future proof reference implantation in the fast evolving crypto landscape, compared to languages that are proprietorially used by smaller blockchains. For the creation of the prototype the smart contracts are deployed on the Ethereum Blockchain and utilise ETH as their underlying currency fro Gas Fees and payments.

\subsection{Contract Factory}
There are two design approaches to utilise a smart contract to manage all critical ride events. Firstly it is possible to create a single smart contract that can be used by all customers and ride providers to log their rides. The advantage of this approach are lower gas fees because no new contract is generated for every trip. The downside of this approach is, that it drastically increases the complexity of the contract to ensure that all trips stay separated inside the contract. It also increases the impact of security loopholes in the contract, because it cloud possibly allow users to influence the rides of other users. 

The second approach would utilise the concept a contract factory. A contract factory allows the generation of smart contracts based on a predefined table through a second smart contract, the so called contract factory. The main disadvantages of a contract factory are increased gas prices because the deployment of a new smart contract is generally more expensive then the interaction with an existing one. But the contract factory approach also provides a number of upsides. It allows for each ride contract to be capsuled into its own smart contract which improves security and decreases the complexity of the ride contract itself. 

Because data privacy and security are most important to the design of the GETACAR Platform it is decided to flow the contract factory approach.

The contract factory that is designed to generate ride contracts looks as follows:

At the center of the contract exists the createContract() function that takes the amount of ETH (that represents the maximum ride cost) as a deposit. The deposit holder will be the newly created ride contract which will only return the deposit if the right circumstances are meet. createContract() also executes some additional code that helps authentication services track newly created contracts. Each contact gets assigned with a contract number determined by the contract counter and is mapped to a timestamp that represents its creation date. Additionally the contract gets registered to the matching service smart contract. This function will be explained in detail at XXX.

\lstset{
  basicstyle=\footnotesize\ttfamily,
  breaklines=true,
  numbers=left,
  firstnumber=43
}

\begin{lstlisting}
    function createContract(uint256 _amount) public payable {
        require(msg.value == _amount, "Sent value does not match the specified amount.");
        Contract newContract = new Contract{value: _amount}(msg.sender);
        userContracts[msg.sender].push(newContract);

        // Increment contract counter and map new contract's address to the counter
        contractCounter++;
        contractsByID[contractCounter] = address(newContract);
        
        // Store the current block's timestamp
        timestampByID[contractCounter] = block.timestamp;

        // Call registerNewContract with the new contract's address
        this.registerNewContract(address(newContract));

        emit ContractCreated(msg.sender, newContract, contractCounter);
    }
\end{lstlisting}

The contact factory also contains a number of helper functions that allow authentication services to better track newly created contracts as mentioned above.

\lstset{
  basicstyle=\footnotesize\ttfamily,
  breaklines=true,
  numbers=left,
  firstnumber=57
}

\begin{lstlisting}

    function getContractsByUser(address user) public view returns (Contract[] memory) {
        return userContracts[user];
    }

    function getContractByID(uint256 contractID) public view returns (address) {
        return contractsByID[contractID];
    }

    // Fetch the timestamp by contract ID
    function getContractTimestampByID(uint256 contractID) public view returns (uint256) {
        return timestampByID[contractID];
    }

\end{lstlisting}




\subsection{Ride Contract}
Now that we have shown how to a ride contract is created through the ride contract factory it is important to look at the ride contract itself.

The Solidity smart contract under consideration allows the ride provider and the customer to interact with each other and tracks these interactions as decibed in XXX.

The contract starts with the initiation of the variables that are used to track the status of the ride through a constructor. The constructor also sets the wallet address of the customer who initiated the contract through the contract factory as party1. The address that is mapped to party one represnts the customer inside the contract.

\lstset{
  basicstyle=\footnotesize\ttfamily,
  breaklines=true,
  numbers=left,
  firstnumber=78
}

\begin{lstlisting}
    constructor(address _party1) payable {
        party1 = _party1;
        rideProviderAcceptedStatus = false;
        rideProviderArrivedAtPickupLocation = false;
        userReadyToStartRide = false;
        rideProviderStartedRide = false;
        rideProviderArrivedAtDropoffLocation = false;
        userMarkedRideComplete = false;
        userCanceldRide = false;
        rideProviderCanceldRide = false;

    }
\end{lstlisting}

After the contract is created by the customer, the address of the contract is shared via the matching service with the ride provider. The ride provider then uses the signContract() function to co-sign the contract as party2 and thereby activates the ride contract. As described in XXX the ride provider also has to deposit 10\% of predefined  maximum ride cost into the contract as part of the deposit trust mechanism.

\lstset{
  basicstyle=\footnotesize\ttfamily,
  breaklines=true,
  numbers=left,
  firstnumber=67
}

\begin{lstlisting}
    function signContract() public payable {
        require(party2 == address(0), "Party2 has already signed the contract.");
        require(!isActive, "Contract is already active.");
        require(!userCanceldRide, "User cannceld ride ");
        require(msg.sender != party1, "Party2 cannot be identical to Party1.");
        
        party2 = msg.sender;
        isActive = true;

        uint256 tenPercent = (address(this).balance * 10) / 100;
        require(msg.value >= tenPercent, "Party2 must deposit an amount equal to 10% of the contract balance.");

        // Refund any excess amount deposited by party2
        if (msg.value > tenPercent) {
            payable(msg.sender).transfer(msg.value - tenPercent);
        }
    }
\end{lstlisting}

Now that both parties have signed the contract the actual ride flow can start, as decided in XXX with the ''Ride Provider accepts ride'' event. All event functions are structured similarly. Therefore we will use the setRideProviderAcceptedStatus() function as an example to showcase how each event is represented by a function inside the smart contract. The function takes a message as an input that will later be written onto the chain permanently. This message is used by customer and the ride provider to exchange encrypted information on the blockchain as decided in XXX. The function also contains a number of checks utilising the require() function to make sure that it is only used as intended. Among others, these checks include the requirement to only be called by the right entity (depending on the event this can be the customer or the ride provider). Additionally a check is applied to ensure that the events are called in the ride order as designed in the ride flow. At last three more checks make sure that the contract is active, not canceled and a ride provider has accepted the contract. If all checks are successfully it can be ensured that the function is used as intended and the event is written onto the chain as a secure and auditable prof that it accrued. This is done though calling the UpdatePosted() function.

\lstset{
  basicstyle=\footnotesize\ttfamily,
  breaklines=true,
  numbers=left,
  firstnumber=165
}
\begin{lstlisting}
    function setRideProviderAcceptedStatus(string memory _message) public {
        require(isActive, "Contract is not active.");
        require(msg.sender == party2, "Only Party2 can set the ride provider accepted status.");
        require(!rideProviderAcceptedStatus, "Ride Provider Accepted Status can only be set once.");

        require(!rideProviderCanceldRide, "Ride Provider Canceld Ride Status can only be set once.");
        require(!userCanceldRide, "User Canceld Ride Status can only be set once.");

        rideProviderAcceptedStatus = true;
        emit UpdatePosted(msg.sender, _message, "rideProviderAcceptedStatus");
    }
\end{lstlisting}

The UpdatePosted() function works as follows: It gets called by an event function and takes the wallet of the entity calling the function, the message and the name of the event function and emits it as an event onto the chain. These events are permanent and can not be changed.

\lstset{
  basicstyle=\footnotesize\ttfamily,
  breaklines=true,
  numbers=left,
  firstnumber=165
}
\begin{lstlisting}
 event UpdatePosted(address indexed author, string message, string functionName);
\end{lstlisting}

Now that the functions that enable the general ride flow are shown, it is important to take a look at edge-cases that are handled on-chain. As described in XXX, both customer and ride provider have the ability to cancel the ride and if one of the parties cancels the ride the other party gets automatically deposited the money managed by the ride contract. The ride contract provides these features through a setUserCanceldRide() and as setRideProviderCanceldRide() function. In the flowing we will describe their functionality using the setUserCanceldRide() function as an example. If the customer calls the function to cancel their ride two if clauses check if the contract is active and therefore a ride provider has already signed the contract. If this is the case all money inside the contract gets deposited to the ride provider. If for some reason no ride provider is signing the contract the customer gets their money back. This ensures that the customer does not gets their deposit stuck inside a ride contract if a ride provider decides not sign a contract. Additionally the customer has the possibility to emit a message onto the chain that contains information on why the ride was canceled.
Lastly the contract gets marked with the status userCanceldRide = true.

\begin{lstlisting}
    function setUserCanceldRide(string memory _message) public {
        require(msg.sender == party1, "Only Party1 can set the user canceld ride status.");
        
        if(!isActive) {
            uint256 balance = address(this).balance;
            payable(party1).transfer(balance);
            return;
        }

        require(!rideProviderCanceldRide, "Ride Provider Canceld Ride Status can only be set once.");
        require(!userCanceldRide, "User Canceld Ride Status can only be set once.");

        userCanceldRide = true;
        
        if(isActive) {
            uint256 balance = address(this).balance;
            payable(party2).transfer(balance);
        }
        
        emit UpdatePosted(msg.sender, _message, "userCanceldRide");
    }
\end{lstlisting}

As described in XXX, the rating is also managed via the smart contract. Both customer and ride provider can post their rating through similar functions. The customer can use the setRideRating() function for this process that takes the selected rating as input.
The function ensures, besides other checks, that the rating is in the predefined range of allowed ratings. 

\lstset{
  basicstyle=\footnotesize\ttfamily,
  breaklines=true,
  numbers=left,
  firstnumber=375
}
\begin{lstlisting}
  function setRideRating(uint _rating) public {
        require(msg.sender == party1, "Only Party1 can set the ride rating.");
        require(!isRideRatingSet, "Ride rating can only be set once.");
        require(_rating >= 0 && _rating <= 5, "Rating must be between 0 and 5.");
        require(isActive, "Contract is not active.");
        rideRating = _rating;
        isRideRatingSet = true;
    }
\end{lstlisting}

It is also possible for the customer to rate passengers. Therefore following function is used by the ride providers to add passengers to the ride contract. This will allow customers to rate their passengers. For privacy reasons the customer will only see the pseudonyms known to the ride provider for the passengers. The user can then rate the passengers based on their seating position inside the vehicle and the starting time of their ride. Through this system it is possible for customers sharing a vehicle to rate each other without the need to share personal information like a name or a profile picture. 

\lstset{
  basicstyle=\footnotesize\ttfamily,
  breaklines=true,
  numbers=left,
  firstnumber=375
}
\begin{lstlisting}
    function addPassenger(string memory _passengerID, uint _seatingPosition, string memory _startTime) public {
        require(isActive, "Contract is not active.");
        require(msg.sender == party2, "Only Party2 can add passengers.");

        Passenger memory newPassenger = Passenger({
            passengerID: _passengerID,
            seatingPosition: _seatingPosition,
            startTime: _startTime,
            rating: 5
        });

        passengers.push(newPassenger);
    }
\end{lstlisting}

The claimETH() function is used by the ride provider to get the applicable amount of money to cover the ride cost from the smart contract. The function can only be triggered once the customer has marked the ride as successfully completed. After all checks are successfully completed the contact sends 10\% of the total amount of money on hold inside the contract to a predefined address managed by the GETACAR foundation as described in XXX. Afterwards the remaining money gets deducted by the amount of money that is claimed by the ride provider to cover the ride cost. This amount is then transfered to the wallet of the ride provider. The remaining deposit left inside the smart contract then gets send back to the wallet of the customer. 

\lstset{
  basicstyle=\footnotesize\ttfamily,
  breaklines=true,
  numbers=left,
  firstnumber=375
}
\begin{lstlisting}
    function claimETH(uint256 amount) public {
        require(isActive, "Contract is not active.");
        require(msg.sender == party2, "Only Party2 can claim the deposited ETH.");
        require(userMarkedRideComplete, "User must mark the ride complete before claiming the deposited ETH.");
        require(amount <= address(this).balance, "Requested amount exceeds the contract balance.");
        
        address payable hardcodedAddress = payable(0xE39a3085CB78341547F30a1C6bD12977d51aa967);  // Address of the GETACAR Foundation

        uint256 balance = address(this).balance;
        uint256 tenPercent = balance / 10;
        uint256 remainder = balance - tenPercent;

        hardcodedAddress.transfer(tenPercent);

        uint256 payback = remainder - amount;
        remainder -= payback;

        payable(party1).transfer(payback);
        payable(party2).transfer(remainder);
    }
\end{lstlisting}

Conclusion XXX


\subsection{Matching Contract}

The Matching Contract provides the fully on-Chain rating and load balancing for the off-chain matching services as described in XXX. At its core the function provided by the Matching contract is simple: A customer can provide an array of matching services and a minimum rating and the smart contract returns the matching service out of this array that complies with the rating requirements and has the lowest number of handled matches so far. This allows for a load balancing between the matching services so that no service can collect too much data.

All matching services that are available to the GETACAR platform are registered inside the Matching contract with a struct that contains the name of the service, the number of matches that where handled by the matching service (that resulted in completed rides) and the number of requests that where managed by the service. To ensure that the request counter is accurate, the contract counts a request every time a specific matching service gets suggested by the Matching Contract, as the design assumes that the customer will follow the suggestion and and post their ride request to the suggested matching service. For example: A customer provides an array containing the names of matching service A, B and C without defining a minimum rating. The matching contract determines that matching service A has had the lowest amount of requests out of the three so far and therefor suggests to use matching service A to the user. This suggestion is counted as a request and the request counter of matching service A is increased by one.

\lstset{
  basicstyle=\footnotesize\ttfamily,
  breaklines=true,
  numbers=left,
  firstnumber=375
}
\begin{lstlisting}
    struct MatchingServiceObject {
        string name;
        uint256 matches;
        uint256 requests;
    }
\end{lstlisting}

The actual recommendation function looks as follows: 

\lstset{
  basicstyle=\footnotesize\ttfamily,
  breaklines=true,
  numbers=left,
  firstnumber=375
}
\begin{lstlisting}
    function getMatchingService(string[] memory names) public {
        uint256 lowestMatches = type(uint256).max;

        string memory lowestMatchServiceName = "";
        uint256 lowestMatchServiceRating;

        for (uint i = 0; i < names.length; i++) {
            for (uint j = 0; j < services.length; j++) {
                if (keccak256(bytes(services[j].name)) == keccak256(bytes(names[i]))) {
                    if (services[j].matches < lowestMatches) {
                        lowestMatches = services[j].matches;
                        lowestMatchServiceName = services[j].name;
                        lowestMatchServiceRating = (services[j].matches * 100) / services[j].requests; // Multiply by 100 for two decimal places
                        services[j].requests += 1;
                    }
                }
            }
        }
        // Emit the event with the result
        emit LowestMatchService(lowestMatchServiceName, lowestMatchServiceRating);
    }
\end{lstlisting}


After explaining the request counter it is important to take a look at how the match counter works as it is equally necessary for calculate the rating of the matching service. What makes the calculation of this value more complicated is the fact that the contract needs to verify that only rides are counted that where officially handled by the GETACAR platform. If this is not the case it would open up doors for rating manipulations through external parties. In order to prevent that the addMatch() function is not called by a user that has successfully completed a ride but can only be called by the ride contracts themselves. The function utilises a for loop that checks a list that contains all contracts that are created by the official contract factory. If the contract that calls the function is on the list, the value of the utilised matching service is increased by one.

\lstset{
  basicstyle=\footnotesize\ttfamily,
  breaklines=true,
  numbers=left,
  firstnumber=375
}
\begin{lstlisting}
    function addMatch(string memory serviceName) external onlyRegisteredContracts {
        for (uint i = 0; i < services.length; i++) {
            if (keccak256(bytes(services[i].name)) == keccak256(bytes(serviceName))) {
                services[i].matches += 1;
            }
        }
    }
\end{lstlisting}

To enforce that only the contract factory can add ride contract addresses to the list of verified addresses a onlyFactory() modifier is implemented. Similarly a onlyRegisteredContracts() modifier is used to ensure that only contracts from the list of verified contracts can add successful matches to the rating services.

\lstset{
  basicstyle=\footnotesize\ttfamily,
  breaklines=true,
  numbers=left,
  firstnumber=375
}
\begin{lstlisting}
    modifier onlyFactory() {
        require(msg.sender == FACTORY_ADDRESS, "Only the factory can call this");
        _;
    }

    modifier onlyRegisteredContracts() {
        require(registeredContracts[msg.sender], "Only registered contracts can call this");
        _;
    }
\end{lstlisting}








\section{Customer Frontend and Virtual Vehicle}\label{sec:Frontend}
\begin{figure}[h]
    \centering
    
    \begin{minipage}{0.45\linewidth}
        \centering
        \includegraphics[width=\linewidth]{data/ffss/1.png}
        \caption{Frontend: Welcome Screen}
        \label{fig:WelcomeScreen}
    \end{minipage}
    \hfill
    \begin{minipage}{0.45\linewidth}
        \centering
        \includegraphics[width=\linewidth]{data/ffss/2.png}
        \caption{Frontend: Map Screen}
        \label{fig:MapScreen}
    \end{minipage}
    
\end{figure}



\begin{figure}[h]
    \centering
    
    \begin{minipage}{0.45\linewidth}
        \centering
        \includegraphics[width=\linewidth]{data/ffss/3.png}
        \caption{Frontend: Map Trip Screen}
        \label{fig:MapTripScreen}
    \end{minipage}
    \hfill
    \begin{minipage}{0.45\linewidth}
        \centering
        \includegraphics[width=\linewidth]{data/ffss/4.png}
        \caption{Frontend: Search Ride Screen}
        \label{fig:SearchRideScreen}
    \end{minipage}
    
\end{figure}


\begin{figure}[h]
    \centering
    
    \begin{minipage}{0.45\linewidth}
        \centering
        \includegraphics[width=\linewidth]{data/ffss/5.png}
        \caption{Frontend: Ride Overview Screen}
        \label{fig:FWS1}
    \end{minipage}
    \hfill
    \begin{minipage}{0.45\linewidth}
        \centering
        \includegraphics[width=\linewidth]{data/ffss/6.png}
        \caption{Frontend: Awaiting Confirmation Screen}
        \label{fig:AwaitingConfirmationScreen}
    \end{minipage}
    
\end{figure}



\begin{figure}[h]
    \centering
    
    \begin{minipage}{0.45\linewidth}
        \centering
        \includegraphics[width=\linewidth]{data/ffss/7.png}
        \caption{Frontend: Pickup Location Drive Screen}
        \label{fig:PickupLocationDriveScreen}
    \end{minipage}
    \hfill
    \begin{minipage}{0.45\linewidth}
        \centering
        \includegraphics[width=\linewidth]{data/ffss/8.png}
        \caption{Frontend: Vehicle Arrived Screen}
        \label{fig:VehicleArrivedScreen}
    \end{minipage}
    
\end{figure}



\begin{figure}[h]
    \centering
    
    \begin{minipage}{0.45\linewidth}
        \centering
        \includegraphics[width=\linewidth]{data/ffss/9.png}
        \caption{Frontend: Driving Screen}
        \label{fig:DrivingScreen}
    \end{minipage}
    \hfill
    \begin{minipage}{0.45\linewidth}
        \centering
        \includegraphics[width=\linewidth]{data/ffss/10.png}
        \caption{Frontend: Destination Screen}
        \label{fig:DestinationScreen}
    \end{minipage}
    
\end{figure}



\begin{figure}[h]
    \centering
    
    \begin{minipage}{0.45\linewidth}
        \centering
        \includegraphics[width=\linewidth]{data/ffss/11.png}
        \caption{Frontend: Rate Ride Provider Screen}
        \label{fig:RateRideProviderScreen}
    \end{minipage}
    \hfill
    \begin{minipage}{0.45\linewidth}
        \centering
        \includegraphics[width=\linewidth]{data/ffss/12.png}
        \caption{Frontend: Rate Passenger Screen}
        \label{fig:RatePassengerScreen}
    \end{minipage}
    
\end{figure}


\begin{figure}[h]
    \centering
    \includegraphics[width=0.4\linewidth]{data/ffss/13.png}
    \caption{Frontend: Settings Screen}
    \label{fig:SettingsScreen}
\end{figure}
\section{Matching Service}
After showcasing the implementation of the smart contracts and the customer and ride provider frontend, the last component that is part of the prototype implementation of the GETACAR platform is the Matching Service. The matching service is written in Node.js\footnote{https://nodejs.org/en} and the bids are stored inside MongoDB\footnote{https://www.mongodb.com} as a NoSQL database. The design decision of utilising a NoSQL database in connection with the matching service was made for several reasons. NoSQL databases provide high performance and simplify the storage of the ride requests and ride bids that are stored as simple JSON objects. Lastly, there is also no need for complex SQL functions inside the matching service that would promote the usage of an SQL database.


\subsection{Endpoints and Data}
The Node.js application provides four core endpoints for customers and ride providers to handle all interactions with the matching service. Following ride flow, the first endpoint that is typically utilised is the POST \texttt{/requestRide} endpoint. The customer frontend utilises this endpoint to post ride requests to the matching service. The request contains the following data points:

\begin{description}
    \item[userId:] The customer pseudonym provided by an authentication service
    \item[pickupLocation:] The cloaked pickup location
    \item[dropoffLocation:] A cloaked dropoff location
    \item[userRating:] The rating of the customer
    \item[rating:] The rating of the customer requesting the ride
    \item[userPublicKey:] A newly generated public key from the customer used for the Diffie-Hellman Key Exchange
    \item[maxWaitingTime:] The maximum time the customer is willing to wait for the arrival of the ride provider
    \item[minRating:] The minimum rating necessary for a ride provider to have to be allowed to manage the ride
    \item[minPassengerRating:] The minimum rating for passengers to have to be allowed to share the ride with the customer
    \item[maxPassengers:] The maximum amount of passengers the customer is willing to have at once
\end{description}


The Matching Service adds the following data points to the ride request and writes them onto the database:

\begin{description}
    \item[rideRequestId:] A unique id to identify the ride request
    \item[gridLocation:] The grid square from where the ride request came from. This makes it easier for ride provider to find fitting ride requests if a matching service is deployed for a number of different grid squares.
    \item[auctionStartedTimestamp:] The timestamp represents the moment the data was posted onto the matching service visible for ride providers to bid on the ride request.
    \item[auctionStatus:] The status of the auction. The auction can have one of four different statuses: 'open', 'determining-winner', 'waiting-for-signature' or 'closed'.
    \item[auctionWinner:] The winner of the auction. If the winner was not determined yet, this field is empty.
    \item[winningBid:] The unique identifier of the winning bid. If the winner was not determined yet, this field is empty.
    \item[p:] The prime number used for Diffie-Hellman key exchange.
    \item[g:] The base used for Diffie-Hellman key exchange.
\end{description}


Once the complete ride request is available in the database, it can be read by ride providers. Ride providers can use the GET \texttt{/rideRequests} endpoint to receive a JSON object containing all ride requests with open auctions. The ride provider can search this dataset and, once they find a fitting ride request, bid on it.
To bid on a ride request, the ride provider can utilise the POST \texttt{/bid} endpoint. A bid request contains the following data points:

\begin{description}
    \item[rideRequestId:] The id of the ride request that this bid is for
    \item[rideProviderId:] The ride provider pseudonym provided by an authentication service
    \item[amount:] The maximum ride cost that the ride provider is willing to offer the ride for
    \item[rating:] The rating of the ride provider
    \item[model:] The model of the vehicle
    \item[estimatedArrivalTime:] The time to get to the customer pickup location
    \item[passengerCount:] The number of passengers inside the vehicle when arriving at the pickup location
    \item[vehiclePublicKey:] A newly generated public key from the ride provider used for the Diffie-Hellman Key Exchange 
\end{description}

Before a bid can be posted to the database, the Matching Service compares the timestamp of the bid with the timestamp of the ride request the bid is associated with to ensure that the auction is truly open. For this prototype, the time frame for this is set to 30 seconds. The bid also gets extended with additional data points by the matching service itself before it gets written onto the database. These are the additional data points:

\begin{description}
    \item[bidId:] A unique id to identify the bid
    \item[bidPlacedTimestamp:] Timestamp representing the moment the bid is posted
\end{description}

With ride requests and associated bids being written to the database, the next step for the matching service is handling the running auction. Each auction is posted with the status ''open''. A function managed by the Matching Service continuously crawls the database for ride requests where the auction is older than 30 seconds. If such an auction is found, the auction status changes from ''open'' to ''determining-winner''. A second function then takes the ride request and analyses all bids connected with the request to determine the winning bid based on the principles of the second price auction. The winning bid is then written into the ride request itself, changing the auction status to ''waiting-for-signature''. 

The customer is able to check the status of their auction through the GET \texttt{/rideRequest/:rideRequestId} endpoint by providing the identifier of their ride request inside the URL. This endpoint returns the status of the auction, and in case a winning bid is found, it additionally returns the winning bid itself. Based on the winning bid, the customer can then decide if they want to take the ride or not. If the customer decides to take the ride, they follow the ride flow and create a ride contract through the contract factory that contains the maximum ride cost as a deposit. They then need to use the GET \texttt{setContractAddress/:rideRequestId/:contractAddress} endpoint to update their ride request with the contact's address on the blockchain. As the creation of the contract is understood as the initial signing of the ride contract, the status of the auction changes from 'waiting-for-signature' to 'closed'. 

The GET \texttt{/rideRequest/:rideRequestId} endpoint also enables the ride provider to track the status of the auction. Through the endpoint, the auction winner is able to receive the address of the ride contract and is, therefore, able to co-sign the contract as a ride provider.

\subsection{Grid System}
As described in \ref{subsec:MatchingService}, the design of the Matching Service includes a map grid that is utilised to assign matching services to specific jurisdiction zones and to cloak the exact pickup and dropoff locations of customers. There are many possible ways to create a map grid that would allow for this use case. The GETACAR prototype implementation uses H3\footnote{https://h3geo.org}, a hexagonal hierarchical geospatial indexing system that provides a predefined grid for the platform to use. The advantage of H3 is that it provides a number of grid resolutions, as each grid is made up of hexagons and pentagons that themself are made up of smaller hexagons and pentagons, as seen in figure \ref{fig:H3Visualisation}. An algorithm allows one to easily check if a hexagon/ pentagon of a smaller resolution is contained within a hexagon/ pentagon of a higher resolution. This approach would allow GETACAR to utilise dynamic resolutions for the  jurisdiction zones and the location cloaking with higher resolutions used for crowded areas with high traffic, like cities and lower resolutions that cover larger, less crowded  areas with less traffic.~\cite{H3Geo.} The prototype uses fixed resolutions with Res 9 for the location cloaking and Res 6 for the  jurisdiction zones of the matching services. The available resolutions are displayed in table \ref{tab:resolutions}.

\begin{figure}[h]
    \centering
    \includegraphics[width=\linewidth]{data/11.png}
    \caption{H3 Grid Visualisation ~\cite{H3Geo.}}
    \label{fig:H3Visualisation}
\end{figure}

\begin{table}[h]
\centering
\begin{tabular}{|c|r|r|r|}
\hline
\textbf{Res} & \textbf{Average Hexagon Area (km$^2$)} & \textbf{Pentagon Area (km$^2$)} & \textbf{Ratio (P/H)} \\
\hline
0 & 4,357,449.416078381 & 2,562,182.162955496 & 0.5880 \\
1 & 609,788.441794133 & 328,434.586246469 & 0.5386 \\
2 & 86,801.780398997 & 44,930.898497879 & 0.5176 \\
3 & 12,393.434655088 & 6,315.472267516 & 0.5096 \\
4 & 1,770.347654491 & 896.582383141 & 0.5064 \\
5 & 252.903858182 & 127.785583023 & 0.5053 \\
6 & 36.129062164 & 18.238749548 & 0.5048 \\
7 & 5.161293360 & 2.604669397 & 0.5047 \\
8 & 0.737327598 & 0.372048038 & 0.5046 \\
9 & 0.105332513 & 0.053147195 & 0.5046 \\
10 & 0.015047502 & 0.007592318 & 0.5046 \\
11 & 0.002149643 & 0.001084609 & 0.5046 \\
12 & 0.000307092 & 0.000154944 & 0.5046 \\
13 & 0.000043870 & 0.000022135 & 0.5046 \\
14 & 0.000006267 & 0.000003162 & 0.5046 \\
15 & 0.000000895 & 0.000000452 & 0.5046 \\
\hline
\end{tabular}
\caption{H3 Grid Resolution ~\cite{H3Geo.}}
\label{tab:resolutions}
\end{table}

\label{sec:Matching Service}

\chapter{Evaluation}
After designing the GETACAR platform and developing a prototype based on this design it is important to evaluate the platform to prove its viability. To do so we will validate the GETACAR platform against the objects defined at the beginning of this research in section \ref{sec:objectives}. Afterwards, we will conduct a final privacy assessment of the platform and the prototype to ensure that we meet all privacy goals. At last, we will showcase the results from testing the completed GETACAR prototype. 

\section{Validation against Research Objectives}
\textbf{Prototypical Realization of the Decentralized Platform}

!!!! Privacy tabelle von paper x nutzen 

\section{Privacy Considerations}
Insuring privacy throughout the platform is one of the most important aspects of GETACAR. While we showed the general spread of user information across the components of the platform, as shown in \ref{tab:CustomerDataPrivacyMatrix}, it is important to put the platform through a privacy assessment. This assessment is meant to ensure that the privacy design at its core does not contain any loopholes that could endanger user privacy. OMAR et al. provide such an assessment for an anonymity-oriented privacy-preserving reputation system, as it is implemented into GETACAR ~\cite{HasanOmar}.

GETACAR fulfils the requirements for this assessment as the true identity of users are hidden on the platform, interactions stay anonymous, users are represented by multiple pseudonyms, and transactions can get carried out anonymously. 

The paper itself focuses on the privacy preservation of rating systems but the assessment itself works on a platform level, as the same systems that preserve privacy for users when they interact with the rating systems are also in place for all other interactions on GETACAR. Table \ref{tab:privacyAssessment} shows all twelve  points of assessment and how the GETACAR platform is evaluated for each.


\label{tab:privacyAssessment}
\begin{longtable}{|p{3cm}|p{4.5cm}|p{5cm}|p{1.5cm}|}
\caption{User Anonymity-Oriented Privacy-Preserving Reputation System Properties ~\cite{HasanOmar}} \\
\hline
\textbf{Property} & \textbf{Description} & \textbf{Evaluation} & \textbf{Fulfilled} \\
\hline
\endfirsthead

\multicolumn{4}{c}%
{{\bfseries \tablename\ \thetable{} -- continued from previous page}} \\
\hline
\textbf{Property} & \textbf{Description} & \textbf{Evaluation} & \textbf{Fulfilled} \\
\hline
\endhead

\hline \multicolumn{4}{|r|}{{Continued on next page}} \\
\hline
\endfoot

\hline
\endlastfoot

Multiple Pseudonyms & A user can assume multiple pseudonyms, either per context or per transaction. & Every user can take on a new pseudonym for each new transaction. For off-chain interactions, the authentication service provides the pseudonyms directly, for on-chain transactions the user is able to generate their own new wallet, which then gets registered with the authentication service. & Yes \\
\hline
User-Pseudonym Unlinkability & The true identity of a user is not linkable to any pseudonym they use. & By knowing the identity of a user, it is not possible to identify pseudonyms that belong to the user as there is no information contained in the pseudonym that would allow to make this connection. & Yes\\
\hline
Pseudonym-Pseudonym Unlinkability & Two different pseudonyms of the same user cannot be linked. & The pseudonyms are not linked directly to each other. Therefore it is not possible to conclude which pseudonyms belong to the same user. & Yes\\
\hline
Rater Anonymity & A user can rate another user without revealing their true identity. & On GETACAR the rater stays anonymous as they use a newly generated wallet as their pseudonym for the ride flow and to post their rating on the blockchain & Yes \\
\hline
Ratee Anonymity & A user can receive a rating without their real identity being disclosed. & GETACAR also provides Ratee anonymity, as users only rate other users based on their wallet pseudonyms on-chain.& Yes\\
\hline
Inquirer Anonymity & A user can inquire about another user's reputation anonymously. & Every user can request the rating of another user without exposing their identity. To get a rating it is only necessary to provide the pseudonym of the user to the authentication service. The authentication service can then return the rating of the user connected to the pseudonym. & Yes\\
\hline
Reputation Transfer and Aggregation & A ratee can transfer and aggregate reputation among their pseudonyms. & As each pseudonym is connected to a single user by the authentication service, the rating transfers between all pseudonyms & Yes\\
\hline
Reputation Unforgeability & A ratee cannot show reputation higher than their pseudonyms' cumulative reputation. & Reputation forgeability is not possible as a user does not provide their rating themself but through the authentication service that represents a trusted authority & Yes\\
\hline
Distinctness & Reputation of a ratee is an aggregate of votes from distinct raters. & As all ratings are posted to a smart contract on the blockchain that logs the rater and ratee and verifies that both parties are part of the platform distinctness of ratings is ensured. & Yes\\
\hline
Accountability & Users are accountable for adversarial actions. & The rating systems keep users accountable for their actions and allow for the revelation of the identity of a user in edge cases.& Yes\\
\hline
Authorizability of Ratings & Only users who have had a transaction with the ratee are allowed to rate her. & The smart contracts ensure that all ratings are valid, as customers and ride providers can only rate each other after signing a ride contract on-chain.& Yes\\
\hline
Verifiability by Ratee & A ratee should be able to identify all published feedback linked to their identity and verify their authenticity. &As each user knows all their pseudonyms themselves, they can calculate their rating on their on to verify the calculations of the authentication service.  & Yes\\
\hline
\end{longtable}

As seen in the table \ref{tab:privacyAssessment} GETACAR is able to satisfy all properties of the assessment. While the fulfilment of some of these points relies on the implementation of the authentication service, which is not completely realised in this paper, this is still a very notable achievement, as the paper ~\cite{HasanOmar} does not identify a single source out of 26 analysed research papers, that fulfils all of the properties of the assessment.  
\section{Testing and Results}
\begin{table}[h]
\centering
\begin{tabular}{|l|c|}
\hline
\textbf{Transaction} & \textbf{Gas Cost} \\
\hline
Determining best Matching Service for current Hexagon/Pentagon & 734.18 \\
\hline
Creation of Ride Contract on-chain from the Contract Factory& 47915.23 \\
\hline
Posting the ''Start driving'' Event& 391.55 \\
\hline
Posting the ''Ride completed'' Event & 683.45 \\
\hline
Rating Ride Provider & 802.55 \\
\hline
Rating one Passenger & 577.86 \\
\hline
\hline
\textbf{Sum}  & \textbf{51104.82} \\
\hline
\end{tabular}
\caption{Gas consumption for the Ride Customer transactions}
\end{table}

Test123

\begin{table}
\centering
\begin{tabular}{|l|c|}
\hline
\textbf{Blockchain} & \textbf{Price per Unit of Gas in Gwei}\\
\hline
Ethereum & 11 \\
\hline
BNB Smart Chain & 3 \\
\hline
Polygon & 90 \\
\hline
Avalanche & 25 \\
\hline
\end{tabular}
\caption{Gas price of different Blockchains}
\end{table}


\chapter{Conclusion \& Outlook}
With an expected increase in overall traffic in the near future caused by the widespread adaptation of autonomous vehicles, ride-pooling solutions are needed to decrease the number of individual trips on the road. As current ride-pooling platforms have shown to be insufficient for tackling this problem because of their centralised nature and how they deal with data privacy, this paper introduces GETACAR, a decentralised, privacy-preserving ride-pooling platform. The design of the GETACAR ride-pooling platform is based on a comprehensive literature analysis that outlines the current state and shortcomings of decentralised ride-pooling platforms in scientific research. The design of the GETACAR platform combines the findings from the research analysis with best practices from the industry and contributes its own design suggestions to the discourse. The design of the GETACAR platform demonstrates how blockchain can be used to track rides,  how the interaction flow between customers and ride providers is supposed to look like and how payments and service fees can be managed throughout the platform. Additionally, the design covers concepts that ensure privacy and anonymity for all users as well as decentralised trust mechanisms. GETACAR also describes how security is ensured throughout the platform and how edge cases can be handled off-chain. 

The design of GETACAR is verified through the creation of a prototype that showcases all relevant features of the platform and allows a simulation of the complete ride flow between the customer, platform and ride provider. The prototype proves the feasibility of the GETACAR platform design and decentralised, privacy-preserving ride-pooling in general. 

Therefore, the findings gained in this work can be used as a basis for future research. The design can be used as a blueprint to create market-ready,  privacy-preserving ride-pooling platforms that favour user interests over corporate gains. We hope that GETACAR contributes to the research landscape of decentralised ride-pooling and, therefore, helps to solve the problem of increased traffic caused by autonomous vehicles.
\section{Limitations}
While the GETACAR platform design provides an in-depth overview of all relevant components necessary to create a decentralised privacy-preserving ride-pooling platform, it is essential to note that the creation of such a platform is highly complex, and the area still provides room for improvement.
First of all, smart contracts, the platform's backbone, are written highly verbose to better illustrate the key functions of the contracts. Optimising the contracts will result in lower Gas fees, which decrease the operation cost of the platform  overall. Secondly, while the platform is built with decentralised authentication services in mind, the prototype does not include the component. There is also no mash of crypto exchanges connected with the prototype, allowing customers and ride providers to use fiat currency on the platform. These factors currently limit the proof of feasibility.
While the on-chain interaction flow is worked out in detail, there is no predefined communication protocol for the encrypted interaction between customer and ride provider, enabled through the Diffie-Hellman Key Exchange. The technical implementation for exchanging encrypted messages is fully realised, but no standardised format for these messages has been defined yet.
Lastly, it is important to note that the privacy and security aspects are verified on an architecture and general platform design level. Professional penetration testing of the prototype is needed to fully verify these aspects of the platform. 
\section{Outlook}
By providing a fully developed platform design and a functional prototype, GETACAR lends itself to future research in the area of decentralized ride-pooling.
A point of interest for future research should be the continuation of the quantitative and qualitative validation of the GETACAR platform design. The quantitative evaluation can be done by creating complex simulations on top of the platform. The simulations should be able to mimic customer and ride provider behaviour at a large scale to evaluate if the platform is able to process these vast amounts of data without running into bottlenecks. While the design of the platform, customer and ride provider flow is heavily based on the findings from the scientific literature, it should be further verified through qualitative testing. Therefore the platform prototype should be given out to potential customers and ride-providers for real world testing and further refinement of the platform.

Further verifying the design of the GETACAR platform is important for the creation of a successful, market-ready decentralised ride-pooling platform but there are also other research topics related to the creation of such a platform that have not been covered in the paper. This includes a detailed legal analysis on how the GETACAR Foundation should be set up and how the foundation should handle tasks like assigning the matching services to the world grid and how the foundations process should look like for verifying new authentication services.
While the basic economic feasibility of the platform is ensured by this paper, a detailed research on this topic is needed, to develop a detailed business case for GETACAR.

\printbibliography

All links were last followed on September 21, 2023.

\appendix

\definecolor{verylightgray}{rgb}{.97,.97,.97}

\lstdefinelanguage{Solidity}{
	keywords=[1]{anonymous, assembly, assert, balance, break, call, callcode, case, catch, class, constant, continue, constructor, contract, debugger, default, delegatecall, delete, do, else, emit, event, experimental, export, external, false, finally, for, function, gas, if, implements, import, in, indexed, instanceof, interface, internal, is, length, library, log0, log1, log2, log3, log4, memory, modifier, new, payable, pragma, private, protected, public, pure, push, require, return, returns, revert, selfdestruct, send, solidity, storage, struct, suicide, super, switch, then, this, throw, transfer, true, try, typeof, using, value, view, while, with, addmod, ecrecover, keccak256, mulmod, ripemd160, sha256, sha3}, % generic keywords including crypto operations
	keywordstyle=[1]\color{blue}\bfseries,
	keywords=[2]{address, bool, byte, bytes, bytes1, bytes2, bytes3, bytes4, bytes5, bytes6, bytes7, bytes8, bytes9, bytes10, bytes11, bytes12, bytes13, bytes14, bytes15, bytes16, bytes17, bytes18, bytes19, bytes20, bytes21, bytes22, bytes23, bytes24, bytes25, bytes26, bytes27, bytes28, bytes29, bytes30, bytes31, bytes32, enum, int, int8, int16, int24, int32, int40, int48, int56, int64, int72, int80, int88, int96, int104, int112, int120, int128, int136, int144, int152, int160, int168, int176, int184, int192, int200, int208, int216, int224, int232, int240, int248, int256, mapping, string, uint, uint8, uint16, uint24, uint32, uint40, uint48, uint56, uint64, uint72, uint80, uint88, uint96, uint104, uint112, uint120, uint128, uint136, uint144, uint152, uint160, uint168, uint176, uint184, uint192, uint200, uint208, uint216, uint224, uint232, uint240, uint248, uint256, var, void, ether, finney, szabo, wei, days, hours, minutes, seconds, weeks, years},	% types; money and time units
	keywordstyle=[2]\color{teal}\bfseries,
	keywords=[3]{block, blockhash, coinbase, difficulty, gaslimit, number, timestamp, msg, data, gas, sender, sig, value, now, tx, gasprice, origin},	% environment variables
	keywordstyle=[3]\color{violet}\bfseries,
	identifierstyle=\color{black},
	sensitive=true,
	comment=[l]{//},
	morecomment=[s]{/*}{*/},
	commentstyle=\color{gray}\ttfamily,
	stringstyle=\color{red}\ttfamily,
	morestring=[b]',
	morestring=[b]"
}

\lstset{
	language=Solidity,
	backgroundcolor=\color{verylightgray},
	extendedchars=true,
	basicstyle=\footnotesize\ttfamily,
	showstringspaces=false,
	showspaces=false,
	numbers=left,
	numberstyle=\footnotesize,
	numbersep=9pt,
	tabsize=2,
	breaklines=true,
	showtabs=false,
	captionpos=b
}




\lstset{
  basicstyle=\footnotesize\ttfamily,
  breaklines=true,
  numbers=left,
  firstnumber=1,
}

\begin{lstlisting}
// SPDX-License-Identifier: MIT
pragma solidity ^0.8.0;

contract MatchingService {

    struct MatchingServiceObject {
        string name;
        uint256 matches;
        uint256 requests;
    }


    MatchingServiceObject[5] public services;

    address public FACTORY_ADDRESS;
    bool public isFactoryAddressSet = false;

    mapping(address => bool) public registeredContracts;
    address[] public registeredContractsList; 

    // Declare the event
    event LowestMatchService(string serviceName, uint256 serviceRating);

    modifier onlyFactory() {
        require(msg.sender == FACTORY_ADDRESS, "Only the factory can call this");
        _;
    }

    modifier onlyRegisteredContracts() {
        require(registeredContracts[msg.sender], "Only registered contracts can call this");
        _;
    }

    //Hardcoded Dummy Matching Services 
    constructor() {
        services[0] = MatchingServiceObject("ms1", 10, 15);
        services[1] = MatchingServiceObject("ms2", 15, 20);
        services[2] = MatchingServiceObject("ms3", 20, 30);
        services[3] = MatchingServiceObject("ms4", 5, 10);
        services[4] = MatchingServiceObject("ms5", 8, 12);
    }

    function setFactoryAddress(address _factoryAddress) external {
        require(!isFactoryAddressSet, "Factory address is already set");
        FACTORY_ADDRESS = _factoryAddress;
        isFactoryAddressSet = true;
    }

    function registerContract(address contractAddress) external onlyFactory {
        require(!registeredContracts[contractAddress], "Contract is already registered"); // Additional check to prevent duplicate addresses
        registeredContracts[contractAddress] = true;
        registeredContractsList.push(contractAddress); 
    }

    function getAllRegisteredContracts() external view returns (address[] memory) {
        return registeredContractsList;
    }

    function getMatchingService(string[] memory names) public {
        uint256 lowestMatches = type(uint256).max;

        string memory lowestMatchServiceName = "";
        uint256 lowestMatchServiceRating;

        for (uint i = 0; i < names.length; i++) {
            for (uint j = 0; j < services.length; j++) {
                if (keccak256(bytes(services[j].name)) == keccak256(bytes(names[i]))) {
                    if (services[j].matches < lowestMatches) {
                        lowestMatches = services[j].matches;
                        lowestMatchServiceName = services[j].name;
                        lowestMatchServiceRating = (services[j].matches * 100) / services[j].requests; // Multiply by 100 for two decimal places
                        services[j].requests += 1;
                    }
                }
            }
        }
        // Emit the event with the result
        emit LowestMatchService(lowestMatchServiceName, lowestMatchServiceRating);
    }

    function addMatch(string memory serviceName) external onlyRegisteredContracts {
        for (uint i = 0; i < services.length; i++) {
            if (keccak256(bytes(services[i].name)) == keccak256(bytes(serviceName))) {
                services[i].matches += 1;
            }
        }
    }
}
\end{lstlisting}


\lstset{
  basicstyle=\footnotesize\ttfamily,
  breaklines=true,
  numbers=left,
  firstnumber=1,
}

\begin{lstlisting}
// SPDX-License-Identifier: MIT
pragma solidity ^0.8.0;

import "./contract.sol";
import "./matching.sol";  // Assuming both contracts are in the same directory

contract ContractFactory {

    MatchingService private matchingServiceInstance;
    address[] public registeredContracts;

    uint256 public contractCounter = 0;  // Counter to keep track of contract IDs

    // Mapping from contract ID to contract address
    mapping(uint256 => address) public contractsByID;

    // Mapping from contract ID to contract timestamp
    mapping(uint256 => uint256) public timestampByID;


    constructor(address _matchingServiceAddress) {
        matchingServiceInstance = MatchingService(_matchingServiceAddress);

        // Set this contract as the factory address in the MatchingService contract
        matchingServiceInstance.setFactoryAddress(address(this));
    }

    mapping(address => Contract[]) public userContracts;
    event ContractCreated(address indexed user, Contract newContract, uint256 contractID);  // Added contractID to the event

    function registerNewContract(address _contractAddress) external {
        // Call the registerContract() function on the MatchingService contract
        matchingServiceInstance.registerContract(_contractAddress);

        // Optionally, store the registered contract's address in this factory for record-keeping
        registeredContracts.push(_contractAddress);
    }

    function createContract(uint256 _amount) public payable {
        require(msg.value == _amount, "Sent value does not match the specified amount.");
        Contract newContract = new Contract{value: _amount}(msg.sender);
        userContracts[msg.sender].push(newContract);

        // Increment contract counter and map new contract's address to the counter
        contractCounter++;
        contractsByID[contractCounter] = address(newContract);
        
        // Store the current block's timestamp
        timestampByID[contractCounter] = block.timestamp;

        // Call registerNewContract with the new contract's address
        this.registerNewContract(address(newContract));

        emit ContractCreated(msg.sender, newContract, contractCounter);
    }

    function getContractsByUser(address user) public view returns (Contract[] memory) {
        return userContracts[user];
    }

    function getContractByID(uint256 contractID) public view returns (address) {
        return contractsByID[contractID];
    }

    // Fetch the timestamp by contract ID
    function getContractTimestampByID(uint256 contractID) public view returns (uint256) {
        return timestampByID[contractID];
    }
}

\end{lstlisting}


\lstset{
  basicstyle=\footnotesize\ttfamily,
  breaklines=true,
  numbers=left,
  firstnumber=1,
}

\begin{lstlisting}
// SPDX-License-Identifier: MIT
pragma solidity ^0.8.0;

interface IMatchingService {
    function addMatch(string memory serviceName) external;
}

contract Contract {
    address public party1;
    address public party2;
    bool public isActive;
    bool public rideProviderAcceptedStatus;
    bool public rideProviderArrivedAtPickupLocation;
    bool public userReadyToStartRide;
    bool public rideProviderStartedRide;
    bool public rideProviderArrivedAtDropoffLocation;
    bool public userMarkedRideComplete;
    bool public userCanceldRide;
    bool public rideProviderCanceldRide;

    uint public userRating;
    uint public rideRating;
    bool public isUserRatingSet;
    bool public isRideRatingSet;

    //Hard Coded Address of the Matching Service to bump up Matching Count of Rating Service
    address constant MATCHING_SERVICE_ADDRESS = 0x0991df810C73d820c776b024Eb720d39e9CfBb1a;


    constructor(address _party1) payable {
        party1 = _party1;
        rideProviderAcceptedStatus = false;
        rideProviderArrivedAtPickupLocation = false;
        userReadyToStartRide = false;
        rideProviderStartedRide = false;
        rideProviderArrivedAtDropoffLocation = false;
        userMarkedRideComplete = false;
        userCanceldRide = false;
        rideProviderCanceldRide = false;

    }

    struct Passenger {
        string passengerID;
        uint seatingPosition;
        string startTime;
        uint rating;
    }


    Passenger[] public passengers;

    function addPassenger(string memory _passengerID, uint _seatingPosition, string memory _startTime) public {
        require(isActive, "Contract is not active.");
        require(msg.sender == party2, "Only Party2 can add passengers.");

        Passenger memory newPassenger = Passenger({
            passengerID: _passengerID,
            seatingPosition: _seatingPosition,
            startTime: _startTime,
            rating: 0
        });

        passengers.push(newPassenger);
    }

    function addPassengerRating(uint _passengerIndex, uint _rating) public {
        require(isActive, "Contract is not active.");
        require(msg.sender == party1, "Only Party1 can rate passengers.");
        require(_rating >= 0 && _rating <= 5, "Rating must be between 0 and 5.");
        require(_passengerIndex < passengers.length, "Passenger not found.");

        passengers[_passengerIndex].rating = _rating;
    }

    function signContract() public payable {
        require(party2 == address(0), "Party2 has already signed the contract.");
        require(!isActive, "Contract is already active.");
        require(!userCanceldRide, "User cannceld ride ");
        require(msg.sender != party1, "Party2 cannot be identical to Party1.");
        
        party2 = msg.sender;
        isActive = true;

        uint256 tenPercent = (address(this).balance * 10) / 100;
        require(msg.value >= tenPercent, "Party2 must deposit an amount equal to 10% of the contract balance.");

        // Refund any excess amount deposited by party2
        if (msg.value > tenPercent) {
            payable(msg.sender).transfer(msg.value - tenPercent);
        }
    }

    event UpdatePosted(address indexed author, string message, string functionName);


    function setRideProviderAcceptedStatus(string memory _message) public {
        require(isActive, "Contract is not active.");
        require(msg.sender == party2, "Only Party2 can set the ride provider accepted status.");
        require(!rideProviderAcceptedStatus, "Ride Provider Accepted Status can only be set once.");

        require(!rideProviderCanceldRide, "Ride Provider Canceld Ride Status can only be set once.");
        require(!userCanceldRide, "User Canceld Ride Status can only be set once.");

        rideProviderAcceptedStatus = true;
        emit UpdatePosted(msg.sender, _message, "rideProviderAcceptedStatus");
    }

    function setRideProviderArrivedAtPickupLocation(string memory _message) public {
        require(isActive, "Contract is not active.");
        require(msg.sender == party2, "Only Party2 can set the ride provider arrived status.");
        require(rideProviderAcceptedStatus, "Ride Provider Accepted Status must be set before setting arrived status.");
        require(!rideProviderArrivedAtPickupLocation, "Ride Provider Arrived Status can only be set once.");

        require(!rideProviderCanceldRide, "Ride Provider Canceld Ride Status can only be set once.");
        require(!userCanceldRide, "User Canceld Ride Status can only be set once.");
        
        rideProviderArrivedAtPickupLocation = true;
        emit UpdatePosted(msg.sender, _message, "rideProviderArrivedAtPickupLocation");
    }

    function setUserReadyToStartRide(string memory _message) public {
        require(isActive, "Contract is not active.");
        require(msg.sender == party1, "Only Party1 can set the user ready to start ride status.");
        require(rideProviderArrivedAtPickupLocation, "Ride Provider Arrived Status must be set before setting user ready to start ride status.");
        require(!userReadyToStartRide, "User Ready To Start Ride Status can only be set once.");

        require(!rideProviderCanceldRide, "Ride Provider Canceld Ride Status can only be set once.");
        require(!userCanceldRide, "User Canceld Ride Status can only be set once.");

        userReadyToStartRide = true;
        emit UpdatePosted(msg.sender, _message, "userReadyToStartRide");
    }

    function setRideProviderStartedRide(string memory _message) public {
        require(isActive, "Contract is not active.");
        require(msg.sender == party2, "Only Party2 can set the ride provider started ride status.");
        require(userReadyToStartRide, "User Ready To Start Ride Status must be set before setting ride provider started ride status.");
        require(!rideProviderStartedRide, "Ride Provider Started Ride Status can only be set once.");

        require(!rideProviderCanceldRide, "Ride Provider Canceld Ride Status can only be set once.");
        require(!userCanceldRide, "User Canceld Ride Status can only be set once.");

        rideProviderStartedRide = true;
        emit UpdatePosted(msg.sender, _message, "rideProviderStartedRide");
    }

    function setRideProviderArrivedAtDropoffLocation(string memory _message) public {
        require(isActive, "Contract is not active.");
        require(msg.sender == party2, "Only Party2 can set the ride provider arrived at dropoff location status.");
        require(rideProviderStartedRide, "Ride Provider Started Ride Status must be set before setting ride provider arrived at dropoff location status.");
        require(!rideProviderArrivedAtDropoffLocation, "Ride Provider Arrived At Dropoff Location Status can only be set once.");

        require(!rideProviderCanceldRide, "Ride Provider Canceld Ride Status can only be set once.");
        require(!userCanceldRide, "User Canceld Ride Status can only be set once.");

        rideProviderArrivedAtDropoffLocation = true;
        emit UpdatePosted(msg.sender, _message, "rideProviderArrivedAtDropoffLocation");
    }

    function setUserMarkedRideComplete(string memory _message) public {
        require(isActive, "Contract is not active.");
        require(msg.sender == party1, "Only Party1 can set the user marked ride complete status.");
        require(rideProviderArrivedAtDropoffLocation, "Ride Provider Arrived At Dropoff Location Status must be set before setting user marked ride complete status.");
        require(!userMarkedRideComplete, "User Marked Ride Complete Status can only be set once.");

        require(!rideProviderCanceldRide, "Ride Provider Canceld Ride Status can only be set once.");
        require(!userCanceldRide, "User Canceld Ride Status can only be set once.");

        userMarkedRideComplete = true;
        //Call Matching Service, ms1 is hardcoded. For a real implementation this value would be provided by the forntend when calling the setUserMarkedRideComplete() function
        IMatchingService(MATCHING_SERVICE_ADDRESS).addMatch("ms1");
        emit UpdatePosted(msg.sender, _message, "userMarkedRideComplete");
    }

    function setUserCanceldRide(string memory _message) public {
        require(msg.sender == party1, "Only Party1 can set the user canceld ride status.");
        
        if(!isActive) {
            uint256 balance = address(this).balance;
            payable(party1).transfer(balance);
            return;
        }

        require(!rideProviderCanceldRide, "Ride Provider Canceld Ride Status can only be set once.");
        require(!userCanceldRide, "User Canceld Ride Status can only be set once.");

        userCanceldRide = true;
        
        if(isActive) {
            uint256 balance = address(this).balance;
            payable(party2).transfer(balance);
        }
        
        emit UpdatePosted(msg.sender, _message, "userCanceldRide");
    }

    function setRideProviderCanceldRide(string memory _message) public {
        require(isActive, "Contract is not active.");
        require(msg.sender == party2, "Only Party2 can set the ride provider canceld ride status.");
        
        require(!rideProviderCanceldRide, "Ride Provider Canceld Ride Status can only be set once.");
        require(!userCanceldRide, "User Canceld Ride Status can only be set once.");

        rideProviderCanceldRide = true;

        uint256 balance = address(this).balance;
        payable(party1).transfer(balance);
        
        emit UpdatePosted(msg.sender, _message, "rideProviderCanceldRide");
    }

    function setUserRating(uint _rating) public {
        require(msg.sender == party2, "Only Party2 can set the user rating.");
        require(!isUserRatingSet, "User rating can only be set once.");
        require(isActive, "Contract is not active.");
        require(_rating >= 0 && _rating <= 5, "Rating must be between 0 and 5.");
        userRating = _rating;
        isUserRatingSet = true;
    }

    function setRideRating(uint _rating) public {
        require(msg.sender == party1, "Only Party1 can set the ride rating.");
        require(!isRideRatingSet, "Ride rating can only be set once.");
        require(_rating >= 0 && _rating <= 5, "Rating must be between 0 and 5.");
        require(isActive, "Contract is not active.");
        rideRating = _rating;
        isRideRatingSet = true;
    }


    function claimETH(uint256 amount) public {
        require(isActive, "Contract is not active.");
        require(msg.sender == party2, "Only Party2 can claim the deposited ETH.");
        require(userMarkedRideComplete, "User must mark the ride complete before claiming the deposited ETH.");
        require(amount <= address(this).balance, "Requested amount exceeds the contract balance.");
        
        address payable hardcodedAddress = payable(0xE39a3085CB78341547F30a1C6bD12977d51aa967);  // replace with the actual GETACAR Foundation address

        uint256 balance = address(this).balance;
        uint256 tenPercent = balance / 10;
        uint256 remainder = balance - tenPercent;

        hardcodedAddress.transfer(tenPercent);

        uint256 payback = remainder - amount;
        remainder -= payback;

        payable(party1).transfer(payback);
        payable(party2).transfer(remainder);
    }

}
\end{lstlisting}


\pagestyle{empty}
\renewcommand*{\chapterpagestyle}{empty}
\Versicherung
\end{document}
