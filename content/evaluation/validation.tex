It is important to validate the results of this research against our initial objectives to determine if the research is a success. The following table \ref{tab:roAssessment} lists all five research objectives that are defined in \ref{sec:objectives} with the corresponding evaluation. 


\label{tab:roAssessment}
\begin{longtable}{|p{3cm}|p{4.5cm}|p{5cm}|p{1.5cm}|}
\caption{Research Objective Assessment} \\
\hline
\textbf{Research Objectives} & \textbf{Description} & \textbf{Evaluation} & \textbf{Fulfilled} \\
\hline
\endfirsthead

\multicolumn{4}{c}%
{{\bfseries \tablename\ \thetable{} -- continued from previous page}} \\
\hline
\textbf{Research Objectives} & \textbf{Description} & \textbf{Evaluation} & \textbf{Fulfilled} \\
\hline
\endhead

\hline \multicolumn{4}{|r|}{{Continued on next page}} \\
\hline
\endfoot

\hline
\endlastfoot

Design of the Components and Interaction Flow between the Platform, Customer, and Ride
Provider & The research needs to provide a design blueprint for the decentralised ride-pooling platform. This design should communicate the general vision of the platform and explain the key concepts. At its
core, the platform is an ecosystem of components interacting with each other. Therefore it is also
necessary to design a streamlined, secure, and efficient flow for these interactions. The objective
here is to develop an interaction flow that: Ensures Seamless ride booking, facilitates trustworthy
transactions and preserves privacy. & Section \ref{sec:DesignOfThePlatform} providers a detailed overview of the platform, including a component overview and detailed descriptions of the design of the customers frontend, ride provider frontend, authentication service, matching service, the ride contracts and the integration of crypto exchanges. The interaction flow for each component is described and additional diagrams are provided. By basing the overall customer and ride provider flow on already established and tested while also utilising the advantages of blockchain technology GETACAR provides seamless ride booking, while also facilitating trustworthy transactions and preserves privacy through out the platform. & Yes\\
\hline
Design of a Trust Mechanism for Customer and Ride-Providers & Trust is a necessity for every platform but is especially relevant for decentralised platforms as these
platforms are not managed by a single owner that can single-handedly settle disputes or resolve
unexpected edge cases. Therefore it is necessary for the platform to have a robust trust system that
sanctions malicious behaviour and promotes good behaviour. & The GETACAR platform design includes a detailed design of trust mechanism, as described in \ref{sec:PrivacyAndTrustMechanism}. This includes the rating trust mechanisms that allow customers to rate their ride providers and passengers as well as allowing ride providers to rate their customers. All ratings are posted anonymously and every user of the platform can request ratings of other users based on their pseudonyms. This method allows transparent ratings throughout the platform without exposing the identities of the users. Additionally the platform utilises deposit trust mechanisms that ensure that monetary incentives are in place for all parties to act according to the predefined ride flow.   & Yes\\
\hline
Evaluation of Customer and Ride-Provider Anonymity and Privacy throughout the Platform & One of the disadvantages of blockchain-based platforms is that the high level of transparency can
result in a neglect of customer and ride-provider anonymity and privacy. This counts especially
for ride pooling platforms where large amounts of personal data like location and transaction data
get exchanged. That is why it is important to assess the platform design regarding privacy and
anonymity to show that no entity can collect critical amounts of data from the platform. & Ensuring ride-provider anonymity and privacy is especially important on a blockchain based platform as all transactions are publicly visible. Therefore GETACAR implements a robust privacy mechanisms, as described in \ref{sec:PrivacyAndTrustMechanism}. GETACAR utilises authentication services that provide pseudonyms for users and verifies newly generated wallet addresses. This allows for ride providers and customers to take on new identifies  any time they interact with the platform be it on-chain or off-chain. This system provides a solid foundation for ensuring that the identities of users are not exposed when interacting with the platform & Yes\\
\hline
Proposal of Solutions for Physical Issues and Edge Cases & While the general focus of the research lies in creating digital processes that allow handling as much
of the user flow through the platform as possible, it is important to also design solutions for potential
damage to vehicles by passengers or inappropriate actions by individuals towards other passengers
that need to be handled outside the platform. Therefore, the aim is to ensure accountability and
conceptualise reporting mechanisms. & Ensuring privacy throughout the platform while also enforcing accountability for all users is a complex endeavour. The GETACAR platform solves this problem by utilising authentication services that are the only entities inside the platform that can match pseudonyms and wallet addresses to their owners as described in \ref{subsec:AuthService}. To ensure decentralisation everyone can get verified by the GETACAR Foundation to host an authentication service instance. Thereby each authentication service only holds a subset of all users registered on the platform. This system enables flows for solving physical issues on vehicles or other edge cases, like if a ride provider wants to make an insurance claim because a customer has damaged a vehicle, they can do so by providing the pseudonym of the user to the authentication service with a request for revealing the identity to the insurance company. & Yes\\
\hline
Prototypical Realisation of the Decentralised Platform & Based on the theoretical design, a prototype implementation of the platform is constructed. By
building the platform, it is possible to simulate real-world scenarios, understand unforeseen
challenges, and refine the design in response to them. & The design of the GETACAR platform is verified through a prototype that showcases the core functions of the GETACAR platform and the interaction between the components. The prototypical realisation is showcased in chapter \ref{chap:PrototypeImplementation}.  & Yes\\
\hline
\end{longtable}
