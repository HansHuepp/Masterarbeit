While autonomous driving is set to redefine our general mobility, it simultaneously brings forth a number of challenges. Chief among them is the anticipated surge in traffic and privacy and centralisation concerns surrounding the conventional ride-pooling solutions. As we are coming closer to this future of autonomous transportation, it becomes necessary to address these challenges. This section outlines the primary objectives this research seeks to accomplish, ensuring that the advantages of autonomous mobility are maximized while minimizing potential drawbacks.


\textbf{Prototypical Realization of the Decentralized Platform}

The main goal of this research is the creation of a decentralized platform for ride-pooling.  To archive this  the research follows two objectives:

1. Conceptualization: Before diving into implementation, this research will cover the theoretical design of the framework for the platform. This involves establishing a general vision of the platform, mapping out user flows, and ensuring the platform's privacy considerations and robustness against potential threats.

2. Prototyping: Based on the theoretical design, we will proceed to construct a prototype of the platform. By building the platform,  we can simulate real-world scenarios, understand unforeseen challenges, and refine the design in response to these challenges.


\textbf{Design of an Interaction Protocol between the Platform, Users, and Providers}

The platform, at its core, is an ecosystem of interaction flows.  A streamlined, secure, and efficient protocol for these interactions is necessary for the success of the platform. The objective here is to develop an interaction flow that: Ensures Seamless ride booking, facilitates trustworthy transactions and preserves privacy.


\textbf{Design of a Trust Mechanism for Users and Providers}

Trust is a necessity for any successful platform. In decentralized systems, where there is not a central authority to manage disputes or verify participants, the importance of trust is high. The objective is to instate mechanisms that: verifies participants, incentives good behavior and penalize misconduct.


\textbf{Evaluation of User and Provider Anonymity in Transactions}

An integral part of the platform's advantage is the security that users and providers can have transaction with each other without directly knowing each other's identity. This objective involves:
Showing how the platform hides the identities of participants during transactions and insuring that no party is able to collect critical amounts of data.


\textbf{Proposal of Solutions for Physical Issues and Edge Cases}

While much of the research focuses on the digital side of the platform, the handling of unpredictable issues that accrue in the real-world can not be ignored. Prominent concerns in ride-pooling, especially with autonomous vehicles, are the potential damage to vehicles by passengers or inappropriate actions by individuals towards other passengers. Therefor, the aim is to: Ensure accountability and conceptualise reporting mechanisms.

To ensure a realistic scope fir this research we have excluded some aspects of the platform. Therefore this research will not cover the creation of a decentralised authentication service for the platform. The reason for this is that an authentication service is a generic component that is used for all kinds of decentralised platforms. However the platform will discuss the general authentication flow that the platform will utilise. While ensuring that the platform is generally economically feasible, this research will also not cover the economic intricacies of running a decentralised ride pooling platform.

 Through the defined objectives, the goal is to construct a platform that shows that the concept of a decentralised, privacy protecting ride-pooling platform is feasible.