The landscape of transportation is on the cusp of a transformative leap. As elucidated in the problem statement, while autonomous driving is set to redefine the way we perceive mobility, it simultaneously brings forth a myriad of challenges. Chief among them is the anticipated surge in traffic and the privacy and monopolistic concerns surrounding the conventional ride-pooling solutions. As researchers and stakeholders invested in the future of transportation, it becomes our prerogative to address these challenges head-on. This section delineates the primary objectives we seek to accomplish in this study, ensuring that the benefits of autonomous mobility are maximized while minimizing potential drawbacks.

\subsection{Prototypical Realization of the Decentralized Platform}

The heart of our research lies in the creation of a decentralized platform for ride-pooling. Unlike traditional systems where power and control are concentrated, decentralized platforms spread power across nodes, ensuring equitable control and reducing the risk of any single entity's dominance.

The objective is twofold:

Conceptualization: Before diving into implementation, we will lay down a theoretical framework for the platform. This involves establishing parameters for participation, mapping out user journeys, and ensuring the system's robustness against potential threats.

Prototyping: Armed with a comprehensive theoretical design, we will proceed to construct a prototype of the platform. By building a tangible system, we can simulate real-world scenarios, understand unforeseen challenges, and refine the design in response to these challenges.

\subsection{Design of an Interaction Protocol between the Platform, Users, and Providers}

Every platform, at its core, is an ecosystem of interactions. Within the context of our decentralized ride-pooling platform, these interactions encompass:

Users seeking rides
Providers offering ride services
Transactions facilitating the above exchanges
A streamlined, secure, and efficient protocol for these interactions is paramount for the success of the platform. The objective here is to develop a protocol that:

Ensures Seamless Integration: The protocol should allow new users and providers to easily join the platform and existing ones to leave without causing disruptions.

Facilitates Trustworthy Transactions: Every transaction must ensure that the provider is fairly compensated and the user receives the promised ride.

Preserves Privacy: Given the emphasis on privacy, the protocol should guarantee that the interactions between users and providers remain untraceable by external entities.

\subsection{Design of a Trust Mechanism for Users and Providers}

Trust is the linchpin of any successful platform. In decentralized systems, where there isn't a central authority to arbitrate disputes or verify participants, the importance of trust is magnified. Our objective is to instate mechanisms that:

Verify Participants: A mechanism to authenticate and verify new entrants to the platform to prevent malicious actors.

Enable Feedback: Allow users and providers to rate and review each other, ensuring a self-regulating community.

Incentivize Good Behavior: Introduce rewards or recognitions for participants who consistently adhere to platform guidelines and receive positive feedback.

Penalize Misconduct: Conversely, there should be deterrents and penalties for those engaging in fraudulent or malicious activities.

\subsection{Evaluation of User and Provider Anonymity in Transactions}

An integral part of our platform's promise is the assurance that users and providers can transact without directly knowing each other's identity. This objective involves:

Anonymity Verification: Rigorous testing to ensure that the implemented measures successfully obfuscate the identities of participants during transactions.

Privacy Audits: Periodic checks to ascertain that no external or internal entities can trace or link transactions back to individual participants.

\subsection{Proposal of Solutions for Physical Issues, such as Vehicular Damage}

While much of our research leans heavily on the digital side of the platform, we cannot ignore the real-world, tangible issues. A prominent concern in ride-pooling, especially in an autonomous setting, is the potential damage to vehicles. Thus, we aim to:

Develop Damage-Reporting Mechanisms: Allow users to report any damages or issues they encounter during their ride.

Ensure Accountability: While preserving user privacy, create a system that holds individuals accountable for any damages they cause.

Integrate Insurance or Damage Control Measures: Explore partnerships with insurance providers or other mechanisms to safeguard against substantial damages.

In conclusion, the objectives outlined here form the backbone of this research. They not only signify our commitment to advancing the realm of transportation but also underscore our dedication to ensuring that advancements cater to societal well-being, both in terms of mitigating traffic congestion and safeguarding user privacy and rights. Through these objectives, we hope to sculpt a future where autonomous driving and ride-pooling coalesce harmoniously, amplifying the strengths of each other while concurrently addressing their individual challenges.