The landscape of transportation is transforming rapidly. As stated in the problem statement, while autonomous driving is set to redefine the way we perceive mobility, it simultaneously brings forth a number of challenges. Chief among them is the anticipated surge in traffic and privacy and centralisation concerns surrounding the conventional ride-pooling solutions. As we are coming closer to this future of autonomous transportation, it becomes necessary to address these challenges head-on. This section outlines the primary objectives I seek to accomplish in this study, ensuring that the benefits of autonomous mobility are maximized while minimizing potential drawbacks.

\subsection{Prototypical Realization of the Decentralized Platform}

The main goal of this research lies in the creation of a decentralized platform for ride-pooling. Unlike traditional systems where power and control are concentrated, decentralized platforms spread power across nodes, ensuring distributed control and reduced risk of  single entity dominance.

The objective is twofold:

Conceptualization: Before diving into implementation, I will lay down a theoretical framework for the platform. This involves establishing parameters for participation, mapping out user flows, and ensuring the system's robustness against potential threats.

Prototyping: Based on the theoretical design, I will proceed to construct a prototype of the platform. By building the system, I can simulate real-world scenarios, understand unforeseen challenges, and refine the design in response to these challenges.

\subsection{Design of an Interaction Protocol between the Platform, Users, and Providers}

The platform, at its core, is an ecosystem of interaction flows. These interactions include:

\begin{itemize}
    \item Users seeking rides
    \item Providers offering ride services
    \item Transactions facilitating the above exchanges
    \item <TODO: Complete List>
\end{itemize}

A streamlined, secure, and efficient protocol for these interactions is necessary for the success of the platform. The objective here is to develop a protocol that:

Ensures Seamless Ride Booking: The protocol should allow an easy to use ride booking flow for users as well as for ride providers which can compete with existing, centralised platforms.

Facilitates Trustworthy Transactions: Every transaction must ensure that the provider is fairly compensated and the user receives the promised ride.

Preserves Privacy: Given the emphasis on privacy, the protocol should ensure that the interactions between users and providers remain untraceable by external entities.

<TODO: Complete List>

\subsection{Design of a Trust Mechanism for Users and Providers}

Trust is a necessity for any successful platform. In decentralized systems, where there is not a central authority to manage disputes or verify participants, the importance of trust is magnified. The objective is to instate mechanisms that:

Verify Participants: A mechanism to authenticate and verify new entrants to the platform to prevent malicious actors.<Might need to rephrase this because i am not doing the Auth Service>

Incentivize Good Behavior: Introduce incentives for platform participants who consistently adhere to platform guidelines and receive positive ratings.

Penalize Misconduct: Conversely, there should be negative effects for those engaging in fraudulent or malicious activities.

<TODO: Complete List>

\subsection{Evaluation of User and Provider Anonymity in Transactions}

An integral part of the platform's advantage is the assurance that users and providers can transact without directly knowing each other's identity. This objective involves:

Anonymity Verification: Showing how the platform hides the identities of participants during transactions.

Data Collection Prevention: Insuring that no party is able to collect critical amounts of data.

<TODO: Complete List>

\subsection{Proposal of Solutions for Physical Issues and Edge Cases}

While much of the research focuses on the digital side of the platform, the handling of unpredictable issues that accrue in the real-world can not be ignored. Prominent concerns in ride-pooling, especially with autonomous vehicles, are the potential damage to vehicles by passengers or inappropriate actions by individuals towards other passengers. Therefor, the aim is to:

Ensure Accountability: While preserving user privacy, create a system that holds individuals accountable for any damages or misconduct.

Develop Reporting Mechanisms: Allow users and ride providers to report any damages or misconduct they encounter during their ride.

<TODO: Complete List>

 Through these objectives, the goal is to construct a platform that shows that the concept of a decentralised, privacy protecting ride-pooling platform is feasible.