To achieve the overall goal of this research, creating a decentralised ride pooling platform that provides an alternative to the centralised solutions, several research objectives have to be met:

\textbf{Design of the Components and Interaction Flow between the Platform, Customer, and Ride Provider}

The research needs to provide a design blueprint for the decentralised ride-pooling platform. This design should communicate the general vision of the platform and explain the key concepts.
At its core, the platform is an ecosystem of components interacting with each other. 
Therefore, it is also necessary to design a streamlined, secure, and efficient flow for these interactions. It is also important to showcase how the individual components are deployed, especially in regard to the off-chain components that make up the platform. The objective here is to develop an interaction flow that ensures seamless ride booking, facilitates trustworthy transactions, and preserves privacy.


\textbf{Design of a Trust Mechanism for Customer and Ride-Providers}

Trust is a necessity for every platform but is especially relevant for decentralised platforms as these platforms are not managed by a single owner that can single-handedly  settle disputes or resolve unexpected edge cases. Therefore, it is necessary for the platform to have a robust reputation system that sanctions malicious behaviour and promotes  good behaviour.


\textbf{Evaluation of Customer and Ride-Provider Anonymity and Privacy throughout the Platform}

One of the disadvantages of blockchain-based platforms is that the high level of transparency can result in a neglect of customer and ride-provider anonymity and privacy. This counts especially for ride pooling platforms where large amounts of personal data like location and transaction data get exchanged. That is why it is important to assess the platform design regarding privacy and anonymity to show that no entity can collect critical amounts of data from the platform.

\newpage
\textbf{Proposal of Solutions for Physical Issues and Edge Cases}

While the general focus of the research lies in creating digital processes that allow handling as much of the user flow through the platform as possible, it is important to also design solutions for  potential damage to vehicles by passengers or inappropriate actions by individuals towards other passengers that need to be handled outside the platform. Therefore, the aim is to ensure accountability and conceptualise reporting mechanisms.

\textbf{Prototypical Realisation of the Decentralised Platform}

Based on the theoretical design, a prototype implementation of the platform is constructed. By building the platform,  it is possible to simulate real-world scenarios, understand unforeseen challenges, and refine the design in response to them.

To ensure a realistic scope for this research, there are exclusions to some aspects of the platform. Therefore, this research will not cover creating a decentralised authentication service for the platform. The reason for this is that an authentication service is a generic component that is used for all kinds of decentralised platforms. However, the platform will discuss the general authentication flow that the platform will utilise. While ensuring the platform is generally economically feasible, this research will not cover the economic intricacies of running a decentralised ride-pooling platform.
Through the defined objectives, the goal is to construct a platform that shows that the concept of a decentralised, privacy-protecting ride-pooling platform is feasible.