In recent years, developments surrounding transportation have changed dramatically with the surge in technologies such as autonomous driving. While experts and industry stakeholders advertise \gls{av} for their ability to introduce more efficiency, convenience, and safety to traveling, there lies an inherent problem that has impact on the traffic conditions.

The rapid integration and adoption of autonomous driving, as projected, will inevitably lead to a upswing in traffic. This increase comes from the upsides that make an \gls{av} appealing, its ease and convenience. With the absence of the need to drive manually, more individuals will choose a personal \gls{av} over other  transport methods, creating stagnation and traffic jams. Therefor, while the technology promises a number of advantages, without proper intervention, it can also have negative effects on  travel, above all in the form of  increased road traffic.

Ride-Pooling, or shared mobility, emerges as a potential solution to this problem. The idea of ride-pooling revolves around utilising a single vehicle to transport multiple passengers that are headed in similar directions. This ensures optimal usage of a car's seating capacity, reducing the number of individual rides, and thereby decreasing overall traffic. However, in practice ride-pooling faces a number of challenges.

A fundamental challenge lies in the current monopolized state of the ride-pooling market that is controlled by a view companies. Such centralization not only puts vast amounts of data and power into the hands of one or few entities but also makes it harder for new market entrants to compete. New competitors and smaller businesses find themselves blocked out, leading to a decrease in innovation,  price inflation, and reduced customer choice.

Furthermore, centralized platforms, by their nature, consolidate vast amounts of personal and transactional data. This leads to concerns regarding user privacy. If a user’s ride details, routes, payment information, and behavior on the platform are stored without adequate privacy measures, it exposes them to potential risks.

Given this backdrop, the concept of a decentralized, trust-based platform for ride-pooling becomes not just preferable but essential. The \gls{dlt}, commonly known as blockchain, has demonstrated its potential in recent years as a tool for creating decentralized platforms. By design blockchain offers transparency, immutability, and decentralization. These attributes can address many of the problems created by centralized platforms. On the other hand, factors such as the  transparency of data running through the blockchain create challenges in regards to the privacy and anonymity of user activity on the platform.

Therefore, this study conceptualizes, designs, and evaluates a platform for ride-pooling based on  blockchain technology that preserves privacy. This platform will allow users and providers to interact seamlessly, allowing the privacy of transactions and creating an environment where parties only share necessary information with each other.

In summary, autonomous driving presents challenges that threaten to reverse its benefits. The situation demands innovative solutions, one of which is ride-pooling. However, for the privacy preserving execution of ride-pooling, a departure from traditional centralized platforms to decentralized ones is necessary. This research aims to contribute to the creation of such platforms.




