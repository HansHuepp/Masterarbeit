Technological breakthroughs in the area of autonomous driving have accelerated in recent years.
While autonomous vehicles provide the ability to make travelling more convenient and efficient, they also can create problems regarding general traffic conditions ~\cite{Riel.2022}.

The fast adoption of autonomous vehicles, once they enter the mass market, will inevitably result in more vehicles on the roads. This increase can be explained through the number of upsides that make autonomous vehicles appealing. Without the need to drive manually, more people will decide to travel with a personal autonomous vehicle and against public transport methods, creating stagnation and traffic jams. Therefore, while the technology promises several advantages, without proper intervention, it can also negatively affect  travel ~\cite{Riel.2022}.

Ride-pooling is one solution to tackle this problem. The concept centres around the idea of using a single vehicle to transport multiple passengers who all travel in a similar direction. Thereby, ride-pooling reduces the overall traffic of individual vehicles on the roads by improving the utilisation of seating capacity inside the vehicles. However, in practice, ride-pooling faces many challenges ~\cite{Riel.2022}.

One of the big challenges with current ride-pooling platforms is the centralised nature of the available platforms. This centralisation allows companies to collect huge amounts of personal and transaction data, containing highly sensitive information like residential addresses, payment information, and travel habits that can be sold and utilised for targeted advertising. Having a single entity in control of the ride-pooling platforms also results in worse conditions for ride providers and higher prices for customers ~\cite{Badr.}. 

With this background, it becomes clear that a new approach towards ride-pooling platforms is needed to tackle the industry's problems and further popularise the concept of ride-pooling. This paper promotes the creation of a decentralised, trust-based ride-pooling platform which can tackle the problems of the current Industry.

Blockchain technology has proven itself to be a capable tool for creating decentralised platforms in recent years. Platforms built with blockchain technology offer  data immutability, transaction transparency, and decentralisation  over a network of worker nodes by design. Blockchain technology addresses many of the problems inherent in centralised platforms. While the technology provides several advantages, it also brings with it several unique challenges regarding the privacy and anonymity of user activity on the platform, as all transactions running through the blockchain network are public by design ~\cite{Mahmoud.2022}.

Therefore, this study takes up the challenge of conceptualising, designing, prototyping and evaluating a platform for ride-pooling based on blockchain technology that preserves privacy. This platform will bring customers and ride providers together to allow for seamless ride-pooling that ensures the privacy of transactions and creates an environment where parties only share necessary information with each other.





