In recent years, the narrative surrounding transportation has transformed dramatically with the surge in technologies such as autonomous driving. While experts and industry stakeholders champion autonomous vehicles (AVs) for their ability to introduce efficiency, convenience, and potential safety to our roadways, there lies an inherent problem that may exacerbate current traffic conditions.

The rapid integration and adoption of autonomous driving, as projected, would inevitably lead to a massive upswing in vehicular traffic. This increase stems from the very allure that makes autonomous vehicles appealing – their ease and convenience. With the absence of the need to drive manually, more individuals might be inclined to choose personal autonomous vehicles over other modes of transport, creating urban congestion. Thus, while the technological transition promises a host of advantages, without proper intervention, it may also cause counterproductive repercussions, chiefly in the form of escalated traffic.

Ride-pooling, or shared mobility, emerges as a potential solution to this quandary. The principle of ride-pooling revolves around leveraging a single vehicle to transport multiple passengers headed in similar directions, ensuring optimal usage of a car's seating capacity, reducing the number of individual trips, and consequentially, mitigating traffic. However, the transition from theory to practice in ride-pooling reveals a suite of challenges.

A fundamental challenge lies in the potential monopolization of the ride-pooling market. Traditional models that venture to offer ride-pooling services could inadvertently centralize the platform. Such centralization not only puts an inordinate amount of data and power into the hands of one or few entities but also stifles competition. New market entrants or smaller businesses find themselves barricaded out, leading to diminished innovation, potential price inflations, and reduced consumer choices.

Furthermore, centralized platforms, by their nature, tend to be opaque, consolidating vast amounts of personal and transactional data. This leads to pronounced concerns regarding user privacy. If a user’s ride details, routes, payment information, and behavior on the platform are stored without adequate privacy measures, it exposes them to potential risks.

Given this backdrop, the concept of a decentralized, trust-based platform for ride-pooling becomes not just preferable but essential. Distributed ledger technology (DLT), commonly known as blockchain, has demonstrated its mettle in recent years as a potent tool in creating decentralized systems. By design, DLT offers transparency, immutability, and decentralization – attributes that can address many of the concerns posed by centralized systems.

Therefore, this study seeks to embark on a mission to conceptualize, design, and evaluate a platform for ride-pooling based on such distributed technology. By doing so, it endeavors to craft a system where users and providers can interact seamlessly, ensuring the privacy of transactions and enabling an environment wherein neither party necessarily has direct cognizance of the other. Inherent to this mission are several research questions, such as the means to realize this platform cognizant of stringent privacy concerns, the limitations it might confront, and the measures to ensure privacy in such a digital ecosystem.

In summary, the impending ubiquity of autonomous driving, though promising, presents challenges that threaten to reverse its benefits. The situation demands innovative solutions, one of which is ride-pooling. However, the successful execution of ride-pooling mandates a departure from traditional centralized platforms to decentralized ones that preserve privacy and engender trust. This research hopes to contribute significantly to this paradigmatic shift.




