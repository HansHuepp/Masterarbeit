With the prospect of autonomous driving becoming a reality in the foreseeable future, a surge in traffic is anticipated due to the increased accessibility and convenience offered by this technology.

In recent years, developments surrounding transportation have changed dramatically with the surge in technologies such as autonomous driving. While experts and industry stakeholders advertise autonomous vehicles (AVs) for their ability to introduce efficiency, convenience, and potential safety to our roadways, there lies an inherent problem that may impair current traffic conditions.

The rapid integration and adoption of autonomous driving, as projected, would inevitably lead to a massive upswing in vehicular traffic. This increase stems from the upsides that make autonomous vehicles appealing, their ease and convenience. With the absence of the need to drive manually, more individuals might be inclined to choose personal autonomous vehicles over other modes of transport, creating stagnation and traffic jams. Therefor, while the technological transition promises a number of advantages, without proper intervention, it may also have negative effects on  travel, above all in the form of  increased road traffic.

Ride-pooling, or shared mobility, emerges as a potential solution to this problem. The principle of ride-pooling revolves around utilising a single vehicle to transport multiple passengers headed in similar directions, ensuring optimal usage of a car's seating capacity, reducing the number of individual trips, and consequentially, decreasing overall traffic. However, in practice ride-pooling faces a number of challenges.

A fundamental challenge lies in the potential monopolization of the ride-pooling market. <Text on my why monopolisation in the industry happens>. Such centralization not only puts a vast amount of data and power into the hands of one or few entities but also makes it harder for parties to compete. New market entrants or smaller businesses find themselves blocked out, leading to diminished innovation, potential price inflation, and reduced consumer choices.

Furthermore, centralized platforms, by their nature, tend to consolidate vast amounts of personal and transactional data. This leads to concerns regarding user privacy. If a user’s ride details, routes, payment information, and behavior on the platform are stored without adequate privacy measures, it exposes them to potential risks.

Given this backdrop, the concept of a decentralized, trust-based platform for ride-pooling becomes not just preferable but essential. Distributed ledger technology (DLT), commonly known as blockchain, has demonstrated its potential in recent years as a tool for creating decentralized systems. By design, DLT offers transparency, immutability, and decentralization – attributes that can address many of the concerns posed by centralized systems. On the other hand, factors such as the  transparency of data running through a distributed ledger poses challenges in regards to the privacy of personal data and the anonymity of user activity on the platform.

Therefore, this study seeks to conceptualize, design, and evaluate a platform for ride-pooling based on such distributed technology. By doing so, it endeavors to craft a system where users and providers can interact seamlessly, ensuring the privacy of transactions and enabling an environment where neither party has direct knowledge of the other.

In summary, the ubiquity of autonomous driving, though promising, presents challenges that threaten to reverse its benefits. The situation demands innovative solutions, one of which is ride-pooling. However, the successful execution of ride-pooling mandates a departure from traditional centralized platforms to decentralized ones that preserve privacy and engender trust. This research hopes to contribute significantly to this paradigmatic shift.




