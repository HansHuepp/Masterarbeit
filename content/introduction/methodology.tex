Addressing the complexities inherent in the development and implementation of a decentralized ride-pooling platform makes a structured, systematic approach necessary. The methodology, informed by best practices in research and technological development, is comprised of a series of stages, each building upon the previous, to ensure the project's integrity and viability. This section describes the steps it will take to accomplish the research objectives and to answer the research questions.

\begin{enumerate}
    \item \textbf{Literature Research about Current Solutions}: 
    The first step is to get an understanding of the current state of academic literature. I will review academic papers, industry reports, and white-papers about ride-pooling, decentralized systems, and related technologies. This phase will provide a base, offering insights into what has been achieved, the methodologies employed by other researchers, and where gaps and opportunities might lie. The literature review will also help with bench-marking the results of this research by setting standards against which I can measure my own solution.

    \item \textbf{Identification of Shortcomings}: 
    Building on the knowledge acquired in the literature review, I will undertake a critical analysis to identify shortcomings, limitations, or unaddressed issues in existing solutions. This ranges from technological challenges to concerns about privacy, scalability, user adoption, and more. Highlighting these gaps will provide a clear direction for the proposed solution and ensure that the platform offers tangible improvements over existing models.

    \item \textbf{Proposal of a Solution Design}:
    With a comprehensive understanding of the landscape and the identified gaps, I will move to the design phase. This involves conceptualizing the architecture of the decentralized platform, defining interaction protocols, and outlining trust mechanisms, among other components. Detailed flowcharts, system diagrams, and interaction flows are developed during this phase to visualize and refine the design.

    \item \textbf{Prototypical Implementation of the Solution Design}: 
    Once the design is solidified, the next step is to build a prototype. This phase involves actual coding, leveraging technologies like the Ethereum Blockchain and developing smart contracts, user interfaces, the matching service and more. The prototype serves as a tangible representation of the design, allowing for real-world testing and iterative refinement.

    \item \textbf{Evaluation Whether the Previously Set Requirements Are Met}: 
    Post-implementation, the evaluation phase starts. The prototype will be assessed against the requirements and objectives set for this research. This evaluation involves functional testing, ensuring the system operates as intended, as well as non-functional testing, which involve assessments of scalability, performance, and privacy preservation. <Feedback from Stakeholders / Experts could be possible>

    \item \textbf{Identification of Limitations and Proposal of Possible Improvements}:
After evaluating the prototype, I will identify any potential shortcomings, challenges, or areas for enhancements. Based on this, I will propose possible improvements, setting the stage for subsequent iterations and future research.
\end{enumerate}

In summary, this iterative methodology insures an in depth academic research proceeding while still accounting for the needs of a real-world application. By progressing systematically through these stages, the aim is to develop a decentralized ride-pooling platform that utilises state of the art technologies and architecture to respond to real-world challenges and needs.
