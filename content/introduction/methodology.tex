It is important to utilise a structured approach when designing the decentralised ride-pooling platform. The following methodology provides a step-by-step process where each step builds upon the previous one, ensuring the platform viability, resulting in the design and prototypical implementation of a decentralised, privacy-preserving ride-pooling platform that showcases the current state of technology and scientific research in that field.


\begin{enumerate}

    \item \textbf{Literature Research about Current Solutions}: 
    First, it is important to understand the current state of scientific research. Therefore this paper will conduct an in-depth analysis of the current state of scientific literature, reviewing academic papers, industry reports, and white papers about ride-pooling, decentralized systems, and related technologies. The output of this step is a comprehensive overview of what has been achieved in the area of decentralised ride-pooling so far.

    \item \textbf{Identification of Shortcomings}: 
    Building upon the previous stage, this research will work out potential shortcomings of the current research landscape, point them out and propose solutions to balance out these shortcomings. This allows for the decentralised ride-pooling platform built through this research to not only be a gathering of existing research findings but also to provide added value to the research landscape.

    \item \textbf{Proposal of a Solution Design}:
    The next step is to create a comprehensive design for the platform based on the findings of the research analysis. This phase includes designing the architecture of the decentralized platform, defining interaction flows and outlining trust and privacy mechanisms. 
    

    \item \textbf{Prototypical Implementation of the Solution Design}: 
    To prove the viability of the design, it is necessary to build a prototype that can showcase that the core functions and components work as intended. Therefore this step includes the programming of smart contracts, user interfaces and other components and enabling them to communicate with each other. The finished prototype allows for real-world testing and iterative refinement.

    \item \textbf{Evaluation Whether the Previously Set Requirements Are Met}: 
    Based on the results from the working prototype, it is then important to analyse if all research objectives set for the decentralised ride-pooling platform are met. Based on this evaluation, it is possible to determine if the research is a success and can provide a significant contribution to research.

    \item \textbf{Identification of Limitations and Proposal of Possible Improvements}:
    The evaluation is also meant to bring up shortcomings of the research and aspects of the platform that require further investigation. These shortcomings and possible improvements are clearly pointed out so that future researchers can take the findings of this work and use it as the base for their research.

\end{enumerate}

In summary, this iterative methodology ensures a scientific step-by-step approach to developing the  privacy-preserving ride-pooling platform and meeting all research objectives set for this work.