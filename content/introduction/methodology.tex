Addressing the complexities inherent in the development and implementation of a decentralized ride-pooling platform mandates a structured, systematic approach. Our methodology, informed by best practices in research and technological development, comprises a series of stages, each building upon the previous, to ensure the project's integrity, viability, and efficacy. This section delineates the steps we will undertake to accomplish our objectives and answer the research questions posited.

\begin{enumerate}
    \item \textbf{Literature Research about Current Solutions}: 
    Our first step is immersing ourselves in the wealth of knowledge that already exists. We'll review academic papers, industry reports, and whitepapers pertinent to autonomous driving, ride-pooling, decentralized systems, and related technologies. This phase will provide a grounding, offering insights into what has been achieved, the methodologies employed by other researchers, and where gaps or opportunities might lie. The literature review will also aid in benchmarking, setting standards against which we can measure our own solutions.

    \item \textbf{Identification of Shortcomings}: 
    Building on the knowledge acquired in the literature review, we will undertake a critical analysis to identify shortcomings, limitations, or unaddressed issues in existing solutions. This could range from technological challenges to concerns about privacy, scalability, user adoption, and more. Highlighting these gaps will provide a clear direction for our proposed solution and ensure that our platform offers tangible improvements over existing models.

    \item \textbf{Proposal of a Solution Design}:
    With a comprehensive understanding of the landscape and the identified gaps, we'll move to the design phase. This involves conceptualizing the architecture of the decentralized platform, defining interaction protocols, and outlining trust mechanisms, among other components. Detailed flowcharts, system diagrams, and user journey maps might be developed during this phase to visualize and refine the design.

    \item \textbf{Prototypical Implementation of the Solution Design}: 
    Once our design is solidified, the next step is to bring it to life through a prototype. This phase involves actual coding, leveraging technologies like Hyperledger Fabric or comparable platforms, and developing smart contracts, user interfaces, matching services, and more. The prototype serves as a tangible representation of our design, allowing for real-world testing and iterative refinement.

    \item \textbf{Evaluation Whether the Previously Set Requirements Are Met}: 
    Post-implementation, a rigorous evaluation phase ensues. The prototype will be assessed against the requirements and objectives set at the outset. This evaluation encompasses functional testing, ensuring the system operates as intended, as well as non-functional testing, which could involve assessments of scalability, performance, and privacy preservation. Feedback from potential users or stakeholders might also be solicited during this phase to ensure the platform's practicality and user-friendliness.

    \item \textbf{Identification of Limitations and Proposal of Possible Improvements}:
    Every solution, regardless of how meticulously designed or implemented, will have limitations. Recognizing these limitations is crucial, not as an admission of inadequacy, but as an avenue for continuous improvement. After evaluating our prototype, we will identify any potential shortcomings, challenges, or areas for enhancement. Based on this, we'll propose possible improvements, setting the stage for subsequent iterations or future research endeavors.
\end{enumerate}

In summary, the methodology is iterative, reflexive, and geared towards excellence. It champions both the depth of academic rigor and the pragmatism of real-world application. By progressing systematically through these stages, we aim to develop a decentralized ride-pooling platform that is not only technologically advanced but also responsive to real-world needs and challenges.
