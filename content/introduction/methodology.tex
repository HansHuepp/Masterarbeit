The complexities of the development and implementation of a decentralized ride-pooling platform makes a structured, systematic approach necessary. The methodology is comprised of a series of stages, each building upon the previous, to ensure the platforms viability. This section describes the steps this research will take to accomplish the its objectives.

\begin{enumerate}
    \item \textbf{Literature Research about Current Solutions}: 
    The first step is to get an understanding of the current state of academic literature. A review of academic papers, industry reports, and white-papers about ride-pooling, decentralized systems, and related technologies will be conducted. This phase will provide a base, offering insights into what has been achieved.

    \item \textbf{Identification of Shortcomings}: 
    Building on the knowledge gained through the literature review, this research will undertake a critical analysis to identify shortcomings, limitations, or unaddressed issues in existing solutions. This ranges from technological challenges to concerns about privacy, scalability, user adoption, and more. Highlighting these gaps provides a clear direction for the proposed solution and ensures that the platform offers real improvements over existing solutions.

    \item \textbf{Proposal of a Solution Design}:
    With an understanding of the landscape and the identified gaps, we will move to the design phase. This involves conceptualizing the architecture of the decentralized platform, defining interaction flows, and outlining trust mechanisms, among others. Detailed flowcharts, system diagrams, and interaction flows are developed during this phase to visualize the design.

    \item \textbf{Prototypical Implementation of the Solution Design}: 
    Once the design is finished, the next step is to build a prototype. This phase involves actual coding, utilising technologies like the Ethereum Blockchain and developing smart contracts, user interfaces, the matching service and more. The prototype serves as a proof of the design, allowing for real-world testing and iterative refinement.

    \item \textbf{Evaluation Whether the Previously Set Requirements Are Met}: 
After the implementation, the evaluation phase starts. The prototype will be assessed against the requirements and objectives set for this research. This evaluation involves functional testing, ensuring the system operates as intended, as well as non-functional testing, which involve assessments of scalability, performance, and privacy preservation.

    \item \textbf{Identification of Limitations and Proposal of Possible Improvements}:
After evaluating the prototype, we will identify any potential shortcomings, challenges, or areas for enhancements. Based on this, we will propose possible improvements for  future research.
\end{enumerate}

In summary, this iterative methodology insures an in depth academic research proceeding while still accounting for the needs of a real-world application. By progressing systematically through these stages, the aim is to develop a decentralized ride-pooling platform that utilises state of the art technologies to respond to real-world challenges.
