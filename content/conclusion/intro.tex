With an expected increase in overall traffic in the near future caused by the widespread adaptation of autonomous vehicles, ride-pooling solutions are needed to decrease the number of individual trips on the road. As current ride-pooling platforms have shown to be insufficient for tackling this problem because of their centralised nature and how they deal with data privacy, this paper has introduced GETACAR, a decentralised, privacy-preserving ride-pooling platform. The design of the GETACAR ride-pooling platform is based on a comprehensive literature analysis that outlines the current state and shortcomings of decentralised ride-pooling platforms in scientific research. The design of the GETACAR platform combines the findings from the research analysis with best practices from the industry and contributes its own design suggestions to the discourse. The design of the GETACAR platform demonstrates how blockchain can be used to track rides,  how the interaction flow between customers and ride providers is supposed to look like and how payments and service fees can be managed throughout the platform. Additionally, the design covers concepts that ensure privacy and anonymity for all users as well as decentralised trust mechanisms. GETACAR also describes how security is ensured throughout the platform and how edge cases can be handled off-chain. 
The design of GETACAR is verified through the creation of a prototype that showcases all relevant features of the platform and allows a simulation of the complete ride flow between the customer, platform and ride provider. The prototype proves the feasibility of the GETACAR platform design and decentralised, privacy-preserving ride-pooling in general. Therefore, the findings gained in this work can be used as a basis for future research. The design can be used as a blueprint to create market-ready,  privacy-preserving ride-pooling platforms that favour user interests over corporate gains. 

We hope that GETACAR contributes to the research landscape of decentralised ride-pooling and therefore helps to solve the problem of increased traffic caused by autonomous vehicles.