While the GETACAR platform design provides an in-depth overview of all relevant components necessary to create a decentralised privacy-preserving ride-pooling platform, it is essential to note that the creation of such a platform is highly complex, and the area still provides room for improvement.
First of all, smart contracts, the platform's backbone, are written highly verbose to better illustrate the key functions of the contracts. Optimising the contracts will result in lower Gas fees, which decrease the operation cost of the platform  overall. Secondly, while the platform is built with decentralised authentication services in mind, the prototype does not include the component. There is also no mash of crypto exchanges connected with the prototype, allowing customers and ride providers to use fiat currency on the platform. These factors currently limit the proof of feasibility.
While the on-chain interaction flow is worked out in detail, there is no predefined communication protocol for the encrypted interaction between customer and ride provider, enabled through the Diffie-Hellman Key Exchange. The technical implementation for exchanging encrypted messages is fully realised, but no standardised format for these messages has been defined yet.
Lastly, it is important to note that the privacy and security aspects are verified on an architecture and general platform design level. Professional penetration testing of the prototype is needed fully verify these aspects of the platform. 