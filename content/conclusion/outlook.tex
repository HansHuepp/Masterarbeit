By providing a fully developed platform design and a functional prototype, GETACAR lends itself to future research in the area of decentralized ride-pooling.
A point of interest for future research should be the continuation of the quantitative and qualitative validation of the GETACAR platform design. The quantitative evaluation can be done by creating complex simulations on top of the platform. The simulations should be able to mimic customer and ride provider behaviour at a large scale to evaluate if the platform is able to process these vast amounts of data without running into bottlenecks. While the design of the platform, customer and ride provider flow is heavily based on the findings from the scientific literature, it should be further verified through qualitative testing. Therefore the platform prototype should be given out to potential customers and ride-providers for real world testing and further refinement of the platform.

Further verifying the design of the GETACAR platform is important for the creation of a successful, market-ready decentralised ride-pooling platform but there are also other research topics related to the creation of such a platform that have not been covered in the paper. This includes a detailed legal analysis on how the GETACAR Foundation should be set up and how the foundation should handle tasks like assigning the matching services to the world grid and how the foundations process should look like for verifying new authentication services.
While the basic economic feasibility of the platform is ensured by this paper, a detailed research on this topic is needed, to develop a detailed business case for GETACAR.