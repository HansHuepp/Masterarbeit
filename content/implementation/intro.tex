To validate the design of the GETACAR Platform, it is important to showcase that an actual implementation of the services is feasible. Therefore, in the following chapter, we showcase how each component is built. The goal of this implementation is not to create a ready-to-release platform to prove that the core features and functions of the platform work as previously described. Because of that, the aim is to build this prototype with as many standardised and commonly utilised technologies as possible. While some specialist or upcoming technologies like newer cutting-edge blockchain protocols might allow to tackle shortcomings of the platform or allow for additional features, this is not the focus of this prototype. The goal is to present a prototype that can be used as a blueprint to build and release a marked-ready decentralised ride-pooling platform. Therefore, it should be easy to adapt the design of the GETACAR platform with a variety of different underlying tech stacks.

The prototype we describe on the following pages does cover the smart contracts that enable the ride flow, the matching service and both the frontend for the customer and the interface for the ride provider. All components interact with each other and provide a complete ride experience. Not part of the prototype is the authentication service. As described in the introduction of this paper, the realisation of a decentralised authentication service is not part of the scope of this research. Additionally, the prototype does not fully implement the Crypto Exchange component, as it would require corporations with multiple companies to provide crypto exchange services. Therefore this prototype utilises a crypto wallet with an integrated crypto exchange to allow for the manual simulation of the buying and selling crypto currency on the GETACAR platform.
