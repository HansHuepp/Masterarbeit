Having delved into the details of the current research landscape surrounding decentralised ride-pooling platforms, we can now construct our own platform based on the findings of the scientific literature. The goal is not to replicate existing platforms but to set a benchmark that encapsulates best practices from various research papers while also proposing improvements to current methodologies.

We want to look at the platform holistically and incorporate all the attributes in our design that are necessary to operate the platform, including Blockchain utilisation, transparency, user and provider interaction protocols, payments and service fees, privacy and anonymity, security and resilience, trust mechanisms, off-chain edge cases, and a prototypical realisation.

Therefore, in this chapter, we provide an overview of the conceptual design of the decentralised platform, discuss the inner workings of each component that is part of the platform in detail and showcase the data privacy and trust mechanisms of the platform.
Through these explanations, we aim to provide a comprehensive blueprint for a decentralised ride-pooling platform that meets and exceeds the expectations set by the current academic and industry standards.

For convenience, the proposed platform will, from now on, be called GETACAR. We chose the name for its ability to describe the core offering of the platform, to get-a-car ride, while also being short, recognisable and easy to remember.

(The name also sounds phonetically similar to the title of the 1997 science fiction movie Gattaca, starring Ethan Hawke, Uma Thurman, and Jude Law, which the author of this paper immensely enjoyed.)