For ride-pooling platforms, the user experience for customers and ride providers is most important. From a customer perspective, the GETACAR platform's conceptual design is intentionally straightforward. Like other research papers, GETACAR aligns its user flow with established centralised solutions such as Uber Pool and Lyft.

The reason behind this design decision is evident. Platforms like Uber and Lyft have invested significant resources to optimise user flow. Over the years, they have collected valuable insights and established best practices that have proven effective. It is counterproductive to reinvent the wheel when these robust ride flow models already exist. Instead, by basing the ride flow on these best practices, GETACAR aims to provide an experience that is not only familiar to users but also efficient and intuitive. 
One of the primary objectives of GETACAR is to provide an offer that rivals (if not surpasses) current centralised solutions. GETACAR aims to get users to transition from centralised platforms to a decentralised counterpart by emulating the ride flow of these established platforms. To further promote the transition to GETACAR, the platform offers increased privacy and competitive pricing.
Competitive pricing is achieved by offering reduced platform fees, which translates to better end-user prices. 

\begin{figure}[H]
    \centering
    \includegraphics[width=0.99\linewidth, height=0.85\textheight, keepaspectratio]{data/ride-flow.svg}
    \caption{Ride Booking Flow}
    \label{fig:RideBookingFlow}
\end{figure}

As shown in diagram \ref{fig:RideBookingFlow}, a generic customer-ride provider flow looks like as follows:
The customer journey begins by opening the app. This is followed by entering their destination. They then choose the type of ride they prefer by adjusting their ride settings and confirming their pickup location. Once these steps are completed, they request a ride and wait for a driver match. After receiving a match, the customer can track the driver's location in real time. The customer enters the vehicle when the driver arrives and confirms their readiness to start the ride. This is a crucial step with autonomous vehicles; it lets customers signal their readiness. At the destination, the customer exits the car, ends the ride, and has the opportunity to rate and review their experience.

The process starts on the ride provider's side by opening the app if the ride provider is human. Autonomous vehicles will just need to connect to the platform and set their status to online. The ride provider then waits for a ride request. After receiving a request, they can choose to accept it. If accepted, the ride provider navigates to the pickup location. Upon arrival, they confirm to  the customer that they arrived at the pickup location. The provider begins the drive to the dropoff location after the customer confirms they are ready to start the ride. Once the ride is concluded, the driver receives their payment and, like the customer, has the chance to rate the experience. After this, the driver waits for the subsequent ride request, completing the cycle.

Offering rides and interacting with the platform as a ride provider must also be designed to be as straightforward and user-friendly as possible. To compete with existing centralised platforms, which already employ huge fleets of ride providers, GETACAR needs to ensure a massive influx of ride providers. Therefore, it is essential to ensure that it is easy to connect autonomous vehicles to the platform and allow human ride providers to interact with it. 

The decision to primarily focus on the interaction between a single ride provider and a single customer, without putting extra emphasis on the ride pooling aspect of the platform, was done so on purpose. For our vision of a decentralised ride-pooling platform, the difference between a classic taxi service and a ride-pooling service is minimal. From a customer point of view, the only difference between a taxi service and our ride pooling service is that the customer can select how many passengers are allowed to ride with them, what their minimum rating is supposed to be and how much time they are willing to take into account for detours to pick up other passengers. While a customer does not need to share their ride with other customers, they are incentivised to do so, as it results in lower overall ride cost and higher chances of finding matches. 

On the other hand, this operation model creates a very competitive market for ride providers. Ride providers can maximise their sales volume by taking on multiple customers at once for a ride pool. But with an increased number of customers joining a ride, the chances of delays also increase, which can result in bad customer reviews. This constant decision-making about which customer to offer a ride to and which not allows for various operation strategies by the ride providers and promotes calculated decision-making and risk-taking. 

\subsection{Organisational Overview}\label{subsec:OrganisationalOverview}
While our research focuses on creating the technical architecture and design for our feature-complete decentralised ride-pooling platform, it is necessary to consider the organisational structure behind the platform. For the platform to grow, it is most important to have some form of supervising entity capable of developing updates for the technical components, representing and promoting the platform to external parties and verifying the trustworthiness of some components. When implementing this kind of supervising entity, it is necessary to ensure that its centralised nature does not revert to the decentralised nature of the ride-pooling platform. Luckily, many examples from open-source software projects successfully showcase how to enable such an organisational structure. One example of  this is the Open Source Initiative. OSI is a California-based public benefit corporation that has been educating about and advocating for the benefits of open source since 1998. Therefore, we will assume, for the design of the technical components, that a so-called GETACAR Foundation exists that can handle organisational tasks necessary for creating a feature-complete ride pooling platform. The exact structure of this organisation should be part of future research in this field.

\subsection{Component Overview}\label{subsec:ComponentOverview}
As previously discussed, ensuring that the user experience remains intuitive and seamless for all parties is important. To achieve this, the GETACAR platform is built upon several interconnected, decentralised components, each serving a distinct purpose to allow for an intuitive user flow that does not comprise security, transparency and privacy. The components can be seen in figure \ref{fig:stern}.

In the following, we will provide an overview of these components:

\begin{enumerate}
    \item \textbf{Customer Frontend}:
    \begin{itemize}
        \item \textbf{Purpose}: This is the primary interface for customers to interact with the platform.
        \item \textbf{Features}:
        \begin{itemize}
            \item Allows customers to request rides.
            \item Enables customers to rate ride-providers and fellow passengers.
            \item Provides settings for customers to specify preferences, such as the minimum acceptable rating for ride providers.
        \end{itemize}
        \item \textbf{Mode of decentralisation}: Self-hosted 
    \end{itemize}

    \item \textbf{Ride Provider Frontend}:
    \begin{itemize}
        \item \textbf{Purpose}: This interface is tailored for ride providers to manage their services.
        \item \textbf{Features}:
        \begin{itemize}
            \item Enables ride providers to view and bid on open ride requests.
            \item Allows ride providers to rate passengers.
            \item Provides settings for ride providers to specify preferences, like the minimum acceptable rating of passengers.
        \end{itemize}
        \item \textbf{Mode of decentralisation}: Self-hosted 
    \end{itemize}

    \item \textbf{Authentication Service}:
    \begin{itemize}
        \item \textbf{Purpose}: To ensure the security and privacy of user data.
        \item \textbf{Features}:
        \begin{itemize}
            \item Manages customer and ride provider accounts and their associated ratings.
            \item Ensures that customer and ride provider pseudonyms are kept separate from their real identities, providing an added layer of privacy.
        \end{itemize}
        \item \textbf{Mode of decentralisation}: Self-hosted 
    \end{itemize}

    \item \textbf{Matching Service}:
    \begin{itemize}
        \item \textbf{Purpose}: To optimise the pairing of customers with ride providers.
        \item \textbf{Features}:
        \begin{itemize}
            \item Facilitates an auction mechanism where customers post ride requests visible to all ride providers in the vicinity.
            \item Ride providers can anonymously bid on these requests, ensuring competitive pricing and optimal matching.
        \end{itemize}
        \item \textbf{Mode of decentralisation}: Self-hosted / Blockchain based
    \end{itemize}

    \item \textbf{Ride Contract Service}:
    \begin{itemize}
        \item \textbf{Purpose}: To manage the intricacies of the ride and payment process.
        \item \textbf{Features}:
        \begin{itemize}
            \item Manages each phase of the ride process, from initiation to completion.
            \item Handles the ride and payment process, ensuring secure and auditable transactions between customers and ride providers.
        \end{itemize}
        \item \textbf{Mode of decentralisation}: Blockchain based
    \end{itemize}

    \item \textbf{Crypto Exchange}:
    \begin{itemize}
        \item \textbf{Purpose}: To bridge the gap between traditional fiat currency and the cryptocurrency used within the platform.
        \item \textbf{Features}:
        \begin{itemize}
            \item Allows customers and ride providers to transact in fiat currency for their real-world needs.
            \item Facilitates the conversion between fiat and cryptocurrency, ensuring that all platform transactions remain crypto-based for added security and transparency.
        \end{itemize}
        \item \textbf{Mode of decentralisation}: Blockchain based
    \end{itemize}
\end{enumerate}

Each of the components is decentralised, either in the sense that the component is designed so that an individual or a group is able to run and host it themself, to interact with the network, or the component runs decentralised on the blockchain. After providing a small overview of the important components of the GETACAR platform, we will use the following Sections of this Chapter to dive deeper into the concepts and architecture behind each component. 

\begin{figure}[h]
    \centering
    \includegraphics[width=\linewidth]{data/10.svg}
    \caption{UML Component Diagram: GETACAR Platform}
    \label{fig:stern}
\end{figure}

\subsection{Customer Frontend}\label{subsec:UserFrontend}
In the context of creating a decentralised ride-pooling platform, the user interface plays an important role in ensuring a seamless experience for both customers and ride providers. Therefore, the GETACAR frontend also draws inspiration from established platforms like Uber and Lyft. To build a comprehensive ride booking frontend, three distinct views are needed:

The primary point of interaction for customers is the booking view. This view lets customers easily input their desired pickup and dropoff locations through a search bar. Once the locations are set, this view previews the planned route on a map. This allows the customers to have a visual representation of their upcoming trip. Alongside the map, customers are presented with some additional details, such as the expected arrival time at the trip destination and the total distance of the ride. A button allows customers to commit to the route and request their ride. This initiates the matching service to find a fitting match for the customer.

Once a match is found, the frontend switches to the on-ride view, guiding the customer through the ride experience. The view first presents important information about the matched ride provider. This information includes the vehicle type, the ride provider rating, the number of passengers already in the vehicle, the expected pickup time and the maximum price that the ride will cost. Two distinct buttons allow customers to either accept the ride offer or decline it. The customer has a fixed amount of time to accept the offer before it gets automatically declined. 

Accepting the offer results in the customer depositing the maximum ride cost. After that, the view  transitions the customer to the next phase of the on-ride flow, where the customer receives real-time status updates about the ride provider's current status. This includes ''Vehicle is on its way'', ''Arrived at Pickup Location'' and ''Arrived at Dropoff Location''. Once inside the vehicle, customers can confirm they're ready to start the journey and, upon reaching the destination, confirm the ride's successful completion. After completing the ride, the customer will get paid back any additional money they deposited that exceeded the actual ride cost. At any point between ride confirmation and completion, the customer has the option to abort the ride. The complete ride flow from the customer's point of view can be seen in figure \ref{fig:ridebookingflowUML}. In that case, the money automatically gets deposited to the ride provider. If the reason for the aborted ride lies with the ride provider, the customer can contact the GETACAR foundation to initiate a complaint. Aborting rides because of the misconduct of ride providers is considered an edge case, especially when considering that most ride providers are expected to be autonomous vehicles.  After the ride, customers can rate the ride provider and potential passengers, which is a crucial part of the platform's trust mechanisms.

\begin{figure}[h]
    \centering
    \includegraphics[width=\linewidth]{data/3.svg}
    \caption{UML Activity Diagram: Ride Booking Flow}
    \label{fig:ridebookingflowUML}
\end{figure}

A settings view allows customers to manage their accounts and customise their ride experience. A customer can look up details about their account, including their rating. They can also adjust their ride preferences by specifying a minimum rating for ride providers. This ensures they are only matched with ride providers that meet their standards. Similarly, customers can adjust the minimum rating required for passengers to share a ride with them. This setting allows the customer to ensure a comfortable ride environment. This section also allows customers to manage other account settings. This includes payment methods and other privacy preferences. In conclusion, the frontend design consisting of the booking view, the on-ride view and the settings view aims to provide an intuitive experience for customers through a focus on customer needs. The design of the GETACAR frontend will contribute to the platform's success and customer adoption.

To ensure transparency, the code of the frontend is open source. This allows everyone who is interested in it to take a look at the structure of this component in detail. This also allows customers to make personal adjustments to the application and to compile the frontend themselves.

\subsection{Ride Provider Frontend}\label{subsec:RideProviderFrontend}
The design and functionality of the Ride Provider Frontend largely depend on the nature of the ride provider, be it a human pilot or an autonomous vehicle. To ensure a wide adoption by ride providers, the GETACAR Platform aims to cater to both types of providers.

For human ride providers, GETACAR provides a frontend similar to the customer frontend. This interface presents a view that allows ride providers to scroll through open ride requests. This view allows them to check ride details and place bids on ride requests. When an auction for a ride request results in the provider winning the ride opportunity and the customer confirming the ride, the frontend informs the ride provider that they have indeed won the auction. 
To confirm the ride and to receive the exact pickup and dropoff location of the customer, the ride provider needs to deposit 10\% of the customer deposit themselves to ensure that they will actually carry out the ride. The ride provider can claim their deposit back once the ride is successfully completed.

After this confirmation, a second view becomes visible that allows the ride provider to manage the actual ride process. This flow mirrors the customer's journey. The provider confirms their commitment to the customer to take on the ride request and confirms their drive to the pickup location. When reaching the pickup point, the ride provider informs the customer of their arrival. The customer can then board the vehicle and initiate the ride. The ride provider then starts the travel to the dropoff location. After reaching the destination, the ride provider confirms the arrival via the frontend. This  concludes the ride from their perspective. Once the customer leaves the vehicle and confirms the ride's end, the provider is able to claim the actual ride cost. This cost is deducted from the maximum ride cost deposit made by the customer at the ride's start. 
Just like the customer, the provider retains the option to abort the ride at any point between the ride's confirmation and the confirmation of arrival at the pickup location. The complete ride flow is shown in figure \ref{fig:SwimLaneCustomer}

Like the customer frontend, the Ride Provider frontend also includes a view offering general information and settings, mirroring the customer frontend. It is worth noting that all the functionalities are present, even if the ride provider is an autonomous vehicle. The distinguishing factor here is the absence of a visual user interface for autonomous vehicles. Instead, the computer that controls the autonomous vehicle can interact directly with the endpoints of the platform.

\subsection{Authentication Service}\label{subsec:AuthService}

The linchpin of security and user anonymity on the platform is the authentication service. Any user, be it a customer, ride provider, or a hoster of a matching service, must first register with an authentication service to interact with the platform. This service is the only component that manages user data such as name, age, address, and the associated rating. It also oversees the registration of new autonomous vehicles and human ride providers, storing details like the number plate, VIN number, the responsible entity (company or individual) for the vehicle, and the vehicle's associated rating.

Once registered, users can use the authentication service to receive login tokens and pseudonyms. This also includes verifying crypto wallets that the user themself generates. The pseudonyms and credentials enable users to validate their permissions to engage with other platform components without sharing personal information. For on-chain interactions, users can use the frequently changing wallets linked to their authentication service account. This approach not only ensures user anonymity during on-chain engagements but also prevents external entities from monitoring the chain and tracking user activities. 

\begin{figure}[h]
    \centering
    \includegraphics[height=0.40\textheight]{data/2.svg}
    \caption{Customer Pseudonyms trough Multiple Components}
    \label{fig:directSVG}
\end{figure}


The architecture also facilitates rating tracking by the authentication service. For example, a customer utilises five different crypto wallets across five rides and receives five ratings from ride providers. Monitoring of the blockchain only reveals five separate wallets, each with a single rating, between a pool of other wallets on the platform with individual ratings. The authentication service, however, is able to connect all five wallets to the customer, as it knows all pseudonyms and wallets linked to the customer, enabling it to compute an aggregate rating. The authentication service also ensures that users can not register multiple times with the platform to escape a series of bad ratings, for example.

Beyond user authentication and rating computation, the service also validates rating information about other users. For example, a ride provider can verify a user's claimed rating by forwarding the rating and the user's pseudonym to the matching service. 

In edge cases, the authentication service can unveil a user's identity, such as when a ride provider seeks to make an insurance claim to a customer due to intentional vehicle damage by a customer or when a passenger wishes to report harassment to law enforcement. While revealing a user's identity should be rare, it's an indispensable feature to ensure platform security and safety.

Given the immense power held by the authentication service, it is important to prevent it from morphing into a centralised control point. Therefore, the design allows anyone to host their authentication service and request it to be verified by the GETACAR foundation. First-time platform registrants can also choose their preferred authentication service from the list of verified services. All verified authentication services communicate with each other, ensuring that users can  not secretly create identities across multiple services. This inter-service exchange is executed without revealing the users managed by each service to one another. 

\subsection{Matching Service}\label{subsec:MatchingService}
The primary responsibility of the matching service is to match ride requests from customers with suitable ride providers. Traditional centralised services, such as Uber and Lyft, employ algorithmic matching systems. These algorithms consider various factors like proximity, availability, and customer preferences to quickly assign a driver to a customer's request. While efficient, this centralised approach often lacks transparency, and the decision-making process is entirely controlled by the platform, potentially leading to biases or unfair advantages for certain drivers or riders.

Recognising these challenges, GETACAR has adopted an auction-based ride-matching system. This approach offers several advantages. Firstly, it introduces a competitive environment where ride providers can bid for rides, ensuring that customers get the most cost-effective offers. Secondly, it promotes transparency by making the selection method known to everyone. Lastly, it empowers ride providers by giving them the autonomy to bid on rides based on their preferences and profitability. The complete matching flow between the customer, ride provider and platform can be seen in figure \ref{fig:maching}.

\begin{figure}[h]
    \centering
    \includegraphics[width=\linewidth]{data/5.svg}
    \caption{BPMN 2.0 Activity Diagram: Customer, Ride Provider Matching Service Flow}
    \label{fig:maching}
\end{figure}


Within this framework, GETACAR employs a Vickrey auction for its matching. The second-price Vickrey auction has distinct advantages. In this model, the highest bidder wins but pays the amount bid by the second-highest bidder. This encourages ride providers to bid their true valuation without the fear of overpaying. It promotes honest bidding, reduces the chances of strategic manipulation, and ensures that customers receive competitive prices while providers are fairly compensated. Building upon the generalised, anonymous second-price auction concept, the intricate auction flow within the matching service unfolds as follows:

A customer initiates the process by submitting a ride request to the matching service. This request encapsulates an approximation of both the pickup and drop-off locations, ensuring that the precise coordinates are exclusively shared between the customer and ride provider once a mutual commitment to the ride is established. Additionally, the request contains ride details such as the customer's rating, the minimum acceptable rating for the ride provider, the minimum rating for potential co-passengers, and the maximum passenger count.

Upon receiving the ride request, the matching service marks it as an open auction, inviting ride providers to place their bids. This auction remains accessible for a predetermined duration, during which ride providers can submit their bids. These bids contain the proposed ride price and are accompanied by additional data like the provider's rating, anticipated arrival time and the vehicle model.

Following the auction's closure, the matching service identifies the winning bid and communicates the result to the customer. Subsequently, the customer is granted a fixed time window to review and accept the offer before it lapses.

Should the customer confirm the winning bid, they generate a ride contract containing the agreed-upon sum from the bid. The winning bidder is then notified of their successful bid and receives the ride contract's address, marking the commencement of their service obligation.

Lastly, the matching service will be used to establish a secure connection to share sensitive information on-chain. Therefore, the matching service initiates a Diffie-Hellman Key Exchange by sharing the necessary prime number, base and submitted customer and ride provider public keys between the two parties. Based on this information, the customer and ride provider can compute a shared secret for encrypted interactions on the blockchain.

This secure connection will allow customers and ride providers to share detailed location data and other bits of personal data via blockchain without exposing the information to the ledger.

Having explained the core concept of the matching service, it is important to also explain how decentralisation is achieved for this service and how we ensure that the matching service can not collect too much transaction data. One conceivable method to ensure decentralisation is to execute the matching process on-chain. Running the matching service on-chain offers transparency, immutability, and a trustless environment, ensuring that all transactions are verifiable and irreversible.

However, this on-chain approach is not without challenges. Even with the use of pseudonyms and location approximations, all ride requests and bids would be publicly accessible on the blockchain. This transparency, while advantageous in some contexts, could possibly expose patterns, preferences, and behaviours of users, potentially compromising their privacy and anonymity. Moreover, executing the intricate matching flow on the blockchain, especially at a large scale, demands significant computational resources. This would inevitably increase the platform's overall costs, thereby inflating ride prices and reducing the profits for ride providers.

To circumvent these challenges, the platform adopts an approach similar to the authentication service. Individuals or organisations can host their independent matching service, which, post-registration, undergoes verification by the GETACAR foundation. 

To ensure an equal distribution of ride requests between the matching services and to prevent matching services from collecting excessive amounts of personal data, a grid-based distribution system is proposed. The grid-based system assigns each service-specific grid field as their jurisdiction zone.

To stop a single service from collecting long-term data within a grid, we mandate that each grid be managed by at least two distinct matching services. The GETACAR Foundation also has the possibility to rearrange the grid jurisdiction zones in time intervals for additional security.

For a customer, the process of finding the right matching service looks as follows: Every customer frontend contains a list of all verified matching services and their corresponding grid jurisdictions. The customer is then able to determine the matching services responsible for their current grid square.  Customers can then provide the list of possible matching services to an on-chain service and receive the best matching service from the available options within their grid. This on-chain service guarantees an even distribution of requests within a grid square by monitoring the number of requests for each service and the amount of successfully completed rides through the matching service. This allows for a load balancing between the matching services so that no service can collect too much data. Furthermore, it allows for the creation of a rating for every matching service, derived by dividing the total number of requests processed by the service by the number of rides successfully completed through it. The complete flow can be seen in figure \ref{fig:getBestMatchingService}.

\begin{figure}[h]
    \centering
    \includegraphics[width=\linewidth]{data/4.svg}
    \caption{BPMN 2.0 Activity Diagram: Customer and Matching Contract Interaction}
    \label{fig:getBestMatchingService}
\end{figure}


\subsection{Ride Contract Service}\label{subsec:RideContractService}

Within the blockchain, various events and statuses related to the ride-pooling process are recorded. Initially, the ride provider accepts the ride, and a ''Ride-provider accepted'' status is written to the blockchain. Following this, the ride provider starts driving to the pickup location, and once they arrive, a ''Ride-provider arrived at pickup location'' status is documented. Meanwhile, the customer indicates they are ready to start the ride, prompting a ''Customer ready to start ride'' status to be written to the blockchain. The ride provider then starts the ride, leading to a ''Ride-provider started ride'' status getting written onto the blockchain. Upon the ride's conclusion, when the ride provider arrives at the drop-off location, a ''Ride-provider arrives at dropoff location'' status is recorded. The customer then marks the ride as complete, resulting in a ''Customer marked ride complete'' status on the blockchain. The complete flow can be seen in \ref{fig:SwimLaneCustomer}

\begin{figure}[h]
    \centering
    \includegraphics[width=\linewidth]{data/1.svg}
    \caption{BPMN 2.0 Swimlane Diagram: Customer, Platform and Ride Provider}
    \label{fig:SwimLaneCustomer}
\end{figure}


Each event that gets posted to the blockchain also contains a custom message that can be determined by the entity posting the event. These messages will be encrypted by the key that was exchanged between customer and ride provider via the Diffie-Hellman Key Exchange. These Messages can contain a number of different information depending on the status of the ride. A typical message sent by the ride provider as part of the initial ``Ride-provider accepted'' event would be a tracking link that allows the customer to track the vehicle. 

The contract also contains a number of additional functions in regards to trust and privacy that are described in detail in \ref{subsec:RatingTrustMechanisms} and \ref{subsec:DepositTrust}.

This decentralised ride-pooling platform ensures transparency and trustworthiness by recording every significant event on the blockchain, with both the customer and the ride provider actively interacting with it, ensuring a smooth and verifiable ride process.


\subsection{Crypto Exchange}\label{subsec:CryptoExchange}

The Crypto Exchange plays an important role in bridging the gap between traditional fiat currencies and cryptocurrencies, ensuring that all transactions within the platform can utilise the advantages of crypto currencies. When a customer books a ride, the frontend instantiates a new wallet. To fund this wallet for the ride, the customer pays a crypto exchange, which then credits the wallet with the required amount of crypto currency. Post-ride, ride providers have the option to convert the crypto currency they have received into fiat currency via the crypto exchange.

This enables the platform to utilise the benefits of blockchain and crypto currency, such as transparency and security. Simultaneously, it offers the flexibility for customers and ride providers to transact in the more universally accepted fiat currency, thereby enhancing usability.

However, a significant privacy concern emerges when considering that most crypto exchanges mandate personal identification for buying and trading crypto currency. This could potentially establish a traceable link between real-world identities and wallet owners, undermining the platform's commitment to privacy. There are several solutions to solving this problem. One approach involves collaborating with an exchange that solely requires verification from a GETACAR authentication service to authorise trades, bypassing the need for personal identification. Alternatively, diversifying transactions across multiple crypto exchanges or employing coin mixers can obfuscate transaction trails. A coin mixer, or crypto currency tumbler, is a service that mixes potentially identifiable crypto currency funds with others, making it challenging to trace specific coins back to their original source. This ensures enhanced privacy and anonymity for users, making transactions less traceable on the blockchain.

Another proposition is for the authentication service function them self as a crypto exchanges. This would further strengthen the platform's services, potentially offering a more streamlined user experience. The decision which approach to flow  depends on multiple external factors including the willingness of cooperation from crypto exchanges and the liquidity of the authentication services.
 