In the journey of conceptualising the GETACAR platform, paramount importance has been accorded to privacy measures and the trust mechanism design, ensuring the protection of users  at every juncture. 

\subsection{Data Privacy from a Customer and Ride Provider Perspective}
Now that we have explained the functions and concepts behind all the components relevant to the GETACAR ride pooling platform in detail, we can analyse how data privacy is  affected by the platform's design.

To assess the data privacy of the platform based on its design, it is important to showcase how the data that customers and ride providers procure is shared between the GETACAR services.

Table \ref{tab:CustomerDataPrivacyMatrix} is a compilation of conventional and GETACAR-specific data points  exchanged in the ride-sharing platform.
To facilitate clarity and ease of interpretation, each data point, in the context of each service, is represented using specific symbols:

\begin{itemize}
    \item \textbf{$\checkmark$}: Symbolises that the service possesses access to the respective information.
    \item \textbf{$\bullet$}: Denotes the necessity for the information to be shared amongst parties.
    \item \textbf{$\times$}: Indicates the service's lack of access to the particular information.
    \item \textbf{($\times$)}: Indicates that the service can access parts of the information.
\end{itemize}

The ensuing table provides a detailed breakdown of these data points about the various services within the platform.

\begin{table}[H]
\centering
\small % Reduce font size
\begin{tabular}{|l|c|c|c|c|c|c|}
\hline
\textbf{Data} & \rotatebox{90}{\textbf{Customer}} & \rotatebox{90}{\textbf{Ride Provider}} & \rotatebox{90}{\textbf{Matching Service}} & \rotatebox{90}{\textbf{Crypto Exchange}} & \rotatebox{90}{\textbf{Authentication Service}} & \rotatebox{90}{\textbf{Publicly Available}} \\
\hline
\multicolumn{1}{|l|}{\textbf{Basic Personal Details:}} & \multicolumn{1}{c}{} & \multicolumn{1}{c}{} & \multicolumn{1}{c}{} & \multicolumn{1}{c}{} & \multicolumn{1}{c}{} & \multicolumn{1}{c|}{} \\
\hline
Full name & $\checkmark$$\bullet$ &  $\times$ & $\times$ & $\times$ & $\checkmark$$\bullet$ & $\times$ \\
\hline
Gender & $\checkmark$$\bullet$ &  $\times$ & $\times$ & $\times$ & $\checkmark$$\bullet$ & $\times$ \\
\hline
Date of birth & $\checkmark$$\bullet$ &  $\times$ & $\times$ & $\times$ & $\checkmark$$\bullet$ & $\times$ \\
\hline
\multicolumn{1}{|l|}{\textbf{Contact Information:}} & \multicolumn{1}{c}{} & \multicolumn{1}{c}{} & \multicolumn{1}{c}{} & \multicolumn{1}{c}{} & \multicolumn{1}{c}{} & \multicolumn{1}{c|}{} \\
\hline
Email address & $\checkmark$$\bullet$ &  $\times$ & $\times$ & $\times$ & $\checkmark$$\bullet$ & $\times$ \\
\hline
Phone number & $\checkmark$$\bullet$ &  $\times$ & $\times$ & $\times$ & $\checkmark$$\bullet$ & $\times$ \\
\hline
Home address & $\checkmark$$\bullet$ &  $\times$ & $\times$ & $\times$ & $\checkmark$$\bullet$ & $\times$ \\
\hline
\multicolumn{1}{|l|}{\textbf{Payment Information:}} & \multicolumn{1}{c}{} & \multicolumn{1}{c}{} & \multicolumn{1}{c}{} & \multicolumn{1}{c}{} & \multicolumn{1}{c}{} & \multicolumn{1}{c|}{} \\
\hline
Credit/debit card details & $\checkmark$$\bullet$ & $\times$ & $\times$ & $\checkmark$$\bullet$ & $\times$ & $\times$ \\
\hline
Bank account details & $\checkmark$$\bullet$ & $\times$ & $\times$ & $\checkmark$$\bullet$ & $\times$ & $\times$  \\
\hline
Payment history & $\checkmark$$\bullet$ & $\times$ & $\times$ & ($\times$) &$\checkmark$ & $\times$ \\
\hline
Billing address & $\checkmark$$\bullet$ & $\times$ & $\times$ & $\checkmark$$\bullet$ & $\times$ & $\times$ \\
\hline
\multicolumn{1}{|l|}{\textbf{Ride Details:}} & \multicolumn{1}{c}{} & \multicolumn{1}{c}{} & \multicolumn{1}{c}{} & \multicolumn{1}{c}{} & \multicolumn{1}{c}{} & \multicolumn{1}{c|}{} \\
\hline
Pickup and drop-off locations & $\checkmark$$\bullet$ & $\checkmark$$\bullet$ &  ($\times$) & $\times$ & $\times$ & $\times$ \\
\hline
Date and time of rides & $\checkmark$$\bullet$ & $\checkmark$$\bullet$ & ($\times$) & $\times$ & $\checkmark$ & $\times$ \\
\hline
Ride preferences & $\checkmark$$\bullet$ & $\checkmark$$\bullet$ & ($\times$) & $\times$ & $\times$ & $\times$ \\
\hline
Ride history & $\checkmark$$\bullet$ & $\times$ & $\times$ & $\times$ & $\times$ & $\times$ \\
\hline
\multicolumn{1}{|l|}{\textbf{Location Data: }} & \multicolumn{1}{c}{} & \multicolumn{1}{c}{} & \multicolumn{1}{c}{} & \multicolumn{1}{c}{} & \multicolumn{1}{c}{} & \multicolumn{1}{c|}{} \\
\hline
Real-time location during a ride & $\checkmark$$\bullet$ & $\checkmark$$\bullet$ & $\times$ & $\times$ & $\times$ & $\times$ \\
\hline
Frequent locations &$\checkmark$$\bullet$ & $\times$ & $\times$ & $\times$ & $\times$ & $\times$ \\
\hline
Route taken during the ride & $\checkmark$$\bullet$ & $\checkmark$$\bullet$ & $\times$ & $\times$ & $\times$ & $\times$ \\
\hline
\multicolumn{1}{|l|}{\textbf{Device Information: }} & \multicolumn{1}{c}{} & \multicolumn{1}{c}{} & \multicolumn{1}{c}{} & \multicolumn{1}{c}{} & \multicolumn{1}{c}{} & \multicolumn{1}{c|}{} \\
\hline
Device type & $\checkmark$$\bullet$ & $\times$ & $\times$ & $\times$ & $\times$ & $\times$ \\
\hline
Operating system & $\checkmark$$\bullet$ & $\times$ & $\times$ & $\times$ & $\times$ & $\times$ \\
\hline
App version & $\checkmark$$\bullet$ & $\times$ & $\times$ & $\times$ & $\times$ & $\times$ \\
\hline
Device identifiers & $\checkmark$$\bullet$ & $\times$ & $\times$ & $\times$ & $\times$ & $\times$ \\
\hline
\multicolumn{1}{|l|}{\textbf{Communication Data: }} & \multicolumn{1}{c}{} & \multicolumn{1}{c}{} & \multicolumn{1}{c}{} & \multicolumn{1}{c}{} & \multicolumn{1}{c}{} & \multicolumn{1}{c|}{} \\
\hline
In-app messages between driver and rider & $\checkmark$$\bullet$ & $\checkmark$$\bullet$ & $\times$ & $\times$ & $\times$ & $\times$ \\
\hline
\multicolumn{1}{|l|}{\textbf{Behavioral Data: }} & \multicolumn{1}{c}{} & \multicolumn{1}{c}{} & \multicolumn{1}{c}{} & \multicolumn{1}{c}{} & \multicolumn{1}{c}{} & \multicolumn{1}{c|}{} \\
\hline
App usage patterns & $\checkmark$$\bullet$ & $\times$ & $\times$ & $\times$ & $\times$ & $\times$ \\
\hline
Click patterns within the app & $\checkmark$$\bullet$ & $\times$ & $\times$ & $\times$ & $\times$ & $\times$ \\
\hline
Features frequently used & $\checkmark$$\bullet$ & $\times$ & $\times$ & $\times$ & $\times$ & $\times$ \\
\hline
\multicolumn{1}{|l|}{\textbf{Safety and Security Data: }} & \multicolumn{1}{c}{} & \multicolumn{1}{c}{} & \multicolumn{1}{c}{} & \multicolumn{1}{c}{} & \multicolumn{1}{c}{} & \multicolumn{1}{c|}{} \\
\hline
Records of any incidents or disputes during rides & $\checkmark$$\bullet$ & $\checkmark$$\bullet$ & $\times$ & $\times$ & $\times$ & $\times$ \\
\hline
\multicolumn{1}{|l|}{\textbf{Ratings: }} & \multicolumn{1}{c}{} & \multicolumn{1}{c}{} & \multicolumn{1}{c}{} & \multicolumn{1}{c}{} & \multicolumn{1}{c}{} & \multicolumn{1}{c|}{} \\
\hline
Ratings provided about drivers & $\checkmark$$\bullet$ & $\times$ & $\times$ & $\times$ & $\checkmark$ & $\times$ \\
\hline
Ratings received from drivers about the user & $\times$ & $\checkmark$$\bullet$ & $\times$ & $\times$ & $\checkmark$ & $\times$ \\
\hline
\multicolumn{1}{|l|}{\textbf{Preferences and Settings: }} & \multicolumn{1}{c}{} & \multicolumn{1}{c}{} & \multicolumn{1}{c}{} & \multicolumn{1}{c}{} & \multicolumn{1}{c}{} & \multicolumn{1}{c|}{} \\
\hline
Language preference & $\checkmark$$\bullet$ & $\times$ & $\times$ & $\times$ & $\times$ & $\times$ \\
\hline
Notification settings & $\checkmark$$\bullet$ & $\times$ & $\times$ & $\times$ & $\times$ & $\times$ \\
\hline
\end{tabular}
\caption{Customer Data Privacy Matrix}
\label{tab:CustomerDataPrivacyMatrix}
\end{table}


The GETACAR platform has been meticulously designed to primarily focus on user privacy. Through its architecture, the platform ensures that user data is managed with the utmost discretion, granting access only where necessary across its specific components. The table provided offers a detailed overview of this data-sharing mechanism.


When it comes to basic personal details such as the full name, gender, and date of birth, these are exclusively accessible by the customer and the authentication service. This design choice ensures that these personal identifiers remain shielded and aren't exposed to other platform facets. Similarly, contact details like the email address, phone number, and home address are safeguarded, with access limited to the customer and the authentication service. This setup ensures that personal contact details remain undisclosed to ride providers or other services. As explained in \ref{subsec:AuthService}, the only time the authentication service exposes these data points to third parties is in extreme edge cases, for example, when a  ride provider seeks to lodge an insurance claim due to intentional vehicle damage by a customer or when a passenger wishes to report harassment to law enforcement.

In the realm of payment information, details like credit/debit card numbers and bank account specifics are primarily shared between the customer and the crypto exchange. This arrangement facilitates smooth transactions while upholding the confidentiality of the user's financial data. Moreover, the payment history, vital for transaction transparency, is accessible only to the customer if they utilise several different crypto exchanges or other methods to hide their payment history, as explained in \ref{subsec:CryptoExchange}.

Ride details form a crucial part of the platform's operations. Information such as pickup and drop-off locations, the date and time of rides, and specific ride preferences are shared between the customer and the ride provider. While most of the ride details are shared only between the customer and ride provider, the date and time of rides are also visible by the authentication service because the service can look up on-chain activities of users because the service knows all wallet addresses connected to one customer.


Additionally, location data, including real-time location during a ride and the route taken, are encrypted and shared only between the customer and the ride provider executing the ride, as decided in \ref{subsec:MatchingService}. This is pivotal for ensuring safety and efficiency during the journey. To some extent, the date and time of rides and Ride preferences are also shared with the matching service. This could be prevented by applying encryption algorithms like homomorphic encryption to the matching flow. For the design of the GETACAR platform, we decided to utilise a different approach detailed in Chapter \ref{subsec:MatchingService}. This approach ensures that no matching service can collect enough data to analyse customer behaviour. It is also important to note that the matching service only receives changing pseudonyms as ride requester; no personal details are shared. 

Device-related information, which encompasses details like the device type, operating system, app version, and unique device identifiers, remains confidential, with access restricted solely to the customer. As decided in \ref{subsec:UserFrontend}, this is ensured by providing the frontend as open-source code with no user data collection features.

Communication, especially in-app messages between the driver and the rider, is kept private, confined to just the two parties, ensuring that all communication remains confidential as described in \ref{subsec:MatchingService}.

Behavioural data, which includes insights into app usage patterns, click patterns within the app, and frequently accessed features, is kept private, with access limited exclusively to the customer. This is ensured by providing the frontend as open-source code with no user data collection features.

Ratings form an integral part of the platform's security mechanism. Customers' Ratings about drivers are accessible to both the customer and the authentication service. On the flip side, drivers' ratings about the user are accessible to the ride provider and the authentication service. This dual-access system ensures transparency in the rating process while upholding user privacy, as described in \ref{subsec:RatingTrustMechanisms}.

Looking at the ride provider data privacy, the table \ref{tab:CustomerDataPrivacyMatrix} can equally be applied to the ride providers. Ride providers have some additional information managed by the authentication service, like the vehicle's number plate, its VIN and, in the case of autonomous vehicles, the individual or company owner that has authorised the vehicle to be part of the GETACAR platform. Like the customer contract information, they are not shared with other services. The same goes for interactions with Matching Services and Crypto Exchanges.

The table and its accompanying explanation underscore the GETACAR platform's commitment to safeguarding user data. Thanks to pseudonyms and changing wallets, there is no publicly available data that unveils a user's identity or allows for long-term tracing to make assumptions about the identity of a user. This is an excellent achievement for a decentralised, transparent, blockchain-based platform.  Additionally, the data available to services that do not explicitly need access to this information is reduced to a minimum and could only be further reduced by applying complex and computationally intensive cryptography like homomorphic encryption, which could potentially impact the user experience.

\subsection{Rating Trust Mechanisms}\label{subsec:RatingTrustMechanisms}
Rating systems are an essential tool for ensuring trust and safety on ride-pooling platforms and are considered a best practice based on the scientific literature. When passengers and drivers are allowed to rate each other, it creates a feedback mechanism that makes both parties accountable. Providing users with the possibility of viewing the ratings of other customers and ride providers allows users to make informed decisions about which ride providers to choose based on previous experiences of other users. Additionally, a rating system can promote better service from ride providers. Equally, customers are encouraged to show better behaviour towards passengers and the ride provider. As it is the goal of this research paper to design and build the technical foundation of the GETACAR platform, we will assume that a generic five-star rating system with time-weighted ratings provides optimal results. The GETACAR platform's design ensures that several rating systems are supported. Determining the best possible rating system for the GETACAR platform should be part of future research. 

As discussed in \ref{subsec:UserFrontend} the platform allows customers to rate their ride providers and passengers and allows ride providers to rate their customers. The ratings are posted onto the ride contract running on the Blockchain through a rating function. This allows for much flexibility regarding the rating system because the smart contract can easily be adjusted to change the rating system, for example from a five start rating to a ten-star rating. 

\begin{figure}[h]
    \centering
    \includegraphics[width=\linewidth]{data/6.svg}
    \caption{Customer and Ride Provider rate each other on Blockchain}
    \label{fig:directSVG}
\end{figure}


\begin{figure}[h]
    \centering
    \includegraphics[width=\linewidth]{data/8.svg}
    \caption{Ride Provider adds Passengers to Ride Contract and Customer rates them}
    \label{fig:directSVG}
\end{figure}


The authentication service monitors all contracts. If a rating is posted on the Blockchain that affects a pseudonym that belongs to a user managed by an authentication service, this instance of the authentication service adds the rating to the user profile. Thereby, the authentication service has a list of all ratings that belong to a single user alongside metadata like the time the rating was posted. This allows the rating service to implement more complex rating systems that provide more accurate ratings.

\begin{figure}[h]
    \centering
    \includegraphics[width=\linewidth]{data/7.svg}
    \caption{Get Rating from Authentication Service as Ride Provider}
    \label{fig:directSVG}
\end{figure}

\begin{figure}[h]
    \centering
    \includegraphics[width=\linewidth]{data/9.svg}
    \caption{Get Rating from Authentication Service as Customer}
    \label{fig:directSVG}
\end{figure}


To ensure that users do not lie about their ratings, the authentication services work as a verification service for the ratings they manage through a request that contains the non-verified rating and the pseudonym connected with it. This by itself creates a small risk that users can find connected pseudonyms by analysing ratings posted on the Blockchain. To counter this, all ratings the authentication service provides (including the rating presented to a user as their own rating) are rounded to 0.3 steps (e.g. 5, 4.7, 4.3, 4, ... ). 


To put it in a nutshell, the combination of on-chain ratings combined with off-chain rating accumulation allows for more complex and accurate rating systems while still utilising the transparency and auditing abilities of a blockchain-based rating system which ensures a secure and trustworthy environment on the platform.

Additionally, the completely blockchain-based rating system is in place for the matching services as described in \ref{subsec:MatchingService}, allowing customers to ensure that they are utilising a well-performing matching service without risking exposing personal information by utilising the same instance of the matching service too often.

\subsection{Deposit Trust Mechanisms}\label{subsec:DepositTrust}
While the rating system provides a solid foundation for ensuring trust in the platform, many research papers utilise money deposits by the customer and ride provider to ensure that the cost of the ride will be paid and to counter malicious activities on the platform. In the following, we will focus on how to implement such deposit functions into the platform with the help of smart contracts. It is not part of this research to determine what the exact amounts of deposited money should be to ensure trust between customer and ride provider. This should be part of future research.

The implementation of the deposit-based trust mechanism in GETACAR looks as follows:
After the customer reviews and accepts the winning bid from a ride provider, they create a ride contract on the blockchain as described in \ref{subsec:RideContractService}. This contract is loaded with a deposit by the customer that is equal to the maximum ride cost set by the ride provider in their bid. This ensures that the ride provider has enough money to pay for the ride. It also ensures that the customer can not be overplayed by the ride provider as the maximum ride cost is limited by the deposit in the contract. 

To ensure that the ride provider has honest intent on completing the ride, they have to deposit 10\% of the maximum ride cost into the contract themselves to be able to be allowed to handle the ride. This deposit will be transferred back to the ride provider after completing the ride. 

After the ride is marked as completed by both the customer and the ride provider, the ride provider can now claim the amount of money from the ride contract that they determined as the actual ride price. The rest gets transferred back to the customer.

By managing all the transactions through a smart contract, we can ensure that no trusted third party is needed as well, and no trust between the customer and ride provider is needed to ensure that all transactions are handled properly.

\subsection{Conclusion}
Based on the current state of the scientific literature and best practices from the industry, we designed the GETACAR ride-pooling platform. This chapter has shown how each of the components of the GETACAR platform works and how components interact with each other. The design was also assessed on how it handles data privacy, and we explained how trust mechanisms are implemented into the platform. Now that we have successfully designed the platform, it is important to show that the design successfully transports into reality. 