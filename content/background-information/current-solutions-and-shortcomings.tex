\subsection{Selection of search parameters and research databases}
Due to the diverse literature regarding decentralised ride-sharing platforms, the proven approach of a systematic literature search according to Vom Brocke was chosen. In this way, quality criteria such as traceability and reproducibility can be ensured through a clearly defined processes. Two common, cross-publisher research databases and one common publisher database were used for the literature search.
The selection of several cross-publisher research databases is intended to ensure that the search provides a representative overview of existing research on decentralised ride-sharing platforms,. The selection of the database of a publisher with a focus on information technology is intended to show how the research topic is treated in the literature from a primarily information technology perspective. The cross-publisher research databases used are Scopus and Ebscohost. The publisher database is IEEE Xplore.
The goal is to obtain research literature as a search result that deals with the development of decentralised ride-sharing platforms. To obtain results covering mainly decentralised platforms the search phrase "decentralised" was used. The following three synonyms were used to obtain search results that deal with the topic of ride-sharing: "ride-sharing", "ride-pooling" and "ride-hailing". Initial tests have shown that results with this search phrase return suitable research papers without noticeable gabs in regards to the topics covering decentralised ride-sharing .
The complete search phrase looks as followed:
("decentralised" AND ("ride-sharing" OR "ride-pooling" OR "ride-hailing"))

For Scopus, Epscohost (all selectable databases included) and IEEE Xplore, the search phrase was applied to the title, abstract and keywords of the publications. Initial tests have shown that restricting the search to title, abstract and keywords is the best compromise between the quantity and quality of the search results. Only literature that was published in the after 2014 (2015 – 2023) was considered for the literature search. This is to ensure that the specialist literature found is of current relevance without overly restricting the scope of the existing research literature. Likewise, after the initial compilation of the search results, all duplicates were removed. In this way it is avoided that publications are counted twice because they are listed in several literature databases.

\subsection{Selection of the findings}
The literature search was carried out in July 2023 and resulted in 86 hits. A criteria-based selection was made beyond the search phrase and the time limit for the publication of the specialist literature. The exclusion criteria used in the criteria-based selection are: no publications in languages other than English, no panels and comments, no literature that only deals with decentralised platforms or ride-sharing. In addition, publications that are not freely available or accessible via a license from the University of Stuttgart had to be excluded. The inclusion criteria used are: Only publications in English, only publications from 2015 onwards, only papers discussing the technical development of decentralised ride-sharing platforms.
Following Bandara, a first check of the actual relevance of the hits for answering the research question was carried out by screening the title, keywords and abstract. A full-text analysis was then carried out on the literature that was still considered relevant after the initial screening. By applying the inclusion and exclusion criteria in the initial screening and the subsequent full-text analysis, <REAL NUMBER> publications were identified from the 86 search hits that are relevant to answering the research question. Figure <FIGURE NUMBER> shows how the relevant research literature is distributed across the research databases.

\begin{table}[h]
\centering
\caption{Results of the Literature Search}
\begin{tabular}{|l|c|c|c|}
\hline
Scientific Database & Search Results & Excluded Literature & Included Literature \\ \hline
Scopus & 54 & 49 & 5 \\ \hline
Epscohost & 2 & 2 & 0 \\ \hline
IEEE & 30 & 25 & 5 \\ \hline
Citation search &  &  & 3 \\ \hline \hline
Total & 86 & 73 & 13 \\ \hline
\end{tabular}
\end{table}

The analysis of the publications shows that a large number of different approaches are discussed in the scientific literature on how decentralised ride sharing plafroms can be build. In order for the results of the literature analysis to be evaluated and interpreted, the results must first be structured. For this purpose, a concept matrix approach according to Webster and Watson is pursued. Based on the concept matrix approach, the specialist literature identified as relevant is assigned to nine topics relevant for the creation of a decentralised ride-sharing platform:

\begin{itemize}
    \item \textbf{Decentralisation}: Decentralisation is the foundational principle behind a distributed ride-sharing platform. It is important to cover how central intermediaries can be removed, to empowers individual participants, ensuring that the platform operates democratically. 
    
    \item \textbf{Transparency}: Transparency ensures that all transactions and interactions on the platform are open and verifiable. This fosters trust among users and providers, as they can independently verify the correctness of any action. It also discourages fraudulent activities and ensures accountability. It is important to show how transparency can be achieved without neglecting other important requirements like privacy and anonymity 
    
    \item \textbf{Security and Resilience}: For a decentralised platform to gain widespread adoption, it must guarantee the safety and security of its users. Ensuring robust security measures prevents malicious attacks, data breaches, and other vulnerabilities. Resilience ensures that the platform can recover and continue to function even after facing unforeseen challenges or disruptions.
    
    \item \textbf{User and Provider Interaction Protocols}: Clear protocols for interactions between users and providers are crucial for smooth platform operations. These protocols must define the rules of engagement, ensuring that both parties have a consistent and predictable experience. 
    
    \item \textbf{Payments and Service Fees}: A decentralised platform requires a transparent and efficient payment system. It is important to covers how payments are processed, the distribution of service fees, and the mechanisms to ensure that providers are compensated fairly for their services.
    
    \item \textbf{Privacy and Anonymity}:  Users and providers must be assured that their personal and transactions data are protected. The implementation of anonymity features has to ensure that users can engage with the platform without revealing their identities, providing an added layer of security.
    
    \item \textbf{Trust Mechanisms}: In the absence of a central authority, trust mechanisms become vital. These mechanisms, such as reviews, ratings, and reputation systems, must be implemented for users and providers gauge the reliability and credibility of one another, allowing for a trustworthy community.
    
    \item \textbf{Off-Chain Edge Cases}: Decentralised platforms often rely on blockchain technology. However, not all transactions or interactions are suitable for on-chain processing due to cost or efficiency concerns. Therfore it is important to address off-chain edge cases. so that the platform can handle scenarios that are outside its primary framework, enhancing its versatility.
    
    \item \textbf{Prototypical Realization}: Before full deployment, a decentralised platform should undergo prototypical realization to test its feasibility, identify potential issues, and gather user feedback. This iterative process is needed to ensures that the final platform is robust, user-friendly, and meets the needs of its target audience.
\end{itemize}


As a result, the concept matrix shows the frequency with which the concepts dealt with in the specialist literature are distributed over the nine topics of decentralised ride-sharing. The assignment of the concepts on the x-axis to authors of the relevant specialist literature on the y-axis can be seen in Figure 2 via the "x" markings.


\begin{longtable}{p{5cm}l||l|l|l|l|l|l|l|l|l}
\caption{Results of the Literature Search} \\
\rotatebox{0}{Research Paper / Topic} & 
\rotatebox{90}{} &
\rotatebox{90}{Blockchain Utilisation} & 
\rotatebox{90}{Transparency} & 
\rotatebox{90}{User and Provider Interaction Protocols} & 
\rotatebox{90}{Payments and Service Fees} & 
\rotatebox{90}{Privacy and Anonymity} & 
\rotatebox{90}{Security and Resilience} & 
\rotatebox{90}{Trust Mechanisms} & 
\rotatebox{90}{Off-Chain Edge Cases} & 
\rotatebox{90}{Prototypical Realization} \\ 
\hline
\endfirsthead

\multicolumn{11}{c}%
{{\bfseries \tablename\ \thetable{} -- continued from previous page}} \\
\hline
\rotatebox{0}{Research Paper / Topic} & 
\rotatebox{90}{} &
\rotatebox{90}{Blockchain Utilisation} & 
\rotatebox{90}{Transparency} & 
\rotatebox{90}{User and Provider Interaction Protocols} & 
\rotatebox{90}{Payments and Service Fees} & 
\rotatebox{90}{Privacy and Anonymity} & 
\rotatebox{90}{Security and Resilience} & 
\rotatebox{90}{Trust Mechanisms} & 
\rotatebox{90}{Off-Chain Edge Cases} & 
\rotatebox{90}{Prototypical Realization} \\ 
\hline
\endhead

\hline \multicolumn{11}{r}{{Continued on next page}} \\
\endfoot

\hline
\endlastfoot

B-Ride: Ride Sharing With Privacy-Preservation, Trust and Fair Payment Atop Public Blockchain & ~\cite{Baza.2021} & X & X & X & X & O & X & X & O & O \\
\hline
Application of Blockchain Technology to Smart City Service: A Case of Ridesharing & ~\cite{Chang.} & X & X & X & O & X & X & O & O & X \\
\hline
Ride-Hailing for Autonomous Vehicles: Hyperledger Fabric-Based Secure and Decentralize Blockchain Platform & ~\cite{Shivers.} & X & X & O & O & X & X & O & X & O \\
\hline
RiderS: Towards a Privacy-Aware Decentralized Self-Driving Ride-Sharing Ecosystem & ~\cite{Bathen.} & X & X & O & O & X & X & O & X & O \\
\hline
A Decentralized Ride-Hailing Mode Based on Blockchain and Attribute Encryption & ~\cite{Zhang.} & X & X & O & O & X & X & O & X & O \\
\hline
Enhancing Blockchain-based Ride-Sharing Services using IPFS & ~\cite{Mahmoud.2022} & X & X & O & O & X & X & O & X & O \\
\hline
BlockWheels - A Peer to Peer Ridesharing Network & ~\cite{Joseph.} & X & X & O & O & X & X & O & X & O \\
\hline
A Light Blockchain-Powered Privacy-Preserving Organization Scheme for Ride Sharing Services & ~\cite{Baza.52520205282020} & X & X & O & O & X & X & O & X & O \\
\hline
\pagebreak
BlockV: A Blockchain Enabled Peer-Peer Ride Sharing Service & ~\cite{Pal.} & X & X & O & O & X & X & O & X & O \\
\hline
Blockchain-Based Ride-Sharing System with Accurate Matching and Privacy-Preservation & ~\cite{Badr.} & X & X & X & O & X & X & O & O & X  \\
\hline
Towards Blockchain-based Ride-sharing Systems & ~\cite{Vazquez.} & X & X & O & O & X & X & O & X & O \\
\hline
Co-utile P2P ridesharing via decentralization and reputation management & ~\cite{Sanchez.2016} & X & X & O & O & X & X & O & X & O \\
\hline
\end{longtable}

\subsection{Scientific Literature findings}


\textbf{B-Ride: Ride Sharing With Privacy-Preservation, Trust and Fair Payment Atop Public Blockchain}

The research paper introduces B-Ride, a decentralized ride-sharing service built on public Blockchain. Unlike traditional centralized systems, B-Ride doesn't rely on a trusted third party, addressing vulnerabilities and high service fees. B-Ride ensures ride data privacy for both drivers and riders. To counter malicious users exploiting blockchain's anonymity, the system introduces a time-locked deposit protocol using smart contracts and zero-knowledge set membership proof. This ensures trust and commitment from all participants. A unique "pay-as-you-drive" methodology is proposed for fair payment, where drivers are compensated based on the distance covered. Additionally, B-Ride features a decentralized reputation management mechanism, rating drivers on past behavior, incentivizing them to maintain good conduct. The system was successfully implemented and tested on the Ethereum blockchain, highlighting its real-world applicability.

\textbf{Application of Blockchain Technology to Smart City Service: A Case of Ridesharing}

The authors, Shuchih Ernest Chang and Chi-Yin Chang, highlight the challenges faced by traditional ridesharing platforms, such as Uber and Lyft, including regulatory and management issues. To address these challenges, the SmaRi system leverages blockchain technology and smart contracts. This approach not only ensures secure and automated transactions but also promotes decentralized decision-making, shifting power from centralized entities to individual users. The research emphasizes the potential of blockchain in reshaping ridesharing services, aligning with the core values of the sharing economy and moving society closer to the vision of smart cities.

\textbf{Ride-Hailing for Autonomous Vehicles: Hyperledger Fabric-Based Secure and Decentralize Blockchain Platform}

To address the problems of centralised ride-sharing platforms, the authors propose a decentralized approach using blockchain technology, allowing individual AV owners to contribute their vehicles to a community-driven fleet when not in use. The chosen blockchain platform for this endeavor is Hyperledger Fabric, prized for its permissioned and secure nature. This framework isn't just limited to AVs; it's designed to be versatile, catering to both AV and traditional human-driven ride-hailing systems. Through rigorous evaluations, the framework demonstrated robust security and efficient performance under heavy network loads. In essence, the paper underscores the potential of a decentralized, blockchain-driven ride-hailing system, paving the way for a more secure, scalable, and community-centric transportation future.

\textbf{RiderS: Towards a Privacy-Aware Decentralized Self-Driving Ride-Sharing Ecosystem}

The research paper introduces "RiderS," a groundbreaking decentralized self-driving ride-sharing ecosystem. Central to this is the emphasis on user privacy, achieved through a privacy-first biometric technology. Instead of traditional passwords, users become their own unique identifier, ensuring genuine system interactions. To fortify this ecosystem, blockchain technology is employed, offering benefits like decentralization and auditability. Each participant, whether a rider or an autonomous vehicle (AV), accesses the system via a "Wallet." This software client manages credentials, facilitates transactions, and serves as the primary gateway into the blockchain. Monetary exchanges within this ecosystem utilize a stable coin named "Mobi," anchored to various cryptocurrencies and fiat currencies. The system supports diverse transactions, from registering biometrics to confirming rides. A standout feature is the emphasis on privacy. Users can generate single-use addresses, ensuring anonymity for each ride. Furthermore, to validate transactions, AVs provide biometric proofs for pickups and drop-offs, ensuring the correct client is serviced

\textbf{A Decentralized Ride-Hailing Mode Based on Blockchain and Attribute Encryption}
The research paper presents a novel ride-hailing approach using blockchain and attribute encryption. The system comprises:

Decentralized Blockchain-Based Ride-Hailing Mode: This mode has roles such as the \textit{Passenger}, who generates encrypted ride details; the \textit{Driver}, who decrypts and decides on ride acceptance; the \textit{Location Prover (LP)}, verifying the driver's location; and the \textit{Authentication Center (AC)}, distributing keys and authenticating identities.
Matching Phase: Passengers and drivers match on the blockchain. Passengers submit encrypted ride details, and drivers with matching attributes can decrypt and decide on ride acceptance.

Deposit Payment Phase: A protocol requiring deposits to deter malicious behavior. Drivers send a deposit to a smart contract upon reaching the designated location. Failure to arrive results in the passenger receiving the deposit.

Fair Payment Phase: The journey is segmented, and passengers pay for each segment, ensuring fairness.

Reputation Calculation Phase: Smart contracts compute user reputation based on driver performance metrics.

In summary, the paper introduces a decentralized ride-hailing model leveraging blockchain and attribute encryption, emphasizing security, transparency, and fairness.


\textbf{Enhancing Blockchain-based Ride-Sharing Services using IPFS}

The authors propose a decentralized ride-sharing system to address challenges in centralized services, such as security concerns and single points of failure. The solution integrates blockchain with the Interplanetary File System (IPFS). Instead of storing all ride-sharing data on the blockchain, the system moves this data to IPFS and only retains a compact hash on the blockchain. This approach reduces data storage on the blockchain, leading to faster processing and lower costs. The system uses smart contracts on the Ethereum platform for management, and experimental results highlight its scalability and efficiency. In essence, the paper offers a secure and efficient decentralized ride-sharing solution by combining blockchain and IPFS.

\textbf{BlockWheels - A Peer to Peer Ridesharing Network}

The paper introduces a novel ridesharing system built on the Ethereum blockchain, aiming to revolutionize the traditional centralized model. The system is inherently decentralized, ensuring no single entity has overarching control, thereby enhancing user trust. This decentralization also guarantees user privacy, as data isn't exposed to any unnecessary external entities. Payments are streamlined and secure, facilitated through cryptocurrency wallets, eliminating the need for physical cash. One of the standout features is its transparency, especially in fare calculations, addressing the common confusion users face with fare surges in traditional platforms. The system boasts quick transactions, with riders and drivers connecting directly without intermediaries, making the process efficient and cost-effective. A sophisticated ride-matching system is in place, utilizing geolocation tools to swiftly pair riders with nearby drivers. Lastly, while the blockchain stores essential user and ride details, auxiliary data, like public keys, are stored in external databases to ensure operational efficiency. In sum, BlockWheels offers a transparent, efficient, and user-centric approach to ridesharing, addressing many pitfalls of its centralized counterparts.

\textbf{A Light Blockchain-Powered Privacy-Preserving Organization Scheme for Ride Sharing Services}

 The authors introduce a decentralized system using public blockchain, eliminating the central third-party vulnerabilities. This system ensures location and time privacy by employing spatial and temporal cloaking techniques, allowing riders and drivers to share generalized locations and time intervals instead of exact details. The matching process between riders and drivers is transparently conducted on the blockchain, with riders publishing their requests and drivers submitting encrypted offers. To further enhance privacy, participants use changing pseudonyms for each trip, ensuring untraceability. The entire scheme has been practically implemented and tested on the Ethereum platform, showcasing its feasibility and effectiveness in real-world scenarios.

\textbf{BlockV: A Blockchain Enabled Peer-Peer Ride Sharing Service}

The authors introduces a decentralized ride-sharing solution using blockchain. BlockV ensures fairness in ride-sharing in two main ways:

Payment Fairness: It allows any network peer to compute the ride cost based on path details.
Ride Fairness: In case of disputes, the system collaborates with Road Side Units (RSUs) to determine and penalize any malicious activity by drivers or riders.
The BlockV system involves four key participants: the DRIVER, RIDER, BlockV, and RSUs. The process starts with riders selecting a route and fare from a decentralized database. Once chosen, they confirm the ride and lock in the fare. At the ride's end, riders can either complete the ride, releasing funds, or raise complaints if unsatisfied. The system then verifies complaints using RSUs and takes appropriate action.

In essence, BlockV offers a transparent and fair solution for ride-sharing, potentially benefiting platforms like Uber and Lyft by leveraging blockchain technology.

\textbf{Blockchain-Based Ride-Sharing System with Accurate Matching and Privacy-Preservation}

At the heart of their proposal is a decentralized system that eliminates the vulnerabilities of a central trusted unit, a common weak point in traditional ride-sharing platforms. Their ingenious method divides the ride-sharing coverage area into small cells using an overlapping grids strategy. This ensures that drivers and riders are matched with pinpoint accuracy, as they report their locations by cell numbers. When their exact locations coincide within a common cell across any grid, a match is made.

But what truly sets this system apart is its commitment to user privacy. Each participant, whether a driver or a rider, encrypts their ride requests or offers using a lightweight cryptosystem before dispatching them to the blockchain. A smart contract on the blockchain then takes over, executing a matching algorithm on these encrypted requests. This process organizes shared rides without ever decrypting the actual data, ensuring that while rides are matched seamlessly, the privacy of the users remains uncompromised.

The team's contributions to the field are twofold. Firstly, they've achieved unparalleled accuracy in matching ride requests and offers, thanks to their unique overlapping grids and time vectors strategy. Secondly, they've set a new standard in privacy preservation in the ride-sharing domain. Their system not only guarantees location privacy for both drivers and riders but does so without sacrificing the precision of ride matching.

\textbf{Towards Blockchain-based Ridesharing Systems}
Instead of relying on a centralized system, where a single authority manages all transactions, their proposal distributes data across multiple network nodes, enhancing security and transparency. Central to this system are smart contracts, which are self-executing digital agreements with terms directly written in code. These contracts not only facilitate ride requests but also ensure that transactions between drivers and passengers are automated and secure. To address privacy concerns, the system employs spatial cloaking, a technique that blurs exact location data, ensuring users' privacy is maintained. When a passenger requests a ride, an off-blockchain algorithm matches them with suitable drivers based on this cloaked data. To foster a sense of trust, both parties, the driver and the passenger, post a deposit fee through a smart contract. This deposit acts as a commitment, and if either party defaults, the other is automatically compensated. As the journey progresses, the system uses smart contracts to make regular partial payments to the driver, ensuring a transparent payment process. Finally, upon the ride's completion, a smart contract verifies the journey's success and transfers the appropriate fee to the driver. The infrastructure of this innovative system assumes the use of smartphones by both drivers and passengers, equipped with GPS and internet connectivity, and also incorporates a location-based service provider to validate the driver's whereabouts. Through this blockchain-based approach, the authors envision a more transparent, secure, and trustworthy ride-sharing experience for all users.

\textbf{Co-utile P2P ridesharing via decentralization and reputation management}
The research paper delves into the challenges that the ridesharing industry faces, especially the paradoxical decline in its adoption. This decline is attributed to growing privacy concerns and a palpable lack of trust among users. To address these challenges, the authors introduce a fully decentralized Peer-to-Peer (P2P) ridesharing system. This system is devoid of central matching agencies, which often compile, aggregate, and potentially exploit users' data. By decentralizing, the system not only assuages privacy concerns but also eradicates a central point of vulnerability susceptible to external attacks.

A standout feature of this system is its commitment to preserving user privacy. In practice, this means that only when a driver's and passenger's trips align will they be privy to each other's identity, desired trip details, and reputation. This selective disclosure ensures that personal data remains confidential. But addressing privacy alone isn't enough; trust is equally pivotal. The authors tackle this by weaving in a decentralized reputation management mechanism. Post a shared ride, both drivers and passengers have the liberty to rate each other. This allows peers to gauge the aggregated reputation of others, based on historical ratings, in a manner that's both transparent and trustworthy.

To underpin the system's integrity, the authors employ the principle of co-utility. This principle posits that individuals maximize their benefits when they assist others in maximizing theirs. Translated to the realm of ridesharing, it ensures that users, driven by rationality, find it in their best interest to adhere to the system's protocols. In essence, the authors believe that this blend of decentralization, privacy preservation, trust-building through reputation management, and the co-utility principle can rejuvenate the ridesharing landscape, making it more appealing and trustworthy for all participants.

