Due to the diverse literature regarding decentralised ride-pooling platforms, the proven approach of a systematic literature search according to Vom Brocke was chosen~\cite{vomBrocke.2009}. In this way, quality criteria such as traceability and reproducibility can be ensured through a clearly defined processes. Two common, cross-publisher research databases and one common publisher database were used for the literature search.
The selection of several cross-publisher research databases is intended to ensure that the search provides a representative overview of existing research on decentralised ride-pooling platforms. The selection of the database of a publisher with a focus on information technology is intended to show how the research topic is treated in the literature from a primarily information technology perspective. The cross-publisher research databases used are Scopus and Ebscohost. The publisher database is IEEE Xplore.
The goal is to obtain research literature as a search result that deals with the development of decentralised ride-pooling platforms. To obtain results covering mainly decentralised platforms the search phrase ``decentralised'' was used. The following three synonyms were used to obtain search results that deal with the topic of ride-pooling: ``ride-sharing'', ``ride-pooling'' and ``ride-hailing''. Initial tests have shown that results with this search phrase return suitable research papers without noticeable gaps in regards to the topics covering decentralised ride-pooling.

The complete search phrase looks as followed:
\begin{verbatim}
("decentralised" AND ("ride-pooling" OR "ride-sharing" OR "ride-hailing"))
\end{verbatim}

For Scopus, Epscohost (all selectable databases included) and IEEE Xplore, the search phrase was applied to the title, abstract and keywords of the publications. Initial tests have shown that restricting the search to title, abstract and keywords is the best compromise between the quantity and quality of the search results. Only literature that was published in the after 2014 (2015 – 2023) was considered for the literature search. This is to ensure that the specialist literature found is of current relevance without overly restricting the scope of the existing research literature. Likewise, after the initial compilation of the search results, all duplicates were removed. In this way it is avoided that publications are counted twice because they are listed in several literature databases.

\subsection{Selection of the findings}
The literature search was carried out between the 28. July and the 18. of August 2023, resulting in 86 hits. A criteria-based selection was made beyond the search phrase and the time limit for the publication of the specialist literature. The exclusion criteria used in the criteria-based selection are no publications in languages other than English, no panels and comments, and no literature dealing with decentralised platforms or ride-pooling. In addition, publications that are not freely available or accessible via a license from the University of Stuttgart had to be excluded. The inclusion criteria used are Only publications in English, only publications from 2015 onwards, and only papers discussing the technical development of decentralised ride-pooling platforms~\cite{Bandara.2015}.
Following Bandara, a first check of the actual relevance of the hits for answering the research question was carried out by screening the title, keywords and abstract. A full-text analysis was then carried out on the literature that was still considered relevant after the initial screening. Applying the inclusion and exclusion criteria in the initial screening and the subsequent full-text analysis, 10 relevant publications were identified from the 86 search hits for answering the research question. Additionally, two more relevant papers could be identified by following citations from the relevant literature.  Table \ref{tab:litSearchResults} shows how the relevant research literature is distributed across the research databases. 


\begin{table}[h]
\centering
\caption{Results of the Literature Search}
\label{tab:litSearchResults}
\begin{tabular}{|l|c|c|c|}
\hline
Scientific Database & Search Results & Excluded Literature & Included Literature \\ \hline
Scopus & 54 & 49 & 5 \\ \hline
Epscohost & 2 & 2 & 0 \\ \hline
IEEE & 30 & 25 & 5 \\ \hline
Citation search &  &  & 2 \\ \hline \hline
Total & 86 & 73 & 12 \\ \hline
\end{tabular}
\end{table}

The analysis of the publications shows that many different approaches are discussed in the scientific literature on how decentralised ride-pooling platforms can be built. For the results of the literature analysis to be evaluated and interpreted, the results must first be structured. For this purpose, a concept matrix approach, according to Webster and Watson is pursued~\cite{Webster.2002}. Based on the concept matrix approach, the specialist literature identified as relevant is assigned to eight topics relevant for the creation of a decentralised ride-pooling platform. These eight topics are derived from a general analysis of the topics covered by the scientific literature combined with topics relevant to fulfilling the research objectives:

\begin{itemize}
    \item \textbf{Blockchain Utilisation}: Blockchain is the underlying technology used for the creation of the decentralised ride-pooling platform. The literature needs to show in detail how blockchain technology is utilised by smart contracts and cryptocurrencies to build a ride-pooling platform.
    
    \item \textbf{Payments and Service Fees}: The decentralised ride-pooling platform must manage ride payments and general service fees. Therefore it is important for the literature to show how these financial transactions can be implemented and how to ensure that ride providers are compensated fairly for their services inside the decentralised ride-pooling platform.
    
    \item \textbf{Privacy and Anonymity}: Using blockchain technology demands a robust architecture that ensures privacy and anonymity for all users inside the platform. The scientific literature must showcase how users can engage with the platform and other users without revealing their identity directly or implicitly by sharing too much personal data with the platform over a longer time period.
    
    \item \textbf{Security and Resilience}: For a decentralised platform to gain widespread adoption, it must guarantee the safety and security of all parties. While the blockchain itself already provides many security features by design, it is important for the literature to show how the off-chain components are hardened and how to prevent the off-chain components from providing false information to the on-chain components.
    
    \item \textbf{Trust Mechanisms}: As decentralised platforms can not rely on a central trusted authority, robust trust mechanisms become essential. The research papers must explain how community trust mechanisms can be successfully implemented into a decentralised platform.
    
    \item \textbf{Off-Chain Edge Cases}: It is impossible to handle every edge case through the decentralised platform. As there is no central authority, it is important to provide alternative solutions to solve these problems without contradicting the decentralised nature of the platform. The research needs to recognise the existence of these edge cases and has to provide solutions to handle them.
    
    \item \textbf{Customer and Ride Provider Interaction Flow}: The customer and ride provider interaction flow stands in the centre of the decentralised ride pooling platform. The literature needs to provide insides into how this flow should look to utilise the advantages of blockchain technology.
    
    \item \textbf{Prototypical Realization}: Before building a market-ready version, the decentralised ride-pooling platform should be built as a prototype that showcases the most important aspects of the platform and proves its feasibility. Therefore it is important for the literature to include a prototypical realisation of the platform that provides important insides that are not which cannot be derived from the architecture alone.
\end{itemize}


As a result, the concept matrix shows the frequency with which the concepts dealt with in the specialist literature are distributed over the nine topics of decentralised ride-pooling. The assignment of the concepts on the x-axis to authors of the relevant specialist literature on the y-axis can be seen in Table \ref{tab:litSearchResultsMatrix}. If a research paper covers a topic in detail, it is marked with $\checkmark$ $\checkmark$. if a research paper covers some aspects of a topic, it is marked with a $\checkmark$ . If a paper does not cover a topic at all or in a way that does not align with the objectives of this research, it is marked with a $\times$. 


\begin{longtable}{p{5cm}l|l|l|l|l|l|l|l|l}
\caption{Results of the Literature Search} \\
\label{tab:litSearchResultsMatrix}
Research Paper / Topic & 
&
\rotatebox{90}{Blockchain Utilisation} & 
\rotatebox{90}{Customer and Ride Provider Interaction Flow} & 
\rotatebox{90}{Payments and Service Fees} & 
\rotatebox{90}{Privacy and Anonymity} & 
\rotatebox{90}{Security and Resilience} & 
\rotatebox{90}{Trust Mechanisms} & 
\rotatebox{90}{Off-Chain Edge Cases} & 
\rotatebox{90}{Prototypical Realization} \\ 
\hline
\endfirsthead

\multicolumn{9}{c}%
{{\bfseries \tablename\ \thetable{} -- continued from previous page}} \\
\hline
Research Paper / Topic & 
&
\rotatebox{90}{Blockchain Utilisation} & 
\rotatebox{90}{Customer and Ride Provider Interaction Flow} & 
\rotatebox{90}{Payments and Service Fees} & 
\rotatebox{90}{Privacy and Anonymity} & 
\rotatebox{90}{Security and Resilience} & 
\rotatebox{90}{Trust Mechanisms} & 
\rotatebox{90}{Off-Chain Edge Cases} & 
\rotatebox{90}{Prototypical Realization} \\ 
\hline
\endhead

\hline \multicolumn{9}{r}{{Continued on next page}} \\
\endfoot

\hline
\endlastfoot


B-Ride: Ride Sharing With Privacy-Preservation, Trust and Fair Payment Atop Public Blockchain & ~\cite{Baza.2021} & $\checkmark$ $\checkmark$ & $\checkmark$ $\checkmark$ & $\checkmark$  & $\times$ & $\times$ & $\checkmark$ & $\times$ & $\checkmark$  \\
\hline
Application of Blockchain Technology to Smart City Service: A Case of Ridesharing & ~\cite{Chang.} & $\checkmark$ & $\checkmark$ & $\times$ & $\checkmark$  & $\checkmark$ $\checkmark$ & $\times$ & $\times$ & $\times$ \\
\hline
Ride-Hailing for Autonomous Vehicles: Hyperledger Fabric-Based Secure and Decentralize Blockchain Platform & ~\cite{Shivers.} & $\checkmark$ & $\checkmark$ & $\times$ & $\checkmark$ & $\times$ & $\times$ & $\times$ & $\checkmark$  \\
\hline
RiderS: Towards a Privacy-Aware Decentralized Self-Driving Ride-Sharing Ecosystem & ~\cite{Bathen.} & $\checkmark$ & $\checkmark$ & $\checkmark$ & $\checkmark$$\checkmark$ & $\checkmark$ & $\times$ & $\times$ & $\checkmark$ \\
\hline
A Decentralized Ride-Hailing Mode Based on Blockchain and Attribute Encryption & ~\cite{Zhang.} & $\checkmark$& $\checkmark$ & $\checkmark$ $\checkmark$ & $\checkmark$ & $\times$ & $\times$ & $\times$ & $\checkmark$  \\
\hline
Enhancing Blockchain-based Ride-Sharing Services using IPFS & ~\cite{Mahmoud.2022} & $\checkmark$$\checkmark$ & $\checkmark$$\checkmark$ & $\checkmark$ & $\checkmark$ & $\checkmark$ & $\times$ & $\times$ & $\checkmark$  \\
\hline
BlockWheels - A Peer to Peer Ridesharing Network & ~\cite{Joseph.} & $\checkmark$ & $\checkmark$$\checkmark$ & $\checkmark$ & $\checkmark$ & $\times$  & $\times$  & $\times$ & $\times$  \\
\hline
A Light Blockchain-Powered Privacy-Preserving Organization Scheme for Ride Sharing Services & ~\cite{Baza.52520205282020} & $\checkmark$ & $\checkmark$ $\checkmark$ & $\checkmark$ & $\checkmark$ & $\times$ & $\times$ & $\times$ & $\times$ \\
\hline
\pagebreak
BlockV: A Blockchain Enabled Peer-Peer Ride Sharing Service & ~\cite{Pal.} & $\checkmark$ & $\checkmark$ & $\checkmark$ & $\times$  & $\checkmark$ & $\checkmark$  &  $\checkmark$ &  $\checkmark$ \\
\hline
Blockchain-Based Ride-Sharing System with Accurate Matching and Privacy-Preservation & ~\cite{Badr.} &  $\checkmark$ & $\checkmark$ & $\times$ & $\checkmark$  & $\checkmark$ $\checkmark$ & $\times$ & $\times$ & $\times$ \\
\hline
Towards Blockchain-based Ride-sharing Systems & ~\cite{Vazquez.} & $\checkmark$ $\checkmark$ & $\checkmark$ & $\checkmark$ & $\checkmark$ & $\checkmark$ & $\times$ & $\times$ & $\checkmark$  \\
\hline
Co-utile P2P ridesharing via decentralization and reputation management & ~\cite{Sanchez.2016} &  $\checkmark$ &  $\checkmark$ &  $\checkmark$ &  $\checkmark$  $\checkmark$ &  $\checkmark$ &  $\checkmark$ & $\times$ & $\times$   \\
\hline
\end{longtable}

\subsection{Scientific Literature findings}
The concept matrix \ref{tab:litSearchResultsMatrix} shows that the literature review did not identify a single paper that provides detailed coverage of all topics and would thereby allow us to answer all research objectives. The matrix also shows that while many of the papers discuss multiple topics, they often remain on a conceptual level without the goal of developing a feature-complete platform. It is still very important to take a detailed look at the identified literature to discuss their approaches on developing a decentralised ride-pooling platform. In the following we will take a look at the outstanding features that are proposed in each paper and evaluate how they can support the creation of our feature complete ride polling service.


``B-Ride: Ride Sharing With Privacy-Preservation, Trust and Fair Payment Atop Public
introduces'' B-Ride, a decentralized ride-sharing service built on public Blockchain~\cite{Baza.2021}. B-Ride ensures ride data privacy for both drivers and riders. To counter malicious users exploiting blockchain's anonymity, the system introduces a time-locked deposit protocol using smart contracts and zero-knowledge set membership proof. This ensures trust and commitment from all participants. A unique "pay-as-you-drive" methodology is proposed for fair payment, where drivers are compensated based on the distance covered. This 
system has many advantages. It ensures that the ride provider gets paid for the driven distance, and the customer does not have to deposit more money than necessary at once. The problem with this approach is, that it requires so called Location Prover. These hardware devices ensure that the car provides honest location information about its position. While this technology is superior to systems that do not relay on Location Prover, a global network of Location Provers is currently not feasible. Therefore our platform will utilise an upfront deposit of the expected ride cost by the user that can be claimed by the ride provider after completing the ride.
Additionally, B-Ride features a decentralized reputation management mechanism, rating drivers on past behavior, incentivizing them to maintain good conduct. The system was successfully implemented and tested on the Ethereum blockchain, highlighting its real-world applicability. While a rating system is needed to ensure trust on the platform, B-Rides implementation also relies on Location Provers. Therefore we will look at other research papers and their approaches in regards to rating mechanisms.

The authors, Shuchih Ernest Chang and Chi-Yin Chang, highlight in their research paper ``Application of Blockchain Technology to Smart City Service: A Case of Ridesharing'' ~\cite{Chang.} the challenges faced by traditional ridesharing platforms. To address the challenges of traditional ridesharing platforms, the SmaRi system leverages blockchain technology and smart contracts. This approach not only ensures secure and automated transactions but also promotes decentralized decision-making. The research emphasizes the potential of blockchain in reshaping ridesharing services. A notable design decision by the authors is to use an off-chain authentication service called social networking service. This service allows users to utilise social media accounts to share rides with friends and to authenticate against the platform. While this concept is not covered in depth it provides insides into the many advantages of an off-chain authentication service.

The paper ``Ride-Hailing for Autonomous Vehicles: Hyperledger Fabric-Based Secure and Decentralize Blockchain Platform'' addresses the problems of centralised ride-sharing platforms. The authors propose a decentralized approach using blockchain technology, allowing individual AV owners to contribute their vehicles to a community-driven fleet when not in use.~\cite{Shivers.} The chosen blockchain platform for this endeavour is Hyperledger Fabric. The paper is notable for utilizing a private blockchain to tackle the problems in regards to anonymity and privacy, which are inherent downsides of public blockchains. The decision between a public and a private blockchain is one of the core architectural decisions for our own ride-sharing platform.  After taking the arguments by ~\cite{Shivers.} as well as other research papers into consideration, we decided to go forward with a public blockchain for our platform. With privacy being a focus of our ride pooling platform, there should be no possibility to trace individual user activity by monitoring the chain activities, even if it is public. Therefore we prioritise the increased decentralisation of public chains. Using a public chain allows us to utilize generic public nodes to handle smart contracts. Thereby we do not need to build a private network of independent node providers to build a private blockchain. Other research papers also prove the feasibility of decentralised ride pooling platforms on public chains ~\cite{Mahmoud.2022} ~\cite{Joseph.} ~\cite{Baza.52520205282020}

Another research paper introduces ``RiderS, a groundbreaking decentralized self-driving ride-sharing ecosystem'' ~\cite{Bathen.}.Central to this is the emphasis on user privacy, achieved through a privacy-first biometric technology. Instead of traditional passwords, users become their own unique identifier, ensuring genuine system interactions. To fortify this ecosystem, blockchain technology is employed, offering benefits like decentralization and auditability. Each participant, whether a rider or an autonomous vehicle (AV), accesses the system via a ''Wallet''. This software client manages credentials, facilitates transactions, and serves as the primary gateway into the blockchain. Monetary exchanges within this ecosystem utilize a stable coin named ''Mobi'', anchored to various cryptocurrencies and fiat currencies. This system is very usefull and should be adapted by our platform. By introducing a Crypto Exchnage to the platform we allow the users to pay with a verity of different currencies including fiat currencies while still utilising the advantages of crypto currencies   in our platform. A standout feature is the emphasis on privacy. Users can generate single-use addresses, ensuring anonymity for each ride. This also should be adapted by our platform. Even though the wallet owner is anonymous on the chain, it prevents wallet tracking over a long period of time, which could lead the exposure of the wallet holder.

The research paper ``A Decentralized Ride-Hailing Mode Based on Blockchain and Attribute Encryption'' presents a novel ride-hailing approach using blockchain and attribute encryption.~\cite{Zhang.} 
The system includes a decentralized Blockchain-Based Ride-Hailing Mode: This mode has roles such as the Passenger, who generates encrypted ride details; the Driver, who decrypts and decides on ride acceptance; the Location Prover (LP), verifying the driver's location; and the Authentication Center, distributing keys and authenticating identities. Thereby the paper introduces a number of concepts that help us create our privacy preserving ride pooling platform. First of all the concept of creating a shared secret between customer and ride provider should be used  to share sensitive information on chain, like exact coordinates. With the  Authentication Center the paper also introduces an off chain authentication service, which further promotes the concept of an off chain authority that can verify  wallets to handle on chain nteractions with the ride pooling platform.

The authors of ``Enhancing Blockchain-based Ride-Sharing Services using IPFS''  propose a decentralized ride-sharing system to address challenges in centralized services, such as security concerns and single points of failure.~\cite{Mahmoud.2022}  The solution integrates blockchain with the Interplanetary File System (IPFS). Instead of storing all ride-sharing data on the blockchain, the system moves this data to IPFS and only retains a compact hash on the blockchain. This approach reduces data storage on the blockchain, leading to faster processing and lower costs. The system uses smart contracts on the Ethereum platform for management, and experimental results highlight its scalability and efficiency. This concept should be utalised if the prototype implementation or future iterations of the platforms should struggle with managing the amounts of data necessary to hanlde rides, resulting in high gas prices or slow blockchain performance.


The paper ``BlockWheels - A Peer to Peer Ridesharing Network a ridesharing'' system built on the Ethereum blockchain. ~\cite{Joseph.} introduces a sophisticated ride-matching system, utilizing geolocation tools to pair riders with nearby drivers. While out platform is utilising an auction based approach to match customers with ride providers, this paper showcases the advantages of an off chain matching approach to handle the complex matching with an on chain ride handelig that tracks the actual ride.

The paper ``A Light Blockchain-Powered Privacy-Preserving Organization Scheme for Ride Sharing Services'' introduce a decentralized system using public blockchain, eliminating the central third-party vulnerabilities.~\cite{Baza.52520205282020} This system ensures location and time privacy by employing spatial and temporal cloaking techniques, allowing riders and drivers to share generalized locations and time intervals instead of exact details. This approach should also be utilised with our platform. With a location matching based on approximated data we can ensure that the customer only needs to share their exact location with the ride provider that will fulfill the ride request.
BlockWheels participants also uses changing pseudonyms for each trip, ensuring untraceability. With BlockWheels also promoting this concept it shows that this approach to ensuring untraceability is a best practice in regards to on chain user activities. The entire scheme has been practically implemented and tested on the Ethereum platform, showcasing its feasibility and effectiveness in real-world scenarios.

The authors ~\cite{Pal.}  introduces a decentralized ride-sharing solution using blockchain. BlockV ensures fairness in ride-sharing in two main ways:
Payment Fairness: It allows any network peer to compute the ride cost based on path details.
Ride Fairness: In case of disputes, the system collaborates with Road Side Units (RSUs) to determine and penalize any malicious activity by drivers or riders.
The BlockV system involves four key participants: the DRIVER, RIDER, BlockV, and RSUs. The process starts with riders selecting a route and fare from a decentralized database. Once chosen, they confirm the ride and lock in the fare. At the ride's end, riders can either complete the ride, releasing funds, or raise complaints if unsatisfied. The system then verifies complaints using RSUs and takes appropriate action.
With the RSUs BlockV provides a solution to the problem on how to manage edge-cases like customer complaints. While this concept relies on the existence of RSUs and mainly focuses on the handling of false routes taken by the ride provider it showcases the importance of robust edge case handling.

The paper ``Blockchain-Based Ride-Sharing System with Accurate Matching and Privacy-Preservation'' proposes a method of dividing the ride-sharing coverage area into small cells using overlapping grids ~\cite{Badr.}. This ensures that customers and ride providers are matched with location accuracy, as they report their locations by cell numbers. When their exact locations coincide within a common cell across any grid, a match is made. While this approach does not utilises the planned auction system proposed by our platform it promotes a grid based approach that can help to ensure that potential matching services can be bound to specific areas. By assinging matching services to single tiles in a grid we can assure that each customer can find their local matching service and no matching service can collect data for areas that are too large.


The paper ``Towards Blockchain-based Ridesharing Systems'' addresses privacy concerns, by also utilising spatial cloaking and an off chain matching service.~\cite{Vazquez.} When a passenger requests a ride, an off-blockchain algorithm matches them with suitable drivers based on this cloaked data. To foster a sense of trust, both parties, the ride provider and the customer, post a deposit fee through a smart contract. This deposit acts as a commitment, and if either party defaults, the other is automatically compensated. This flow very much aligns with our vision of interaction flow of our decentralised ride pooling platform. The main advantage of this approach is that it allows for more complex matching algorithms without dramatically increasing gas fees, while still utilising the advantages of blockchain by tracking the actual ride and related payments on chain. 

The research paper ``Co-utile P2P ridesharing via decentralization and reputation management'' focusses on preserving user privacy~\cite{Sanchez.2016}. In practice, this means that only when a driver's and passenger's trips align will they be privy to each other's identity, desired trip details, and reputation. This selective disclosure ensures that personal data remains confidential. This also aligns with the research objectives of our decentralised ride pooling platform and needs to be considered in the final design.
Addressing privacy alone isn't enough; trust is equally important. The authors tackle this by weaving in a decentralized reputation management mechanism. Post a shared ride, both drivers and passengers have the liberty to rate each other. This allows peers to gauge the aggregated reputation of others, based on historical ratings, in a manner that's both transparent and trustworthy. This is a common best practice even with centralised ride sharing platforms. For our decentralised platform the rating should also be managed on chain, as it profits from the tamper proof nature of blockchain.

\subsection{Conclusion}
The detailed literature review show that there are many different approaches on how a decentralised ride pooling platform should be designed, with different authors focusing on different aspects of the platform. While there are many common best practices in regards to safety and user privacy there is also no uniform approach to designing the different components of the platform. While some papers suggest to handle all interactions with the platform on chain others suggest taking some elements off chain to allow for more complex flows. Therefore we can not rely on simply combining the platforms from the research papers into a single, feature complete platform.

Therefore, to create a feature complete ride pooling platform, it will be necessary to make design decisions that will contradict the suggested approaches of some papers to embrace design decisions made by other papers. These decisions will be made based on our research objectives, which state that the maximisation of privacy, security and transparency is the underlying goal of our platform.


