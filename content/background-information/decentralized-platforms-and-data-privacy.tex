Decentralised platforms, at their core, are systems where components, like resources, or operations, are not controlled or managed by a single, central entity. Instead, they are distributed across multiple nodes, with each having equal authority and autonomy. This is in direct contrast with traditional centralised systems, where a single entity or a group of entities holds all the power and control. In the following chapter we will introduce the core concepts behind decentralised platforms and showcase how data privacy can be handled in a system without a central authority ~\cite{Tverdokhlib.2022}.


\subsection{Introduction to Decentralised Platforms}

One of the primary characteristics of decentralised platforms is that it does not have a central point of control. This means that no single entity has the authority to make decisions on their own or changes without consensus from the majority of the network's participants~\cite{SEFRAOUI.2022}. This leads to enhanced security, as the absence of a single point of failure makes the system more resilient to attacks~\cite{Maffiola.2022}. Additionally, decentralised platforms often employ cryptographic techniques to ensure data integrity, privacy, and authentication. This ensures that transactions and interactions on the platform are secure, verifiable, and immutable~\cite{SEFRAOUI.2022}.

Comparing decentralised platforms with centralised systems reveals strong differences in their operational philosophies. Centralised systems, such as traditional databases or web servers, are controlled by a single entity. This central authority has the power to set rules, make changes and grant or deny access~\cite{Maffiola.2022}. While this centralisation can lead to efficiencies in terms of decision-making and simpler system architectures, it also presents vulnerabilities. A single point of failure in a centralised system can lead to the entire system collapse. Moreover, centralisation often results in data silos, where a single entity has control over large amounts of data ~\cite{Maffiola.2022}.

On the other hand, decentralised platforms operate on the principles of democracy and transparency~\cite{SEFRAOUI.2022}. Decisions are made based on consensus algorithms, ensuring that no single participant can dominate or manipulate the system. This democratisation of control can result in trust among users, as the platform operations are transparent ~\cite{Hasan.2022}. Data in decentralised systems is typically stored across multiple nodes, ensuring redundancy and resilience. Even if one or more nodes fail, the system can continue to operate seamlessly ~\cite{Hasan.2022}.

There are several potential benefits for decentralised platforms~\cite{Hasan.2022}. Firstly, they offer enhanced security and resilience due to their distributed nature. The risk of system-wide failures or attacks is significantly reduced~\cite{Maffiola.2022}. Secondly, they promote transparency and trust among users, as decisions are made collectively and openly~\cite{Hasan.2022}. Additionally, decentralised platforms lead to innovations in peer-to-peer transactions, like smart contracts and decentralised applications, that allow for new business models.

However, decentralised platforms come with a set of challenges~\cite{Hasan.2022}. The lack of a central authority can lead to slower decision-making, as achieving consensus can be time-consuming. Additionally, the technology enabling decentralised platforms, such as blockchain, is still maturing, leading to scalability and performance issues~\cite{Hasan.2022}. Interoperability between different decentralised platforms is also a challenge.

\subsection{Data Privacy: Definition and Importance}
Data privacy, at its core, refers to the right of individuals to control or influence what information about them is collected and how it is used. It centres around the rules put in place to protect personal information and ensure that individuals remain in control of it~\cite{Covert.2020}. This concept is crucial for several reasons.

Data privacy is directly linked to personal autonomy and dignity. Personal data can reveal intimate details about an individual's life, preferences, and habits. Ensuring that such information is not misused or mishandled is vital~\cite{Covert.2020}. Without robust data privacy measures, individual rights can be violated, leading to a loss of trust in digital systems and platforms.

Furthermore, in the context of businesses and services, data privacy is important for maintaining consumer trust. Companies that fail to protect user data or misuse it can face significant reputational damage, legal consequences, and financial losses~\cite{Li.2019}. In sectors like decentralised ride-sharing, where users share location data, payment details, and personal preferences, ensuring data privacy can be the difference between a successful platform and one that users abandon due to trust issues~\cite{Li.2019}.

While data privacy is a critical concept, it is essential to differentiate it from related terms like data security and data protection, as they are often used interchangeably but have distinct meanings.
Data security refers to the protective measures and technologies used to secure data from unauthorised access. It focuses on defending data from malicious threats, like hackers, malware, or other cyber-attacks. For instance, using encryption to secure data to prevent unauthorised access is an example of data security practices.

On the other hand, data protection is a broader concept that includes both data privacy and data security. It refers to the policies, procedures, and legal measures designed to ensure that data is collected, stored, and used in a way that respects individual rights and complies with relevant laws and regulations~\cite{Covert.2020}. 

In conclusion, data privacy is the right of individuals to control their personal information and its usage~\cite{Covert.2020}. As we continue to integrate digital platforms into our everyday life, understanding and prioritising data privacy will become even more important.

\subsection{Mechanisms of Data Privacy in Decentralised Blockchain Systems}
Decentralised blockchain systems are a revolutionary technology, offering data transparency, immutability and security. However, the nature of public blockchains, which are open and transparent, creates significant privacy challenges. Every transaction and its associated data are visible to anyone who accesses the blockchain, leading to potential privacy breaches and exposure of sensitive information.

Encryption plays an important role in addressing these challenges. At its core, encryption involves converting data into a code to prevent unauthorised access. In the context of blockchains, wallets consisting of a public key (often linked to an address on the blockchain) and private key are widely used. A public key, visible to everyone, is used to encrypt data, while a private key, known only to the owner, is used to decrypt it. This ensures that only the intended recipient can access the information. Furthermore, end-to-end encryption ensures that data remains encrypted during its entire journey from the sender to the recipient, preventing potential eavesdroppers from accessing the information during transmission ~\cite{Tran.2022b}.

One approach to establishing  an encrypted connection over a public network is the Diffie-Hellman Key Exchange. The Diffie-Hellman Key Exchange, introduced by Whitfield Diffie and Martin Hellman in 1976, is a cryptographic protocol that allows two parties to independently generate a shared secret key over an insecure communication channel. The protocol is based on the mathematical properties of modular arithmetic and discrete logarithm problems. Specifically, given a prime number \( p \) and a base \( g \) (where \( g \) is a primitive root modulo \( p \)), each party selects a private key and computes a public key. The public keys are then exchanged, and each party uses the other's public key along with their own private key to compute the shared secret. The security of the protocol relies on the difficulty of the discrete logarithm problem: while it is computationally easy to generate the public key from the private key, the reverse operation is considered infeasible with current technology when large prime numbers are used ~\cite{1055638}.

Another advanced cryptographic technique employed in blockchains is zero-knowledge proofs (ZKPs). ZKPs allow one party to prove to another that a statement is true without revealing any specific information about the statement itself. For instance, in a transaction, a user can prove they have sufficient funds without revealing the exact amount. This ensures transaction validity while preserving user privacy ~\cite{Tran.2022b}..

Homomorphic encryption offers another layer of privacy. It allows computations to be performed on encrypted data without first decrypting it. The result, when decrypted, remains accurate. This means that blockchain systems can process transactions and maintain data integrity without exposing the actual data, a boon for privacy-centric applications ~\cite{Putra.}.

While public blockchains offer transparency, they create challenges for the development of applications, especially those requiring higher levels of privacy. Private and consortium blockchains emerge as alternatives in such scenarios. Private blockchains restrict participation to selected entities, while consortium blockchains involve multiple organisations governing the network. Both these types limit data visibility to only authorised participants, enhancing data privacy ~\cite{9334132}.

Off-chain storage is another solution to the privacy challenges. Instead of storing all data on the blockchain, only essential information is kept on-chain, while the rest is stored off-chain in secure databases. This reduces the amount of data exposed on the public ledger, ensuring privacy ~\cite{9838289}.

Lastly, layer 2 solutions, built on top of the primary blockchain, offer scalability and privacy improvements. By processing transactions off the main chain and only writing the final state on-chain, these solutions can ensure faster transactions and enhanced privacy ~\cite{10039486}. In conclusion, while decentralised blockchain systems present certain inherent privacy challenges, a combination of cryptographic techniques and architectural solutions can effectively address these concerns.