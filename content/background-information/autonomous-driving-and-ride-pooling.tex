The advent of autonomous driving, once the stuff of science fiction, is now on the cusp of revolutionizing the transportation sector. Coupled with the growing significance of ride-pooling, this technological development promises a transformative impact on urban mobility, energy consumption, and even the very layout of our cities. This section delves into the intricacies of both concepts, exploring their origins, developments, and the potential synergy they hold for the future of transportation.

Autonomous Driving:
Autonomous, or self-driving vehicles, utilize a blend of hardware and software to navigate and control the car without human intervention. Classified into levels 0 to 5, with 5 being fully autonomous, these vehicles rely on intricate systems of sensors, cameras, lidars, and radars. They constantly gather data about their environment, which is then processed by advanced algorithms to make driving decisions.

Historically, the concept of a car driving itself can be traced back to as early as the 1920s, but tangible progress started in the latter half of the 20th century. Projects like the EUREKA Prometheus Project in the 1980s and the DARPA Grand Challenge in the early 2000s played pivotal roles in advancing autonomous technology. Today, major tech firms and automobile companies are in a race, not just to refine this technology, but also to ascertain the ethical, legal, and infrastructural changes required for mass adoption.

The potential benefits of autonomous driving are vast:

Safety: Human error, responsible for a majority of road accidents, could be drastically reduced.
Efficiency: Optimal driving by autonomous cars might reduce traffic congestion and lead to more streamlined traffic flows.
Accessibility: Those unable to drive due to age, disability, or other factors can enjoy independent mobility.
Economic Impact: A reduction in accidents implies decreased costs in healthcare and vehicle repairs.
However, challenges persist. Technical hurdles, legislative barriers, ethical dilemmas (like decision-making in unavoidable accidents), and public skepticism need addressing for a broader acceptance.

Ride-Pooling:
Ride-pooling, distinct from ride-sharing, involves multiple riders sharing a single vehicle trip, where each passenger's destination is likely different, but their routes are similar. It's an evolution of the traditional carpooling concept, made more efficient and scalable by modern technology.

Platforms like UberPool and Lyft Line have popularized ride-pooling in urban environments. The appeal of such services lies in their promise of reduced commuting costs for passengers, decreased traffic congestion, lower carbon emissions due to fewer cars on the road, and the potential for a significant reduction in the need for parking spaces in urban areas.

However, ride-pooling isn't without its set of challenges. Efficient route optimization to ensure minimal detours, balancing demand and supply, and ensuring passenger safety are areas that companies constantly grapple with.

The Synergy of Autonomous Driving and Ride-Pooling:
When these two paradigms converge, we witness a compelling vision for the future of urban mobility.

Efficiency and Cost: With autonomous vehicles at the helm, ride-pooling can achieve unparalleled efficiency. Vehicles can be operational 24/7, reducing the per-trip cost and, by extension, the fare for passengers. The absence of a driver also implies more space for passengers or cargo.

Environmental Impact: Electric autonomous vehicles, combined with ride-pooling, can substantially reduce carbon emissions. Fewer cars would be on the road, and those in operation would be used more efficiently and likely be eco-friendlier.

Urban Planning: The convergence could reshape cities. With fewer vehicles on the road and less need for parking, vast tracts of land could be repurposed for green spaces, recreation, or housing.

Accessibility and Inclusivity: An autonomous ride-pooling system ensures mobility for all, regardless of age, disability, or socioeconomic status, potentially democratizing transportation.

However, the integration is not without potential pitfalls. Job losses, especially for drivers in the ride-sharing industry, the challenge of retrofitting infrastructure to accommodate autonomous vehicles, and the need to build robust, hack-proof systems are among the concerns that need to be addressed.

In conclusion, both autonomous driving and ride-pooling represent not just technological advancements but shifts in our societal approach to transportation. Their convergence holds the promise of a more efficient, eco-friendly, and inclusive transportation landscape. However, the journey there requires careful navigation, balancing the immense potential benefits with the inherent challenges. The next chapters will delve deeper into the technological backbone that can make this vision a reality.