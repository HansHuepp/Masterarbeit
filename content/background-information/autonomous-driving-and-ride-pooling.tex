 The advantages of autonomous driving, combined with the growing significance of ride-pooling, promise to have a strong impact on urban mobility~\cite{Stamadianos.2023}. We dive into the details of both concepts, exploring their origins, developments, and the potential synergy they hold for the future of transportation.

\subsection{Autonomous Driving}
Autonomous, or self-driving vehicles, combine hardware and software to navigate and control the car without human intervention~\cite{Stamadianos.2023}. Classified into levels 0 to 5, with five being fully autonomous, these vehicles rely on intricate systems of sensors, cameras, lidars, and radars. They continuously gather data about their environment, which is then processed by advanced algorithms to make driving decisions~\cite{Hacohen.2022}.
Projects like the EUREKA Prometheus Project in the 1980s and the DARPA Grand Challenge in the early 2000s played essential roles in developing autonomous technologies~\cite{Hacohen.2022}. Today, significant tech and automobile companies compete to build fully autonomous vehicles for mass adoption~\cite{Stamadianos.2023}.
The potential benefits of autonomous driving are vast:
\begin{description}
    \item[Safety:] Human error, responsible for most road accidents, could be drastically reduced~\cite{Hacohen.2022}.
    \item[Efficiency:] Optimal driving by autonomous cars might reduce traffic congestion and lead to more streamlined traffic flows~\cite{Stamadianos.2023}.
    \item[Accessibility:] Those unable to drive due to age, disability, or other factors can enjoy independent mobility~\cite{Hacohen.2022}.
    \item[Economic Impact:] A reduction in accidents implies decreased costs in healthcare and vehicle repairs~\cite{Stamadianos.2023}.
\end{description}

However, challenges still exist. Technical complications, legal barriers, ethical questions (like decision-making in unavoidable accidents), and public scepticism must be addressed for a broader acceptance~\cite{Hacohen.2022}.

\subsection{Ride-Pooling}
Ride-pooling allows for multiple people to share a single vehicle for a trip, where all passengers have different destinations but share a similar route~\cite{Perivier.}. 
Platforms like UberPool and Lyft Line have popularised ride-pooling in urban environments. The appeal of such services lies in their promise of reduced cost of travel for passengers, decreased overall traffic, lower carbon emissions, and the potential reduction of occupied parking spaces~\cite{Shaheen.}. However, ride-pooling is not without its challenges. Efficient route optimisation to ensure minimal detours, balancing demand and supply, and ensuring passenger safety are areas that ride-pooling providers struggle with~\cite{Perivier.}. 

The synergy of autonomous driving and ride-pooling offers a promising vision of the future of urban mobility~\cite{Stamadianos.2023}. Autonomous vehicles offer more efficiency while also reducing the overall cost of operation without the need for a human driver. This also allows for more overall vehicle space for additional passengers and cargo. In addition, autonomous vehicles can potentially work around the clock without downtime or increased prices for night trips ~\cite{Hacohen.2022,Stamadianos.2023}. Regarding environmental impact, the combination of electric and autonomous vehicles, when integrated with ride-pooling services, allows  for a reduction in environmental pollution \cite{Hacohen.2022,Stamadianos.2023}. 
 
Autonomous ride-pooling also has an effect on city planning, as the design of modern cities is largely centred around vehicles moving and parking. Through a reduction of overall traffic and a sharp decline of needed parking spaces, large areas can be repurposed for housing, parks or recreational areas ~\cite{Stamadianos.2023}. Lastly, at the core of these advancements lies an increase in accessibility.  Autonomous ride-pooling systems lower barriers created by age, disability, or socioeconomic status and thereby allow large groups of society to participate in urban mobility that were previously excluded.~\cite{Hacohen.2022}.
However, autonomous ride-pooling is not without potential downsides. Job losses, especially for  human ride providers, the challenge of adjusting infrastructure to accommodate autonomous vehicles, and the need to build robust and safe systems are concerns that need to be addressed~\cite{Hacohen.2022}.

In conclusion, autonomous ride-pooling platforms represent technological advancements and can change our approach towards transportation~\cite{Shaheen.}, promising a more efficient, environmentally friendly, and inclusive transportation landscape~\cite{Stamadianos.2023}. However, the development of such platforms brings up the challenge of balancing the immense potential benefits with the inherent difficulties. 