 The advantages of autonomous driving, coupled with the growing significance of ride-pooling, promise to have transformative impact on urban mobility~\cite{Stamadianos.2023}. We dive into the intricacies of both concepts, exploring their origins, developments, and the potential synergy they hold for the future of transportation.

\subsection{Autonomous Driving}
Autonomous, or self-driving vehicles, blend hardware and software to navigate and control the car without human intervention~\cite{Stamadianos.2023}. Classified into levels 0 to 5, with five being fully autonomous, these vehicles rely on intricate systems of sensors, cameras, lidars, and radars. They constantly gather data about their environment, which is then processed by advanced algorithms to make driving decisions~\cite{Hacohen.2022}.
Historically, the concept of a car driving itself can be traced back to the 1920s, but tangible progress started in the latter half of the 20th century. Projects like the EUREKA Prometheus Project in the 1980s and the DARPA Grand Challenge in the early 2000s played essential roles in developing autonomous technologies~\cite{Hacohen.2022}. Today, significant tech and automobile companies compete to build fully autonomous vehicles for mass adoption~\cite{Stamadianos.2023}.
The potential benefits of autonomous driving are vast:
\begin{itemize}
    \item Safety: Human error, responsible for most road accidents, could be drastically reduced~\cite{Hacohen.2022}.
    \item Efficiency: Optimal driving by autonomous cars might reduce traffic congestion and lead to more streamlined traffic flows~\cite{Stamadianos.2023}.
    \item Accessibility: Those unable to drive due to age, disability, or other factors can enjoy independent mobility~\cite{Hacohen.2022}.
    \item Economic Impact: A reduction in accidents implies decreased costs in healthcare and vehicle repairs~\cite{Stamadianos.2023}.
\end{itemize}

However, challenges still exist. Technical complications, legal barriers, ethical questions (like decision-making in unavoidable accidents), and public scepticism must be addressed for a broader acceptance~\cite{Hacohen.2022}.

\subsection{Ride-Pooling}
Ride-pooling, distinct from ride-sharing, involves multiple riders sharing a single vehicle on a trip, where each passenger's destination is likely different, but their routes are similar~\cite{Perivier.}. Platforms like UberPool and Lyft Line have popularized ride-pooling in urban environments. The appeal of such services lies in their promise of reduced cost of travel for passengers, decreased overall traffic, lower carbon emissions, and the potential reduction of occupied parking spaces~\cite{Shaheen.}.
However, ride-pooling is not without its challenges. Efficient route optimization to ensure minimal detours, balancing demand and supply, and ensuring passenger safety are areas that ride-pooling providers constantly struggle with~\cite{Perivier.}.

The Synergy of Autonomous Driving and Ride-Pooling offer a promising vision of the future of urban mobility~\cite{Stamadianos.2023}.
When considering efficiency and cost, autonomous ride-pooling offers excellent advantages. The possibility of vehicles being available round the clock presents unprecedented efficiency. This constant operation reduces the per-trip cost, which is directly beneficial for passengers through lower prices. This efficiency is further increased by the absence of a driver, which decreases operating costs and provides additional vehicle space for additional passengers or cargo, increasing the utility of the vehicles~\cite{Hacohen.2022,Stamadianos.2023}.
Regarding environmental impact, the combination of electric and autonomous vehicles, when integrated with ride-pooling services, allows  for trips with reduced ecological impact. Fewer cars on the roads result in less pollution\cite{Hacohen.2022,Stamadianos.2023}. The concept of autonomous ride-pooling extends into urban planning as well.  Cities are currently designed with a focus on  moving vehicles and parking. This could undergo a significant transformation. With the reduced need for roads and parking spaces, there is potential for vast areas to be repurposed for green spaces, recreational areas, or new housing units~\cite{Stamadianos.2023}.
Lastly, at the core of these advancements lies an increase in accessibility and inclusivity.  Autonomous ride-pooling systems lower barriers created by age, disability, or socioeconomic status.~\cite{Hacohen.2022}.
However, the integration is not without potential downsides. Job losses, especially for drivers in the ride-sharing industry, the challenge of adjusting infrastructure to accommodate autonomous vehicles, and the need to build robust and safe systems are concerns that need to be addressed~\cite{Hacohen.2022}.

In conclusion, autonomous ride-pooling platforms represent technological advancements and can change our approach towards transportation~\cite{Shaheen.}, promising a more efficient, eco-friendly, and inclusive transportation landscape~\cite{Stamadianos.2023}. However, the development of such platforms brings up the challenge of balancing the immense potential benefits with the inherent difficulties. 