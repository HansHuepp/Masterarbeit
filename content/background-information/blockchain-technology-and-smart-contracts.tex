
The blockchain concept represents a technological breakthrough. The decentralised, immutable ledger  at the core of the technology allows blockchains to be utilised in a number of industries, including finance and supply chain management, where the integrity and immutability of information play an important role ~\cite{Zhou.2023}. The decentralised consensus approach of blockchain ensures that data modifications are only possible through the unanimous approval of all participating systems. This ensures that information that is written onto the ledger becomes immutable. To provide all the important information on blockchain for this research, we provide an overview of the current state of blockchain, explain the technical concepts behind blockchain in more detail, discuss the applications of smart contracts and take a closer look at the security of blockchain~\cite{Tran.2022b}.

\subsection{Introduction to Blockchain}
Blockchain technology as we know it today originated with the introduction of Bitcoin in 2008 ~\cite{Nakamoto.2009}. 
Satoshi Nakamoto, the pseudonymous individual or group behind Bitcoin, introduced the concept as a solution to the double-spending problem in digital currencies~\cite{Nakamoto.2009}. Digital currencies before Bitcoin faced the problem that it was very difficult to ensure that no token could be spent more than once. The solution to this problem presented by Nakamoto is a decentralised ledger, where every transaction gets verified by a network of nodes through a consensus mechanism ~\cite{Tran.2022b}. This technological breakthrough was groundbreaking because it allowed the creation of decentralised currencies that are not controlled by a single entity like a government.


One of the outstanding features of blockchain technology is its decentralisation~\cite{Gencer.2018}. Unlike traditional databases, such as an SQL database operated by a central entity, blockchains operate on a peer-to-peer network~\cite{Gencer.2018}. Every participant (or node) has access to the entire database and the complete history of all transactions. This means that no single participant has control over the data, and all participants collectively maintain the integrity of the data.

Immutability is another critical feature of blockchains. Once a transaction is recorded on the blockchain, it becomes extremely difficult to alter~\cite{Pilkington.}. This is because each block contains a cryptographic hash of the previous block, creating a chain of blocks~\cite{Pilkington.}. To change a single block, one would need to alter all subsequent blocks, which is computationally impractical, especially in large networks~\cite{ContedeLeon.2017}.

Transparency is inherently built into the system due to its open-source nature. Every transaction on the blockchain is visible to anyone who chooses to view it, ensuring full transparency in the network~\cite{Banupriya.2021}. However, personal information about the users conducting the transactions remains private as each user is commonly represented through some form of Public Key~\cite{Wei.2022}. This ensures a balance between transparency and privacy.

While the foundational principles of blockchains remain consistent, there are different types tailored to specific needs~\cite{Ghosh.2021}. Public blockchains, like Bitcoin and Ethereum, are open to anyone and are simply secured by their cryptographic algorithms~\cite{Ghosh.2021}. In contrast, private blockchains, like the Hyperledger Blockchain Projects, can be restricted to a specific group of participants, often used by businesses for internal processes. Private Blockchains can also be used as consortium blockchains or federated blockchains, operated under the leadership of a group ~\cite{Lu.2023}. They provide a balance between the openness of public blockchains and the restrictions of private ones.

\subsection{How Blockchain Works}
Diving deeper into the mechanics of blockchain technology shows the interplay of cryptographic principles, network theory, and consensus algorithms~\cite{Xiong.2022}. At the core of this technology are blocks, which are essentially records of transactions. Each block typically contains a timestamp, a reference to the previous block (known as the parent block), and a list of transactions. These transactions are represented as cryptographic hashes, which are fixed-size strings of characters generated from input data of any size. The advantage of these hashes is that even a small change in the input data results in a completely different hash, ensuring the integrity of transaction records that can be traced back to the very first block, known as the genesis block~\cite{Xiong.2022}.

Central to the operation of a blockchain is the concept of consensus mechanisms~\cite{Tahir.2022}. These are protocols that ensure all participants in the network agree on the validity of transactions. The most well-known consensus mechanism is \gls{pow}. In \gls{pow}, participants, often referred to as "miners", solve complex mathematical problems to validate transactions and create new blocks. This process requires significant computational power and energy. An alternative mechanism, \gls{pos}, determines the creator of a new block based on their stake or ownership of the cryptocurrency. It is seen as a more energy-efficient alternative to \gls{pow}. \gls{dpos} further refines this by allowing coin holders to vote for a few trusted nodes to validate transactions, streamlining the process and reducing the energy footprint~\cite{KUCHKOVSKY.2021}.

The blockchain network is maintained by nodes, which are computers participating in the network~\cite{Xiong.2022}. In general, there are two primary types of nodes: full nodes and light nodes~\cite{Mitra.2021}. Full nodes store the entire blockchain and validate all transactions and blocks. They serve as the network's backbone, ensuring data integrity and consistency. Light nodes, on the other hand, store only a subset of the blockchain and rely on full nodes for transaction validation and other heavy operations~\cite{Mitra.2021}. Their primary role is facilitating faster and more efficient interactions with the blockchain.

Transactions allow users to interact with the blockchain. Once a transaction is initiated, it is broadcast across the blockchain network and placed in a pool of unconfirmed transactions. Worker nodes then take these transactions from the pool and validate them against the ledger history to ensure that they are valid. If a transaction is determined as valid, it is placed in a block, together with other valid transactions. Once the block is full, it is shared with the network for verification through the consensus mechanism. After this, the block is added to the chain, and the transaction becomes a permanent part of the ledger history~\cite{Xiong.2022}.

\subsection{Smart Contracts}
As one part of blockchain technology, smart contracts have emerged as an advanced tool, extending the use-cases of blockchains beyond the record keeping of transactions~\cite{UchaniGutierrez.2023}. A smart contract is a self-executing contract where the terms of agreement or conditions are represented through written lines of code~\cite{Zhou.2022}. They are protocols that verify and enforce credible transactions without the need for third parties~\cite{Zhou.2022}. At their core, they are digital contracts that automatically execute actions when predefined conditions are met.

The concept of smart contracts is not new, but its practical application gained popularity with the advancements of blockchain technology~\cite{Pierro.}. Ethereum, launched in 2015, demonstrated the potential of smart contracts~\cite{Pierro.}. Ethereum's platform is designed specifically to create and execute smart contracts, providing a more flexible scripting language and a platform for creating a \gls{dapp}~\cite{Pierro.}. Since Ethereum's creation, a number of other blockchains have integrated smart contract capabilities, offering unique features and optimisations.

Once deployed, smart contracts operate without human intervention, ensuring that transactions are carried out correctly when conditions are met~\cite{UchaniGutierrez.2023}. This allows for interactions among parties that are not required to trust each other. Since the contract is on a blockchain, all parties can verify the contract's code and monitor its execution~\cite{UchaniGutierrez.2023}. The decentralised nature of blockchains also ensures that smart contracts are secure from tampering, providing an added layer of security~\cite{Zhou.2022}.

Understanding the life cycle of a smart contract provides insights into its operational model. The journey begins with its creation, where the contract's terms are defined and encoded. Once the code is written and tested, it is deployed onto the blockchain, appearing as an immutable part of the ledger~\cite{Tran.2022b}. After the deployment, the contract is now active and can start receiving and processing information. Execution occurs when the conditions specified in the contract are met, triggering the actions encoded in the contract~\cite{Pierro.}. While many smart contracts are designed to run without a predefined end, there are scenarios where they might have a termination condition, ending the contract's active state on the blockchain ~\cite{Tran.2022b}.

Even though smart contracts have many advantages, they also come with their own set of limitations and challenges ~\cite{.2019}. One notable challenge in the Ethereum network is the concept of Gas fees. Every operation, from contract deployment to execution, requires computational resources. Users pay for these resources using so-called Gas, and with increased network activities, these fees can rise. Scalability remains a concern as well. As more complex smart contracts and DApps are developed, there is a growing demand for blockchains to process more transactions per second without compromising on security or decentralisation ~\cite{Tran.2022b}. Lastly, smart contracts are only as good as the code they're written in. Coding errors or oversights can lead to vulnerabilities, potentially allowing malicious actors to exploit the contract~\cite{Zhou.2022}.

In conclusion, smart contracts represent a significant leap in how agreements and transactions can be managed on a decentralised network~\cite{.2019}. While they offer many advantages, it is necessary to consider their challenges to utilise their full potential~\cite{.2019}.

\subsection{Blockchain Security}
Blockchain's decentralised nature, which is often used for its resilience and transparency, also presents unique security challenges. One of the most discussed vulnerabilities is the 51 Percent attack. In such an attack, if a single entity gains control of more than half of the network's nodes, it can potentially double-spend coins and stop or reverse transactions. This undermines the trust and integrity of the blockchain. Similarly, Sybil attacks occur when a single party controls multiple nodes, aiming to flood the network with false transactions or undermine mechanisms that rely on redundancy and trust ~\cite{Singh.2021}.

Smart contracts have their own set of security concerns~\cite{Alkhalifah.2021}. Reentrancy attacks are a prime example, where an attacker drains funds from a contract by repeatedly calling its functions before the initial function call is completed~\cite{Alkhalifah.2021}. Issues like overflow and underflow, where variable values exceed their set limits, can also be exploited, leading to unintended consequences in contract execution~\cite{Guo.2022}. These vulnerabilities underscore the importance of rigorous code audits and testing before deploying smart contracts on a live network.

Most blockchains offer pseudonymity, where transactions are linked to a cryptographic address rather than personal identities. However, through analysis, patterns can emerge, potentially de-anonymizing users~\cite{Kus.2022}.

In essence, while blockchain offers robust security mechanisms at the core of its design, it is not impenetrable. As the technology matures, addressing these vulnerabilities will be an important part of ensuring its general adoption and trustworthiness.
